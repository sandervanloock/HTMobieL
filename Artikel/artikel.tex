%%%% ijcai11.tex

%TODO: hoogte onderaan kleiner maken
%TODO: use short names for frameworks throughout the paper

\typeout{IJCAI-11 Instructions for Authors}

% These are the instructions for authors for IJCAI-11.
% They are the same as the ones for IJCAI-07 with superficical wording
%   changes only.

\documentclass[a4paper]{artikel3}
% The file ijcai11.sty is the style file for IJCAI-11 (same as ijcai07.sty).
\usepackage{ijcai11}

% Use the postscript times font!
\usepackage{times}


% the following package is optional:
%\usepackage{latexsym} 

%%%%%%%%%%%%%% CUSTOM BEGIN %%%%%%%%%%%%%%

% remove if not a draft
\usepackage{draftwatermark}
\SetWatermarkColor[rgb]{0.94,0.94,0.94}

\usepackage[english]{babel}
\usepackage[utf8]{inputenc}
\usepackage[colorlinks]{hyperref}
\usepackage{multirow}
\usepackage{pgfplotstable} %nodig voor CSV to Latex
\usepackage{booktabs} % voor layout comparisontable
\usepackage{amsmath}

%\usepackage[hyphens]{url}


% captions
\usepackage{caption} 
\captionsetup{margin=10pt,font=small,labelfont=bf}

% engelse term die we niet vertalen naar het nederlands
\newcommand{\term}[1]{\emph{#1}}

% code commande
\newcommand{\code}[1]{\texttt{#1}}

% zodat we niet http:// staan hebben in onze tekst, maar de link wel werkt
\newcommand{\exturl}[1]{\href{http://#1}{#1}}
\renewcommand{\url}[1]{\href{#1}{#1}}


% paragraphs anders te feel white space
\usepackage{setspace}
\newcommand{\setspace}[0]{\vspace{2mm}}
\renewcommand{\paragraph}[1]{\setspace \noindent {\bf #1}  }
\newcommand{\framework}[2]{ \emph{#1 (\textbf{#2}): }} %enkel in usage stukje om framework in te leiden

\newcommand{\challenge}[1]{\paragraph{#1}}

\newcommand{\jqm}[0]{jQuery Mobile}
\newcommand{\jqma}[0]{jQM}
\newcommand{\st}[0]{Sencha Touch}
\newcommand{\sta}[0]{ST}
\newcommand{\kendo}[0]{Kendo UI}
\newcommand{\kendoa}[0]{Kendo}
\newcommand{\lungo}[0]{Lungo}
\newcommand{\lungoa}[0]{Lungo}
\newcommand{\tmp}[0]{The-M-Project}
\newcommand{\quo}[0]{QuoJS}
\newcommand{\moobile}[0]{Moobile}
\newcommand{\davinci}[0]{DaVinci}
\newcommand{\jqt}[0]{jQT}
\newcommand{\js}[0]{JavaScript}
\newcommand{\htc}[0]{HTCDesireZ}
\newcommand{\gtab}[0]{GalaxyTab}
\newcommand{\gs}[0]{GalaxyS}
\newcommand{\nexus}[0]{Nexus 7}
\newcommand{\ipadi}[0]{iPad1 WiFi}
\newcommand{\ipadiii}[0]{iPad3 4G WiFi}
\newcommand{\iphoneiii}[0]{iPhone 3GS}
\newcommand{\iphoneiv}[0]{iPhone 4S}
\newcommand*{\uit}[1]{\switch{#1}}
\newcommand*{\chal}[1]{\dothis{en}{#1}}

\usepackage{pdftexcmds}
\makeatletter
\newcommand{\switch}[1]{
  \ifnum\pdf@strcmp{#1}{anatomie}=0
    U1:~Anatomie van pagina%
  \else\ifnum\pdf@strcmp{#1}{toestel}=0
    U2:~Toestelspecifieke lay-out%
  \else\ifnum\pdf@strcmp{#1}{laadscherm}=0
    U3:~Laadscherm en dialoogvenster%
  \else\ifnum\pdf@strcmp{#1}{formulieren}=0
    U4:~Formulieren%
  \else\ifnum\pdf@strcmp{#1}{vullen}=0
    U5:~Automatisch invullen van formulier%
  \else\ifnum\pdf@strcmp{#1}{autoaanvullen}=0
    U6:~Auto-aanvullen%
  \else\ifnum\pdf@strcmp{#1}{afbeelding}=0
    U7:~Toevoegen van afbeelding%
  \else\ifnum\pdf@strcmp{#1}{validatie}=0
    U8:~Formuliervalidatie%
  \else\ifnum\pdf@strcmp{#1}{handtekening}=0
    U9:~Handtekening%
  \else\ifnum\pdf@strcmp{#1}{ajax}=0
    U10:~AJAX%
  \else\ifnum\pdf@strcmp{#1}{lijsten}=0
    U11:~Lijsten%
  \else\ifnum\pdf@strcmp{#1}{pdf}=0
    U12:~Toon PDF%
  \else\ifnum\pdf@strcmp{#1}{offline}=0
    U13:~Offline%
  \else
    [?]%
  \fi\fi\fi\fi\fi\fi\fi\fi\fi\fi\fi\fi\fi}
\makeatother

%%%%%%%%%%%%%% CUSTOM END %%%%%%%%%%%%%%

% Following comment is from ijcai97-submit.tex:
% The preparation of these files was supported by Schlumberger Palo Alto
% Research, AT\&T Bell Laboratories, and Morgan Kaufmann Publishers.
% Shirley Jowell, of Morgan Kaufmann Publishers, and Peter F.
% Patel-Schneider, of AT\&T Bell Laboratories collaborated on their
% preparation.

% These instructions can be modified and used in other conferences as long
% as credit to the authors and supporting agencies is retained, this notice
% is not changed, and further modification or reuse is not restricted.
% Neither Shirley Jowell nor Peter F. Patel-Schneider can be listed as
% contacts for providing assistance without their prior permission.

% To use for other conferences, change references to files and the
% conference appropriate and use other authors, contacts, publishers, and
% organizations.
% Also change the deadline and address for returning papers and the length and
% page charge instructions.
% Put where the files are available in the appropriate places.

%TODO toevoegen header dingen van gonzalo
\title{Comparative study of frameworks for \\ the development of mobile HTML5 applications}
\author{Tim Ameye \\ tim.ameye@student.kuleuven.be \And Sander Van Loock \\ sander.vanloock1@student.kuleuven.be}
\begin{document}

\maketitle

\begin{abstract}
The need to make mobile applications that run on every device is ever-growing.
One solution is to build mobile web applications using HTML5.
To speed up development, different frameworks exist that help in functionality and building a user interface.
Because of the variety of existing frameworks, a thoroughly comparison is needed to know which framework suits the job best.
This paper compares \st{}, \jqm{}, \kendo{} and \lungo{} by using popularity, productivity, usage, support and performance as comparison criteria. 
The conclusion is that \jqm{} is the best framework with great productivity and performance. 
\kendo{} gets the second place and its main asset is usage.
\lungo{} is third and has the fastest download performance.
\st{} ends last but gets the best user experience.
However all four frameworks have good support.
\end{abstract}

%%%%%%%%%%%%%%%%%%%%%%%%%%%%%%%%%%%%%%%%%%%%%%%%%%%%%%%%%%%%%%%%%%%%%%%%%%%%%%%%%%%%%%%
%%%%%%%%%%%%%%%%%%%%%%%%%%%%%%%%%%%%%%%%%%%%%%%%%%%%%%%%%%%%%%%%%%%%%%%%%%%%%%%%%%%%%%%

\section{Introduction} % (1 blz)
\label{sec:introduction}
Since the release of the Apple iPhone in 2007~\cite{David2011}, the demand for smartphones is ever since increasing. 
Today, over one billion smartphones are in use globally~\cite{Yang2012}.
This will double by 2015~\cite{Gillett2012}.
Also tablets conquer market share as mobile devices.
760 million tablets will be in use globally by 2016~\cite{Gillett2012}.
Besides Apple, other companies like Google and Microsoft are on the track too.
All those mobile devices come in different shapes and sizes.
They also include different features like GPS, camera and connectivity like Bluetooth, WiFi and 3G.

On these mobile devices run mobile operating systems~(OS). The two mobile OSs with the greatest market share are iOS and Android with 14.9\% and 75.0\% respectively in Q3 of 2012~\cite{Protalinski2012}.
These mobile OSs run both on smartphones and tablets.

The mobile OS from Apple, called iOS, first appeared in 2007.
Since then, a new version was released every year with the most recent version being iOS~6.1 which was released in January~2013~\cite{Deitel2012,PhilDutson2012,Apple2013}.
The mobile browser from Apple shipped with iOS is called Mobile Safari.
Apple did make sure to catch up with the latest HTML5 specifications in its browser~\cite{Hales2012}.
This was also because of the dropped support for Adobe Flash on their mobile devices~\cite{Jobs2010}.

In 2005, Google bought Android Inc. and released its first stable mobile OS, called Android, in 2008~\cite{Satyesh2012}.
Like with iOS, new versions were released on a yearly basis.
Android~4.2 is the latest version and was released in October 2012~\cite{Sawers2012}.
Android comes with the Android browser.
The implementation of HTML5 specifications has dragged, but as of Android 4.0, support is increasing~\cite{Hales2012}.
It is now also possible to install the mobile version of the Chrome desktop browser on devices with Android 4.0 and later.

It is important to look at the usage of the different OS versions, which is shown in table~\ref{table:mos-versions}.
This is because not every user has (updated to) the latest version on his mobile device.
The table shows that almost 90\% of iOS users already has version iOS~6.x.
In contrast, about a half of the Android users is still using Android~2.x while the latest major version is Android~4.x. 
Those users cannot benefit of the latest bells and whistles, particularly the new features of HMTL5, on their devices.

\begin{table}[t]
\centering
\pgfplotstabletypeset[
  begin table=\begin{tabular}{l l},
  end table=\end{tabular},
  col sep=comma,
  header=true,
  string type,
  skip coltypes=true,
  columns={Versie,Marktaandeel},
  columns/Versie/.style={column name=\textbf{iOS}},  
  columns/Marktaandeel/.style={column name=\textbf{Share (\%)}},
  every head row/.style={
    before row=\toprule,
    after row=\midrule},
  every last row/.style={
    after row=\bottomrule}
]{../Masterproef/tabellen/ios.csv}
\quad
\pgfplotstabletypeset[
  begin table=\begin{tabular}{l l},
  end table=\end{tabular},
  col sep=comma,
  header=true,
  string type,
  skip coltypes=true,
  columns={Versie,Marktaandeel},
  columns/Versie/.style={column name=\textbf{Android}},  
  columns/Marktaandeel/.style={column name=\textbf{Share (\%)}},
  every head row/.style={
    before row=\toprule,
    after row=\midrule},
  every last row/.style={
    after row=\bottomrule}
]{../Masterproef/tabellen/android.csv}
\caption{Usage of iOS versions on May 8, 2013 and Android versions on May 1, 2013 \protect\cite{Smith2013,Android2013}.}
\label{table:mos-versions}
\end{table}

There are three possible ways to make a mobile application~\cite{Accenture2012,Hales2012}.
The first one is a \emph{web application} which runs entirely in the browser.
The second type of application is a \emph{native application} which is installed on the device and programmed in a language specific to the mobile OS.
Lastly, a mix of both can be made which is called a \emph{hybrid application}.
One approach is to pack the web application as a native application.
The other approach is to use one programming language to build native applications for different OSs.

Advantages of web applications are the presence of a browser on mobile devices, so everyone can use it directly without installing the app first and regardless of their mobile OS~\cite{Accenture2012}.
Secondly, the code needs only to be written once and can be run anywhere (WORA)~\cite{Hales2012}.
This is in contrast to native, were a codebase has to be maintained for every mobile OS.
Lastly, web applications do not need to be published to a store (e.g. App Store for iOS and Google Play for Android) prior to release.
Such a store is a place where users can download application to their mobile device.
A disadvantage of web applications is that they are dependent on the mobile OS just like native applications.
This is especially the case when using new features of HTML5 in web applications.

In contrast, native applications provide better performance~\cite{Accenture2012}.
Secondly, it is easier to use features like GPS and camera of the mobile device with a native application.
Thirdly is a security issue were web applications lack behind.
Lastly, publishing applications through a store has two advantages.
On the one hand it increases publicity.
On the other hand stores can also control the quality of the applications, which is not the case for web applications.

The advantage of hybrid applications is that they patch the issues of web applications with the benefits of native applications.
However, as of December 2012, HTML5 went into candidate recommendation~\cite{Jacobs2012}.
This implies the upcoming support of HTML5 in contemporary browsers and that hybrid applications will probably become obsolete over time.

%%%%%%%%%%%%%%%%%%%%%%%%%%%%%%%%%%%%%%%%%%%%%%%%%%%%%%%%%%%%%%%%%%%%%%%%%%%%%%%%%%%%%%%
%%%%%%%%%%%%%%%%%%%%%%%%%%%%%%%%%%%%%%%%%%%%%%%%%%%%%%%%%%%%%%%%%%%%%%%%%%%%%%%%%%%%%%%

\section{Related Work} % Sander 0.25 pg
\label{sec:related-work}

%TODO related work buiten HTML5, native? anders literatuur buiten vergelijkingscriteria?

Many in-depth comparisons of mobile HTML5 frameworks already exist today.  
However, none of them are scientifically published or use a proof of concept (POC) to validate their comparison.  
The idea of using a POC is not new.  
Oehlman~\cite{Oeflman2011} and Kosmaczewski~\cite{Kosmaczewski2012},  for example,  use a small geosocial game and todo list application respectively to present mobile HTML5 frameworks.

Some blog posts \cite{Sarrafi2012a,Ayuso2012,Rozynski2011} all have their own criteria and methodologies to assess different mobile frameworks.  
The overall application of the criteria changes from use case to use case.  
Rozynski~\cite{Rozynski2011} presents the chosen criteria and discusses them for each framework.  
Ayuso~\cite{Ayuso2012} presents 17 criteria but all of them are discussed at once per framework.  
Thereafter,  advantages and disadvantages are subtracted and proposed to the reader.  
Finally, Sarrafi~\cite{Sarrafi2012a} presents their chosen criteria together with a scorecard and an explanation of scores per criterion.  
Each framework gets evaluated based on the scores for each criterion.

All these blog posts compare mobile and hybrid frameworks.  
Some websites~\cite{Bristowe2012,Burris} only focus on two mobile HTML5 frameworks in contrast to the Mobile Frameworks Comparison Chart from from Falk~\cite{Falk2011} that tries to compare as much as frameworks as possible in a large tabular form.

The report about Cross-Platform Developer Tools from Vision Mobile~\cite{Mobile2012} compares $15$ cross-platform tools.
One of these is Sencha.
The evaluation is based on a questionaire with more than 2400 deverlopers worldwide.
Their main focus lies on hybrid applications because they state that hybrid applications try to combine the best of both web and native applications.
This paper avoids this discussion and tries to find the best framework for building web applications.

The ISO 25010 standaard evaluates the productquality and quality in use of software~\cite{Standard2010}.
HTML5 frameworks could also be assessed with this standard.
The general methodology it uses is not applicable for HTML5 frameworks.
Security,  for example,  is not an issue for framework vendors.


%%%%%%%%%%%%%%%%%%%%%%%%%%%%%%%%%%%%%%%%%%%%%%%%%%%%%%%%%%%%%%%%%%%%%%%%%%%%%%%%%%%%%%%
%%%%%%%%%%%%%%%%%%%%%%%%%%%%%%%%%%%%%%%%%%%%%%%%%%%%%%%%%%%%%%%%%%%%%%%%%%%%%%%%%%%%%%%

\section{Comparison criteria} %Sander 0.75 pg
\label{sec:comparisoncriteria}
%TODO nieuwe criteria ipv oude, wel kort zeggen welke dat we eerst initieel gingen doen

%TODO waarom criteria niet gekozen

The comparison study will be driven by a Proof of concept (POC).  
This POC encompasses an application to let employees attach, sign and submit expenses to their employer via tablet or smartphone.
Afterwards, the employee can view the uploaded expenses and download them as PDF.

Five important criteria will be used:  popularity,  productivity,  usage,  support and performance. 
The POC can be used in all but the first.   
A methodology that results in a score for each criterion will be described below.  
Per criterion,  a table or graph will be provided so give an overview of the total scores
At the final stage,  these scores will be plotted in a spider graph~\cite{Few2005}.   

\paragraph{Popularity}
%TODO Tim: oude tekst
%The activity around frameworks in social media has to be considered with a lot of care.  
%If,  for example,  a framework has a lot of questions on Stack Overflow,  one could conclude the framework has a lot of ambiguity.  
%Someone else could argue  the framework has a large active Internet community build around it.  
%Also,  the numbers of Twitter followers and Facebook likes only give an indication of the popularity of the framework,  there is no exact science involved.

Social networks provide a good indication to determine the popularity that goes with the framework.  
%\cite{Sarrafi2012a,Ayuso2012} use 
Twitter followers ($T$),  GitHub stars ($S$),  GitHub forkers ($F$),  Stack Overflow questions ($S$) and Facebook likes ($FB$) will be looked at.
The summation of these five numbers give the score for the framework's popularity.  

% \begin{equation}
%   POP_f = T_f+S_f+F_f+SO_f+FB_f
%   \label{eq:populariteit}
% \end{equation}

\paragraph{Productivity}
Both autors implement the full POC in a two different framework and carefully record the time to implement it.  
However,  this time will be influenced by many factors.
The second implementations will have an advantage because of the gained experience after the first implementation.
Also,  code reuse and possible incomplete implementations because of the limited functionality of the framework, affect this times in a negative way.
A better criterion will be the hours to partially implement the POC in the other's framework.  
The partial implementation contains the POC's login that loads a long list.%, just like~\cite{Burris}.   
This application is not subjected to the influences described above. %TODO verder verklaren?
The time to implement the login application ($t_{login}$) gives the score for the framework's productivity. 

% \begin{equation}
%   PROD_f = t_{f,login}
%   \label{eq:productiviteit}
% \end{equation}

\paragraph{Usage}
The POC will be subdivided in $13$ challenges and $38$ subchallenges.  
The score of this criterion will depend on the completion of the challenges by the framework.  
The assessment of the challenges can be found in table~\ref{table:challenges-scores}.  
The total of all challenges give the score for the framework's usage.

\begin{table}[h]	
\centering
\begin{tabular}{ll}
\toprule
\textbf{Score} & \textbf{Assessment criteria}\\
\midrule
$C_{f,i} = 2$ & Supported by the framework\\
$C_{f,i} = 1$ & A plugin or HTML5 feature is needed\\
%TODO: beter woord vinden
$C_{f,i} = 0$ & No, custom or hacky implementation\\
\bottomrule
\end{tabular}
\caption{Assessment criteria for implementation challenges.}
\label{table:challenges-scores}
\end{table}

\begin{equation}
  \text{US}_f = \sum_{i=1}^{38}{\left(C_{f,i}\right)}
  \label{eq:gebruik}
\end{equation}

\paragraph{Support}
This criterion indicates the correct functionality of the POC implementation on different contexts.
A context is a combination of a particular operatingsystem, device and browser.
Both Android and iOS, smartphones and tablets will be checked.
Only native browsers will be looked at.
In total eight contexts will be checked for support.
%,  similar as the cross-platform capabilities of~\cite{Sarrafi2012a}.  
In each context,  a subset of the challenges from the previous criterion will be checked.  
These are the challenge were the framework actually provided an implementation ($C_{r,i} > 0$).
If the challenge works as expected $C_{f,c,i} = 1$.
If some part of challenge is infeasible $C_{f,c,i} = 0$.
The summation of scores for each context will give the score for the framework's support.

\begin{equation}
  \text{SU}_f = \sum_{c=1}^{8}{\left(\sum_{i=1}^{N_f}C_{f,c,i}\right)}
  \label{eq:ondersteuning}
\end{equation}

\paragraph{Performance}
The loading time of an HTML5-application is a major bottleneck for performance.
Therefor,  the loading time of the POC ($l_{f,c,POC}$) and login application ($l_{f,c,login}$) will be measured on eight devices.
Also,  the loading time for a cached version of both the POC ($l_{f,c,POC_{cache}}$) and login application ($l_{f,c,login_{cache}}$) will be recorded.
The summation of loading times will give the score for the framework's performance.
The average performance becomes:


\begin{equation}
\begin{split}
\text{AvgDownloadTime}_f =  \\ \frac{\sum_{c=1}^{8}\left(l_{f,c,POC}+l_{f,c,POC_{cache}}+ l_{f,c,login} + l_{f,c,login_{cache}}\right)}{8}
  \label{eq:dowloadtime}
\end{split}
\end{equation}

The responsetime of a list with $850$ elements will be measured on each context with two user tests ($\text{experience}_{f,c}$).
The result will be a total ordening of the frameworks per context that indicate the responsiveness.
The summation of the results will provide a factor to weight the average download time.
The formula for performance then becomes:

\begin{equation}
  \text{PERF}_f =  \frac{\text{AvgDownloadTime}_f}{\sum_{c=1}^{8}{\text{experience}_{f,c}}}
  \label{eq:performance}
\end{equation}

%%%%%%%%%%%%%%%%%%%%%%%%%%%%%%%%%%%%%%%%%%%%%%%%%%%%%%%%%%%%%%%%%%%%%%%%%%%%%%%%%%%%%%%
%%%%%%%%%%%%%%%%%%%%%%%%%%%%%%%%%%%%%%%%%%%%%%%%%%%%%%%%%%%%%%%%%%%%%%%%%%%%%%%%%%%%%%%

\section{Frameworks} % (0.5 blz per framework + tabel)
\label{sec:frameworks}
This section introduces the compared mobile HTML5 frameworks: \st{}, \kendo{}, \jqm{} and \lungo{}.
%TODO: tekst over de 3 mogelijkheden om webapps te maken hier naar boven plaatsen?

\subsection{\st} % (0.5 blz Sander)
\label{sec:frameworks-sencha-touch}

\st{}~(\sta{}) is a framework developed by Sencha,  a company founded in 2010 as a composition of Ext JS, jQuery Touch and Raphaël.  
Ext JS is a JavaScript framework for the development of HTML5 applications.  
jQuery Touch is a jQuery plugin for mobile development that adds touch events to jQuery and depends on the WebKit engine.  
Finally,  Raphaël, is a JavaScript library for vector drawings.  
Parts of the first two technologies can be found in the implementation of the \sta{} framework.    

As at the time of writing,  \sta{} is at version 2.2.0~\cite{Inc.}.

\paragraph{Documentation}
%TODO: url verwijderen
All documentation for \sta{} can be found at \exturl{docs.sencha.com/touch/2-0}.  
The most important features are provided with code examples and live previews.
The key concepts of \sta{} are explained in extensive tutorials:  some are texts, some are videos.  

Another handy tool to discover the \sta{} features is the Kitchen Sink~\cite{Inc.2013}.  
This is a HTML5-application that lines up all possibilities of the framework combined with the corresponding code.

\paragraph{Licence}
\sta{} is free within a commercial context in which the developer does not share the code with its users.  
There is also the option to use an open-source version.  
This comes with a GNU GPLv3 license which implies a free code redistribution as most important property.
Other licenses can be found at~\cite{SenchaInc.}.
  
\paragraph{Code and development}
\sta{} is written on top of Ext JS,  and can also be considered as a JavaScript framework.  
All code needs to be written in JavaScript and loaded by one HTML-container.  
Another important aspect of \sta{} is that it supports the Model-View-Controller (MVC) pattern.  
Models group fields to data objects,  views define how the content is presented to the user and controllers connect these based on events.

\sta{} contains all UI-elements as JavaScript objects.  
Just like object-oriented programming,  those objects are part of a class system.  
Classes can be defined and/or created.  
Single inheritance and overriding is also possible.    

To enhance performance,  it is the programmer's task to create components before they are used.  
In this way,  the programmer partially determines the performance of the application.

\paragraph{Browser support}
Just like jQuery Touch,  \sta{} is based upon the WebKit browser engine.  
This forms the major requirement for browser support.  
Although most mobile browsers contain this engine,  some like FireFox Mobile and Opera Mobile lack behind~\cite{JohnEClark2012}.  
The next release of the Opera browser will contain the WebKit engine~\cite{Wokke2013}, a trend that most browser vendors will (have to) follow.

The framework offers the programmer methods to ask for the current context where the end-user is working in.
This functionality is comparable with Modernizer~\cite{Modernizr2012}.  

%%%%%%%%%%%%%%%%%%%%%%%%%%%%%%%%%%%%%%%%%%%%%%%%%%%%%%%%%%%%%%%%%%%%%%%%%%%%%%%%%%%%%%%

\subsection{\kendo} % (0.5 blz Sander)
\label{sec:frameworks-kendo}
%TODO \kendob afkorting gebruiken

\kendo{} (\kendob{}) is a framefrom from Telerik and consists of three layers:  \kendo{} Web,  \kendo{} Mobile and \kendo{} DataViz.
The first focusses on the development of webapplications,  the second makes them mobile and the last facilitates data representation.
\kendo{} is an hybrid form between JavaScript and markupdriven.
It is build upon the MVVM (Model-View-ViewModel) architecture and requires the jQuery library.

As at the time of writing,  \kendo{} is at version 2013 Q1~\cite{Telerik}. 

\paragraph{Documentation}
%TODO: url verwijderen
The documentation can be found at \url{http://docs.kendoui.com}~\cite{Telerikd}.
Two important sections are the API and Getting Started.
Both sections follow the same structure and each feature from the API is explained in detail in the Getting Started section.
Also,  every feature can be viewed in a demo with the corresponding code.

\paragraph{Licence}
A licence for \kendo{} Complete costs $\$699$ per developper.
Each layer can be bought separately for a different license fee.
\kendo{} also provides server-side wrappers that can automatically create client-side JavaScript code.
As at the time of writing, PHP,  JSP and ASP.NET MVC are supported as server-side technologies.
A license for a server-side wrapper costs $\$300$~\cite{Telerik}.

\paragraph{Code and development}
Both JavaScript and HTML-code needs to be writter because \kendo{} is both JavaScript and markupdriven.

\kendo{} is dependent on the jQuery library.
Just like \jqm{} there are tree strategies for building webapplications:  single-page application,  view per page or a combination of both.

The MVVM architecture is imposed by \kendo{}.
The views and models are similar with the views and models of the MVC architecture from \st{}.
The view-model is an object that binds views with models in both directions and is called \code{ObservableObject} in \kendo{}.
The binding is annotated in HTML with data tags.
\kendo{} supports bindings based on different properties from JavaScript objects.
Examples of the properties are \code{checked,  value} or \code{visible}.
If a view is bound to a model,  the model is changed when a user changes the content of a view or the view is changed when the model is programmatically updated.

\paragraph{Browser support}
One of the most important aspects of \kendo{} is the mimicking of the native look-and-feel of the current operatingsystem.
Supported platforms are iOS,  Android,  BlackBerry and Windows Phone~8.

All widgets are from \kendo{} Web support progressive enhancement.
Also,  the features of \kendo{} Mobile that depend on HTML5 support progressive enhancement.
The styling of forms,  for example,  will still work on older platforms, but the functionality will be limited to text input only.

%%%%%%%%%%%%%%%%%%%%%%%%%%%%%%%%%%%%%%%%%%%%%%%%%%%%%%%%%%%%%%%%%%%%%%%%%%%%%%%%%%%%%%%


\subsection{\jqm{}} % (0.5 blz Tim)
\label{sec:jqm}

\jqm{}~(\jqma{}) is a mobile HTML5 user interface~(UI) framework that was announced in 2010~\cite{Resig2010}. 
In November 2011 version~1.0 was released~\cite{Parker2011} and one year later in October, version~1.2 was released~\cite{Parker2012}. 
As at the time of writing, \jqma{} released version~1.3 \cite{Parker2013a}.
The framework is controlled by the jQuery Project that also manages jQuery Core. 
The latter is a JavaScript library where \jqma{} is dependent on~\cite{JQuery2012}. 
\jqma{} is among other things sponsored by Adobe, Nokia, BlackBerry and Mozilla~\cite{JQuery2012a}.

\paragraph{Licence}
As of September 2012 it is only possible to use \jqma{} under the Massachusetts Institute of Technology~(MIT) licence~\cite{Dmethvin2012}. 
This means that the code is released as open source and can also be used in proprietary projects~\cite{PhilDutson2012}.

\paragraph{Documentation}
The documentation site of version~1.2~\cite{JQuery2012b} is a catalog of all possible elements that \jqma{} offers. 
It contains an overview of all possible UI components. 
By checking the source code, you can find out what code to write to get the same result. 
Furthermore it explains the API on how to configure defaults, use events, methods, utilities, data attributes and theme the framework \cite{JQuery2012b}.

\paragraph{Code and development}
\jqma{} is a mark-up driven UI framework and thus provides mainly UI components. 
\jqma{} provides 6 categories of components: pages and dialogs, toolbars, buttons, content formatting, form elements and litsviews~\cite{JQuery2012b}. 
One can obtain these components by writing HTML5 with \jqma{} specific \code{data-}* attributes. 
When running the application, \jqma{} will add the extra necessary code to correctly show these components by doing progressive enhancement.

There are three ways of writing a HTML5 application in \jqma{}~\cite{Broulik2012}. 
The first one is to write the full application on one single web page.
The advantage is that there are initially less requests to the server.
The second option is to write a web page for each screen. 
The advantage here is that the first viewed screen is downloaded more quickly. 
However, with each transition, the next screen has to be fetched which can delay navigation.
This is done with AJAX by default in \jqma{}.
Lastly, you can mix the two above to find an optimum by putting the most likely viewed screens on one web page and the less likely viewed on separated pages.  

\paragraph{Browser support}
\label{sec:jqm-browser-support}
\jqma{} divides browsers into three grades: A, B and C. 
In an A-graded browser, the application is fully enhanced with AJAX based animated page transitions.
In a B-graded browser, the application has an enhanced experience, but there are no AJAX navigation features.
Lastly in a C-graded browser, the application has a basic, non-enhanced HTML experience, but is still functional~\cite{JQuery2012d}.

%%%%%%%%%%%%%%%%%%%%%%%%%%%%%%%%%%%%%%%%%%%%%%%%%%%%%%%%%%%%%%%%%%%%%%%%%%%%%%%%%%%%%%%

\subsection{\lungo} % (0.5 blz Tim)
\label{sec:frameworks-lungo}

\lungo{} is a mark-up drive framework of which version~1.0 appeared in 2011~\cite{TapQuo2011}.
The framework is maintained by the Spanish company TapQuo that is specialised in mobile user experience~\cite{TapQuo2013a}.
\lungo{} depends on  \quo{} which is a \js{} library optimised for mobile.
%\lungo{} biedt vooral GI-elementen aan, maar daarnaast zijn er ook \term{wrappers} voor cache, opslag en SQL beschikbaar~\cite{TapQuo2013}.
%Er wordt geen programmeerstijl zoals MVC afgedwongen.
As of the time of writing, \lungo{} is at version~2.1~\cite{TapQuo2013}.

\paragraph{Documentation}
The documentation site~\cite{Lungo2013} first shows how a typical \lungo{} application looks like.
There are eight other pages explaining briefly the various UI components and API functionality.
On these pages the source code is shown for each example.
One can also view a live preview of each example.
The documentation site~\cite{TapQuo2013c} of the underlying \js{} library exists of only one page with a very brief summary of the API.

\paragraph{Licence}
The framework is released by the GPLv3 licence.
A commercial licence is also available, but because of the lack of information, the authors contacted TapQuo via email to obtain more info.
However as of writing, they did not receive any answer.

\paragraph{Code and development}
Programming a \lungo{} application is done by annotating the HTML code with CSS classes and \code{data-*} attributes.
There is no architecture like MVC enforced by the framework.
To manipulate the DOM, the developer has to use \quo{}.
This \js{} library that is optimised for mobile does not contain methods for desktop users therefore making it smaller than traditional \js{} libraries like jQuery.
The library does not enforce an architecture.

One can write a single page web application or create multiple pages for each screen.
The \code{<article>} and \code{<section>} HTML5 tags are used to separate the different screens of the application.
\lungo{} supports a multipage web application by providing an asynchronous loader when initializing the application.
Only the code between the \code{<body>} tags of the screens need to be saved on different web pages.

\paragraph{Browser support}
The framework indicates on their website to have support for iOS, Android, Blackberry and FirefoxOS.
They also want to have a consistent experience across mobiles, tvs and desktops~\cite{TapQuo2013}. 

%%%%%%%%%%%%%%%%%%%%%%%%%%%%%%%%%%%%%%%%%%%%%%%%%%%%%%%%%%%%%%%%%%%%%%%%%%%%%%%%%%%%%%%
%%%%%%%%%%%%%%%%%%%%%%%%%%%%%%%%%%%%%%%%%%%%%%%%%%%%%%%%%%%%%%%%%%%%%%%%%%%%%%%%%%%%%%%

%TODO enkel nieuwe evaluatie, niet de oude
\section{Evaluation}
This section evaluates the four mobile HTML5 frameworks by five criteria, namely popularity~(\ref{sec:evaluation-popularity}), productivity~(\ref{sec:evaluation-productivity}), usage~(\ref{sec:evaluation-usage}), support~(\ref{sec:evaluation-support}) and performance~(\ref{sec:evaluation-performance}).

\subsection{Popularity} % (0,5 blz)
\label{sec:evaluation-popularity}

\begin{table}[t]
\centering
\resizebox{\columnwidth}{!} {
\pgfplotstabletypeset[
  col sep=comma,
  string type,
  header=true,
  columns={Popularity,ST,Kendo,jQM,Lungo},
  columns/Popularity/.style={column name=\textbf{Popularity}, column type={l}},  
  columns/ST/.style={column name=\textbf{\sta}, column type={c}},
  columns/ST/.style={column name=\textbf{\sta}, column type={c}},
  columns/jQM/.style={column name=\textbf{\jqma}, column type={c}},
  columns/Lungo/.style={column name=\textbf{\lungoa}, column type={c}},
  columns/Kendo/.style={column name=\textbf{\kendoa}, column type={c}},
  every head row/.style={
    before row=\toprule,
    after row=\midrule},
  every last row/.style={
  	before row=\midrule,
    after row=\bottomrule}
]{../Masterproef/tabellen/populariteit.csv}
}
\caption{Popularity for \st{}~(\sta), \kendo{}~(\kendoa), \jqm{}~(\jqma) and \lungo{}~(\lungoa).}
\label{tabel:evaluatie-popularity}
\end{table}

The scores for popularity can be found in table~\ref{tabel:evaluatie-popularity}. 
\kendob{} takes the first place due to the large amount of \fb{} likes.
\jqma{} and \sta{} take respectively second and third place despite the fact that these two framework are the most popular in literature~\cite{David2011,Firtman2013,Hales2012,Oeflman2011}. 
The last place goes to \lungo{} with a remarkable low popularity on \so{} and \fb.
When looking at the total, there are two groups of frameworks.
On the one hand there is \kendob{} together with \jqma{} and the other hand there is \sta{} together with \lungo{}.

\jqma{} has the most followers on Twitter, followed by \kendob.
\lungo{} is the penultimate, but the ratio tweets to followers indicate that it is the most active one. 
\jqma{} and \kendob{} have a similar ratio.
This is in contrast to \sta{} which has send only one tweet and also has the least number of followers. 

Only \jqma{} and \lungo{} are on \gh{}. 
If the \gh{} stars and \gh{} forkers are left out of popularity, the ranking stays the same.

\kendob{} references in the support menu on their website directly to \so{}. 
However the popularity of \kendob{} is lower than \jqma{}.
\sta{} is the penultimate, but more surpassingly is that \lungo{} only has about 30 questions on \so{} and therefore takes the last place.

\kendob{} and \jqma{} both have created a fan page on \fb{} in respectively November, 2011 and August, 2010.
The fan page of \kendob{} thus has gained more \fb{} likes in a shorter period than the earlier created fan page of \jqma{}.
An explanation is that the different layers of \kendo{} are aggregated on one fan page.
This is in contrast to \jqma{} where a separate fan page exists for jQuery that has about 53,000 likes.
This number is similar to that of \kendo{}.
%TODO interessepagina in het engels
\sta{} and \lungo{} only have a \term{interessepagina} on \fb.
This explains the great difference in \fb{} likes.
If the likes are left out of popularity, \kendob{} and \jqma{} switch places.

%%%%%%%%%%%%%%%%%%%%%%%%%%%%%%%%%%%%%%%%%%%%%%%%%%%%%%%%%%%%%%%%%%%%%%%%%%%%%%%%%%%%%%%

\subsection{Productivity} % Sander 0,5 blz
\label{sec:evaluation-productivity}

% TODO nieuwe cijfers vermelden en nieuwe data

This section will investigate the productivity of the different frameworks based on the time to implement the POC and the login application.
Table~\ref{tabel:evaluatie-productiviteit} contains the scores for this criterion.
The data shows that \lungo{} is the obvious winner,  followed by \kendo{} en \jqm{}.
\st{} appears to be the least productive.

\begin{table}[t]
\centering
\pgfplotstabletypeset[
  col sep=comma,
  string type,
  header=true,
  columns={Productivity,ST,Kendo,jQM,Lungo},
  columns/Productivity/.style={column name=\textbf{Productivity}, column type={l}},  
  columns/ST/.style={column name=\textbf{\sta}, column type={c}},
  columns/ST/.style={column name=\textbf{\sta}, column type={c}},
  columns/jQM/.style={column name=\textbf{\jqma}, column type={c}},
  columns/Lungo/.style={column name=\textbf{\lungoa}, column type={c}},
  columns/Kendo/.style={column name=\textbf{\kendoa}, column type={c}},
  every head row/.style={
    before row=\toprule,
    after row=\midrule},
  every last row/.style={
  	before row=\midrule,
    after row=\bottomrule}
]{../Masterproef/tabellen/productiviteit.csv}
\caption{Productivity for \st{}~(\sta), \kendo{}~(\kendoa), \jqm{}~(\jqma) and \lungo{}~(\lungoa).}
\label{tabel:evaluatie-productiviteit}
\end{table}

Two important factor need to be taken into account when analyzing this data.
First,  the implementations of the POC and login application were first executed with \jqm{} and \st{}.
The gained experience helped the developpers in the second implementation phase.
Also,  problems with the backend were resolved in the first implementation phase.

The imposed architecture with \st{} and \kendo{} increases the learning curve.
Also, the fact that a framework is JavaScript driven complicates the implementation.
This is why \st{} is the least productive.

There are also other factor that influence the productivity.
Firstly,  tools can speed up the production process.
Sencha provides Sencha Architect,  a graphical environment to visualy build applications from the ground up.
Version 2.1 of Sencha Architect was used to build the views but was not of much use to build more sophisticated functionality.
Secondly,  boilarplate code can be used to initialize a new project.
All frameworks beside \lungo{} describe such code in their documentation.
\st{} even offers Sencha Cmd to automate this process.
Also, the quality and quantity of the documentation influence the productivity.
A well-engineered search functionality,  real-time code examples and structure classification is crucial for a good documentation.
\kendo{} is the only framework that which documentation has all three components.  
An other factor is the available literature.
Safari Books Online contains $12$ books that specifically discuss \jqm{},  \lungo{} has no books at all.
The final factor is the different places where a developper can go to ask questions.
Stack Overflow or fora are examples of such places. %TODO fora?
As presented in the previous criterion,  \jqm{} expells in the amount of Stack Overflow questions

%%%%%%%%%%%%%%%%%%%%%%%%%%%%%%%%%%%%%%%%%%%%%%%%%%%%%%%%%%%%%%%%%%%%%%%%%%%%%%%%%%%%%%%

\subsection{Usage}
\label{sec:evaluation-usage}
In this section the usage is evaluated by evaluating 13 challenges.
The total score per challenge can be found in table~\ref{tabel:evaluatie-usage}.

\begin{table}
\centering
\resizebox{\columnwidth}{!} {
\pgfplotstabletypeset[
  col sep=comma,
  string type,
  header=true,
  columns={Challenge,Max,ST(abs),Kendo(abs),jQM(abs),Lungo(abs)},
  columns/Challenge/.style={column name=\textbf{Challenge}, column type={l}},
  columns/Max/.style={column name=\textbf{Max}, column type={l}},    
  columns/jQM(abs)/.style={column name=\textbf{\jqma}, column type={c}},
  columns/ST(abs)/.style={column name=\textbf{\sta}, column type={c}},
  columns/Lungo(abs)/.style={column name=\textbf{\lungoa}, column type={c}},
  columns/Kendo(abs)/.style={column name=\textbf{\kendoa}, column type={c}},
  every head row/.style={
    before row=\toprule,
    after row=\midrule},
  every last row/.style={
  	before row=\midrule,
    after row=\bottomrule}
]{../Masterproef/tabellen/gebruik.csv}
}
\caption{Usage for \st{}~(\sta), \kendo{}~(\kendoa), \jqm{}~(\jqma) and \lungo{}~(\lungoa).}
\label{tabel:evaluatie-usage}
\end{table}

The best framework in usage is \kendob{}, followed closely by \sta{}.
This can be explained because they both enforce an architecture, namely MVVM and MVC respectively.
The lack of enforcing an architecture by the two other frameworks result in a laborious approach, where points are lost due to the lack of features by the framework.
Both \kendob{} and \sta{} are controlled by a company and because of the fact that the POC is a typical application for a company, this can also explain the good results.
\kendob{} obtains the maximum score for \chal{formulieren}, but has a smaller score for \chal{offline} than \sta.

\jqma{} has a bit more than half of the maximum score.
It gets 0 points on \chal{vullen} and \chal{lijsten}.
This is because of the lack of enforcing an architecture, the developer has to do those things manually. 

The last place goes to \lungo{} that fails the usage criterion.
The same problems of \jqma{} are also applicable for \lungo{}.
Furthermore it gets a zero for \chal{validatie} and also has problems with the more advances form elements in \chal{formulieren}.
One has to have the luck of finding a plugin or implement the functionality by hand.

\kendob{} and \sta{} both have six perfect challenges.
\kendob{} fails on two challenges, while \sta{} only fails on one.
\jqm{} has three perfect challenges and fails on four.
\lungo{} has only one perfect challenge and fails on seven.

It can be generally stated that each of the four frameworks have great support for \chal{ajax}, except for one case particular case with \lungo{}.
Also \chal{laadscherm} is fully supported by each framework.

%%%%%%%%%%%%%%%%%%%%%%%%%%

%\challenge{\chal{toestel}}
%\begin{itemize}
%  \item C2.1 Recognising smartphone or tablet
%  \item C2.2 Show left sided menu on tablet 
%  \item C2.3 Enabling smartphone menu.
%\end{itemize}
%
%\framework{\sta}{}
%Detection of a smartphone context is done via the \code{Ext.os.is.Phone} method.  
%If this method returns false,  we assume to be in tablet mode.    
%The main screen of the POC requires a splitted view in tablet mode.  
%A \code{vbox} layout splits the viewport with a vertical axis.  
%The \code{flex} property defines the ratio of the sizes of both resulting components. 
%In smartphone mode,  the left screen is made invisible and an extra menubutton in the header is created.  
%Making the sub header clickable is not possible  
%
%\framework{\kendoa}{}
%\kendo{} also provides methods to query the context.
%\code{kendo.support.tablet},  for example,  returns if a tablet is being used.
%The splitview for the tablet layout is supported by \kendo{} and can be used by defining your view as \code{splitview} as value of the data attribute \code{data-role}.
%The smartphone layout needs to be build by manipulating the DOM with jQuery methods.
%
%\framework{\jqma}{}
%No functionality is provided by the framework, so a search for plugins was started~\cite{Deering2012}.
%However it was shown that the use of CSS3 media queries can accomplish the same without a plugin~\cite{Hadlock2012}.
%The documentation of \jqma{}~1.2 also used a similar approach~~\cite{JQuery2012b}.
%This approach is also encouraged in the documentation of \jqma{}~1.3~\cite{JQuery2013e}.
%The smartphone menu is always accessible on both tablet and smartphone devices when tapping on the sub header.
%
%\framework{\lungoa}{}
%The framework has the function \code{Lungo.Core.isMobile()} to know if the current device is a mobile device or not.
%\quo{} has extra attributes for \code{browser}, \code{os.name}, \code{os.name}, \code{env.os.version} and \code{screen}.
%The latter only gives the width and height, so the developer still needs to program it manually.
%As a solution, the same method as for \jqma{} was used, namely CSS3 media queries.
%The smartphone menu is always accessible on both tablet and smartphone devices when tapping on the sub header.
%
%%%%%%%%%%%%%%%%%%%%%%%%%%%
%
%\challenge{\chal{formulieren}}
%
%\begin{itemize}
% \item \chal{formulieren-placeholders}
% \item \chal{formulieren-types}
% \item \chal{formulieren-optieveld}
% \item \chal{formulieren-datepicker}
% \item \chal{formulieren-monthpicker}
% \item \chal{formulieren-schakelaar}
% \item \chal{formulieren-wissen}
%\end{itemize}
%
%\framework{\sta}{}
%Placeholders, text, email and number fields are supported by the framework and can be easily created.  
%Labels can be avoided by not defining them.  
%To create an option field, an element with type selectfield has to be included.
%Creating custom datepickers is not supported.  
%It is impossible to ignore the days field and only years can be delimited.  
%A flipswitch is possible with the togglefield type.
%Clearing the form after it was send, has to be programmed manually.
%
%\framework{\kendoa}{}
%All form types need to be defined with HTML-attributes and follow the HTML-standard.
%A placeholder can be defined and labels can be omitted.
%%TODO option fields? :Option fields have 
%A datepickers is contained within \kendo{} Web and can be included as widgets.
%The fields and depth of the datepicker can be configured to provide the custom datepickers.
%The \code{data-role} switch can initiate a flip switch.
%Clearing a model that is bound to a form (view) will clear the form automatically.
%
%\framework{\jqma}{}
%Placeholders can be achieved with the HTML5 \code{placeholder} attribute.
%Labels are mandatory in jQM for assistive technologies, but can be hidden via the \code{ui-hide-label} class~\cite{JQuery2013}. 
%The types of form elements used were text, number and email.
%The \code{date} type could not be used, because of the lack of support in mobile browsers~\cite{Deveria2013b}.
%Also the need to specify a range the user can choose from, justifies the use of the Date \& Time Picker of Mobiscroll~\cite{Mobiscroll2013}.
%The need for a custom datepicker with only a month and year needed to be hardcoded.
%Clearing the form after it was send, has to be programmed manually.
%A switch and option field are directly supported by \jqma.
%
%\framework{\lungoa}{}
%Adding placeholders is accomplished by using the HTML5 \code{placeholder} attribute.
%Labels are not mandatory in \lungo{}, so they could be left out of the code.
%
%%De types \code{text}, \code{number} en \code{email} voor formuliervelden worden verwezenlijkt door deze als type voor de \code{input}-tags mee te geven in het formulier.
%%Gelijkaardig met de twee aangehaalde problemen voor \jqm{}, werd niet gekozen voor het \code{date}-type, maar een plug-in om de functionaliteit met datums op te lossen.
%%De \code{date-picker} werd gebruikt van de plug-in pagina van Lungo zelf~\cite{TapQuo2013b}.
%%Bij deze plug-in is al voorbeeldcode aanwezig die nodig is om automatisch een \term{datepicker} te openen en de aangeklikte datum in het formulierveld te zetten.
%%De plug-in laat echter niet toe om een bereik op te geven.
%%De datum met enkel een maand en jaar diende handmatig geprogrammeerd te worden omdat de aangeboden plug-in hiervoor geen ondersteuning bood.
%%
%%Een optieveld werd niet aangeboden door \lungo{} en werd vervangen door een \term{dropdown}menu. 
%%Een schakelaar daarentegen werd dan weer wel aangeboden.
%%Het legen van een formulier gebeurt in \lungo{} door de \code{reset}-functie in \js{} op te roepen op dat formulier.
%
%
%%%%%%%%%%%%%%%%%%%%%%%%%%%
%
%\challenge{\chal{vullen}}
%\begin{itemize}
%  \item C5.1 Filling in the form elements with object
%  \item C5.2 Make the form elements read-only
%\end{itemize}
%%Back button to go back to the list (C3). %TODO niet van toepassing
%
%\framework{\sta{}}{}
%%To show the list of clickable expenses,  ST provides a navigation view.  
%%This mechanism allows to create the list of expenses together with a handler that gets triggers after an item selection.  
%The controller can fill an empty form with a model instance.
%Field names that correspond to the object's properties are linked and filled in.
%Setting the value of the correct radiobutton does not work,  a new fieldset has to be created on which the \code{setGroupValue} method needs to be applied.  
%Making form elements read-only can be done by setting the \code{readOnly} or \code{disabled} property to true.
%
%\framework{\kendoa{}}{}
%The view-model can bind a view with a form and can fill it automatically with a model instance.
%The mapping between properties of the model en fieldnames of the form is done implicitly.
%Making a formelement read-only had to be done with the data-attribute \code{readOnly} or \code{disabled}.
%
%\framework{\jqma{}}{}
%There is no automatic mapping between the data and the form.
%Setting the form elements read-only can be done easily for the types \code{input} and \code{textarea} with the \code{readonly} attribute.
%With a fieldset, the \code{disabled} attribute made is read-only.
%Lastly with a select menu, the other options were deleted from the menu.
%%Adding a back button was no problem.
%
%\framework{\lungoa{}}{}
%TODO
%
%%%%%%%%%%%%%%%%%%%%%%%%%%%
%
%% \chal{autoaanvullen}
%% Fetch suggestions based on input (C1).
%% Show suggestions in clickable drop down (C2).
%% 
%% \framework{jQM}{1,1}
%% As of version~1.3, jQM has built-in autocomplete support~\cite{JQuery2013c}.
%% Because the POC used 1.2, the plugin of Matthews~\cite{Matthews2013} was used.
%% To show only five suggestion, a little customisation by the \code{slice} function was needed because the plugin did not provid this.
%% 
%% \framework{ST}{1,0}
%% A plugin can be found at~\cite{Mysamplecode2012} to create an autocomplete field.  
%% A request to the backend with a keyword returns a JSON array named \code{data}.  
%% ST can parse this array in two ways:  with a \code{JsonReader} or an \code{ArrayReader}.  
%% The first requires that a JSON key precedes each item,  the latter assumes each item in the array maps to a field of a model.  
%% Both strategies cannot be used to parse the array and create separate model instances for each array item.  
%% This implies that no clickable drop-down could be implemented.
%% 
%%%%%%%%%%%%%%%%%%%%%%%%%%%
%
%\challenge{\chal{afbeelding}}
%\begin{itemize}
%  \item C7.1 Choose image
%  \item C7.2 Convert image to base64
%  \item C7.3 Preview image
%\end{itemize}
%
%\framework{\sta{}}{}
%A plugin to upload files can be found~\cite{Smirnov2012} to create buttons with the \code{fileupload} component.  
%This button enables users to select an image and passes it to a PHP file.  
%This file uploads the image and converts it to base64.
%A requirement is that your server is able to run that PHP file.    
%A ST image can be created based on this base64 string.
%
%\framework{\kendoa{}}{}
%\kendo{} Web provides a widget that creates a file upload field from a form element with input type \code{file}.
%The other two challenges depend on the FileReaderAPI and canvas from HTML5.
%The FileReaderAPI can load the canvas with the image.
%Conversion to base64 can be done by calling the \code{toDataURL} method on the canvas.
%The canvas is able to show a preview of the selected image.
%
%\framework{\jqma{}}{}
%A form element of type \code{file} is used to let the user attach an image.
%Depending on the mobile device, the options are to attach a local image or to take one with the camera.
%The attached image is read by the FileReaderAPI, converted to an image and imported on the canvas.
%The base64 encoding can be retrieved by calling the \code{.toDataURL()} on the latter and is stored in session storage.
%However the FileReaderAPI is not supported on Android~2.3 and lower or iOS less than 6.0~\cite{Deveria2013a}.
%Another limitation is the file size that can be imported on a canvas on iOS devices which is depend on the RAM size~\cite{Apple2012}.
%The final limitation is the size of the encoded evidence in base64, which can exceeded the limit for session storage, that is depend on the mobile browser~\cite{Gonzalez2012}.
%
%\framework{\lungoa{}}{}
%TODO
%
%%%%%%%%%%%%%%%%%%%%%%%%%%%
%
%
%\challenge{\chal{validatie}}
%\begin{itemize}
% \item C8.1 Validation rules for a required number and email field 
% \item C8.2 Validation depends on custom conditions 
% \item C8.3 Retrieve error messages of invalid fields
% \item C8.4 Red borders around invalid fields 
%\end{itemize}
%
%\framework{\sta{}}{}
%Required fields and email validation can be assigned to a model.  
%To add custom validation rules or messages,  the \code{validate} method of \code{Ext.data.Model} needs to be overridden.  
%This method accepts a model instance and returns the possible validation errors.  
%The errors must be iterated to concatenate the validation messages.
%The red borders are created by adding a custom CSS class to the form element.
%
%\framework{\kendoa{}}{}
%The validations from the first challenge can be assigned to form elements by declaring them as data-attribute in the HTML-code.
%\kendo{} can create a \code{validator} object from selected form elements.
%The \code{validator} can check the validation rules from the attributes or can be expanded with custom validation rules.
%A validation message can be provided with the \code{validationMessage} and retrieved with the \code{errors} method of a \code{validator}.
%Red borders are not supported.
%
%\framework{\jqma{}}{}
%Because of the lack of support on mobile browsers for the \code{required} attribute~\cite{Deveria2013}, the plugin of Zaefferer~\cite{Zaefferer2013} was used to check for required fields.
%This plugin also provides built-in validation methods that were needed: \code{number}, \code{email} and \code{date} and custom validation conditions.
%The plugin shows errors below the corresponding field, so customisation of the plugin was needed to show them in a dialog.
%To give the invalid fields a red border, CSS was used directly for \code{input} and \code{textarea}.
%Because of the progressive enhancement for \code{select} and \code{fieldset}, a bit of DOM traversal was needed.
%
%\framework{\lungoa{}}{}
%TODO
%
%%%%%%%%%%%%%%%%%%%%%%%%%%%
%
%% \chal{handtekening}
%% Draw a signature with finger or pen (C1).
%% 
%% \framework{jQM}{1}
%% At first, Signature Pad~\cite{Bradley2013} was used as plugin.
%% Because of the time spend on changing the default layout, jSignature~\cite{Systems2013} was used instead.
%% Two advantage over the former were the automatic scaling to 100\% width and the basic layout.
%% However, the plugin does not work on Android~2.3 and lower~\cite{Systems2013}.
%% 
%% \framework{ST}{1}
%% Drawing a signature is handled by a plugin~\cite{SimFla2011}.  
%% Plugins can easily be added to the framework by placing the plugin file in the \code{ux} folder and loading it in the main JavaScript file.  
%% This plugin could be used as-is by using the newly \code{signaturefield} component.  
%
%%%%%%%%%%%%%%%%%%%%%%%%%%%
%
%\challenge{\chal{ajax}}
%\begin{itemize}
%  \item C10.1 Fetch plain text
%  \item C10.2 Fetch and parse JSON
%  \item C10.3 Send JSON payload
%  \item C10.4 Fetch and parse XML
%\end{itemize}
%
%\framework{\sta{}}{}
%AJAX requests can be done either explicitly via a direct \code{Ext.Ajax.request} call or implicitly via stores.  
%The plain text of the AJAX response can be used in the callback.   
%Stores can be configured with a model to define the structure of the recorded objects.  
%A proxy configures readers and writers that define where the data can be read or written.  
%This can be locally at the client side or via a remote server.  
%Readers and writers contain the format of the data - JSON or XML - and automatically parse this data to fields of the corresponding model. 
%Sending JSON payload must be done via an AJAX request where the \code{jsonData} is encoded via \code{Ext.encode}.
%
%\framework{\kendoa{}}{}
%AJAX-requests can be done with a \code{dataSource}.
%This object supports CRUD operations for internal (JavaScript objects and arrays) and external (XML,JSON and JSONP) data.
%The object is build upon the \code{\$.ajax} method of jQuery and therefore is the configuration of an AJAX-request similar to jQuery.
%
%How the data needs to be parsed, is described in the \code{scheme} property of the \code{dataSource}.
%Both JSON and XML are supported as parsing schemes.
%There are two possibilities:  a new model can be instantiated every time data is retrieved or the contents of a single model can be updated after each successful AJAX-request.
%
%A JSON payload can be created by calling the \code{toJSON} method of a model and serialized it with the JSON library.
%
%\framework{\jqma{}}{}
%AJAX requests are made via the jQuery library where jQM is dependent on.
%Handling plain data and JSON as response is straightforward and the data can be used directly.
%When using XML, the data has to be traversed like with HTML, this means by using selectors.
%Sending JSON can be achieved by converting the instance into a string object using the vanilla JavaScript function \code{JSON.stringify} and add it as data to the jQuery AJAX request.
%
%\framework{\lungoa{}}{}
%TODO
%
%%%%%%%%%%%%%%%%%%%%%%%%%%%
%
%\challenge{\chal{lijsten}}
%\begin{itemize}
% \item C11.1 Load data into list and styling list items
% \item C11.2 Clickable items with custom action or linking to record of itemize
% \item C11.3 Sorting data
%\end{itemize}
%
%
%\framework{\sta{}}{}
%A list can be created by setting the type of a component to \code{list}.
%\code{Ext.XTemplate} is de \st{} class responsible for defining a template for listitems.
%A template can be used in a \code{list} by setting the \code{itemTpl} property.
%Also,  a list can be coupled with a \code{Store}.
%Elements of the list are hence coupled with a model instance from the \code{Store}.
%A \code{Controller} can catch the event when an item is clicked and show a view that contains the data of the corresponding modelinstance.
%
%Sorting the \code{Store} will sort the list that is connected with the \code{Store}.
%
%\framework{\kendoa{}}{}
%Lists can be defined by putting \code{listview} as \code{data-role}.
%The corresponding JavaScript instance can be configured with a \code{dataSource} and a template.
%Each instance in the \code{dataSouce} will create a listitem that is styled with the template.
%Linking listitems with a record is no standard feature in \kendo{}.
%A parameterized view can be used to link the listitem with a view.
%Sorting the \code{dataSource} will sort the list that is connected with the \code{dataSouce}.
%
%\framework{\jqma{}}{}
%TODO
%
%\framework{\lungoa{}}{}
%TODO
%
%%%%%%%%%%%%%%%%%%%%%%%%%%%
%% SANDER: zou ik weglaten
%% \chal{pdf}
%% \begin{itemize}
%%  \item C12.1 Requesting a PDF with POST parameters 
%%  \item C12.2 Show the PDF
%% \end{itemize}
%% 
%% \framework{\sta{}}{}
%% 
%% A plugin for a PDF viewer can be found at~\cite{Fiedler2012}.  
%% Some modifications were necessary to made it compliant with the POC.  
%% The PDF must be fetched from the backend with a parameterized POST request instead of a simple GET request.  
%% The plugin automatically creates views for every PDF page.  
%% 
%% \framework{\kendoa{}}{}
%% 
%% To download the PDF, a hidden form is used because AJAX is not the preferred way for fetching raw data.
%% 
%% \framework{\jqma{}}{}
%% 
%% To download the PDF, a hidden form is used because AJAX is not the preferred way for fetching raw data.
%% When the users taps an expense form in the list, this hidden form is submitted with the correct parameters to download the PDF.
%% The mobile device opens the corresponding native PDF viewer to show the PDF.
%% 
%% \framework{\lungoa{}}{}
%% TODO
%
%%%%%%%%%%%%%%%%%%%%%%%%%%%
%
%\challenge{\chal{offline}}
%
%\begin{itemize}
% \item \chal{offline-data}
% \item \chal{offline-app}
%\end{itemize}
%
%
%\framework{\sta{}}{}
%Within \st{} a \code{Store} can be configured with a proxy to read and write data in the \code{sessionStorage} or \code{localStorage} of the browser.
%However,  \code{Stores} can not be configured with two proxies so if data is provided by an external source and needs to be stored locally,  two \code{Stores} need to be created and kept synchronized.
%
%Making your application offline is a feature of Sencha Cmd.
%Building your application for production will create the manfist automatically.
%
%\framework{\kendoa{}}{}
%The framework does not provide features to faciliate offline storage.
%The \code{localStorage} needs to be manipulated by the programmer.
%Creating a manifest file must be done manually or can be automated with a tool like Yeoman. %TODO referentie
%
%\framework{\jqma{}}{}
%TODO
%
%\framework{\lungoa{}}{}
%TODO


%%%%%%%%%%%%%%%%%%%%%%%%%%%%%%%%%%%%%%%%%%%%%%%%%%%%%%%%%%%%%%%%%%%%%%%%%%%%%%%%%%%%%%%

\subsection{Support} % 1 blz
\label{sec:evaluation-support}

Dit section will evaluate the support of the frameworks on $8$ mobile devices.
The formula for this criterion can be found in equation~\ref{eq:ondersteuning}.
The scores for the four frameworks are summarized in table~\ref{tabel:evaluatie-ondersteuning-u}.

\begin{table*}
\centering
\pgfplotstabletypeset[
	column type=l,
	every head row/.style={
		before row={%
			\toprule
			\textbf{Challenge}
			& \multicolumn{2}{c}{\textbf{\sta}}
			& \multicolumn{2}{c}{\textbf{\kendoa}} 
			& \multicolumn{2}{c}{\textbf{\jqma}}
			& \multicolumn{2}{c}{\textbf{\lungoa}} \\
			\cmidrule(r){2-3}
			\cmidrule(r){4-5}
			\cmidrule(r){6-7}
			\cmidrule(r){8-9}
		},
		after row=\midrule,
		},
  	every last row/.style={
  		before row=\toprule,
 		after row=\bottomrule},
	columns={Challenge,ST(abs),ST(max),Kendo(abs),Kendo(max),jQM(abs),jQM(max),Lungo(abs),Lungo(max)},
	begin table=\begin{tabular}{lcccccccc},
	end table=\end{tabular},
	header=true,
	skip coltypes=true,
	columns/Challenge/.style ={column name=},
	columns/ST(abs)/.style ={column name=Score},
	columns/ST(max)/.style={column name=Max},
	columns/Kendo(abs)/.style ={column name=Score},
	columns/Kendo(max)/.style={column name=Max},
	columns/jQM(abs)/.style ={column name=Score},
	columns/jQM(max)/.style={column name=Max},
	columns/Lungo(abs)/.style ={column name=Score},
	columns/Lungo(max)/.style={column name=Max},
	col sep=comma,
	string type,
]{../Masterproef/tabellen/ondersteuning-u.csv}
\caption{Overview of support per challenge.}
\label{tabel:evaluatie-ondersteuning-u}
\end{table*}

As mentioned in section \ref{sec:comparisoncriteria},  only the functionality of the framework is tested so custom or hacky implementation are not checked for support.
This is the reason why the maximum score for each framework is different.
The challenges that could be implemented using the framework are described in the previous criterion.
De maximum scores voor \sta{},  \kendob{},  \jqma{} and \lungo{} are respectively $96$, $104$, $104$ and $80$.
Their result for support is respectively $83$, $91$, $95$ and $62$.
Looking at a relative scores,  \st{} supports $86\%$,  \kendo{} $88\%$,  \jqm{} $91\%$ and \lungo $78\%$.
A first conclusion is that all frameworks have a remarkable high score.
This means that de choice of a framework is independent of support.
The reason why the scores are more or less identical is that the frameworks all depend on the same underlying technology,  HTML5.
Altough differences are small,  \jqma{} provides the best support,  followed by \kendob{} and \sta{}.
\lungo{} has the least support.
The major problems for \lungo{} can be found in \chal{formulieren}.
The navigation through the POC did not worked on the \htc{} and the \gtab{}. %TODO waarom niet?
Also,  the tap event was used to implement the navigation.
When only tapped once,  the system recorded a double tap.
Therefor each tap also resulted in a tap on the next screen and possibly caused a undesirable action.

In general,  Android 2.3 devices - the \htc{} and \gtab{} - had the least support.
All other devices scored an average of $95\%$.
The differences in operatingsystem were remarkable:  Android has $79\%$ support while iOS has $95\%$.
Only version $2.3$ and $4$ of Android and version $5$ and $6$ of iOS were tested.
The two oldest devices were the \iphoneiii{} and \gs{} respectively released in june 2009 and march 2010~\cite{Staff2009,Gideon2010}.
Although both devices are upgraded to a later operatingsystem,  it looks like Android devices offer updates less frequently.%TODO referentie?
This can be explained with the large hetrogenity between Android devices.

All frameworks but \jqma{} provide methods to query the context.
\jqma{} implemented this feature with CSS3 media queries.
This implementations worked on $7$ of $8$ devices except for the \gtab.
However,  \kendob{} was able to recognize this device as a tablet and achieved the maximum score.
Uploading an image requires support for both the local storage and File API.  
Also the \code{file} input type is necessary to indicate the upload component.
The four implementations of this challenge all fail for the \htc{}, \gtab{} and \ipadi{} because the lack of support for one or more of these dependencies~\cite{Deveria2013c}.
HTML5 provides features for form validation but non of the $8$ devices support these features.
All frameworks, however,  implement a custom validation mechanism that is supported on each device.
All four implementations for the signature depend on the \code{canvas} feature of HTML5 and is supported for each device.
\kendob{}, \jqma{} and \lungo{} reuse the code to implement the \chal{pdf}, \sta{} has a plugin available that depend on the PDF.JS,  a PDF-renderer from Mozilla~\cite{Gal2010}.
The first implementation failed to work on the \gs{} while latter implementation showed the PDF successfully.
Finally, offline capabilities also are supported on all $8$ devices~\cite{Deveria2013c} so each framewok achieved the maximum score.

% samenvattende tekst vertalen % Sander
  
%%%%%%%%%%%%%%%%%%%%%%%%%%%%%%%%%%%%%%%%%%%%%%%%%%%%%%%%%%%%%%%%%%%%%%%%%%%%%%%%%%%%%%%

\subsection{Performance} % 1 blz % Tim
\label{sec:evaluation-performance}
The performance of both the total download time and user experience of each framework can be found in table \ref{tabel:evaluatie-performantie}.
\jqma{} takes first place, shortly followed by \lungo{}.
This is because both frameworks do not enforce an architecture, which means that less \js{} code has to be downloaded.
This is in contrast to \kendob{} and \sta{} that take respectively second last and last place.

\begin{table}[b]
\centering
\pgfplotstabletypeset[
  begin table=\begin{tabular}{p{3cm} p{0.8cm} p{0.8cm} p{0.8cm} p{0.8cm} p{0.3cm}},
  end table=\end{tabular},
  skip coltypes=true,
  col sep=comma,
  string type,
  header=true,
  columns={Performance,ST,Kendo,jQM,Lungo},
  columns/Performance/.style={column name=\textbf{Performance}, column type={l}},
  columns/ST/.style={column name=\textbf{\sta}, column type={l}},  
  columns/jQM/.style={column name=\textbf{\jqma}, column type={l}},    
  columns/Kendo/.style={column name=\textbf{\kendoa}, column type={l}},   
  columns/Lungo/.style={column name=\textbf{\lungoa}, column type={l}}, 
  every head row/.style={
    before row=\toprule,
    after row=\midrule},
  every last row/.style={
  	before row=\midrule,
    after row=\bottomrule}
]{../Masterproef/tabellen/performantie.csv}
\caption{Performance for \st{}~(\sta), \kendo{}~(\kendoa), \jqm{}~(\jqma) en \lungo{}~(\lungoa). Less is better.}
\label{tabel:evaluatie-performantie}
\end{table}

%%%%%%%%%%%%%%%%%%

\paragraph{Average download time}
Figure \ref{fig:evaluatie-performantie-gebruikerservaring} shows the average download time for POC, POC from cache, login application, login application from cache for the four frameworks. 

\begin{figure}[!ht]
  \centering
  \includegraphics[width=\columnwidth]{../Masterproef/figuren/performance.pdf}
  \caption{Average download time for POC, POC from cache, login application, login application from cache for the four frameworks. Less is better.}
  \label{fig:evaluatie-performantie-gebruikerservaring}
\end{figure}

\lungo{} takes first place and only needs a third of the download time needed by \sta{}, the slowest of the four.
\jqma{} and \kendob{} take the second and third place respectively.
Because not everything was implemented in the POC of \lungo{}, one could argue that that is the reason of the first place of \lungo{}.
But when looking at the login application, \lungo{} still takes first place.
It is even more than half faster than \jqma{}, \kendob{} or \sta{}.
Those three frameworks have about the same average download time.
The reason that \kendob{} and \sta{} are respectively second last and last is because they enforce an architecture and therefore the libraries to download are bigger than those of \lungo{} and \jqma{}.

When viewing the cached versions of the POC and login application, \kendob{}, \jqma{} en \lungo{} have about the same average download time.
However \st{} has a much higher download time when using the cached version.
The first reason is that the first three frameworks are only using HTML5 Application Cache.
\sta{} also uses its own caching mechanism in addition to HTML5 Application Cache.
The second reason is that the application in the first three frameworks are compiled using Yeoman~\cite{Yeoman2013} and \st{} with Sencha Cmd~\cite{Sencha2012}.

%%%%%%%%%%%%%%%%%%

\paragraph{User experience}
Table \ref{tabel:evaluatie-performantie-gebruikerservaring} shows the total score for user experience.
\sta{} takes first place with the maximal score.
\jqma{} was voted six time the second best for user experience.
\kendob{} was better on the \htc{} and \lungo{} on the \ipadiii{}.
The generation of the 850 list items was impossible for \kendob{} on all the iOS devices.
The browser crashed an the device returned to the home screen.
%De reden alsook de grens waarom \kendo{} niet crasht op iOS-toestellen werd door tijdsbudget niet gecontroleerd.
%Een mogelijke denkpiste is dat \kendo{} een overhead genereerd die het maximale toegelaten geheugen voor het iOS-besturingssysteem overschrijdt.
In contrast, list generation did work on all Android devices.
%De score van \kendo{} is dus slechts voor vier apparaten.

\begin{table}[!ht]
\centering
\pgfplotstabletypeset[
  begin table=\begin{tabular}{p{3cm} p{0.8cm} p{0.8cm} p{0.8cm} p{0.8cm} p{0.3cm}},
  end table=\end{tabular},
  skip coltypes=true,
  col sep=comma,
  string type,
  header=true,
  columns={Device,ST,Kendo,jQM,Lungo},
  columns/Device/.style={column name=\textbf{Device}, column type={l}},
  columns/ST/.style={column name=\textbf{\sta}, column type={l}},  
  columns/jQM/.style={column name=\textbf{\jqma}, column type={l}},    
  columns/Kendo/.style={column name=\textbf{\kendoa}, column type={l}},   
  columns/Lungo/.style={column name=\textbf{\lungoa}, column type={l}},   
  every head row/.style={
    before row=\toprule,
    after row=\midrule},
  every last row/.style={
  	before row=\midrule,
    after row=\bottomrule}
]{../Masterproef/tabellen/performantie-gebruikerservaring.csv}
\caption{User experience for \st{}~(\sta), \kendo{}~(\kendoa), \jqm{}~(\jqma) en \lungo{}~(\lungoa).}
\label{tabel:evaluatie-performantie-gebruikerservaring}
\end{table}

%%%%%%%%%%%%%%%%%%%%%%%%%%%%%%%%%%%%%%%%%%%%%%%%%%%%%%%%%%%%%%%%%%%%%%%%%%%%%%%%%%%%%%%

\subsection{Comparison overview} % ? blz
\label{sec:evaluation-overview}

Figure \ref{fig:spidergraph} shows the spidergraph with the scores for the five criteria for the four frameworks.

\begin{figure}
  \centering
  \includegraphics[width=\columnwidth]{../Masterproef/figuren/spidergraph-final.pdf}
  \caption{Overview with the five comparison criteria for \sta{},  \kendob{},  \jqma{} en \lungo{}.}
  \label{fig:spidergraph}
\end{figure}

The areas of the four pentagons are:
\begin{description}
 \item [\jqm{}] $1.87$
 \item [\kendo{}] $1.32$
 \item [\lungo{}] $0.88$
 \item [\st{}] $0.73$
\end{description}

The area unit is the unit as can be seen on the axes.

\jqma{} has the largest area ($1.87$) and can be seen as the winner.
One of the major factors that make \jqma{} successful is that it does not forces an architecture.
This decreases the learning curve in favor of productivity.
Also,  the extensive documentation and large community make \jqma{} more productive.
\lungo{} also forces no architecture but is extremely unpopular.
The absence of architecture also has two disadvantages.
First,  the framework provides less functionality but many plugins and HTML5-features act as substitutes.
Secondly,  the code that needs to be written is verbose and thereby error prone.

\kendob{} is second with an area of $1.32$.
The MVVM architecture makes \kendob{} more usable but less productive in comparison with \jqma{}.
However,  the MVVM arcitecture seems more intuitive than MVC.
The main drawback of \kendob{},  which can not be seen on the figure, is the high license kost.
The expectation that open-source frameworks are more popular is not reflected in the popularity scores.
\kendob{} also is the least performant.
The crashes on iOS devices caused a decrease in the user experience tests and hence a drop in performance.
The native look-and-feel and corresponding benefits are not been studied although it is an important factor for \kendob{} in particular.

\lungo{} has an area of ($0.88$) and claims the third place.
In tree out of five criteria,  \lungo{} is last.
For the other criteria,  \lungo{} is second.
Again, no architectur is imposed which can be seen in productivity and usability.
However,  the advantage in productivity is smaller and disadvantage is usability is larger in comparison with \jqma{}.
The lack of familiarity,  as can be seen with the popularity,  is an explanation of these differences.
A positive aspect is that \lungo{} achieved the best average download times.  
This is because \quo{},  the core library of \lungo{},  is optimised for mobile devices.
The user experience tests weakend the score for performance


\sta {} is has an area of ($0.73$) and is worst framework following these criteria.
The combination of the MVC architecture and the fact that it is \js-driven makes \sta{} both the least productive and least performant in comparison to the other frameworks.
Alle HTML code is generated by the framework and therefor the \js-files are very large.
As a result,  the average downloadtimes were remarkably large.
The user experience tests gave an opposite result because after the initial waiting time,  \sta{} had the fastest responstimes.
The tools provided by Sencha to support the developper are not able to close the gap in productivity.
The tools require a learning process on their own.
As a last remark,  \sta{} is depending on the WebKit engine.
Devices with a default browser that do not contain this engine,  like Windows Phone, are not investigated.

%%%%%%%%%%%%%%%%%%%%%%%%%%%%%%%%%%%%%%%%%%%%%%%%%%%%%%%%%%%%%%%%%%%%%%%%%%%%%%%%%%%%%%%
%%%%%%%%%%%%%%%%%%%%%%%%%%%%%%%%%%%%%%%%%%%%%%%%%%%%%%%%%%%%%%%%%%%%%%%%%%%%%%%%%%%%%%%

\section{Future work} % 0,5 blz % Tim
\label{sec:future_work}
Firsty, one can look for deeper understanding why certain results are remarkable and why other things failed on particular devices.
An example of the former is why the average download time of the login from cache was higher than the POC from cache for both \jqma{} and \lungo{}.
An example for the latter is to search why the 850 list items of \kendob{} crashed on all iOS devices.
Furthermore research can be done to know the limit of list items when it starts crashing.

Secondly, one can add new frameworks to the comparison.
This will enlarge the comparison itself, but will also keep checking the method of comparing.
Furthermore, new versions of the compared framework will be published.
It may also be worth plotting the comparison over time.
This is because the ranking can change due to new features in framework or new plugins.

Thirdly, the five criteria are driven by the POC.
This POC can be further extended with extra features like a list that when it is pulled down, it automatically refreshes and fetches the new list items.
Similar to this action, one can also use other events than the tap event.
Examples of these are double tap, swipe, hold, and also events that require multiple fingers like a rotate event.
Furthermore the integration of HTML5 features likes GPS, push events, drag and drop, video and audio in frameworks can be researched.

Fourthly, new criteria can be added to the comparison.
It is off interest to incorporate extensibility which states how easy it is to extend an existing application.
A hypothesis is that framework that enforce an architecture will have better extensibility than framework who do not.
Furthermore this new criterion could change the current ranking of the four frameworks.
Another possible new criterion is to look at the final result of the application created by the framework.
Some framework make application that imitate the nativeness of the OS where they are viewed.
Other frameworks make application with an out-of-the-box modern lay-out.

Lastly, one can take a step back and research how web applications cope with battery usage.
This data can also be compared with native and hybrid applications.
This comparison can be more generalised and one can also compare between web, native and hybrid applications.

%%%%%%%%%%%%%%%%%%%%%%%%%%%%%%%%%%%%%%%%%%%%%%%%%%%%%%%%%%%%%%%%%%%%%%%%%%%%%%%%%%%%%%%
%%%%%%%%%%%%%%%%%%%%%%%%%%%%%%%%%%%%%%%%%%%%%%%%%%%%%%%%%%%%%%%%%%%%%%%%%%%%%%%%%%%%%%%

\section{Conclusion} % 0,5 blz % Sander
\label{sec:conclusion}

HTML5 is a newly,  not-standarised technology that targets mobile devices.
This technology provides an answer to the hetrogenity in devices, operatingsystems and browsers.
Many frameworks exist that are build upon HTML5 and facilitate the development of mobile web applications.

This paper described a comparative study between \sta{},  \kendob{},  \jqma{} and \lungo{}.
\sta{} is build with the MVC architecture and is \js-driven.
\kendob{} enforces the MVVM architecture and is both \js- as markupdriven.
\jqma{} and \lungo{} do not have an architecture and are both markupdriven.

Five criteria were chosen to execute the comparative study:  popularity,  productivity,  usage,  support and performance.
Each criterion was provided with a forumla to calculate a score for the criterion.
A POC was formalised and checked with the criteria in as many ways as possible.
The activity of the framework on social networks, determined the popularity.
The productivity was measured by timetracking the time to implementing a loginapplication.
The POC was subdivided in $13$ challenges and $38$ subchallenges to test the usability of the framework.
Next,  a subset of these challenges were tested on $8$ different devices to check the support.
Finally,  the load- and responsetime of the POC and loginapplication determined the performance criterion.
All the scores were plotted in a spidergraph.

After evaluation, \jqma{} was the winner,  followed by \kendob{}, \lungo{} and \sta{}.
The order was determined by the areas of the pentagons on the spidergraph.


%%%%%%%%%%%%%%%%%%%%%%%%%%%%%%%%%%%%%%%%%%%%%%%%%%%%%%%%%%%%%%%%%%%%%%%%%%%%%%%%%%%%%%%
%%%%%%%%%%%%%%%%%%%%%%%%%%%%%%%%%%%%%%%%%%%%%%%%%%%%%%%%%%%%%%%%%%%%%%%%%%%%%%%%%%%%%%%

% % Define a new 'leo' style for the package that will use a smaller font.
% \makeatletter
% \def\url@leostyle{%
%   \@ifundefined{selectfont}{\def\UrlFont{\sf}}{\def\UrlFont{\small\ttfamily}}}
% \makeatother
% %% Now actually use the newly defined style.
% \urlstyle{leo}

%\bibliographystyle{named}
\bibliographystyle{abbrv}
\bibliography{../Referenties/alles-en}

\end{document}