\chapter{Besluit}
\label{chap:besluit}

\section{Conclusie} % 1 pagina % Sander
% - wat we gedaan hebben, onze doelen, hebben we de beste gevonden?
% - we hebben dat gevonden, ….
HTML5 is een nieuwe,  niet-gestandardiseerde technologie die zich richt op mobiele apparaten.
Deze technologie moet een antwoord bieden op de verscheidenheid van apparaten,  besturingssystemen en browsers.
Vele raamwerken bestaan die verderbouwen op HTML5 om de ontwikkeling van mobiele web applicaties te vergemakkelijken.

In deze thesis werd een vergelijkende studie uitgevoerd tussen \st{}, \kendo{},  \jqm{} en \lungo{}.
\st{} bouwt op de MVC-architectuur en is \js-gedreven.
Het raamwerk is gratis binnen een commerciele context.
\kendo{} dwingt de MVVM-architectuur af en is zowel \js- als opmaakgedreven.
Een licentie voor het gebruik van \kendo{} kost $\$699$.
\jqm{} en \lungo{} hebben geen architectuur en zijn beide opmaakgedreven.
Beide raamwerken zijn \term{open-source}.

Vijf criteria werden gekozen om de vergelijkende studie uit te voeren:  populariteit,  productiviteit,  gebruik,  ondersteuning en performantie.
Elk criterium werd voorzien van een formule om een score te berekenen voor het criterium.
Een POC die werknemers toelaat onkosten toe te voegen, werd geformaliseerd en op zoveel mogelijk vlakken getest met de opgelegde criteria
Om populariteit te meten werd naar de activiteit van de raamwerken op sociale netwerken gekeken.
De tijd om een loginapplicatie te ontwikkelen, bepaalde de productiviteit.
De POC werd onderverdeeld in $13$ uitdagingen en $38$ deeluitdagingen om de functionaliteit van het raamwerk te testen en het gebruikscriterium te quoteren.
Vervolgens werd een subset van de uitdagingen getest op $8$ verschillende mobiele toestellen om de ondersteuning te controleren.
Ten slotte bepaalden de laad- en responstijden van de POC en loginapplicatie de performantie.
De responstijden werden opgementen met gebruikservaringstesten.
De scores van de $5$ criteria voor de $4$ raamwerken werden in één spinnenweb ondergebracht.

Na evaluatie bleek \jqm{} het beste raamwerk op basis van de gekozen criteria, gevolgd door \kendo{}, \lungo{} en \st{}.
Deze volgorde werd bepaald door de oppervlakte van de vijfhoeken op het spinnenweb te berekenen.

\section{Geleerde lessen} % 1 pagina 
% beter en vollediger uitwerken van criteria op voorhand! niet enkel bekijken als je effectie moet gaan evalueren, het aantal ongewenste verrassingen blijft zo beperkt
% toggle entries beter bijhouden, veel meer algemener (als je zo een methode gebruikt, beter doen)

% elk 3 nieuwe dingen

\pagebreak
\section{Verder onderzoek} % 1 pagina % Tim

%TODO: moet ik die onderstaande details ook nog uitschrijven, ik zit al aan 1 bladzijde? en die zijn zo gedetailleerd t.o.v. onze bigger picture ;-)
% onderzoeken ophalen icons in cache 
% onderzoeken crash kendo op ios

% nieuwe frameworks toevoegen + updates van huidige frameworks blijven controleren (resultaten ook updaten)
% methodologie blijven verder toetsen
Enerzijds kunnen nieuwe raamwerken worden toegevoegd aan de vergelijking.
Hierdoor vergroot ten eerste de grootte van de vergelijking, maar kan ten tweede ook de methode telkens opnieuw worden getoetst met deze nieuwe raamwerken.
Daarnaast komen van de reeds vergeleken raamwerken geregeld nieuwe versies uit.
Zo is het ook mogelijk om de evolutie in de rangschikking van de vier vergeleken raamwerken over de tijd te bekijken.
Mogelijk kan de rangschikking veranderen bij het uitbrengen van nieuwe versies of plug-ins.

% POC updaten (pull-to-refresh,  meer items laden,  ...) HTML5 features meer toevoegen (GPS, audio,  drag and drop (herorden lijst, lang duwen), carousel met swipe, push eventes 
De huidige methode omvat vijf vergelijkingscriteria die worden gedreven door de POC.
Verder onderzoek kan deze POC uitbreiden met extra kenmerken zoals een lijst wanneer deze naar beneden wordt getrokken, de lijstelementen vernieuwd worden.
Dit loopt in de lijn om ook andere \term{events} te gebruiken dan alleen maar het \term{tap} \term{event}.
Andere \term{events} zijn bijvoorbeeld \term{double tap}, \term{swipe}, \term{hold}, maar ook \term{events} waar meerdere vingers voor nodig zijn zoals \term{rotate}.
Daarnaast is ook de integratie van HTML5-kenmerken zoals GPS, \term{push events}, \term{drag and drop}, video en audio in de raamwerken zeker het onderzoeken waard.

% windows en blackberry ondersteunen?
Naast het toevoegen van extra kenmerken aan de POC, kunnen criteria ook op andere manieren gecontroleerd worden.
Nu worden bij ondersteuning enkel apparaten met een Android- of iOS-besturingssysteem gebruikt.
Dit kan worden vervangen of uitgebreid naar andere besturingssystemen zoals Windows Phone en BlackBerry~OS.
% hoe zit het met de downloadsnelheid als je iets anders dan WiFi gebruikt? 3G 4G…?
Een andere voorbeeld is dat voor de downloadtijd bij performantie Wifi werd gebruikt voor de verbinding.
Andere verbindingsmogelijkheden zoals 3G kunnen worden gebruikt en hierdoor kunnen andere resultaten bekomen worden.
% subjectieve gebruikservaringstesten met > 5 mensen
Een laatste voorbeeld is de gebruikerservaring die wordt gebruikt bij performantie.
% TODO: klopt die grens van 5 personen nu wel?
Deze werd getest door twee personen, maar had eigenlijk door meer dan vijf personen moeten worden getest.
Het is ook mogelijk om een geheel andere manier te zoeken om de gebruikerservaring nog meer objectiever te maken.

% ook het criterium uitbreidbaarheid erbij betrekken, want nu komen ST en Kendo niet helemaal tot uiting in onze spidergraph
Een andere onderzoekspiste is om nieuwe criteria toe te voegen.
Zo kan enerzijds het criterium uitbreidbaarheid worden onderzocht.
Dit criterium omvat hoe gemakkelijk het gaat om de bestaande applicatie geïmplementeerd in een bepaalde raamwerk uit te breiden.
Een te onderzoeken hypothese hierbij is dat raamwerken die een architectuur afdwingen beter zullen scoren dan raamwerken zonder architectuur.
Een bijkomende hypothese is dat de totale score van raamwerken die een architectuur afdwingen zal stijgen en deze van raamwerken zonder architectuur zal dalen.
Dit komt doordat de ene worden afgestraft op productiviteit en de andere op uitbreidbaarheid.
Mogelijk kan de rangschikking van de vier onderzochte raamwerken veranderen.
% het finale resultaat van de framework bekijken (look-and-feel van kendo, nice dialogs van lungo,...)
Anderzijds kan ook een criterium worden toegevoegd die kijkt naar het finale resultaat van het raamwerk.
Zo kunnen bepaalde raamwerken de \term{native look-and-feel} van mobiele besturingssystemen  nabootsen, andere raamwerken bieden dan weer standaard een frisse hedendaagse lay-out.

% is battery use an issue for web applications?
Andere onderzoeksvragen kunnen een stap terugnemen door bijvoorbeeld af te vragen of het baterijverbruik door webapplicaties een probleem vormt.
De bekomen data kan worden vergeleken met \term{native} en hybride applicaties.
% vergelijking web / hybrid / native
Deze laatste vergelijking kan zelfs veralgemeend worden waardoor een vergelijking tussen web-, \term{native} en hybride applicaties zich opdringt. 


%%% Local Variables: 
%%% mode: latex
%%% TeX-master: "masterproef"
%%% End: 
