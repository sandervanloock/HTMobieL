\section{Performantie}
\label{sec:evaluatie-performantie}

De bekomen performantie voor de vier raamwerken wordt weergegeven in tabel \ref{tabel:evaluatie-performantie}.
De performantie wordt bepaald volgens formule~\ref{eq:performantie-enhanced} die de gemiddelde downloadtijd en de gebruikerservaring bevat.
Voor de score van performantie geldt hoe kleiner, hoe beter.

\begin{table}
\centering
\pgfplotstabletypeset[
  begin table=\begin{tabular}{p{8.5cm} p{0.8cm} p{0.8cm} p{0.8cm} p{0.8cm} p{0.3cm}},
  end table=\end{tabular},
  skip coltypes=true,
  col sep=comma,
  string type,
  header=true,
  columns={Performantie,ST,Kendo,jQM,Lungo},
  columns/Performantie/.style={column name=\textbf{Performantie}, column type={l}},
  columns/ST/.style={column name=\textbf{\sta}, column type={l}},  
  columns/jQM/.style={column name=\textbf{\jqma}, column type={l}},    
  columns/Kendo/.style={column name=\textbf{\kendoa}, column type={l}},   
  columns/Lungo/.style={column name=\textbf{\lungoa}, column type={l}}, 
  every head row/.style={
    before row=\toprule,
    after row=\midrule},
  every last row/.style={
  	before row=\midrule,
    after row=\bottomrule}
]{tabellen/performantie.csv}
\caption{Overzicht van performantie.}
\label{tabel:evaluatie-performantie}
\end{table}

Op vlak van performantie scoort \lungo{} het beste, kort gevolgd door \jqm{} en \st{}.
\kendo{} heeft een beduidend lage score en is hierdoor laatst.
Dit wordt verklaard doordat de eerste twee raamwerken geen ontwerppatroon afdwingen waardoor er ook een kleinere \js{}-code dient te worden gedownload ten opzichte van raamwerken die wel een ontwerppatroon afdwingen (zie ook tabel \ref{tabel:raamwerken-tabel}).
\st{} scoort het slechtst op gemiddelde downloadtijd, maar behaalt de beste score op gebruikerservaring.

Eerst zal in sectie~\ref{sec:evaluatie-downloadtijd} de gemiddelde downloadtijd gedetailleerd worden besproken.
Hieropvolgend zal in sectie~\ref{sec:evaluatie-gebruikerservaring} de gebruikerservaring worden besproken.
Als laatste wordt in sectie~\ref{sec:evaluatie-performantie-duiding} de performantie getoetst aan andere metrieken.


%%%%%%%%%%%%%%%%%%

\subsection{Gemiddelde downloadtijd}
\label{sec:evaluatie-downloadtijd}

\begin{table}
\centering
\pgfplotstabletypeset[
  begin table=\begin{tabular}{p{8.5cm} p{0.8cm} p{0.8cm} p{0.8cm} p{0.8cm} p{0.3cm}},
  end table=\end{tabular},
  skip coltypes=true,
  col sep=comma,
  string type,
  header=true,
  columns={Performantie,ST,Kendo,jQM,Lungo},
  columns/Performantie/.style={column name=\textbf{Downloadtijden}, column type={l}},
  columns/ST/.style={column name=\textbf{\sta}, column type={l}},  
  columns/jQM/.style={column name=\textbf{\jqma}, column type={l}},    
  columns/Kendo/.style={column name=\textbf{\kendoa}, column type={l}},   
  columns/Lungo/.style={column name=\textbf{\lungoa}, column type={l}}, 
  every head row/.style={
    before row=\toprule,
    after row=\midrule},
  every last row/.style={
  	before row=\midrule,
    after row=\bottomrule}
]{tabellen/performantie-downloadtijd.csv}
\caption{Gemiddelde downloadtijd van de loginapplicatie.}
\label{tabel:evaluatie-performantie-downloadtijd}
\end{table}

De gemiddelde downloadtijd wordt bepaald volgens formule \ref{eq:totale-downloadtijd} die zowel de loginapplicatie als de loginapplicatie uit cache in rekening brengt.
De uitkomsten hiervan worden weergegeven in tabel~\ref{tabel:evaluatie-performantie-downloadtijd}.
Om de spreiding van deze downloadtijd op de acht apparaten beter te kunnen bekijken, wordt op figuur \ref{fig:performantie-login-boxplot} de downloadtijd uitgezet in een boxplot.
Voor de bespreking van uitschieters wordt verwezen naar appendix~\ref{app:performantie}. 
Eerst zal deze van de loginapplicatie worden besproken en vervolgens van de loginapplicatie uit cache.

\begin{figure}
  \centering
  \subfloat[Loginapplicatie]{
    \includegraphics[width=0.85\textwidth]{figuren/performantie-login.pdf}
    \label{fig:performantie-login}
  }
  \quad
  \subfloat[Loginapplicatie uit cache]{
    \includegraphics[width=0.85\textwidth]{figuren/performantie-login-cache.pdf}
    \label{fig:performantie-login-cache}
  }
  \caption{Downloadtijden van de loginapplicatie.}
  \label{fig:performantie-login-boxplot}
\end{figure}

\paragraph{Loginapplicatie}
\lungo{} behaalt de snelste downloadtijd (zie figuur \ref{fig:performantie-login}).
\jqm{} en \kendo{} behalen respectievelijk een tweede en derde snelste downloadtijd.
\lungo{} is meer dan de helft sneller dan \jqm{}, \kendo{} of \st{}.
Deze drie raamwerken behalen quasi dezelfde downloadtijd.
\st{} is het traagst.
De verwachting is dat de downloadtijd evenredig is met de downloadgrootte van de applicatie.
De afhankelijkheden van het raamwerk nemen het merendeel in van de totale downloadgrootte.
Dit komt omdat er voor de loginapplicatie relatief weinig code moest worden geschreven.
Tabel \ref{tabel:evaluatie-performantie-groottes} toont voor alle raamwerken de grootte van de afhankelijkheden die nodig zijn om de loginapplicatie op te zetten.
De implementatie van de loginapplicatie in \jqm{} gebruikt een plug-in voor validaties, vandaar dat ook de grootte van deze plug-in is opgenomen.
Deze gegevens omvatten drie conclusies.
Ten eerste kan gezien worden dat \quo{}, de \js{}-bibliotheek van \lungo{},  geoptimaliseerd is voor mobiele apparaten.
Verder is het al dan niet aanwezig zijn van een ontwerppatroon niet uit de grootte van de \js{}-bestanden af te leiden.
\kendo{} en \jqm{} hebben namelijk beide afhankelijkheden die ongeveer even groot zijn.
Ten slotte kan geconcludeerd worden dat \st{} de meeste \js-code gebruikt in zijn afhankelijkheden.
Dit komt omdat \st{} een \js-gedreven raamwerk is waarbij alle HTML-code ter plaatse moet worden gegenereerd.

De verwachte evenredigheid tussen downloadtijd (zie figuur \ref{fig:performantie-login-boxplot}) en downloadgrootte (zie tabel \ref{tabel:evaluatie-performantie-groottes}) wordt bevestigd door \kendo{}, \jqm{} en \lungo{}.
\st{} gebruikt uitgesteld parsen van \js{}-code (zie sectie \ref{sec:evaluatie-performantie-duiding}).
Hierdoor zal de browser niet moeten wachten op het parsen en dus sneller alle afhankelijkheden downloaden.

\begin{table}
\centering
\pgfplotstabletypeset[
  begin table=\begin{tabular}{p{8.5cm} p{0.8cm} p{0.8cm} p{0.8cm} p{0.8cm} p{0.3cm}},
  end table=\end{tabular},
  skip coltypes=true,
  col sep=comma,
  string type,
  header=true,
  columns={Bibliotheek,ST,Kendo,jQM,Lungo},
  columns/Bibliotheek/.style={column name=\textbf{Afhankelijkheden}, column type={l}},
  columns/ST/.style={column name=\textbf{\sta}, column type={l}},  
  columns/jQM/.style={column name=\textbf{\jqma}, column type={l}},    
  columns/Kendo/.style={column name=\textbf{\kendoa}, column type={l}},   
  columns/Lungo/.style={column name=\textbf{\lungoa}, column type={l}}, 
  every head row/.style={
    before row=\toprule,
    after row=\midrule},
  every last row/.style={
  	before row=\midrule,
    after row=\bottomrule}
]{tabellen/bibliotheken.csv}
\caption{Grootte van de afhankelijkheden van de raamwerken voor het implementeren van de loginapplicatie.}
\label{tabel:evaluatie-performantie-groottes}
\end{table}


\paragraph{Loginapplicatie uit cache}
Als naar de versie uit cache wordt gekeken, scoren \kendo{}, \jqm{} en \lungo{} hetzelfde (zie figuur \ref{fig:performantie-login-cache}).
Daarentegen behaalt \st{} telkens een veel tragere tijd.
Enerzijds komt dit doordat de drie eerstgenoemde raamwerken enkel gebruik maken van HTML5 Application Cache.
\st{} voegt het Delta Update mechanisme toe aan HTML5 Application Cache.
Dit mechanisme wil voorkomen dat bij een kleine aanpassing in de code,  alle bestanden opnieuw moeten worden opgehaald die in het \code{manifest}-bestand staan opgelijst.
De \code{microloader} is verantwoordelijk voor het asynchroon ophalen van alle benodigde \js{}- en CSS-bestanden.
Na het bouwen van een applicatie met Sencha Cmd,  zullen de gewijzigde bestanden gearchiveerd worden en worden de veranderingen tussen elke versie opgeslagen.
Na het laden van de applicatie, zal de \code{microloader} met een GET-verzoek controleren op wijzigingen.
Dit GET-verzoek zal de grootste tijd voor zijn rekening nemen bij de downloadtijden bij applicaties uit de cache.
\st{} heeft er dus voor gekozen om aan performantie in te boeten ten voordele van het update mechanisme~\cite{Nguyen2012}.
\kendo{}, \jqm{} en \lungo{} gebruiken Yeoman om de applicatie te bouwen.
De webapplicaties gemaakt in \st{} gebruiken daarentegen Sencha Cmd.

%%%%%%%%%%%%%%%%%%

\subsection{Gebruikerservaring}
\label{sec:evaluatie-gebruikerservaring}

\begin{table}
\centering
\pgfplotstabletypeset[
  begin table=\begin{tabular}{p{8.5cm} p{0.8cm} p{0.8cm} p{0.8cm} p{0.8cm} p{0.3cm}},
  end table=\end{tabular},
  skip coltypes=true,
  col sep=comma,
  string type,
  header=true,
  columns={Apparaat,ST,Kendo,jQM,Lungo},
  columns/Apparaat/.style={column name=\textbf{Apparaat}, column type={l}},
  columns/ST/.style={column name=\textbf{\sta}, column type={l}},  
  columns/jQM/.style={column name=\textbf{\jqma}, column type={l}},    
  columns/Kendo/.style={column name=\textbf{\kendoa}, column type={l}},   
  columns/Lungo/.style={column name=\textbf{\lungoa}, column type={l}},   
  every head row/.style={
    before row=\toprule,
    after row=\midrule},
  every last row/.style={
  	before row=\midrule,
    after row=\bottomrule}
]{tabellen/performantie-gebruikerservaring.csv}
\caption{Gebruikerservaring van het scrollen door een lange lijst.}
\label{tabel:evaluatie-performantie-gebruikerservaring}
\end{table}

In tabel \ref{tabel:evaluatie-performantie-gebruikerservaring} wordt de totaalscore voor de gebruikerservaring getoond.
\st{} behaalt de maximale score.
Dit wil zeggen dat op alle toestellen het scrollen door de lijst van \st{} het vlotst werd ervaren.
\jqm{} werd zes keer als tweede beste beoordeeld. 
Op de \htc{} liep \kendo{} vlotter,  op de \ipadi{} was \lungo{} nummer twee.
De lijst genereren met \kendo{} op iOS-toestellen was onmogelijk omdat de applicatie de browser liet crashen.
De reden waarom alsook de grens van het aantal lijstelementen wanneer \kendo{} crasht op iOS-toestellen werd door tijdsbudget niet gecontroleerd.
Een mogelijke denkpiste is dat \kendo{} een overhead genereert die het maximale toegelaten geheugen voor het iOS-besturingssysteem overschrijdt.
Op Android-toestellen kon de lijst echter wel worden getoond.
De score van \kendo{} is dus slechts voor vier apparaten.

%%%%%%%%%%%%%%%%%%

\subsection{Duiding}
\label{sec:evaluatie-performantie-duiding}

In wat volgt zullen metrieken worden besproken die de score van de performantie zullen duiden.
Enerzijds wordt de gemiddelde downloadtijd getoond voor zowel de POC als loginapplicatie en anderzijds wordt naar de score van Google PageSpeed gekeken.

\begin{figure}
 \centering
 \includegraphics[width=\textwidth]{figuren/performance-nl.pdf}
 \caption{Gemiddelde downloadtijd van POC,  POC uit cache,  loginapplicatie en loginapplicatie uit cache voor elk raamwerk.}
 \label{fig:performantie}
\end{figure}

\paragraph{Downloadtijd POC versus loginapplicatie}
Op figuur \ref{fig:performantie} wordt de gemiddelde downloadtijd van de POC en loginapplicatie, zowel gewoon als uit cache, voor de vier raamwerken getoond.
Voor de gemiddelde downloadtijd per apparaat en per raamwerk wordt verwezen naar appendix \ref{app:performantie}.

De tragere downloadtijd van \st{},  zoals geconcludeerd in sectie \ref{sec:evaluatie-downloadtijd}, wordt hier bevestigd.
De verschillen tussen \st{} en de andere raamwerken zijn groter bij de POC dan bij de loginapplicatie.
Dit komt omdat bij de loginapplicatie een afhankelijkheid met het strikt minimale (\code{sencha-touch.js}) werd gebruikt.  
Het volledige raamwerk (\code{sencha-touch-all.js}) werd daarentegen bij de POC gebruikt.
Het Delta Update mechanisme voor het updaten van de cache zal \st{} nog meer vertragen bij de POC.

Indien \st{} buiten beschouwing wordt gelaten, duurt de eerste keer laden van de POC gemiddeld $5,73\unit{s}$. 
Het laden van de POC uit cache duurt gemiddeld $400\unit{ms}$.
De eerste keer laden van de loginapplicatie duurt gemiddeld $3,32\unit{s}$.
Indien deze uit cache komt, duurt dit gemiddeld nog $420\unit{ms}$.

Indien enkel de downloadtijd in rekening wordt gebracht en \st{} buiten beschouwing wordt gelaten, kan er gezien worden dat de downloadtijden $< 10\unit{s}$,  dit zijn aanvaardbare tijden in het domein van gebruiksvriendelijkheid~\cite{Nielsen1993}.
De gebruiker zal de vertraging waarnemen maar de aandacht niet verliezen.
Het initieel laden van de POC implementatie met \st{} moet van een laadscherm worden voorzien omdat de downloadtijd $> 10\unit{s}$.

\paragraph{Google Page Speed}
De score op 100 die Google Page Speed~\cite{Morgan2011} aan de loginapplicatie toekent is $96$ voor \st{}, gevolgd door \lungo{}~($82$),  \kendo{}~($73$) en \jqm{}~($71$).
Er kan geconcludeerd worden dat Sencha Cmd de applicatie optimaal weet te bouwen,  er is het minst plaats voor verbetering.
Hoewel \st{} de meeste tijd vraagt om te laden, zal het hierna sneller werken.
Dit wordt bevestigd in de voorbije testen.

Alle implementaties worden door Google Page Speed aangeraden een tekenset te specificeren en gebruik te maken van het cachegeheugen van de browser.
Dit laatste is een suggestie om de maximale duur van documenten in de cache een week in de toekomst te zetten.
De huidige implementaties hebben een vervaldatum van slechts tien minuten.
Beide werkpunten hebben volgens Google Page Speed een lage prioriteit.
Alle raamwerken buiten \st{} worden aangeraden om de \js-code uitgesteld te parsen.
Dit is inherent aan de raamwerken en niet aan de tool die de applicatie bouwt.
Bij \kendo{} en \jqm{} heeft dit werkpunt een hogere prioriteit in tegenstelling tot \lungo{}.
Dit komt omdat \lungo{} maar een kleiner aantal bytes moet verwerken.
De implementatie van \st{} wordt ook aangeraden om vraagtekens uit URLs te verwijderen.
Dit zou de oorzaak kunnen zijn voor het niet-cachen van bestanden.