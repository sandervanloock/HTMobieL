\chapter{Evaluatie}
\label{chap:evaluatie}

In dit hoofdstuk wordt de vergelijking uitgevoerd op basis van de vijf vergelijkingscriteria uit hoofdstuk \ref{chap:vergelijking}, namelijk gemeenschap~(\ref{sec:evaluatie-gemeenschap}), productiviteit~(\ref{sec:evaluatie-productiviteit}), gebruik~(\ref{sec:evaluatie-gebruik}), ondersteuning~(\ref{sec:evaluatie-ondersteuning}) en performantie~(\ref{sec:evaluatie-performantie}). 

%%%%%%%%%%%%%%%%%%%%%%%%%%%%%%%%%%%%%%%%%%%%%%%%%%%%%%%%%%%%%%%%%%%%%%%%

\section{Gemeenschap}
\label{sec:evaluatie-gemeenschap}

% jQM
% Met 7.400 volgers op GitHub~\cite{GitHub2012} en 11.200 volgers op Twitter~\cite{Twitter2012} komt de grote kracht van jQuery Mobile van zijn community . Dit heeft grotendeels te maken met het feit dat jQuery Mobile geniet van het succes van jQuery, dat ook zeer populair is~\cite{Hales2012}.
%TODO Sander: dit komt in de vergelijkingscriteria

% ST
% We kunnen vaststellen dat Sencha over een grote community beschikt.  Met meer dan 2 miljoen ontwikkelaars wereldwijd is Sencha de grootste provider van een open-source web applicatie~\cite{Inc.}.  

\begin{table}[H]
\centering
\pgfplotstabletypeset[
  col sep=comma,
  string type,
  header=true,
  columns={Gemeenschap,jQM,ST,Kendo,Lungo},
  columns/Gemeenschap/.style={column name=\textbf{Gemeenschap}, column type={l}},  
  columns/jQM/.style={column name=\textbf{\jqm}, column type={c}},
  columns/ST/.style={column name=\textbf{\st}, column type={c}},
  columns/Kendo/.style={column name=\textbf{\kendo}, column type={c}},
  columns/Lungo/.style={column name=\textbf{\lungo}, column type={c}},
  every head row/.style={
    before row=\toprule,
    after row=\midrule},
  every last row/.style={
    after row=\bottomrule}
]{tabellen/gemeenschap.csv}
\caption{Samenvattende tabel voor gemeenschapscriterium}
\label{tabel:evaluatie-gemeenschap}
\end{table}

%%%%%%%%%%%%%%%%%%%%%%%%%%%%%%%%%%%%%%%%%%%%%%%%%%%%%%%%%%%%%%%%%%%%%%%%

\section{Productiviteit}
\label{sec:evaluatie-productiviteit}

\begin{table}[H]
\centering
\pgfplotstabletypeset[
  col sep=comma,
  string type,
  header=true,
  columns={Productiviteit,jQM,ST,Kendo,Lungo},
  columns/Productiviteit/.style={column name=\textbf{Productiviteit}, column type={l}},  
  columns/jQM/.style={column name=\textbf{\jqm}, column type={c}},
  columns/ST/.style={column name=\textbf{\st}, column type={c}},
  columns/Kendo/.style={column name=\textbf{\kendo}, column type={c}},
  columns/Lungo/.style={column name=\textbf{\lungo}, column type={c}},
  every head row/.style={
    before row=\toprule,
    after row=\midrule},
  every last row/.style={
    after row=\bottomrule}
]{tabellen/productiviteit.csv}
\caption{Samenvattende tabel voor productiviteitscriterium}
\label{tabel:evaluatie-productiviteit}
\end{table}

%%%%%%%%%%%%%%%%%%%%%%%%%%%%%%%%%%%%%%%%%%%%%%%%%%%%%%%%%%%%%%%%%%%%%%%%

\section{Gebruik}
\label{sec:evaluatie-gebruik}
%Gebruik wordt opgedeeld per uitdaging.

\subsection{U1: Formulieren}

\paragraph{\jqm} 
%Voor het toevoegen van \term{placeholders} in de formuliervelden kon beroep worden gedaan op het \code{placeholder} attribuut in HTML5. 
%Er dienden geen labels te worden gezet voor de velden. 
%Deze labels zijn echter wel verplicht in jQuery Mobile, maar kunnen onzichtbaar worden gemaakt met de \code{ui-hide-label} CSS klasse~\cite{JQuery2013}. 
%Wat wel opmerkelijk is wanneer men een formulier invult, daarna verstuurt en dan terugkeert, het formulier nog alle waarden bevat. 
%Men moet na het formulier te hebben verstuurd, zelf het formulier altijd leegmaken. 
%Dit kan met behulp van JavaScript via de \code{reset()}-functie op het formulier.
% 
%Voor de types van de formuliervelden werd beroep gedaan op de volgende types: \code{text}, \code{number} en \code{email}. 
%Deze zorgen ervoor dat op de mobiele apparaten aangepaste toetsenborden te voorschijn komen. 
%Het \code{date} type werd echter niet gebruikt. 
%Enerzijds was hiervoor een slechte ondersteuning naar mobiele browsers toe~\cite{Deveria2013b}. 
%Dit betekende onder andere dat Android 2.3 dit niet ondersteunde. 
%Een ander probleem was het ontbreken van een \term{placeholder} in het veld op iOS 6 en Android 4.2. 
%Hierdoor weet de gebruiker in eerste instantie niet wat hij hier moet invullen, omdat er ook geen labels aanwezig moesten zijn. 
%Daarnaast is een \code{placeholder} instellen onmogelijk voor een \code{date} type~\cite{Berjon2012}. 
%Anderzijds moest het ook mogelijk zijn om de gebruiker maar een bepaald bereik van data op te leggen, wat met het \code{date} type onmogelijk is. 
%Hierdoor werd gebruik gemaakt van de Date \& Time Picker van Mobiscroll~\cite{Mobiscroll2013} die ook aangepaste lay-out heeft conform met die van jQuery Mobile. 
%Het veld heeft dan het type \code{text}. 
%Het is dus in principe mogelijk om iets anders dan een datum in te geven. 
%Dit wordt belet door ook nog eens een datum validatie te doen op dit tekstveld mocht de plug-in het al niet afgedwongen hebben.
% 
%Het was ook nodig om enkel de maand en jaar in te geven.  
%Ook hier kon niet het \code{date} type gebruikt worden, omdat daar ook een dag voor nodig is. 
%Daardoor werden de maanden handmatig geprogrammeerd als vaste lijstitems. 
%De jaren zijn dynamisch en zijn telkens dit jaar, het volgende en het vorige jaar. 
%Deze functionaliteit had ook met de Mobiscroll plug-in kunnen worden verwezenlijkt.

\paragraph{\st} 
% Placeholders, text, email and number fields are supported by the framework and can be easily created.  
% Labels can be avoided by not defining them.  
% Creating custom datepickers is not supported.  
% It is impossible to ignore the days field and only years can be delimited.  
% Clearing the form after it was send, has to be programmed manually.
Een formulier wordt in \st{} \code{fielset} genoemd.
Een \code{view} van een formulier voorzien kan door in de rij van elementen een object met \code{xtype} \code{fieldset} te maken.
Dit object kan op zijn beurt voorzien worden van een rij van elementen.
Volgende velden worden aangeboden in \st{}:
\begin{enumerate}
  \item \code{textfield}        Ext.field.Text
  \item \code{numberfield}      Ext.field.Number
  \item \code{emailfield}	 Ext.field.Email			
  \item \code{textareafield}    Ext.field.TextArea
  \item \code{hiddenfield}      Ext.field.Hidden
  \item \code{radiofield}       Ext.field.Radio
  \item \code{checkboxfield}    Ext.field.Checkbox
  \item \code{selectfield}      Ext.field.Select	
  \item \code{togglefield}      Ext.field.Toggle
  \item \code{fieldset}         Ext.form.FieldSet
\end{enumerate}

Tekst-, email en nummervelden worden bij het renderen tot HTML5-invoertypes omgevormd en bijgevolg worden op mobiele toestellen bijhorende virtuele toetsenborden weergegeven.
Een placeholder toevoegen kan door een veld met \code{placeholder} eigenschap te voorzien en de waarde aan de gewenste placeholder gelijk te stellen. 
Een label toevoegen verloopt analoog,  deze weglaten zal geen label renderen.

Een aangepaste \term{datepicker} maken is niet standaard voorzien.
Een standaard \term{datepicker} is echter wel voorzien met \code{datepicker} als xtype.
Deze kan enkel geconfigureerd worden door een begin- en eindjaar in te stellen.
Een \term{datepicker} maken waarbij het bereik kleiner is dan een jaar, is niet mogelijk.
Ook is het onmogelijk om enkel een maand- en jaarveld te tonen.

%TODO challenge bekijken (enkel reset oproepen normaal ok )
Het leegmaken van een formulier gebeurt niet automatisch wanneer het verzonden wordt.
Hiervoor moet de \code{reset} methode op het bijhorende \code{formpanel} worden opgeroepen.

\paragraph{\kendo}
 TODO
 
\paragraph{\lungo} 
TODO

\begin{table}[H]
\centering
\pgfplotstabletypeset[
  begin table=\begin{tabular}{p{8cm} p{1cm} p{1cm} p{1cm} p{1cm}},
  end table=\end{tabular},
  skip coltypes=true,
  col sep=comma,
  string type,
  header=true,
  skip coltypes=true,
  columns={Uitdaging,jQM,ST,Kendo,Lungo},
  columns/Uitdaging/.style={column name=\textbf{Uitdaging}, column type={l}},  
  columns/jQM/.style={column name=\textbf{\jqma}, column type={c}},
  columns/ST/.style={column name=\textbf{\sta}, column type={c}},
  columns/Lungo/.style={column name=\textbf{\lungoa}, column type={c}},
  columns/Kendo/.style={column name=\textbf{\kendoa}, column type={c}},
  every head row/.style={
    before row=\toprule,
    after row=\midrule},
  every last row/.style={
  	before row=\midrule,
    after row=\bottomrule}
]{tabellen/gebruik/u1.csv}
\caption{Scores voor U1:  Formulieren}
\label{tabel:evaluatie-gebruik-u1}
\end{table}

\subsection{U2: Invullen van formulier}

\paragraph{\jqm}
TODO

\paragraph{\st}
% The controller can fill an empty form with a model instance.
% Field names that correspond to the object's properties are linked and filled in.
% Setting the value of the correct radiobutton does not work,  a new fieldset has to be created on which the \code{setGroupValue} method needs to be applied.  
% Making form elements read-only can be done by setting the \code{readOnly} or \code{disabled} property to true.
Het invullen van een formulier wordt ondersteund door de MVC-architectuur.
Twee verschillende methoden worden in de POC gebruikt.
Een eerste maakt gebruik van de \code{setRecord} methode van een \code{formpanel}.
De modelinstantie die het formulier zal invullen als parameter worden meegeven.
\st{} zal automatisch de velden invullen waarbij de naam gelijk is aan de eigenschap van het model.
Zo kan tekst op tekstvelden worden gemapt,  nummers op numerieke velden en \code{booleans} op \code{togglefields}.
Een opmerking over het invullen van een \code{radiofield} moet worden gemaakt.
Een model kan worden voorzien met eigenschappen met volgende types:
\begin{enumerate}
  \item auto (Default, implies no conversion)
  \item string
  \item int
  \item float
  \item boolean
  \item date
\end{enumerate}
Er bestaat dus geen vlekkeloze mapping tussen een eigenschap van een model en een \code{radiofield}.
Hetzelfde geldt voor een \code{checkboxfield}.
Om deze in te vullen moet de \code{setGroupValue} van het veld worden aangesproken.


%TODO u13 lijsten en click invullen van formulier,  hier ook...
De tweede methode voor het invullen van formulieren maakt gebruikt van een \code{navigationview} en wordt in U13: Lijsten besproken.

Velden read-only maken kan door objecten te voozien van de \code{readOnly} eigenschap en de waarde op \code{true} te zetten.
Bij \code{radiofields} en \code{checkboxfields} heet deze eigenschap \code{disabled}.


\paragraph{\kendo}
TODO

\paragraph{\lungo}
TODO

\begin{table}[H]
\centering
\pgfplotstabletypeset[
  begin table=\begin{tabular}{p{8cm} p{1cm} p{1cm} p{1cm} p{1cm}},
  end table=\end{tabular},
  skip coltypes=true,
  col sep=comma,
  string type,
  header=true,
  columns={Uitdaging,jQM,ST,Kendo,Lungo},
  columns/Uitdaging/.style={column name=\textbf{Uitdaging}, column type={l}},  
  columns/jQM/.style={column name=\textbf{\jqma}, column type={c}},
  columns/ST/.style={column name=\textbf{\sta}, column type={c}},
  columns/Lungo/.style={column name=\textbf{\lungoa}, column type={c}},
  columns/Kendo/.style={column name=\textbf{\kendoa}, column type={c}},
  every head row/.style={
    before row=\toprule,
    after row=\midrule},
  every last row/.style={
  	before row=\midrule,
    after row=\bottomrule}
]{tabellen/gebruik/u2.csv}
\caption{U2: Invullen van formulier}
\label{tabel:evaluatie-gebruik-u2}
\end{table}

\subsection{U3: Formuliervalidatie}

\paragraph{\jqm}
% Sommige formuliervelden waren verplicht in te vullen, terwijl andere niet. 
% Hiervoor werd eerst gedacht om het \code{required} attribuut in HTML5 te gebruiken. 
% Het probleem is echter dat er geen ondersteuning is voor mobiele browsers~\cite{Deveria2013}. 
% Daarnaast was het ook nodig om de velden te valideren op hun waarde. 
% Validatie is echter niet standaard aanwezig in jQuery Mobile. 
% Als oplossing werd de plug-in van Jörn Zaefferer gebruikt~\cite{Zaefferer2013}. 
% Deze plug-in loste ook het probleem met de verplichte velden op. 
% Deze plug-in kan op twee manieren gebruikt worden: enerzijds annoteren van de formuliervelden met speciale CSS klassen ofwel anderzijds door programmatie met JavaScript. 
% Beide aanpakken werden getest doorheen de POC. 
% 
% De plug-in bevatte de volgende ingebakken validatieregels nodig in de POC: \code{required}, \code{number}, \code{email} en \code{date}.
% Daarnaast was het nodig dat een veld verplicht was enkel indien een bepaalde optie aangevinkt was.
% Zo'n afhankelijkheidsrelatie is standaard aanwezig in de plug-in.
% 
% Bij fouten tegen validatie moest een dialoogvenster worden weergegeven met daarin een beschrijving van alle foute velden.
% Aangezien de plug-in standaard onder het foute formulierveld een foutboodschap toont, diende de plug-in  te worden aangepast.
% Door de uitgebreide API van de plug-in die ook uitvoerig gedocumenteerd is, konden alle foutboodschappen samen in een dialoogvenster worden weergegeven.
% 
% Een specifiek mobiel probleem was bij het tonen van het dialoogvenster, waarbij de plug-in op de achtergrond de cursor op het eerste veld zette. 
% Hierdoor verscheen het toetsenbord op het scherm van het mobiele apparaat wanneer het dialoogvenster tevoorschijn kwam, wat niet de bedoeling is. 
% Dit werd opgelost door \code{focusInvalid:false} in te stellen in de plug-in.
% 
% Bij het sluiten van het dialoogvenster dienden de foute velden met een rode rand te worden gemarkeerd.
% Aangezien de validatie plug-in de foute velden annoteert met de \code{error} CSS klasse, kon de rode rand in CSS worden geprogrammeerd. 
% Dit ging voor \code{input} en \code{textarea}, maar gaf problemen voor \code{select} en \code{fieldset}.
% Door de extra code die jQuery Mobile genereert rond deze velden, moest via de DOM de omringende code geannoteerd worden om de rode rand te bekomen. 
% Deze functie kon aangehaakt worden op de \code{highlight} en \code{unhighlight} functies van de plug-in.

\paragraph{\st}
% Required fields and email validation can be assigned to a model.  
% To add custom validation rules or messages,  the \code{validate} method of \code{Ext.data.Model} needs to be overridden.  
% This method accepts a model instance and returns the possible validation errors.  
% The errors must be iterated to concatenate the validation messages.
% The red borders are created by adding a custom CSS class to the form element.
Een model kan worden voorzien van validatieregels.
Deze regels worden als objecten in een rij aan de \code{validations} eigenschap van een model toegekend.
Volgende validatieregels zijn ingebouwd:
\begin{description}
  \item [presence] verzekert dat het het veld een waarde heeft waarbij nul als geldig wordt beschouwd,  lege tekst niet.
  \item [length] verzekert dat een text een minimale en/of maximale waarde heeft.
  \item [format] verzekert dat een text voldoet aan een opgegeven reguliere expressie.
  \item [inclusion] verzekert dat de waarde van een veld gelijk is aan een element van een gespecifieerde set.
  \item [exclusion] verzekert dat de waarde van een veld zeker niet gelijk is aan een element van een gespecifieerde set.
\end{description}
De controle of een opgegeven waarde een nummer is kan met de \code{format} regel en de \code{/\d+/} reguliere expressie.
Om een bepaalde modelinstantie te valideren moet de \code{validate} methode op de instantie worden opgeroepen.
Om eigen validatieregels toe te laten moet de implementatie van deze methode worden overschreven. \footnote{Informatie gevonden op \exturl{www.sencha.com/forum/showthread.php?122680-Conditional-fields-validations}}
Deze functionaliteit zit dus niet standaard in \st{}.
Met de nieuwe \code{validate} methode kan een \code{validator} aan een validatieregel worden toegevoegd.
Dit is een functie die de programmeur zelf bepaalt en \code{true} of \code{false} teruggeeft bij het al dan niet slagen van een conditie.

Het opbouwen van een foutenboodschap kan door te itereren over de fouten die na validtie werden teruggevonden.
Een specifieke foutenboodschap kan aan elke validatieregel worden toegekend.

Invalide formulierelementen aanduiden met een rode rand wordt niet door \st{} ondersteund.
Hiervoor moet CSS worden gebruikt.
Foutief ingevulde formulierelementen moeten na valiatie met een CSS-klasse worden aangevuld.

\paragraph{\kendo}
TODO

\paragraph{\lungo}
TODO

\begin{table}[H]
\centering
\pgfplotstabletypeset[
  begin table=\begin{tabular}{p{8cm} p{1cm} p{1cm} p{1cm} p{1cm}},
  end table=\end{tabular},
  skip coltypes=true,
  col sep=comma,
  ignore chars={\"},
  verb string type,
  header=true,
  columns={Uitdaging,jQM,ST,Kendo,Lungo},
  columns/Uitdaging/.style={column name=\textbf{Uitdaging}, column type={l}},  
  columns/jQM/.style={column name=\textbf{\jqma}, column type={c}},
  columns/ST/.style={column name=\textbf{\sta}, column type={c}},
  columns/Lungo/.style={column name=\textbf{\lungoa}, column type={c}},
  columns/Kendo/.style={column name=\textbf{\kendoa}, column type={c}},
  every head row/.style={
    before row=\toprule,
    after row=\midrule},
  every last row/.style={
  	before row=\midrule,
    after row=\bottomrule}
]{tabellen/gebruik/u3.csv}
\caption{Scores voor U3: Formuliervalidatie}
\label{tabel:evaluatie-gebruik-u3}
\end{table}

\subsection{U4: Handtekening}

\paragraph{\jqm}
% Er werd gezocht naar een plug-in om deze functionaliteit te bekomen, doordat jQuery Mobile dit niet standaard aanbiedt. 
% Eerst werd gewerkt met Signature Pad van Thomas Bradley~\cite{Bradley2013}. 
% Door de lange tijd die werd besteed aan het aanpassen van layout, werd overgestapt naar jSignature van Willow Systems~\cite{Systems2013}. 
% Deze laatste gaf ook het voordeel dat de breedte van het gebied om te handtekening in te zetten, zich automatisch naar 100\% schaalde. 
% De plug-in maakt gebruik van het HTML5 canvas element en de \code{.toDataURL()} methode.
% Deze wordt echter niet ondersteund op Android versies 2.3 en lager~\cite{Systems2013} waardoor de functionaliteit op die toestellen niet werkt.

\paragraph{\st}
% Drawing a signature is handled by a plugin~\cite{SimFla2011}.  
% Plugins can easily be added to the framework by placing the plugin file in the \code{ux} folder and loading it in the main JavaScript file.  
% This plugin could be used as-is by using the newly \code{signaturefield} component.  
Het tekenen van een handtekening steunt op een plug-in van SimFla~\cite{SimFla2011} en is in de Sencha Market te vinden op \exturl{market.sencha.com/extensions/signature-pad-field}.
Een plug-in aan het raamwerk toevoegen kan door het \js-bestand in de touch/src/ux folder te plaatsen.
Vervolgens moet de plug-in worden geladen bij het initialiseren van de applicatie.

De plug-in maakt een nieuw xtype \code{signaturefield} beschikbaar dat als veld in een formulier kan worden gebruikt.

De plug-in maakt gebruikt van het HTML5-canvas en retourneert de handtekening als geëncodeerde base64 text.


\paragraph{\kendo}
TODO

\paragraph{\lungo}
TODO

\begin{table}[H]
\centering
\pgfplotstabletypeset[
  begin table=\begin{tabular}{p{8cm} p{1cm} p{1cm} p{1cm} p{1cm}},
  end table=\end{tabular},
  skip coltypes=true,
  col sep=comma,
  string type,
  header=true,
  columns={Uitdaging,jQM,ST,Kendo,Lungo},
  columns/Uitdaging/.style={column name=\textbf{Uitdaging}, column type={l}},  
  columns/jQM/.style={column name=\textbf{\jqma}, column type={c}},
  columns/ST/.style={column name=\textbf{\sta}, column type={c}},
  columns/Lungo/.style={column name=\textbf{\lungoa}, column type={c}},
  columns/Kendo/.style={column name=\textbf{\kendoa}, column type={c}},
  every head row/.style={
    before row=\toprule,
    after row=\midrule},
  every last row/.style={
  	before row=\midrule,
    after row=\bottomrule}
]{tabellen/gebruik/u4.csv}
\caption{Scores voor U4: Handtekening}
\label{tabel:evaluatie-gebruik-u4}
\end{table}

\subsection{U5: Toon PDF}

\paragraph{\jqm}
% Het is niet aangeraden om ruwe data, zoals een PDF, op te halen via AJAX. 
% Hierdoor werd gebruik gemaakt van een verborgen formulier met de nodige parameters die de PDF ophaalt bij de backend. 
% Bij het klikken op een lijstitem in het overzicht, wordt dit verborgen formulier opgestuurd naar de backend die dan een PDF teruggeeft in de browser. 
% Het weergeven van de PDF wordt overgelaten aan het mobiel apparaat dat de correcte applicatie hiervoor opstart.

\paragraph{\st}
% A plugin for a PDF viewer can be found at~\cite{Fiedler2012}.  
% Some modifications were necessary to made it compliant with the POC.  
% The PDF must be fetched from the backend with a parameterized POST request instead of a simple GET request.  
% The plugin automatically creates views for every PDF page.  
Het tonen van een PDF steunt op een plug-in van Fiedler~\cite{Fiedler2012} en kan op de Sencha Market gevonden worden op \exturl{market.sencha.com/extensions/pdf-viewer-panel}.
Het tonen van een PDF-bestand kan door de huidige \code{view} te wijzigen naar een \code{Ext.ux.PDF view}.
Deze \code{view} bestaat uit een paneel met een hoofdtekst.
Het paneel toont één pagina van het PDF-bestand,  de hoofdtekst bevat de navigatie naar andere pagina's.

Om de plug-in in de POC in te passen waren echter twee aanpassingen noodzakelijk.
Het PDF-bestand moet via een POST verzoek worden opgehaald waarbij parameters het exacte PDF-bestand aanduiden.
Ook moest er een terugknop in de hoofdtekst van het paneel worden aangebracht om terug naar het overzicht van doorgestuurde formulieren te gaan.
Beide aanpassingen moesten in het \js-bestand van de plug-in worden aangebracht.

\paragraph{\kendo}
TODO

\paragraph{\lungo}
TODO

\begin{table}[H]
\centering
\pgfplotstabletypeset[
  begin table=\begin{tabular}{p{8cm} p{1cm} p{1cm} p{1cm} p{1cm}},
  end table=\end{tabular},
  skip coltypes=true,
  col sep=comma,
  string type,
  header=true,
  columns={Uitdaging,jQM,ST,Kendo,Lungo},
  columns/Uitdaging/.style={column name=\textbf{Uitdaging}, column type={l}},  
  columns/jQM/.style={column name=\textbf{\jqma}, column type={c}},
  columns/ST/.style={column name=\textbf{\sta}, column type={c}},
  columns/Lungo/.style={column name=\textbf{\lungoa}, column type={c}},
  columns/Kendo/.style={column name=\textbf{\kendoa}, column type={c}},
  every head row/.style={
    before row=\toprule,
    after row=\midrule},
  every last row/.style={
  	before row=\midrule,
    after row=\bottomrule}
]{tabellen/gebruik/u5.csv}
\caption{Scores voor U5: Toon PDF}
\label{tabel:evaluatie-gebruik-u5}
\end{table}

\subsection{U6: Toevoegen van afbeelding}

\paragraph{\jqm}
% Het opladen van een bestand kan gebeuren door \code{file} als invoertype van het formulierveld te gebruiken. 
% In versie 1.2 wordt dit veld nog niet opgemaakt met lay-out, maar dit gebeurt wel in versie 1.3~\cite{JQuery2013d}. 
% Voor het kan worden doorgestuurd naar de backend, moet het bewijs eerst lokaal worden omgevormd naar base64. 
% Dit werd geïmplementeerd met de FileReaderAPI en het canvas, wat beide HTML5 specificaties zijn. 
% Het aangeklikte bestand wordt gelezen door middel van de FileReaderAPI, waarna het als afbeelding wordt opgeslagen en geïmporteerd wordt op het canvas. 
% Eenmaal geïmporteerd, kan men de \code{.toDataURL()} oproepen op het canvas om de geïmporteerde afbeelding om te vormen naar base64. 
% Deze aanpak werkt correct op recente mobiele apparaten. 
% De FileReaderAPI wordt echter niet ondersteund op Android versies 2.3 en lager of iOS versies lager dan 6.0~\cite{Deveria2013a} waardoor het opladen van een bewijs niet werkt.
% 
% Het voorvertonen van het geüploade bestand hangt af van het mobiele besturingssysteem.
% Zo wordt op iOS 6 een miniatuurafbeelding getoond, terwijl op Android de bestandsnaam wordt getoond.
% Het is natuurlijk ook mogelijk om de preview na conversie zelf te tonen op het scherm.
% Bij iOS zouden er dan twee voorvertoningen te zien zijn op hetzelfde scherm.

\paragraph{\st}
% A plugin to upload files can be found~\cite{Smirnov2012} to create buttons with the \code{fileupload} component.  
% This button enables users to select an image and passes it to a PHP file.  
% This file uploads the image and converts it to base64.
% A requirement is that your server is able to run that PHP file.    
% A ST image can be created based on this base64 string.
Het opladen van een afbeelding steunt op een plug-in van Smirnov~\cite{Smirnov2012} en kan in de Sencha Market gevonden worden op \exturl{market.sencha.com/extensions/file-uploading-component-for-sencha-touch}.
De plug-in is generiek voor het opladen van elk type bestand,  niet uitsluitend afbeeldingen.
Het \js-bestand moet in de touch/src/ux folder worden geplaatst en de \code{Ext.ux.Fileup} klasse moet worden geïnitialiseerd.
Het xtype \code{img} wordt dan beschikbaar voor \st{} componenten.

De plug-in voorziet twee modes voor het opladen van bestanden: lokale als base64 of extern naar een server.
De eerste laat toe afbeeldingen in het DOM of \term{local storage} te laden.
Dit laatste is een aspect van de POC.

Nadat een bestand is opgeladen kunnen twee gebeurtenissen zich voordoen:  \code{loadsuccess} of \code{loadfailure}.
Het is de taak van een \code{controller} om deze gebeurtenissen op te vangen en een bijhorende methode te definiëren.
De succes functie krijgt de base64 text mee en kan een voorbeeld van de afbeelding laten weergeven.

\paragraph{\kendo}
TODO

\paragraph{\lungo}
TODO

\begin{table}[H]
\centering
\pgfplotstabletypeset[
  begin table=\begin{tabular}{p{8cm} p{1cm} p{1cm} p{1cm} p{1cm}},
  end table=\end{tabular},
  skip coltypes=true,
  col sep=comma,
  string type,
  header=true,
  columns={Uitdaging,jQM,ST,Kendo,Lungo},
  columns/Uitdaging/.style={column name=\textbf{Uitdaging}, column type={l}},  
  columns/jQM/.style={column name=\textbf{\jqma}, column type={c}},
  columns/ST/.style={column name=\textbf{\sta}, column type={c}},
  columns/Lungo/.style={column name=\textbf{\lungoa}, column type={c}},
  columns/Kendo/.style={column name=\textbf{\kendoa}, column type={c}},
  every head row/.style={
    before row=\toprule,
    after row=\midrule},
  every last row/.style={
  	before row=\midrule,
    after row=\bottomrule}
]{tabellen/gebruik/u6.csv}
\caption{Scores voor U6: Toevoegen van afbeelding}
\label{tabel:evaluatie-gebruik-u6}
\end{table}

\subsection{U7: Auto-aanvullen}

\paragraph{\jqm}

% Hoewel versie 1.3 automatische aanvulling ter beschikking heeft~\cite{JQuery2013c}, werd tijdens de implementatie gebruik gemaakt van versie 1.2 die dit niet had. 
% Daarom werd de plug-in van Andy Matthews gebruikt~\cite{Matthews2013}. 
% Dit is een zeer gemakkelijk te integreren plug-in die zowel met lokale data als data op afstand kan werken. 
% Daarnaast dienden enkel vijf suggesties te worden getoond. 
% Deze functionaliteit zat niet in de plug-in, maar werd geïmplementeerd met de JavaScript \code{slice} functie.

\paragraph{\st}
% A plugin can be found at~\cite{Mysamplecode2012} to create an autocomplete field.  
% A request to the backend with a keyword returns a JSON array named \code{data}.  
% ST can parse this array in two ways:  with a \code{JsonReader} or an \code{ArrayReader}.  
% The first requires that a JSON key precedes each item,  the latter assumes each item in the array maps to a field of a model.  
% Both strategies cannot be used to parse the array and create separate model instances for each array item.  
% This implies that no clickable drop-down could be implemented.
Het automatisch aanvullen van een formulierelement steunt op een plug-in van Tajur. %TODO referentie https://github.com/martintajur/sencha-touch-2-autocomplete-textfield
Deze plug-in is niet op de Sencha Market terug te vinden.
Door het \js-bestand toe te voegen wordt het xtype \code{autocompletefield} beschikbaar.
Een object met dit xtype kan een \code{proxy} definiëren die de server kan aanspreken om suggesties asynchroon op te halen.
Ook is het mogelijk het maximaal aantal suggesties vast te leggen.

De \term{backend} server die bij de POC hoort geeft bij een bepaald sleutelwoord suggesties in een JSON-rij terug.
De rij is voorzien van een sleutel maar alle elementen van de rij hebben geen sleutel.
\st{} voorziet vier methoden om de resultaten van een \code{proxy} te parsen naar modelinstanties:
\begin{description}
 \item [\code{JsonReader}] parst JSON-sleutels naar model velden.
 \item [\code{XmlReader}] parst XML-tags naar model velden.
 \item [\code{ArrayReader}] mapt elementen van een rij op velden van een model.
\end{description}
Geen van voorgaande methoden was in staat de rij met suggesties te parsen van rij-element naar modelinstantie.
Hierdoor kon geen klikbare dropdownmenu worden getoond.

\paragraph{\kendo}
TODO

\paragraph{\lungo}
TODO

\begin{table}[H]
\centering
\pgfplotstabletypeset[
  begin table=\begin{tabular}{p{8cm} p{1cm} p{1cm} p{1cm} p{1cm}},
  end table=\end{tabular},
  skip coltypes=true,
  col sep=comma,
  string type,
  header=true,
  columns={Uitdaging,jQM,ST,Kendo,Lungo},
  columns/Uitdaging/.style={column name=\textbf{Uitdaging}, column type={l}},  
  columns/jQM/.style={column name=\textbf{\jqma}, column type={c}},
  columns/ST/.style={column name=\textbf{\sta}, column type={c}},
  columns/Lungo/.style={column name=\textbf{\lungoa}, column type={c}},
  columns/Kendo/.style={column name=\textbf{\kendoa}, column type={c}},
  every head row/.style={
    before row=\toprule,
    after row=\midrule},
  every last row/.style={
  	before row=\midrule,
    after row=\bottomrule}
]{tabellen/gebruik/u7.csv}
\caption{Scores voor U7: Auto-aanvullen}
\label{tabel:evaluatie-gebruik-u7}
\end{table}

\subsection{U8: AJAX}

\paragraph{\jqm}

% Het maken van oproepen via AJAX gebeurt via de jQuery bibliotheek waar jQuery Mobile op steunt. 
% Dit gebeurt met de functie \code{\$.ajax} waar onder andere kan ingesteld worden wat het te verwachten antwoord is (zoals tekst, JSON of XML). 
% Bij het succesvol uitvoeren van de oproep wordt de \code{succes} functie opgeroepen, bij faling de \code{error} functie waarna een relevante foutboodschap wordt getoond.
% Het afmelden zonder antwoord, het aanmelden voor het bekomen van het token (tekst), het ophalen van de gebruikersgegevens (JSON), het ophalen van de uitgaveformulieren (XML) en het ophalen van de omwisselingskoersen (XML) ging zonder enig probleem.

% In jQuery is er de functie \code{parseJSON} aanwezig, maar aangezien we in de AJAX oproep instellen dat we JSON verwachten, parst jQuery al automatisch het antwoord. 
% Hierdoor hebben we \code{parseJSON} niet nodig en kunnen we direct omgaan met het antwoord.
% 
% Het is ook nodig om JSON te versturen als oproep naar de backend. 
% Dit wordt gedaan vanuit JavaScript zonder een jQuery nodig te hebben. 
% Eerst wordt een object met de nodige inhoud aangemaakt, waarop daarna de functie \code{JSON.stringify} opgeroepen wordt die het object in een string omzet.
% Deze is daarna klaar om te worden verstuurd als data via een AJAX oproep met behulp van jQuery.

% Net zoals bij JSON het geval was, was het ook niet nodig om expliciet de \code{parseXML} functie te gebruiken. 
%Het doorlopen en opvragen van gegevens uit het XML-bestand vraagt meer werk. 
%Waar je bij JSON direct aan de data kon, moet je bij XML de data ophalen net zoals je dat zou doen uit een HTML-pagina. 
%Dit betekent dus met selectoren aan de hand van de jQuery bibliotheek.

\paragraph{\st}
% AJAX requests can be done either explicitly via a direct \code{Ext.Ajax.request} call or implicitly via stores.  
% The plain text of the AJAX response can be used in the callback.   
% Stores can be configured with a model to define the structure of the recorded objects.  
% A proxy configures readers and writers that define where the data can be read or written.  
% This can be locally at the client side or via a remote server.  
% Readers and writers contain the format of the data - JSON or XML - and automatically parse this data to fields of the corresponding model. 
% Sending JSON payload must be done via an AJAX request where the \code{jsonData} is encoded via \code{Ext.encode}.
AJAX-verzoeken kunnen zowel expliciet via een directe oproep met \code{Ext.Ajax.request} als impliciet via \code{stores} worden uitgevoerd.
De expliciete oproep is gelijkaardig aan de \code{\$.ajax} methode van jQuery.
Een enige uitzondering is te vinden bij kruis-domein AJAX-verzoeken.
Om aan de CORS-standaarden (Cross-Origin Resource Sharing) te voldoen moet de eigenschap \code{useDefaultXhrHeader} op \code{false} worden gezet.
%TODO referentie cors + opzoeken options request

De tweede,  impliciete,  methode voor AJAX-verzoeken is via \code{stores}.
Een \code{store} wordt voorzien van een \code{proxy}.  
Deze kan data aan de client of server zijde opslaan.  
Een \code{proxy} voor opslag aan client zijde kan zowel in het RAM-geheugen als in de \term{local storage} en \term{session storage} van de browser opslaan.  
Een \code{proxy} voor server opslag kan data verzenden via AJAX (zelfde domein) of JSONP (verschillende domeinen).  
Een \code{proxy} kan ook geconfigureerd worden met \code{readers} en \code{writers} om data van de server te lezen of naar de server te schrijven.

Het verzenden van een JSON-\term{payload} moet via een expliciet AJAX-verzoek gebeuren.
Data kan via \code{Ext.encode} naar JSON worden geëncodeerd en via de \code{jsonData} eigenschap aan het verzoek worden gekoppeld.

\paragraph{\kendo}
TODO

\paragraph{\lungo}
TODO

\begin{table}[H]
\centering
\pgfplotstabletypeset[
  begin table=\begin{tabular}{p{8cm} p{1cm} p{1cm} p{1cm} p{1cm}},
  end table=\end{tabular},
  skip coltypes=true,
  col sep=comma,
  string type,
  header=true,
  columns={Uitdaging,jQM,ST,Kendo,Lungo},
  columns/Uitdaging/.style={column name=\textbf{Uitdaging}, column type={l}},  
  columns/jQM/.style={column name=\textbf{\jqma}, column type={c}},
  columns/ST/.style={column name=\textbf{\sta}, column type={c}},
  columns/Lungo/.style={column name=\textbf{\lungoa}, column type={c}},
  columns/Kendo/.style={column name=\textbf{\kendoa}, column type={c}},
  every head row/.style={
    before row=\toprule,
    after row=\midrule},
  every last row/.style={
  	before row=\midrule,
    after row=\bottomrule}
]{tabellen/gebruik/u8.csv}
\caption{Scores voor U8: AJAX}
\label{tabel:evaluatie-gebruik-u8}
\end{table}

\subsection{U9: Toestelspecifieke lay-out}

\paragraph{\jqm}
TODO

\paragraph{\st}
% Detection of a smartphone context is done via the \code{Ext.os.is.Phone} method.  
% If this method returns false,  we assume to be in tablet mode.    
% The main screen of the POC requires a splitted view in tablet mode.  
% A \code{vbox} layout splits the viewport with a vertical axis.  
% The \code{flex} property defines the ratio of the sizes of both resulting components.  
% %TODO niet meer relevant The right component is the page that changes by clicking the menu items in the left component.  
% %This can be realized via a \code{card} layout where changing the page implies setting the active component of the \code{card} layout.
% In smartphone mode,  the left screen is made invisible and an extra menubutton in the header is created.  
% Making the sub header clickable is not possible  
\st{} ondersteunt het herkennen van de context waarin de applicatie wordt gebruikt.
\st{} kan zowel besturingssysteem, browser als ondersteunde (HTML5-)kenmerken opvragen en herkennen.
Het besturingssysteem kan bevraagd worden via \code{Ext.os.name}.
Deze methode herkent onder andere Android, iOS, Windows en BlackBerry.
Er kan ook gebruik worden gemaakt van de \term{singleton} klasse \code{Ext.os.is}.
Zo geeft \code{Ext.os.is.Android} terug of Android het gebruikte besturingssysteem is of niet.
Het opvragen en herkennen van browser en (HTML5-)kenmerken gebeurt op een analoge manier.


\st{} voorziet vijf lay-outs die aan een component kunnen worden toegekend:
\begin{description}
 \item [\code{HBox}] plaatst de componenten horizontaal naast elkaar.
 \item [\code{VBox}] plaatst de componenten verticaal onder elkaar.
 \item [\code{Card}] plaatst de componenten boven elkaar.
 \item [\code{Fit}] maakt de component passend voor zijn ouder container.
 \item [\code{Docking}] maakt het plaatsen van extra componenten mogelijk in de top-, rechter-, bodem- of linkerrand van zijn ouder container.
\end{description}
De creatie van de tablet lay-out steunt op de \code{HBox} lay-out.
De \code{flex} eigenschap van deze layout definieert de ratio van de groottes van beide componenten.
De creatie van de smartphone lay-out maakt het menu in de linkse component van de lay-out onzichtbaar.
Om naar het menu terug te keren moet een extra knop in de hoofdtekst worden toegevoegd die naar het menu navigeert.
\st{} ondersteund geen klikbare hoofdteksten die deze functionaliteit toelaten.

\paragraph{\kendo}
TODO

\paragraph{\lungo}
TODO

\begin{table}[H]
\centering
\pgfplotstabletypeset[
  begin table=\begin{tabular}{p{8cm} p{1cm} p{1cm} p{1cm} p{1cm}},
  end table=\end{tabular},
  skip coltypes=true,
  col sep=comma,
  string type,
  header=true,
  columns={Uitdaging,jQM,ST,Kendo,Lungo},
  columns/Uitdaging/.style={column name=\textbf{Uitdaging}, column type={l}},  
  columns/jQM/.style={column name=\textbf{\jqma}, column type={c}},
  columns/ST/.style={column name=\textbf{\sta}, column type={c}},
  columns/Lungo/.style={column name=\textbf{\lungoa}, column type={c}},
  columns/Kendo/.style={column name=\textbf{\kendoa}, column type={c}},
  every head row/.style={
    before row=\toprule,
    after row=\midrule},
  every last row/.style={
  	before row=\midrule,
    after row=\bottomrule}
]{tabellen/gebruik/u9.csv}
\caption{Scores voor U9: Toestelspecifieke lay-out}
\label{tabel:evaluatie-gebruik-u9}
\end{table}

\subsection{U10: Offline}

\paragraph{\jqm}
TODO

\paragraph{\st}
TODO

\paragraph{\kendo}
TODO

\paragraph{\lungo}
TODO

\begin{table}[H]
\centering
\pgfplotstabletypeset[
  begin table=\begin{tabular}{p{8cm} p{1cm} p{1cm} p{1cm} p{1cm}},
  end table=\end{tabular},
  skip coltypes=true,
  col sep=comma,
  string type,
  header=true,
  columns={Uitdaging,jQM,ST,Kendo,Lungo},
  columns/Uitdaging/.style={column name=\textbf{Uitdaging}, column type={l}},  
  columns/jQM/.style={column name=\textbf{\jqma}, column type={c}},
  columns/ST/.style={column name=\textbf{\sta}, column type={c}},
  columns/Lungo/.style={column name=\textbf{\lungoa}, column type={c}},
  columns/Kendo/.style={column name=\textbf{\kendoa}, column type={c}},
  every head row/.style={
    before row=\toprule,
    after row=\midrule},
  every last row/.style={
  	before row=\midrule,
    after row=\bottomrule}
]{tabellen/gebruik/u10.csv}
\caption{Scores voor U10: Offline}
\label{tabel:evaluatie-gebruik-u10}
\end{table}

\subsection{U11: Laadscherm en dialoogvenster} 

\paragraph{\jqm}

% Het standaard laadscherm is enkel een \term{spinner} die niet opvallend aanwezig is en ook zonder een tekst eronder ronddraait.
% Door de opties in de API te gebruiken, komt de \term{spinner} duidelijk naar voor en staat er ook een tekst onder.

% Eerst werd gebruik gemaakt van DateBox \cite{Sage2013} als plug-in om op een gemakkelijke manier een dialoogvenster te tonen.
% Uiteindelijk bleek de plug-in niet zo gemakkelijk aanpasbaar en daarenboven zijn dialoogvensters standaard in jQuery Mobile aanwezig.
% Het is dan ook helemaal niet nodig om hiervoor een plug-in te gebruiken.
% Door zelf de dialoogvenster met jQuery Mobile aan te maken, kon de layout minimier aangepast worden.

\paragraph{\st}
TODO

\paragraph{\kendo}
TODO

\paragraph{\lungo}
TODO

\begin{table}[H]
\centering
\pgfplotstabletypeset[
  begin table=\begin{tabular}{p{8cm} p{1cm} p{1cm} p{1cm} p{1cm}},
  end table=\end{tabular},
  skip coltypes=true,
  col sep=comma,
  string type,
  header=true,
  columns={Uitdaging,jQM,ST,Kendo,Lungo},
  columns/Uitdaging/.style={column name=\textbf{Uitdaging}, column type={l}},  
  columns/jQM/.style={column name=\textbf{\jqma}, column type={c}},
  columns/ST/.style={column name=\textbf{\sta}, column type={c}},
  columns/Lungo/.style={column name=\textbf{\lungoa}, column type={c}},
  columns/Kendo/.style={column name=\textbf{\kendoa}, column type={c}},
  every head row/.style={
    before row=\toprule,
    after row=\midrule},
  every last row/.style={
  	before row=\midrule,
    after row=\bottomrule}
]{tabellen/gebruik/u11.csv}
\caption{Scores voor U11: Laadscherm en dialoogvenster}
\label{tabel:evaluatie-gebruik-u11}
\end{table}

\subsection{U12: Dataconversie}

\paragraph{\jqm}
TODO

\paragraph{\st}
TODO

\paragraph{\kendo}
TODO

\paragraph{\lungo}
TODO

\begin{table}[H]
\centering
\pgfplotstabletypeset[
  begin table=\begin{tabular}{p{8cm} p{1cm} p{1cm} p{1cm} p{1cm}},
  end table=\end{tabular},
  skip coltypes=true,
  col sep=comma,
  string type,
  header=true,
  columns={Uitdaging,jQM,ST,Kendo,Lungo},
  columns/Uitdaging/.style={column name=\textbf{Uitdaging}, column type={l}},  
  columns/jQM/.style={column name=\textbf{\jqma}, column type={c}},
  columns/ST/.style={column name=\textbf{\sta}, column type={c}},
  columns/Lungo/.style={column name=\textbf{\lungoa}, column type={c}},
  columns/Kendo/.style={column name=\textbf{\kendoa}, column type={c}},
  every head row/.style={
    before row=\toprule,
    after row=\midrule},
  every last row/.style={
  	before row=\midrule,
    after row=\bottomrule}
]{tabellen/gebruik/u12.csv}
\caption{Scores voor U12: Dataconversie}
\label{tabel:evaluatie-gebruik-u12}
\end{table}

\subsection{U13: Lijsten}

\paragraph{\jqm}
% Het aanmaken van lijsten gebeurt met de \term{listview} widget.

\paragraph{\st}
Navigationview:
Deze \code{view} heeft een \code{push} en \code{pop} methode om een \code{view} op een \code{stack} te plaatsen of af te halen.
De 

\paragraph{\kendo}
TODO

\paragraph{\lungo}
TODO

\begin{table}[H]
\centering
\pgfplotstabletypeset[
  begin table=\begin{tabular}{p{8cm} p{1cm} p{1cm} p{1cm} p{1cm}},
  end table=\end{tabular},
  skip coltypes=true,
  col sep=comma,
  string type,
  header=true,
  columns={Uitdaging,jQM,ST,Kendo,Lungo},
  columns/Uitdaging/.style={column name=\textbf{Uitdaging}, column type={l}},  
  columns/jQM/.style={column name=\textbf{\jqma}, column type={c}},
  columns/ST/.style={column name=\textbf{\sta}, column type={c}},
  columns/Lungo/.style={column name=\textbf{\lungoa}, column type={c}},
  columns/Kendo/.style={column name=\textbf{\kendoa}, column type={c}},
  every head row/.style={
    before row=\toprule,
    after row=\midrule},
  every last row/.style={
  	before row=\midrule,
    after row=\bottomrule}
]{tabellen/gebruik/u13.csv}
\caption{Scores voor U13: Lijsten}
\label{tabel:evaluatie-gebruik-u13}
\end{table}

\subsection{U14: Anatomie van pagina}

\paragraph{\jqm}
TODO

\paragraph{\st}
TODO

\paragraph{\kendo}
TODO

\paragraph{\lungo}
TODO

\begin{table}[H]
\centering
\pgfplotstabletypeset[
  begin table=\begin{tabular}{p{8cm} p{1cm} p{1cm} p{1cm} p{1cm}},
  end table=\end{tabular},
  skip coltypes=true,
  col sep=comma,
  string type,
  header=true,
  columns={Uitdaging,jQM,ST,Kendo,Lungo},
  columns/Uitdaging/.style={column name=\textbf{Uitdaging}, column type={l}},  
  columns/jQM/.style={column name=\textbf{\jqma}, column type={c}},
  columns/ST/.style={column name=\textbf{\sta}, column type={c}},
  columns/Lungo/.style={column name=\textbf{\lungoa}, column type={c}},
  columns/Kendo/.style={column name=\textbf{\kendoa}, column type={c}},
  every head row/.style={
    before row=\toprule,
    after row=\midrule},
  every last row/.style={
  	before row=\midrule,
    after row=\bottomrule}
]{tabellen/gebruik/u14.csv}
\caption{Scores voor U14: Anatomie van pagina}
\label{tabel:evaluatie-gebruik-u14}
\end{table}

\subsection{Overzicht}

\begin{table}[H]
\centering
\pgfplotstabletypeset[
  col sep=comma,
  string type,
  header=true,
  columns={Uitdaging,jQM,ST,Kendo,Lungo},
  columns/Uitdaging/.style={column name=\textbf{Uitdaging}, column type={l}},  
  columns/jQM/.style={column name=\textbf{\jqma}, column type={c}},
  columns/ST/.style={column name=\textbf{\sta}, column type={c}},
  columns/Lungo/.style={column name=\textbf{\lungoa}, column type={c}},
  columns/Kendo/.style={column name=\textbf{\kendoa}, column type={c}},
  every head row/.style={
    before row=\toprule,
    after row=\midrule},
  every last row/.style={
  	before row=\midrule,
    after row=\bottomrule}
]{tabellen/gebruik.csv}
\caption{Samenvattende tabel voor gebruikscriterium}
\label{tabel:evaluatie-gebruik}
\end{table}

\section{Ondersteuning}
\label{sec:evaluatie-ondersteuning}

\begin{table}[H]
\centering
\pgfplotstabletypeset[
  col sep=comma,
  string type,
  header=true,
  columns={Apparaat,jQM,ST,Kendo,Lungo},
  columns/Apparaat/.style={column name=\textbf{Apparaat}, column type={l}},  
  columns/jQM/.style={column name=\textbf{\jqma}, column type={c}},
  columns/ST/.style={column name=\textbf{\sta}, column type={c}},
  columns/Kendo/.style={column name=\textbf{\kendoa}, column type={c}},
  columns/Lungo/.style={column name=\textbf{\lungoa}, column type={c}},
  every head row/.style={
    before row=\toprule,
    after row=\midrule},
  every last row/.style={
  	before row=\toprule,
 	after row=\bottomrule}
]{tabellen/ondersteuning.csv}
\caption{Samenvattende tabel voor ondersteuningscriterium}
\label{tabel:evaluatie-ondersteuning}
\end{table}

\section{Performantie}
\label{sec:evaluatie-performantie}

\begin{table}[H]
\centering
\pgfplotstabletypeset[
  col sep=comma,
  string type,
  header=true,
  columns={Performantie,jQM,ST,Kendo,Lungo},
  columns/Performantie/.style={column name=\textbf{Performantie}, column type={l}},  
  columns/jQM/.style={column name=\textbf{\jqm}, column type={c}},
  columns/ST/.style={column name=\textbf{\st}, column type={c}},
  columns/Kendo/.style={column name=\textbf{\kendo}, column type={c}},
  columns/Lungo/.style={column name=\textbf{\lungo}, column type={c}},
  every head row/.style={
    before row=\toprule,
    after row=\midrule},
  every last row/.style={
    after row=\bottomrule}
]{tabellen/performantie.csv}
\caption{Samenvattende tabel voor performantiecriterium}
\label{tabel:evaluatie-performantie}
\end{table}



% TODO Tim: opruimen hieronder
% \subsubsection{Tabbalk}
% 
% \paragraph{jQuery Mobile} 
% Standaard is er een tabbalk aanwezig in jQuery Mobile, maar de POC impliceerde een tabbalk die niet de volledige breedte innam.
% Daarom werd gekozen voor \code{fieldset} met twee opties.
% 
% \subsubsection{Inlogscherm indien niet aangemeld}
% 
% \paragraph{jQuery Mobile} 
% Indien men een applicatie maakt met meerdere schermen op eenzelfde pagina, laadt jQuery Mobile altijd het eerste scherm in de code in.
% Het startscherm werd als eerste scherm gekozen.
% Indien gemerkt wordt dat de gebruiker niet aangemeld was, dan wordt hij doorverwezen naar het inlogscherm.
% 
% \subsubsection{Detail van toegevoegde uitgave}
% 
% \paragraph{jQuery Mobile} 
% Na het toevoegen van een uitgave, is het mogelijk om deze opnieuw te bekijken (maar niet aan te passen).
% Hiervoor wordt hetzelfde formulier (dat om een uitgave toe te voegen wordt gebruikt) gekopieerd, waarna alle elementen op enkel lezen worden gezet.
% Dit was geen probleem voor velden van het type \code{input} en \code{textarea}. 
% Dit kon echter niet  bij \code{fieldset}. 
% Daar moesten via \code{disabled} de andere opties onmogelijk worden gemaakt.
% Eenzelfde probleem gold voor het \code{select} formuliertype bij een buitenlandse uitgave.
% Daar werd enkel de geselecteerde optie in het lijstje getoond en alle andere opties eruit verwijderd.
% 
% Het invullen van het formulier zelf werd bekomen door de \code{id}'s van de velden op te vragen en hun waarde in te stellen volgens de JSON voorstelling van die uitgave. 
% Er is met andere woorden geen automatische mapping van de JSON data naar de formuliervelden. 
% 
% 
% \subsubsection{Omvormen van valuta}
% 
% \paragraph{jQuery Mobile} 
% De omvorming bij een buitenlandse uitgave dient automatisch te gebeuren bij het ingeven van bedrag en munteenheid.
% Hiervoor wordt aangehaakt op het veranderingsevenement \code{.change} dat jQuery aanbiedt, waarna na omvorming het bedrag direct getoond wordt aan de gebruiker.
% 
% \subsubsection{Sorteren}
% 
% \paragraph{jQuery Mobile} 
% Het sorteren van data werd geïmplementeerd door eerst in JavaScript een vergelijkingsfunctie te schrijven.
% Daarna wordt deze functie meegegeven aan de sorteerfunctie die ook in JavaScript aanwezig is.
% Er komt dus geen functionaliteit van het raamwerk om data te sorteren.
% 
% \subsubsection{Offline} 
% TODO
% 
% %%%%%%%
% 
% \subsubsection{Tablet en smartphone}
% 
% \paragraph{jQuery Mobile} 
% In jQuery Mobile is er standaard geen splitview aanwezig om een menu te tonen voor tablets, maar niet voor smartphones. 
% Eerst werd hiervoor gezocht naar plug-ins aan de hand van~\cite{Deering2012}, wat leidde tot: Splitview~\cite{Rahman2013}, SimpleSplitView~\cite{Yared2013} en Multiview~\cite{Franck2012}. 
% Deze drie mogelijke kanshebbers hadden elk hun tekorten. 
% Zo was de eerste destructief ten opzichte van het raamwerk. 
% Dit betekent dat de bestanden van het raamwerk zelf werden aangepast, wat het moeilijker maakt als men wil updaten naar een nieuwe versie. 
% De tweede plug-in werkte enkel tot versie 1.0.1 van jQuery Mobile. 
% De laatste plug-in had moeite met het zich aanpassen aan veranderende afmetingen van de browser. 
% 
% Uiteindelijk werd van een plug-in afgestapt door \cite{Hadlock2012} waarbij werd aangetoond hoe men via CSS3 media queries hetzelfde kan bereiken. 
% Daarnaast gebruikt de documentatie van jQuery Mobile~1.2 een gelijkaardige layout~\cite{JQuery2012b}. 
% De uiteindelijke oplossing voor het probleem kwam uit te combinatie van deze twee voorgaande oplossingen.
% Ook uit de documentatie van versie 1.3 \cite{JQuery2013e} blijkt dat dit de correcte manier is om hiermee om te gaan.
% 
% Navigatie op een smartphone gebeurt door te klikken op de extra titel onder de koptekst. 
% Hierdoor ga je naar een smartphone vriendelijk menu om naar andere stappen te gaan.
% 
% \subsubsection{Koptekst en voettekst}
% 
% \paragraph{jQuery Mobile}
% Het toevoegen van een koptekst en voettekst ging zonder enig probleem door gebruik te maken van \code{data-role="header"} en \code{data-role="footer"}. 
% Wel moest dezelfde code op ieder scherm worden herhaald. 
% Dit kan worden vermeden door gebruik te maken van eenzelfde \code{data-id} attribuut. 
% Daarnaast werd de voettekst gefixeerd aan de onderkant van het scherm en de bijhorende logo's links en rechts uitgelijnd. 
% Voor dit laatste werd gebruik gemaakt van de zogenaamde \term{grid} die jQuery Mobile aanbiedt. 
% Deze voettekst wordt niet getoond op een smartphone, wat wordt bekomen door gebruik te maken van de CSS3 media queries.
% 
% Bij het toevoegen van een uitgave dient er een extra titel onder de koptekst te komen. 
% Eerst werd geprobeerd om bovenaan een lijstdeler te plaatsen, maar dan schoof de inhoud van de pagina niet mee naar onder. 
% De uiteindelijke oplossing kwam vanuit de documentatie \cite{JQuery2013b} om dit met behulp van de \code{ui-bar} CSS klasse te implementeren. 
% Deze extra titel wordt ook gebruikt om op de smartphone naar de speciale smartphone navigatie te gaan (zie ook vorige sectie).
% 
% \subsubsection{Knoppen}
% 
% \paragraph{jQuery Mobile} 
% De kleur van de knoppen aanpassen kan op twee manieren. 
% Ofwel schrijft men zelf de CSS-code ofwel gebruikt men ThemeRoller~\cite{JQuery2012c}. 
% Deze laatste manier werd gebruikt om de knoppen groen te maken. 
% Men sleept dan eenvoudigweg in die webinterface de groene kleur op de knop en daarna kan de bijhorende CSS-code worden gedownload. 
% Door daarna de knop te annoteren met het \code{data-theme} attribuut activeert men het betreffende thema. 
% Om de knop blauw te maken was er geen nood aan een aanpassing, doordat blauw al één van de standaard thema's was en men die direct kan gebruiken.
% 
% Knoppen toevoegen aan de koptekst gaat ook op een zeer eenvoudige manier.
% % TODO: verder schrijven