%TODO architectuur ook in artikel vervangen door design pattern? Poster: Architectuur => ontwerppatroon
%TODO overal [H] verwijderen, dat lost die spacing problemen op ok (TIM doet dit)
%TODO commandos uitdagingen zoals gebruikt in tekst
\documentclass[master=cws,dutch,masteroption={vs,gs},inputenc=utf8]{kulemt}
\setup{title={Vergelijkende studie van raamwerken voor de ontwikkeling van mobiele HTML5-applicaties},
  author={Tim Ameye\and Sander Van Loock},
  promotor={Prof.\,dr.\,ir.\ E. Duval},
  assessor={Prof.\,dr.\,M.\ Denecker\and F. Van Assche},
  assistant={Ir.\ G.~Parra}} %TODO jan en jannik toevoegen?
% De volgende \setup mag verwijderd worden als geen fiche gewenst is.
\setup{filingcard,
  translatedtitle={Comparative study of frameworks for the development of mobile HTML5 applications},
  udc=681.3,
  shortabstract={} % TODO copy paste van samenvatting
}
% Verwijder de "%" op de volgende lijn als je de kaft wil afdrukken
%\setup{coverpageonly}
% Verwijder de "%" op de volgende lijn als je enkel de eerste pagina's wil
% afdrukken en de rest bv. via Word aanmaken.
%\setup{frontpagesonly}

% Kies de fonts voor de gewone tekst, bv. Latin Modern
\setup{font=lm}

% Tenslotte wordt hyperref gebruikt voor pdf bestanden.
% Dit mag verwijderd worden voor de af te drukken versie.
\usepackage[pdfusetitle,colorlinks,plainpages=false]{hyperref}
\usepackage{kulemtx}
\headstyles{kulemtman}

\renewcommand*\kulemtmanToC{%
\let\cftchapterfont\sffamily
\let\cftchapterdotsep\cftdotsep
\def\cftchapterleader{\normalfont\cftdotfill{\cftchapterdotsep}}%
\def\cftchapterpagefont{}%
\setlength\cftbeforechapterskip{\medskipamount}%
\setlength\cftbeforesectionskip{\smallskipamount}%
\settocdepth{section}%
\addtodef\cftchapterbreak{\par}{}%
\let\cftsubsectionfont\itshape
\def\l@subsection##1##2{%
\leftskip\cftsubsectionindent \rightskip\@tocrmarg \parfillskip\fill
\ifhmode ,\quad \else\noindent\fi \ignorespaces
{\let\numberline\@gobble \cftsubsectionfont ##1}%
~~{\cftsubsectionpagefont ##2}\ignorespaces}}
\kulemtmanToC


\usepackage{pgfplotstable} %nodig voor CSV to Latex
\usepackage{booktabs} % voor lay-out comparisontable
\usepackage{float} %floating tabellen onderdrukken
\usepackage{pdflscape} %tabel in landscape zetten
\usepackage{rotating}
\usepackage{enumitem}
\usepackage{longtable} %tabel over meerdere pagina's
\usepackage{graphicx} %breedte tabel
\usepackage{subfig}
\usepackage{tocvsec2}
\usepackage{amsmath} % voor tekst in math
\newcommand{\unit}[1]{\ensuremath{\, \mathrm{#1}}} % units in math mode
\usepackage{wasysym}
\usepackage{pdfpages}

\newcommand{\term}[1]{\emph{#1}} % engelse term die niet worden vertaald naar het nederlands
\newcommand{\code}[1]{\texttt{#1}} % code command

\hyphenation{platform-onafhankelijke}

% veelgebruikte woorden, apparaten, uitdagingen
\newcommand{\jqm}[0]{jQuery Mobile}
\newcommand{\jqma}[0]{jQM}
\newcommand{\st}[0]{Sencha Touch}
\newcommand{\sta}[0]{ST}
\newcommand{\kendo}[0]{Kendo UI}
\newcommand{\kendoa}[0]{Kendo}
\newcommand{\lungo}[0]{Lungo}
\newcommand{\lungoa}[0]{Lungo}
\newcommand{\tmp}[0]{The-M-Project}
\newcommand{\quo}[0]{QuoJS}
\newcommand{\moobile}[0]{Moobile}
\newcommand{\davinci}[0]{DaVinci}
\newcommand{\jqt}[0]{jQT}
\newcommand{\js}[0]{JavaScript}
\newcommand{\htc}[0]{HTCDesireZ}
\newcommand{\gtab}[0]{GalaxyTab}
\newcommand{\gs}[0]{GalaxyS}
\newcommand{\nexus}[0]{Nexus 7}
\newcommand{\ipadi}[0]{iPad1 WiFi}
\newcommand{\ipadiii}[0]{iPad3 4G WiFi}
\newcommand{\iphoneiii}[0]{iPhone 3GS}
\newcommand{\iphoneiv}[0]{iPhone 4S}
\newcommand*{\uit}[1]{\switch{#1}}
\newcommand*{\chal}[1]{\dothis{en}{#1}}

\usepackage{pdftexcmds}
\makeatletter
\newcommand{\switch}[1]{
  \ifnum\pdf@strcmp{#1}{anatomie}=0
    U1:~Anatomie van pagina%
  \else\ifnum\pdf@strcmp{#1}{toestel}=0
    U2:~Toestelspecifieke lay-out%
  \else\ifnum\pdf@strcmp{#1}{laadscherm}=0
    U3:~Laadscherm en dialoogvenster%
  \else\ifnum\pdf@strcmp{#1}{formulieren}=0
    U4:~Formulieren%
  \else\ifnum\pdf@strcmp{#1}{vullen}=0
    U5:~Automatisch invullen van formulier%
  \else\ifnum\pdf@strcmp{#1}{autoaanvullen}=0
    U6:~Auto-aanvullen%
  \else\ifnum\pdf@strcmp{#1}{afbeelding}=0
    U7:~Toevoegen van afbeelding%
  \else\ifnum\pdf@strcmp{#1}{validatie}=0
    U8:~Formuliervalidatie%
  \else\ifnum\pdf@strcmp{#1}{handtekening}=0
    U9:~Handtekening%
  \else\ifnum\pdf@strcmp{#1}{ajax}=0
    U10:~AJAX%
  \else\ifnum\pdf@strcmp{#1}{lijsten}=0
    U11:~Lijsten%
  \else\ifnum\pdf@strcmp{#1}{pdf}=0
    U12:~Toon PDF%
  \else\ifnum\pdf@strcmp{#1}{offline}=0
    U13:~Offline%
  \else
    [?]%
  \fi\fi\fi\fi\fi\fi\fi\fi\fi\fi\fi\fi\fi}
\makeatother

\begin{document}

% \begin{preface}
%   Dit is mijn dankwoord om iedereen te danken die mij bezig gehouden heeft.
%   Hierbij dank ik mijn promotor, mijn begeleider en de voltallige jury.
%   Ook mijn familie heeft mij erg gesteund natuurlijk.
% \end{preface}

\chapter*{Voorwoord - Tim}
%TODO

\chapter*{Voorwoord - Sander}
%TODO
%prof duval
%gonzalo
%jan en jannik
%ouders
%maaike
%gunter d
%tim

\tableofcontents*

\begin{abstract}
Ontwikkelaars van mobiele applicaties worden geconfronteerd met een variëteit aan mobiele besturingssystemen die op smartphones en tablets aanwezig zijn.
Dit komt doordat een applicatie wordt geprogrammeerd in een programmeertaal die specifiek is aan het besturingssysteem.
%TODO Gonzalo zin
De ontwikkeling van mobiele webapplicatie, gebruikmakend van HTML5, is een mogelijke oplossing.
Om het ontwikkelingsproces van deze mobiele HTML5-applicaties te versnellen worden raamwerken aangeboden.
Deze helpen zowel bij het toevoegen van functionaliteit als de elementen voor de gebruikersinterface. 

Door de variëteit aan mobiele HTML5-raamwerken, dringt een grondige vergelijking zich op om te weten welk raamwerk nu het beste is.
Dit werk vergelijkt \st{}, \kendo{}, \jqm{} en \lungo{} op basis van vijf vergelijkingscriteria:  populariteit,  productiviteit,  gebruik,  ondersteuning en performantie.
Populariteit kijkt naar de activiteit van de raamwerken op sociale netwerken.
Productiviteit wordt opgemeten om aan te tonen hoe lang het duurt om met een raamwerk vertrouwd te raken en er daadwerkelijk iets mee te maken.
Het gebruik van de raamwerken bekijkt de elementen die het raamwerk aanbiedt.
Ondersteuning test de elementen van het raamwerk op verschillende apparaten.
Performantie meet enerzijds de downloadtijd en anderzijds de gebruikerservaring.
De laatstgenoemde meet hoe vlot het gaat om door een lange lijst te scrollen.

De vergelijking toont aan dat \jqm{} het beste raamwerk is.
Daarna volgen \kendo{}, \lungo{} en \st{} op een respectievelijke tweede, derde en vierde plaats.
\jqm{} heeft als belangrijkste troef de hoge productiviteit doordat het enerzijds zeer goed gedocumenteerd is en anderzijds geen ontwerppatroon afdwingt.
Dit laatste is echter ook een nadeel omdat het hierdoor minder scoort op gebruik.
\kendo{} heeft als belangrijkste troef het gebruik doordat het een ontwerppatroon afdwingt.
Het scoort echter ondermaats op performantie.
\lungo{} behaalde enkel de maximumscore voor performantie doordat het raamwerk geoptimaliseerd is voor mobiel gebruik.
\st{} is het minst productief en performant in vergelijking met de andere raamwerken.
Daarentegen scoort \st{} het best op het vlak van gebruikerservaring.
Door het afdwingen van een ontwerppatroon scoort het quasi evengoed als \kendo{} op vlak van gebruik.
Alle onderzochte raamwerken scoren zeer goed op ondersteuning op de onderzochte mobiele apparaten.
\end{abstract}


\selectlanguage{english}
\begin{abstract}
Developers of mobile applications are confronted with a variety of mobile operating systems that are present on smartphones and tablets.
This is because an application has to be programmed in a specific language for the operating system.
%TODO gonzalo, zin bekijken
Developing mobile web applications using HTML5 can be a solution for this problem.
To speed up development,  frameworks are presented.
These help to add functionality and elements for the user interface.

Because of the variety of mobile HTML5 frameworks,  a thorough comparative study is necessary.
This thesis compares \st{}, \kendo{}, \jqm{} and \lungo{} based on five comparison criteria:  popularity,  productivity,  usage,  support and performance.
Popularity looks at the activity of frameworks on social networks.
Productivity is determinded by measuring how long it takes to get acquainted with the framework.
Next,  the usage of the framework is studied by looking which elements are offered by the framework.
The support tests the elements of the framework on different devices.
Performance measures the download time and user experience.
The latter measures the smoothness of scrolling through a long list.

The comparison shows that \jqm{} is the best framework.
Followed by, \kendo{},  \lungo{} and \st{} respectively.
\jqm{} is highly productive,  a major advantage because on the one hand it is well-documented and on the other hand it does not impose an architecture.
The latter,  however,  becomes a disadvantage when looking at the usage.
\kendo{} has an architecture as most important advantage.
However,  it scores below average on performance.   
\lungo{} only achieved a maximum score for performance because the framework is optimised for mobile usage.

\st{} is the least productive and performant in comparison with the other frameworks.
By contrast,  \st{} scores the best in user experience.
By enforcing an architecture,  it scores almost as good as \kendo{} regarding usage.
All frameworks supported the evaluated mobile devices.

\end{abstract}
\selectlanguage{dutch}

%TODO frameworks in captions niet herhalen + minder is beter(TIM)
% Een lijst van figuren en tabellen is optioneel
\listoffigures
\listoftables
% Bij een beperkt aantal figuren en tabellen gebruik je liever het volgende:
%\listoffiguresandtables

%TODO sorteren op afkorting
\chapter{Lijst van afkortingen}
\begin{flushleft}
  \renewcommand{\arraystretch}{1.1}
  \begin{longtable}{p{2cm} l}
     AHP & Analytic Hierarchy Process \\
     AJAX & Asynchronous JavaScript And XML \\
     API & Application Programming Interface \\
     CORS & Cross-origin resource sharing \\
     CRUD & Create Read Update Destroy \\
     CSS(3) & Cascading Style Sheets (3) \\
     DOM & Document Object Model \\
     EDGE & Enhanced Data Rates for GSM Evolution \\
     (G)GI & (Grafische) Gebruikersinterface \\
     GNU GPL & GNU's Not Unix General Public License \\
     GPRS & General Packet Radio Service \\
     GPS & Global Positioning System \\
     GPU & Graphics Processing Unit \\
     GUI & Graphical User Interface \\
     GWT & Google Web Toolkit \\
     HSDPA & High-Speed Downlink Packet Access \\
     HTML(5) & HyperText Markup Language (5) \\
     IDE & Integrated Development Environment \\
     \jqma{} & \jqm{} \\
     JSON(P)  & JavaScript Object Notation (with Padding) \\
     \kendoa{} & \kendo{} \\
     LTE & Long Term Evolution \\
     \lungoa{} & \lungo{} \\
     MIT & Massachusetts Institute of Technology \\
     MVC & Model-View-Controller \\
     MVVM & Model-View-View Model \\
     OEM & Original Equipment Manufacturer \\
     PCAP & Packet CAPture \\
     PDA & Personal Digital Assistant \\
     PDF & Portable Document Format \\
     POC & Proof Of Concept \\
     PPI & Pixels Per Inch \\
     QR & Quick Response \\
     RAM & Random Access Memory \\
     RIA & Rich Internet Application \\
     RWD & Responsive Web Design \\
     SASS & Syntactically Awesome Stylesheets \\
     SDK & Software Development Kit \\
     SEO & Search Engine Optimization \\
     SMS & Short Message Service \\
     \sta{} & \st{} \\
     USB & Universal Serial Bus \\
     UTMS & Universal Mobile Telecommunications System \\
     W3C & World Web Consortium \\
     WHATWG & Web Hypertext Application Technology Working Group \\
     WYSIWYG & What You See Is What You Get \\
     WORA & Write Once, Run Anywhere \\
     X(HT)ML & eXtensible (HyperText) Markup Language \\ 
  \end{longtable}
\end{flushleft}

% Nu begint de eigenlijke tekst
\mainmatter

\chapter{Inleiding} 
\label{inleiding}
%In dit hoofdstuk wordt het werk ingeleid. Het doel wordt gedefinieerd en er wordt uitgelegd wat de te volgen weg is (beter bekend als de rode draad).

\section{Achtergrondinformatie}
% - web apps (cross platform)
% - HTML5/js/css3
% - wat doet framework?
% - om de ontwikkeling vergemakkelijk worden frameworks aangeboden... 

Het gebruik van smartphones en tablets stijgt ontzettend snel in onze samenleving.
Voorheen was er de \term{feature phone} waarop enkel de voorgeïnstalleerde applicaties kon worden gebruikt.
Nu kunnen smartphones en tablets ook extra applicaties vanuit een winkel downloaden en installeren.
Ontwikkelaars van deze mobiele applicaties worden geconfronteerd met de variëteit aan mobiele besturingssystemen die op deze apparaten aanwezig zijn.
Dit komt doordat een applicatie dient te worden geprogrammeerd aan de hand van een SDK (Software Development Kit) die specifiek is voor het besturingssysteem.
Ontwikkelaars zullen dus eenzelfde applicatie in verschillende programmeertalen dienen te programmeren om een zo groot mogelijk publiek te bereiken.
Niet enkel het programmeren, maar ook het onderhoud van de applicaties in verschillende programmeertalen brengt een grote kost met zich mee.

Een oplossing hiervoor is het maken van een mobiele webapplicatie, gebruikmakend van HTML5.
Ten eerste wordt deze rechtstreeks in een webbrowser geopend en dus niet langer vanuit een winkel geïnstalleerd.
Dit betekent dus dat ieder mobiel apparaat dat een webbrowser heeft, de webapplicatie kan openen ongeacht zijn mobiel besturingssysteem.
Ten tweede wordt de applicatie slechts in één programmeertaal geschreven, wat de kost verlaagd.
Om het ontwikkelingsproces van deze mobiele HTML5-applicaties te versnellen worden raamwerken aangeboden die helpen bij de functionaliteit van de applicatie en de elementen voor de gebruikersinterface. 

\section{Probleembeschrijving}
% - heel veel frameworks, nog geen literatuur die vergelijkt (alleen blogs e.d.)
% - eerder voorstelling ipv objectieve verglijking.
% - beste framework?
% - hoe vergelijken?

Mobiele HTML5-raamwerken zijn er in overvloed en ook de verschillende versies van eenzelfde raamwerk volgen elkaar in snel tempo op.
In de huidige literatuur worden er vaak raamwerken aangehaald en besproken, maar niet vergeleken.
Indien deze toch worden vergeleken, gebeurt dit vaak subjectief of worden punten gegeven zonder een gestaafde methode te gebruiken.
Ook bestaat er geen literatuur die vergelijkingen van mobiele HTML5-raamwerken aggregeert.

\section{Doelstellingen}
% Is er een beste framework?
% contribuite:  methodologie uitwerken om OBJECTIEF en VISUEEL raamwerken te vergelijken

Deze thesistekst bestaat uit twee doelstellingen.
Een eerste doel is het definiëren van een methodologie om HTML5-raamwerken met elkaar te vergelijken.
Deze methodologie moet alle belangrijke aspecten van de raamwerken tegen het licht houden.
Ook moet er geprobeerd worden de werkwijze zo objectief mogelijk te laten verlopen en het resultaat van de studie op een eenvoudige,  visuele manier aan de lezer te presenteren.
Het tweede doel omvat de effectieve vergelijking van de raamwerken zelf.
Door de grote verscheidenheid van HTML5-raamwerken moeten de bestudeerde raamwerken zo worden gekozen dat ze zoveel mogelijk aspecten bevatten.
Hier komt ook de afweging tussen het aantal bestudeerde raamwerken en de diepte van de vergelijkende studie de kop op steken.
De raamwerken die worden gekozen moeten vervolgens worden vergeleken met de vooropgestelde methodologie.
Het resultaat moet alle positieve en negatieve aspecten van de raamwerken bevatten.
Vervolgens moet er gekeken worden of er één raamwerk het beste is of er verschillende raamwerken in verschillende situaties als beste kunnen worden bestempeld.

\section{Toepassingsgebied}
% mobiele wereld (mobile = booming)
% web (web = booming)
% kruising tussen web en mobile = super booming!
% bedrijfswereld (capgemini) HTML5 iets nieuws,  bedrijven kunnen nu een met een gerust hart een goede keuze maken (mss beter doelstellingen)

Mobiele HTML5-raamwerken vergemakkelijken de ontwikkeling van mobiele HTML5-applicaties.
Deze applicaties zijn toegankelijk via het web en geoptimaliseerd om op mobiele apparaten te kunnen werken.
Het aanspreken van mobiele applicaties via het web heeft zowel voor- als nadelen.
Zeker wanneer er wordt vergeleken met \term{native} of hybride applicaties.
De focus van deze studie ligt echter niet op het onderzoeken van deze voor- of nadelen.
Wel zullen de verschillende technologieën besproken worden om mobiele applicaties te maken.

Omdat Capgemini deze thesis mee ondersteunt zullen de applicaties gemaakt met HTML5-raamwerken vanuit een bedrijfscontext worden benaderd.
Dit zal vooral naar boven komen in de keuze van vergelijkingscriteria en de methode om deze criteria te testen.

\section{Overzicht}
Eerst wordt in hoofdstuk~\ref{chap:literatuurstudie} de basis van dit werk uitgelegd.
Vervolgens worden in hoofdstuk~\ref{chap:raamwerken} de vier gekozen raamwerken uitvoerig besproken.
Daarna zullen in hoofdstuk~\ref{chap:vergelijkingscriteria} de gekozen vergelijkingscriteria aan bod komen en verantwoord worden.
Hieropvolgend wordt in hoofdstuk~\ref{chap:evaluatie} deze vergelijking uitgevoerd op de gekozen raamwerken aan de hand van de gekozen criteria.
Als laatste wordt in hoofdstuk~\ref{chap:besluit} het besluit geformuleerd.

%%% Local Variables: 
%%% mode: latex
%%% TeX-master: "masterproef"
%%% End: 

%Gebruik van stylesheets term in het nederlands: http://www.bol.com/nl/p/websites-opmaken-met-css/1001004010718921/
	%T: check

%OPMERKING:  het gebruik van 'CSS-stylesheets' slaat op niets (kijk naar de betekenis van CSS :-))
	%T: idd, maar we gebruiken toch nergens te term CSS-stylesheets?
	
%OPMERKING: is het UI-elementen of UI elementen idem voor HTML5-code of HTML5 code?
	%T: het is met een streepje

\chapter{Literatuurstudie}
\label{chap:literatuurstudie}
In sectie \ref{sec:mobiele-apparaten} wordt bekeken welke mobiele apparaten er allemaal bestaan. 
Vervolgens wordt er gekeken wat er onder de motorkap van deze apparaten zit, namelijk welke mobiele besturingssystemen (\ref{sec:mobiele-besturingssystemen}), welke mobiele applicaties (\ref{sec:mobiele-applicaties}) en welke mobiele webbrowsers (\ref{sec:mobiele-webbrowsers}) er bestaan. 
Daarna komen de drie bouwblokken van het web aan bod (\ref{sec:html5-css3-js}), namelijk HTML, CSS en JavaScript.
Hierna wordt ingegaan op vele mobiele HTML5 raamwerken (\ref{sec:mobiele-html5-raamwerken}).  
Ten slotte worden verschillende, reeds bestaande manieren om raamwerken te vergelijken, bekeken (\ref{sec:vergelijken-raamwerken}).

%%%%%%%%%%%%%%%%%%%%%%%%%%%%%%%%%%%%%%%%%%%%%%%%%%%%%%%%%%%%%%%%%%
%%%%%%%%%%%%%%%%%%%%%%%%%%%%%%%%%%%%%%%%%%%%%%%%%%%%%%%%%%%%%%%%%%

% TODO Tim: verwerken pie charts

\section{Mobiele apparaten}
\label{sec:mobiele-apparaten}
Mobiele apparaten vind je in alle soorten en maten, met weinig of veel opties, voor weinig of veel geld. 
Het verdient daarom de aandacht om deze diversiteit onder de loep te nemen. 
Eerst zullen we de soorten mobiele apparaten bekijken volgens \cite{GCF2013} en daarna zullen we ingaan op de kenmerken volgens \cite{PhilDutson2012}.

\subsection{Soorten}
Sinds de voorstelling van de Apple iPhone in 2007~\cite{David2011}, stijgt het gebruik van de \term{smartphone} ontzettend snel in onze samenleving.  
Momenteel zijn er al meer dan 1 miljard \term{smartphones} in gebruik~\cite{Yang2012}. 
Dit zal tegen 2015 verdubbeld zijn~\cite{Gillett2012}.
Foto's of video's nemen, navigeren naar het dichtstbijzijnde restaurant of nog snel het weer voor de komende dagen opzoeken, het is allemaal mogelijk. 
Hoewel Apple de lat hoog heeft gelegd met het uitbrengen van de iPhone, zijn er ook nog andere spelers op de markt. 
Zo hebben we bijvoorbeeld ook de op Google's Android gebaseerde \term{smartphones} zoals de Nexus 4 en de op Windows Phone gebaseerde \term{smartphones} zoals de Nokia Lumia 800.

Niet enkel de \term{smartphone} behoort tot de categorie van mobiele apparaten, maar ook de \term{tablet}. 
Tegen 2016 zulen er 760 miljoen \term{tablets} in gebruik zijn~\cite{Gillett2012}.
Ook hier kan terug gedacht worden aan één van Apple's succesvolle producten, namelijk de in 2010 uitgebrachte iPad~\cite{Apple2010}. 
Er dient echter wel opgemerkt te worden dat tien jaar voordien, Microsoft al eerder een \term{tablet} uitbracht met veel minder succes~\cite{Microsoft2000}.

De \term{e-reader} behoort tot de laatste categorie van mobiele apparaten. 
Deze wordt hoofdzakelijk gebruikt om digitale boeken te lezen, maar betere modellen laten bijvoorbeeld ook toe om te surfen op het Internet. 
Ook hier bestaat er een variëteit aan modellen zoals de Kindle van Amazon en de Reader van Sony.

\subsection{Kenmerken}
Door de vele verschillende soorten en modellen aan mobiele apparaten, is het nodig om op een hoog niveau te bekijken over welke kenmerken deze allemaal (kunnen) beschikken. 
Bij deze bespreking zullen we ingaan op de voornaamste kenmerken van \term{smartphones} en \term{tablets}. De kenmerken en tekst zijn gebaseerd op~\cite{PhilDutson2012}.

\subsubsection{Resolutie en PPI}
Een eerste kenmerk, waar vooral Apple met haar Retina graag mee uitpakt, is de resolutie. 
Dit is het aantal pixels getoond op het beeldscherm en wordt uitgedrukt in breedte $\times$ hoogte. 
Hoe kleiner, hoe minder er op het scherm kan worden getoond. 
Dit is vooral belangrijk wanneer veel informatie op het scherm wordt getoond. 
Indien men maar over een kleine resolutie beschikt, zal men moet scrollen om te rest van de informatie te kunnen zien.
Een overzicht van resoluties van bekende mobiele apparaten wordt getoond op de figuur \ref{fig:resoluties}.

Als men naast de resolutie ook nog eens gaat rekening houden met de fysieke grootte van het scherm, dan kunnen we spreken van over pixels per inch~(PPI). 
De eerste iPhone had een resolutie van 320$\times$480 en een 3,5” scherm, wat neerkomt op 163 PPI. 
De iPhone4 (Retina) daarentegen heeft een resolutie van 640$\times$960 en een 3,5” scherm, wat neerkomt op 326 PPI. 
Met andere woorden zijn er meer pixels op dezelfde fysieke grootte geplaatst, wat een scherper beeld tot resultaat heeft. 

% TODO afbeelding misschien vectorieel maken:  
\begin{figure}
  \centering
  \includegraphics[height=0.8\textwidth]{figuren/mobile-devices-resolutions.png}
  \caption{Resoluties van bekende mobiele apparaten~\cite{Wolfermann2012}.}
  \label{fig:resoluties}
\end{figure}

\subsubsection{Aanraakscherm}
De populaire soorten schermen zijn resistieve en capacitieve aanraakschermen. De eerstgenoemde soort maakt gebruik van twee lagen die gescheiden worden door een tussenruimte. Door druk ontstaat er contact tussen de twee lagen. Meestal wordt bij deze soort schermen een stylus meegeleverd. 

De laatstgenoemde soort maakt gebruik van veranderingen in frequentie. Door het scherm aan te raken met je vinger, dat een geleider is, ontstaat er een kleine verandering in frequentie die gedetecteerd wordt. Niet-geleidende materialen zullen geen frequentieverandering veroorzaken, wat verklaart dat zo'n scherm niet reageert als het wordt aangeraakt met een handschoen.

\subsubsection{GPS}
Met het \term{global positioning system} (GPS) kan de gebruiker zijn locatie opvragen en doorgeven aan een applicatie om zo bijvoorbeeld het dichtstbijzijnde restaurant te vinden. 
Doordat het wat kan duren vooraleer de locatie is vastgesteld via GPS, kan het mobiel apparaat ook gebruik maken van mobiele masten of het Internet om zo, hetzij minder nauwkeurig, sneller de locatie te bepalen.

\subsubsection{Camera}
Praktisch ieder recent mobiele apparaat is uitgerust met een camera. 
Sommige bevatten zelfs twee camera's. 
De camera vooraan is veelal van mindere kwaliteit en wordt gebruikt om videogesprekken te voeren. 
Achteraan het apparaat zit dan een camera met hogere resolutie om mooie foto's te kunnen maken.

Twee andere toepassingen van de camera zijn toegevoegde realiteit en het inscannen van barcodes.
Bij het eerstgenoemde wordt informatie toegevoegd aan het beeld dat door de camera wordt geregistreerd.
Het laatstgenoemde wordt gebruikt om de populaire QR-code in te scannen en te zien wat ze betekent.
Zo'n code kan tekst bevatten, een link naar een website, een telefoonnummer, enzovoort. 

\subsubsection{Verbinding}
%TODO: Wifi, 3G, EDGE

In deze periode wil iedereen met elkaar verbonden zijn, dus ook op mobiele apparaten.
We bespreken kort Wifi, 3G, Bluetooth en infrarood.
Het mobiel apparaat kan meerdere mogelijkheden voorzien om verbinding te maken. 
Enerzijds kan men verbinden via Wi-Fi. Daarnaast zijn er ook nog andere technologieën zoals 3G mogelijk.

% \subsubsection{Oriëntatie}
% Een handig kenmerk is dat vele mobiele apparaten kunnen detecteren hoe ze gehouden worden door de gebruiker. Dit maakt het mogelijk om de informatie zo optimaal mogelijk op het scherm te tonen. Men kan bijvoorbeeld een verschillende lay-out voorzien voor een staand en liggend scherm.
% 
% \subsubsection{Versnellingsmeter}
% Als het mobiele apparaat een versnellingsmeter bevat, is het mogelijk om hiervan in spelletjes e.d. gebruik van te maken.
% 
% \subsubsection{Afstandssensor}
% Niet veel mobiele apparaten beschikken over dit kenmerk, maar dit kan bijvoorbeeld gebruikt worden wanneer een gebruiker met zijn smartphone aan het bellen is, waarbij het scherm zichzelf automatisch uitschakelt als het tegen de wang wordt gehouden.

% \subsubsection{Fysiek toetsenbord}
% Sommige apparaten beschikken ook nog over een fysiek toetsenbord, soms ook in combinatie met een aanraakscherm. 

% \subsubsection{Barometer}
% Sommige mobiele apparaten zijn uitgerust met een barometer. Naast het meten van de druk die kan helpen bij het bepalen van het weer, helpt deze de GPS bij het bepalen van de locatie.

%TODO Sander:  Als er in de tekst naar kenmerken van een device wordt verwezen krijg je altijd het voorbeeld GPS of Camera.  Dat lijken mij ook de twee belangrijkste voor POC.  Deze twee zijn mss voldoende om dan te bespreken?

%TODO Tim: Inderdaad, of we kunnen deze kleine paragrafen allemaal in 1 grote steken.

%%%%%%%%%%%%%%%%%%%%%%%%%%%%%%%%%%%%%%%%%%%%%%%%%%%%%%%%%%%%%%%%%%
%%%%%%%%%%%%%%%%%%%%%%%%%%%%%%%%%%%%%%%%%%%%%%%%%%%%%%%%%%%%%%%%%%

\section{Mobiele besturingssystemen}
\label{sec:mobiele-besturingssystemen}
Net zoals er brede waaier bestaat aan besturingssystemen voor computers, geldt dit ook zo voor mobiele apparaten. We geven hier een overzicht van mobiele besturingssystemen met een significant marktaandeel~\cite{David2011, Hales2012} zoals iOS en Android, maar ook een nieuwkomer op de markt, namelijk Windows Phone.

\subsection{iOS}
Het iPhone besturingssysteem is voor het eerst uitgekomen in juni 2007 tezamen met de iPhone. Later werd het hernoemd naar iPhone OS en uiteindelijk werd het iOS. Het is duidelijk dat iOS gebonden is aan de hardware van Apple. Verschillende versies volgden elkaar op: iOS 2 (juli 2008), iOS 3 (juni 2009), iOS 4 (juni 2010) en iOS 5 (oktober 2011)~\cite{Deitel2012, PhilDutson2012}. 

De nieuwste versie, iOS 6, werd uitgegeven in september 2012. Nieuwigheden zijn onder andere hun eigen Maps (in plaats van Google Maps) en een Pass Kit (de vervanging van het traditionele trein-, cinematicket, enz.). Daarnaast zijn er ook ander andere verbeteringen uitgevoerd met betrekking tot sociale media en spraakcommando's~\cite{Deitel2012}.

Op figuur \ref{fig:marketshare-ios} is te zien dat bijna twee derde van de iOS-gebruikers al iOS 6 gebruikt.

\begin{figure}
  \centering
  \includegraphics[width=0.5\textwidth]{figuren/marketshare-ios-2012-11-14.png}
  \caption{Marktaandeel iOS-besturingssystemen op 14 november 2012~\cite{Sylvain2012}.}
  \label{fig:marketshare-ios}
\end{figure}

Browsen op het web gebeurt met de geïnstalleerde Mobile Safari webbrowser (zie \ref{sec:mobile-safari}). Applicaties kunnen gedownload worden in de App Store, die sinds iOS 2 aanwezig is~\cite{Deitel2012}. 

\subsection{Android}
Android Inc. werd opgericht in 2003 en werd in 2005 overgekocht door Google Inc~\cite{Satyesh2012}. Het is net zoals iOS een mobiel besturingssysteem, maar in tegenstelling tot iOS is het open~\cite{David2011}. De eerste stabiele versie, Android 1.0, kwam uit in september 2008. Ook hier volgden verschillende versies elkaar op: Android 2.0 (oktober 2009), Android 3.0 (februari 2011) en Android 4.0 (oktober 2011)~\cite{Satyesh2012}. Hun nieuwste versie, Android 4.2, werd aangekondigd in oktober 2012~\cite{Sawers2012}. 

Op figuur \ref{fig:marketshare-android} is het marktaandeel te zien van de verschillende Android besturingssystemen, waargenomen over een periode van 14 dagen. Het is duidelijk dat Gingerbread (Android 2.3) meer dan de helft van het marktaandeel inneemt.
Applicaties worden gedownload in Google Play. Android bevat ook een standaard browser (zie \ref{sec:android-browser}).

\begin{figure}
  \centering
  \includegraphics[width=0.7\textwidth]{figuren/marketshare-android-2012-11-01.png}
  \caption{Marktaandeel Android besturingssystemen op 1 november 2012~\cite{Android2012}.}
  \label{fig:marketshare-android}
\end{figure}

\subsection{Windows Phone}
Windows Phone van Microsoft werd aangekondigd in oktober 2010 als vervanging voor Windows Mobile~\cite{Seitz2010,Lieberman2010}. Dit is duidelijk te zien als we kijken naar de versies: de laatste versie was Windows Mobile 6.5.3 en de eerste versie is Windows Phone 7. In 2011 ging Microsoft een partnerovereenkomst aan met Nokia om zo snel de markt te kunnen overwinnen~\cite{Microsoft2011}. De nieuwste versie, Windows Phone 8, werd aangekondigd in oktober 2012~\cite{Reed2012}. 

%%%%%%%%%%%%%%%%%%%%%%%%%%%%%%%%%%%%%%%%%%%%%%%%%%%%%%%%%%%%%%%%%%
%%%%%%%%%%%%%%%%%%%%%%%%%%%%%%%%%%%%%%%%%%%%%%%%%%%%%%%%%%%%%%%%%%

\section{Mobiele applicaties}
\label{sec:mobiele-applicaties}
%TODO Sander: eventueel eerst voorstellen,  dan een paragraafje minivergelijking..
Er zijn drie mogelijkheden om mobiele applicaties te maken~\cite{Accenture2012,Hales2012}. Eén aanpak is het maken van een webapplicatie.
Zo'n applicatie wordt geopend vanuit de webbrowser. Een andere aanpak is een \term{native} applicatie. Hierbij zal de gebruiker de applicatie installeren op zijn apparaat. Als laatste kan een mix van de vorige gemaakt worden en dat wordt een hybride applicatie genoemd.

\subsection{Webapplicaties}
In het rapport 'The (Not So) Future Web'~\cite{Phifer2011} uit juni 2011 wordt gesteld dat tegen 2015 60\% van alle mobiele bedrijfsapplicaties en 40\% van alle mobiele consumentenapplicaties, webapplicaties zullen zijn. Er zijn namelijk veel voordelen~\cite{Accenture2012} verbonden aan webapplicaties.

Ten eerste heeft iedereen die een webbrowser heeft op zijn mobiel apparaat, toegang tot de applicatie.  Dit voordeel gaat niet op voor een native applicatie dat enkel voor een specifiek platform is geschreven. 

Ten tweede, aansluitend bij het bovenstaande voordeel, moet de code slechts eenmaal worden geschreven. Een vaak voorkomende term die dit samenvat is WORA: \term{write once, run anywhere}~\cite{Hales2012}. Dit is in tegenstelling tot een native applicatie die specifiek geschreven is voor bijvoorbeeld iOS, Android en Windows Phone. Daar dient de code driemaal te worden geschreven \'en te worden onderhouden.

Ten derde moeten webapplicaties niet worden geverifieerd vooraleer ze worden uitgebracht. Dit is wel zo bij native applicaties. Hierdoor kan in een webapplicatie een belangrijke update snel doorgevoerd worden, terwijl de native applicatie nogmaals het verificatieproces moet doorlopen.

\subsection{Native applicaties}
Een andere mogelijk is om een native applicatie te schrijven. Voordelen~\cite{Accenture2012} hier zijn onder meer de snelheidswinst doordat de applicatie rechtstreeks met het besturingssysteem kan werken. Aansluitend bij het vorige kan ook worden geargumenteerd dat het over het algemeen een native applicatie gemakkelijker de kenmerken van het mobiel apparaat, zoals de camera of GPS, aan kan spreken. Ten derde blijft beveiliging nog altijd een knelpunt bij webapplicaties. Een native applicatie heeft hier minder problemen. Als laatste kan opgemerkt worden dat het gebruik van een winkel (\term{store}) voor het aanbieden van een applicatie als voordeel kan gezien worden, afgezien van het verificatieproces. De applicatiewinkel zorgt namelijk voor reclame en correcte uitbetaling bij gebruik van de applicatie.

\subsection{Hybride applicaties}
Er bestaat een mix tussen de twee voorgaande soorten van mobiele applicaties, namelijk een hybride applicatie~\cite{Accenture2012}. Hierbij wordt de webapplicatie verpakt in een native applicatie. Hierdoor kan men specifieke kenmerken van het mobiel apparaat benaderen die men vanuit een pure webapplicatie niet kon benaderen.

%%%%%%%%%%%%%%%%%%%%%%%%%%%%%%%%%%%%%%%%%%%%%%%%%%%%%%%%%%%%%%%%%%
%%%%%%%%%%%%%%%%%%%%%%%%%%%%%%%%%%%%%%%%%%%%%%%%%%%%%%%%%%%%%%%%%%

\section{Mobiele webbrowsers}
\label{sec:mobiele-webbrowsers}
Sinds 2008 spreken we van het mobiele web~\cite{Hales2012}. 
Vanuit mobiele webbrowsers op tablets en smartphones wordt het web meer en meer aangesproken. 
Deze mobiele webbrowsers vormen als het ware kleine besturingssystemen, waardoor de browser zelf een platform wordt~\cite{Hales2012}. 
Ze geven namelijk toegang tot allerlei kenmerken van het mobiele apparaat zoals camera en GPS. 
Denk maar aan het heel concreet voorbeeld van Google die het besturingssysteem Chrome OS maakte op basis van de Chrome webbrowser~\cite{Hales2012}.

Vanuit het standpunt om webapplicaties te maken, is het dan ook zeer belangrijk om deze evolutie op te volgen. 
Een webbrowser haalt namelijk webpagina's op die geschreven zijn in HTML en andere technologieën. 
Doordat deze technologieën evolueren (zie \ref{sec:html5-css3-js}), zullen de webbrowsers zelf ook (moeten) mee evolueren. 
Niet iedere browser zal dit op dezelfde manier doen, waardoor er verschillen zullen ontstaan waar men rekening mee zal moeten houden. 
Het is namelijk ongewenst dat een webapplicatie enkel op Mobile Safari werkt als men een zo breed mogelijk publiek wenst te bereiken. 

Hieronder bespreken we enkele mobiele webbrowsers. 
Eerst halen we de twee meest populaire browsers aan, namelijk Mobile Safari en de native Android browser~\cite{Hales2012}. 
Ze zijn beide op de WebKit browser \term{engine} gebaseerd~\cite{Oeflman2011}. 
Zo'n \term{engine} zorgt ervoor dat de code van de opgehaalde webpagina wordt omgezet naar de webpagina die de gebruiker te zien krijgt. 
We bekijken ook kort Internet Explorer Mobile en Opera Mobile. 
Het marktaandeel van de genoemde browsers kunt u zien in tabel \ref{tbl:marktaandeel-browsers}.
% TODO beter beschrijven browser engine

% TODO toevoegen data over IE mobile
\begin{table}
\begin{center}
\begin{tabular}{ll}
\hline
\textbf{Mobiele webbrowser} & \textbf{Marktaandeel} (\%) \\
\hline
\hline
Mobile Safari				& 61.50 \\
Android browser				& 26.09 \\
Opera Mini				& 7.02 	\\
Chrome					& 1.14 	\\
Opera Mobile				& 0.53 	\\
Internet Explorer Mobile & 		\\
Andere					& 		\\
\hline
\end{tabular}
\caption{Marktaandeel mobiele webbrowsers op november 2012~\cite{NetApplications2012}.}
\label{tbl:marktaandeel-browsers}
\end{center}
\end{table}

\subsection{Mobile Safari}
\label{sec:mobile-safari}
Deze webbrowser van Apple zit standaard bij iOS en kan ook enkel op dit besturingssysteem worden gebruikt. 
Apple heeft veel moeite gedaan om telkens de laatste nieuwe specificaties van HTML5 in zijn webbrowsers te implementeren~\cite{Hales2012}. Natuurlijk zal dit ook te maken hebben met het feit dat ze geen Flash meer ondersteunen op hun iPods, iPhones en iPads~\cite{Jobs2010}.

\subsection{Android browser}
\label{sec:android-browser}
Android biedt de native Android browser aan. 
Implementatie van de HTML5 specificaties hebben wat aangesleept, maar vanaf Android 4.0 gaat dit een stuk beter~\cite{Hales2012}. 
Daarnaast is het nu ook mogelijk om de Chrome webbrowser op mobiele apparaten te installeren.

\subsection{Internet Explorer Mobile}
Net zoals je bij Windows ook Internet Explorer krijgt, geldt dit ook voor hun mobiel  besturingssysteem. 
Bij de nieuwe Windows Phone 8 zal Internet Explorer Mobile 10 worden meegeleverd. Deze gebruikt dezelfde \term{engine} als Internet Explorer 10. \term{WebSockets}, \term{Web Workers}, \term{Application Cache} en \term{IndexedDB} worden hierin ondersteund~\cite{Hales2012}, meer daarover in~\ref{sec:html5-css3-js}.

\subsection{Opera Mobile/Mini}
Op het moment van schrijven is Opera Mobile 12.10 de beste mobiele HTML5 browser~\cite{Sights2012}. 
Opera heeft eigenlijk twee aparte browsers, namelijk Opera Mobile en Opera Mini. 
Bij deze laatste staat de browser \term{engine} op servers van Opera, waardoor het niet het mobiel apparaat is die de webpagina verwerkt. 
De server zal, na verwerking, deze webpagina op een gecomprimeerde manier doorsturen naar de browser op het apparaat~\cite{PhilDutson2012}.

%\subsection{Mobile Firefox / Fennec}
%Mobile Firefox 16 sleept op dit moment nog net een podiumplaats in de wacht en eindigt derde, voor Mobile Safari. Mozilla staat bekend voor zijn drijvende community. 

%%%%%%%%%%%%%%%%%%%%%%%%%%%%%%%%%%%%%%%%%%%%%%%%%%%%%%%%%%%%%%%%%%
%%%%%%%%%%%%%%%%%%%%%%%%%%%%%%%%%%%%%%%%%%%%%%%%%%%%%%%%%%%%%%%%%%

\section{HTML5, CSS3 en JavaScript}
\label{sec:html5-css3-js}
De drie bouwstenen voor webontwikkeling zijn HTML5, CSS3 en JavaScript. 
HTML5 is verantwoordelijk voor de inhoud, CSS3 voor de presentatie en JavaScript voor de functionaliteit~\cite{PhilDutson2012}. 
Hieronder zullen we dan ook deze bouwstenen toelichten.

\subsection{HTML5}
Zoals uitgelegd in \cite{MacDonald2011} stopte in 1998 het W3C (World Web Consortium) met het werken aan de HTML standaard en alle energie ging uit naar zijn opvolger: XHTML 1.0, een verbeterde HTML versie die XML-gedreven is. 
XHTML kwam in grote mate overeen met HTML, maar de syntax was veel strikter. 
In het begin kon het zijn naam waarmaken en webontwerpers helpen betere resultaten te boeken doordat ze slechte gewoontes moesten opgeven. 
Jammer genoeg bleven de beloofde voordelen uit. 
Wat veel erger was voor de nieuwe standaard, was dat geen enkele browser klaagde indien deze strikte syntax niet werd gevolgd.

In ~\cite{MacDonald2011} staat ook de reactie die hierop kwam van het W3C.  
Ze brachten een nieuwe versie uit, namelijk XHTML 2.
De manier waarop webpagina's werden geschreven veranderde doordat vele tags waren veranderd of verwijderd. 
Daarenboven sleepte deze nieuwe standaard maar aan en aan, wat ook niet in hun voordeel was. 

In plaats van te onderzoeken wat er mis was met HTML, wat XHTML probeerde te doen, werd in 2004 onderzocht wat er ontbrak. 
Opera Software, Mozilla Foundation en Apple vormden de WHATWG (Web Hypertext Application Technology Working Group). 
Ze wilden HTML niet vervangen, maar uitbreiden en die manier moest achterwaarts compatibel zijn. Na reflectie geloofde ook het W3C in deze aanpak, weliswaar op hun eigen manier.  
Zo werd HTML5 geboren, waarbij versie 5 refereert naar waar de vorige versie, HTML 4.01, gestopt was.

HTML5 is volgens ~\cite{MacDonald2011} nog altijd in ontwerp. 
Hierdoor kunnen nieuwe kenmerken op ieder momenten worden toegevoegd.  
Er is ook nog steeds onduidelijkheid waar HTML5 ons zal brengen.  
Het W3C focust op een unieke HTML5 standaard (verwacht rond 2014) terwijl WHATWG de nieuwe markup-taal ziet als levende taal waarbij voortdurend  nieuwe dingen kunnen worden toegevoegd. 
Een belangrijke opmerking hierbij is dat het laatste woord altijd bij de webbrowserfabrikant ligt, net zoals dat het geval was met de strikte syntax in XHTML. 
Als een kenmerk niet in de browser wordt ondersteund, heeft het ook geen kans op overleven.

\subsubsection{Drie basisprincipes}
Achter HTML5 zit een filosofie die in drie basisprincipes kan worden samengevat~\cite{MacDonald2011}.  
De eerste is achterwaartse comptabiliteit. De standaard mag geen veranderingen invoeren die oudere pagina's zou doen breken. 
Ten tweede moet de standaard geen nieuwe specificaties afdwingen die door de meerderheid op een andere manier worden gedaan. 
Als laatste moeten de specificaties ook een praktisch nut hebben. 
Dit betekent dat daar waar veel vraag naar is, ook het beste opweegt om in de specificaties op te nemen.

\subsubsection{Acht technologieklassen}
HTML5 kan ook bekeken worden als de volgende acht technologieklassen~\cite{W3C2012}. 
Iedere klasse wordt met enkele concrete voorbeelden aangehaald.

\begin{description}
\item [Multimedia] De nieuwe video- en audiotags maken het mogelijk om video- en geluidsfragmenten toe te voegen zonder gebruik te maken van plug-ins van derden zoals Adobe Flash en Microsoft Silverlight.

\item [Offline en opslag]  Mobiele apparaten zijn onstabiel in hun verbinding met het Internet. HTML5 voorziet het offline werken in de \term{cache}, lokale opslag (vroeger kon men enkel gebruik maken van de zogenaamde cookies) en een API om bestanden te manipuleren.

\item [Performantie en integratie]  \term{Web Workers} maken het mogelijk om langdurige JavaScript taken in de achtergrond uit te voeren zodat webapplicaties dynamisch en snel blijven.

\item [Semantiek]  Een hele hoop nieuwe tags zorgen voor meer semantiek binnen webpagina's. Waar voorheen de webpagina bestond uit en verzameling \code{<div>}-elementen, kan nu veel concreter worden aangegeven wat er precies binnen die tags staat. Dit kan voor \term{search engine optimization} (SEO) een grote impact hebben. Daarnaast biedt dit ook mogelijkheden voor \term{e-readers} die nu beter de pagina kunnen analyseren.

\item [CSS3]  Hand in hand met HTML5 gaat CCS3 (zie \ref{ref:css3}). Het laat toe om webpagina's op te maken afhankelijk van het formaat van het mobiele apparaat. Ook kunnen webpagina's met effecten worden uitgebreid. 

\item [3D, grafieken en effecten]  De nieuwe \code{<canvas>}-tag in samenwerking met enkele lijnen JavaScript zijn enorm krachtig om eenvoudig tekeningen en animaties zelf te programmeren.

\item [Verbinding]  \term{Events} aan server zijde kunnen data naar \term{WebSockets} pushen. Hierdoor moet de webpagina niet meer voortdurend de server raadplegen, wat veel efficiënter is.
%TODO:  WebSockets is een naam voor een specifieke technologie.  Web workers is geen eigennaam maar duidt een JavaScript script aan (defined by W3C)  Wat is \term en wat niet?

\item [Toegang tot het apparaat] Webapplicaties kunnen meer en meer kenmerken zoals camera en GPS aanspreken net zoals native applicaties dat kunnen. 
\end{description}

Er dient opgemerkt te worden dat aangehaalde klassen zoals CSS3 en geolocatie niet tot de specificaties van HTML5 behoren. 
Toch worden ze onder de koepel van HTML5 gezien~\cite{MacDonald2011}.

\subsubsection{Kenmerken detecteren en opvullen}
\paragraph{Kenmerken detecteren}
Door enerzijds de levendigheid van HTML5 en anderzijds het verdeelde landschap van browsers en besturingssystemen, worden niet alle kenmerken van HTML5 overal ondersteund. 
Een eerste mogelijkheid is om zelf op te zoeken welke kenmerken op welke apparaten werken. 
Dat kan je bijvoorbeeld controleren op \url{www.caniuse.com} en \url{www.mobilehtml5.org}~\cite{MacDonald2011}. 

%tool is volgens woordenlijst.org een aanvaarde nederlandse term!
Wat nog handiger is, is om op het apparaat zelf te detecteren of het gewenste kenmerk beschikbaar is. 
Een erg handige tool hiervoor is Modernizr~\cite{Modernizr2012}. 
Het toevoegen van dit JavaScript-bestand creëert een JavaScript-object dat voor elk kenmerk teruggeeft of het al dan niet in de gebruikte browser wordt ondersteund.

\paragraph{Kenmerken opvullen}
Wanneer eenmaal gedetecteerd is dat een kenmerk niet aanwezig is, zijn er twee mogelijkheden: ofwel terugvallen op een alternatief of simuleren van dat kenmerk. 
Een voorbeeld van dit eerste kan gebeuren bij het gebruiken van de \code{<video>}-tag. 
Indien dit niet wordt ondersteund, kan men terugvallen op de Adobe Flash plug-in. 
Voor het simuleren van een kenmerk maakt men gebruik van \term{polyfills}. 
Dit zijn alternatieven op basis van JavaScript waarbij de native functionaliteit die normaal moet aanwezig zijn, geëmuleerd wordt~\cite{MacDonald2011,Weyl2011}.

\subsubsection{HTML5e}
Een bedrijf wil enerzijds een stabiele webapplicatie en wil anderzijds ook van deze nieuwe kenmerken zoveel mogelijk gebruik gaan maken. 
De term HTML5e~\cite{Hales2012} omvat de vijf meest ondersteunde HTML5 kenmerken in browsers. 
Op figuur \ref{fig:html5e} vind je een tabel die voor mobiele webbrowsers van toepassing is.

\begin{figure}
  \centering
  \includegraphics[width=0.8\textwidth]{figuren/html5e}
  \caption{HTML5e mobiele ondersteuning~\cite{Hales2012}}
  \label{fig:html5e}
\end{figure}

\subsection{CSS3}
\label{ref:css3}
Hand in hand met HTML5 gaat CSS3, dat zorgt voor de presentatie. 
Het is namelijk het hart van webdesign. 
CSS3 heeft hetzelfde probleem zoals HTML5 als het aankomt op de ondersteuning bij browsers~\cite{MacDonald2011}. 
Ook hier is er dus een brede waaier aan kenmerken die nog niet overal worden ondersteund. 
Kenmerken die enkel in een bepaalde browser ondersteund worden, worden voorafgaan door een browserprefix (zoals \code{-webkit-} voor WebKit gebaseerde browsers en \code{-o-} voor Opera).

In deze sectie zullen we kort belangrijke eigenschappen bespreken zoals \term{media queries}, effecten en lettertypes aan de hand van~\cite{MacDonald2011}.

\subsubsection{Media queries}
Zoals al aangehaald, hebben we verschillende apparaten met verschillende schermen en resoluties. 
Een goeie webpagina bestaat erin deze elementen zo goed mogelijk te benutten. 
Dit kan vanaf nu door gebruik te maken van \term{media queries} in CSS3. 
De website kan zich hiermee aanpassen aan het apparaat waarop het wordt getoond. 
Dit wordt in het Engels omschreven als \term{responsive design}.
% TODO nederlands woord voor responsive design

Ook CSS3 volgt het principe van achterwaartse compatibiliteit. 
Browsers die deze \term{media queries} niet ondersteunen, zullen deze negeren en enkel de gewone lay-out toepassen ongeacht het toestel.

\subsubsection{Effecten}
Transparantie, afgeronde hoeken, schaduw en kleurenverloop zijn maar enkele van de nieuwe kenmerken in CSS3. 
Voorheen moest de webdesigner deze dingen vaak met afbeeldingen oplossen, maar nu kan dit allemaal gebeuren met CSS3. 
Daarnaast hebben we ook effecten als transformaties en transities. 
Zo is het mogelijk wanneer men over een afbeelding gaat, deze ingezoomd en geroteerd kan worden. 

Dit is zeer vooruitstrevend om wille van twee zaken. 
Enerzijds schrijf men dingen makkelijker in CSS dan met JavaScript-code. 
Anderzijds komt er ook meer en meer ondersteuning vanuit de hardware. 
Zo worden 3D transformaties in CSS3 versneld door de \term{graphics processing unit} (GPU)~\cite{Hales2012,Kool2012}.

\subsubsection{Lettertypes}
Een laatste kenmerk in CSS3 is de betere ondersteuning van lettertypes. 
Waar vroeger enkel gewerkt kon worden met veilige lettertypes voor het web, is het nu mogelijk om eigen lettertypes op te laden en te gebruiken op je website.

\subsection{JavaScript}
\label{ref:javascript}
JavaScript gaat terug tot in 1995, toen LiveScript~\cite{McFarland2011}. 
Het heeft een lange weg afgelegd tot nu en is niet altijd even ernstig genomen. 
Dit kwam omdat men niet inzag wat er allemaal mee kon worden gedaan. 

Op dit moment is het maar al te duidelijk waar JavaScript in uitblinkt: het aanpassen van het \term{document object model} of kortweg DOM~\cite{PhilDutson2012}. 
Dit is een API voor HTML-documenten~\cite{Hegaret2004}. 
Hierdoor kunnen dynamische interfaces gecreëerd worden, kan op gebeurtenissen - zoals ergens op klikken - onmiddellijk gereageerd worden en is de website dan ook meer bruikbaar geworden door deze directe feedback~\cite{McFarland2011}.

% TODO referentie verschillende manieren intrepeteren
Het schrijven van JavaScript is niet gemakkelijk om twee redenen~\cite{McFarland2011}. 
Ten eerste, vergelijkbaar met HTML5 en CSS3, kunnen browsers JavaScript op verschillende manieren interpreteren. 
Gelukkig is er de laatste tijd veel gestandaardiseerd, maar toch blijven er nog verschillen. %TODO Sander: is dit niet wat vaag?  `veel gestandardiseerd'.  Concreet maken met referentie?
De ontwikkelaar dient dus tijdens het programmeren met deze verschillen rekening te houden. 
Ten tweede vergt het schrijven van simpele, veel voorkomende taken soms veel code.

Een oplossing voor de bovenstaande pijnpunten is gebruik maken van een bibliotheek. 
Een voorbeeld hiervan is de populaire jQuery Core bibliotheek. 
Het is ook mogelijk om jQuery uit te breiden met verscheidene plug-ins om de functionaliteit nog te vergroten~\cite{McFarland2011}.

%%%%%%%%%%%%%%%%%%%%%%%%%%%%%%%%%%%%%%%%%%%%%%%%%%%%%%%%%%%%%%%%%%
%%%%%%%%%%%%%%%%%%%%%%%%%%%%%%%%%%%%%%%%%%%%%%%%%%%%%%%%%%%%%%%%%%

%TODO Sander:  deze sectie slechts alle frameworks aanhalen (ook diegene dat we niet gaan vergelijken).  In een ander hoofdstuk (analyse?) de gebruikte frameworks grondiger bestuderen en argumenteren waarom deze gekozen hebben.  Of waarom we de andere niet gekozen hebben.

\section{Mobiele HTML5 raamwerken}
\label{sec:mobiele-html5-raamwerken}

% paragraaf per framework
% - welk framework (markup / javascript)
% - korte geschiedenis en versie
% - bedrijf, licentie

\subsection{jQuery Mobile} % TIM

\subsection{Sencha Touch}% SANDER
Sencha Touch wordt ontwikkeld door Sencha,  een bedrijf dat in 2010 is ontstaan als een samensmelting van Ext JS,  jQuery Touch en Raphaël.
Ext JS kan beschouwd worden als de voorganger van Sencha Touch. 
Sencha Touch is net als Ext JS een JavaScript gedreven raamwerk met een MVC architectuur.
All functionaliteiten worden dus in JavaScript geschreven en MVC bepaalt de implementatie.
Het aanroepen van het raamwerk gebeurt door het invoeren van de Sencha Touch bibliotheek binnen \code{<script>}-elementen.
Sencha Touch is gratis binnen een commerciële context waarbij het bedrijf in kwestie de broncode niet deelt voor zijn gebruikers.  
Wanneer je dit wel wil doen bestaat er ook een gratis \term{open-source} versie van Sencha Touch.
Op het moment van schrijven is Sencha Touch aan versie 2.1.1~\cite{Inc.}. 

\subsection{Kendo UI} % SANDER
Kendo UI is een HTML5 raamwerk van de hand van Telerik.
Buiten Kendo UI is Telerik voornamelijk gericht op \term{tools} voor de ontwikkelaar.
Zo ontwikkelen ze DevTools dat een grafische gebruikersinterface aan .NET ontwikkelaar aanbiedt.
Ze voorzien ook Icenium voor de ontwikkeling van hybride applicaties en kan men zien als de tegenhanger van Kendo UI dat webapplicaties aanbiedt.
Kendo UI bestaat uit drie luiken:  Web, Mobile en DataViz.  
Het eerste is gericht op de ontwikkeling van desktop en mobiele applicaties,  het tweede voegt een \term{native look-and-feel} toe aan mobiele applicaties en het laatste zorgt voor data visualisatie met HTML5 en JavaScript technologie.
Kendo UI is een JavaScript gedreven raamwerk met een MVVM architectuur dat steunt op de jQuery bibliotheek.
Verder heeft de ontwikkelaar ook de mogelijkheid om eenvoudig de \term{backend} in integreren aan de klantzijde.
.NET,  PHP en JSP zij momenteel de ondersteunde \term{server side wrappers}.
Een licentie voor Kendo UI waarbij één van voornoemde \term{wrappers} mogelijk is kost $\$999$.
Zonder \term{backend} integratie betaal je $\$300$ minder.
Op het moment van schrijven is Kendo UI aan versie 2013 Q1~\cite{Telerik}. 


\subsection{Lungo} % TIM

\subsection{The M-Project} % SANDER
The M-Project is een JavaScript/HTML5 raamwerk dat bouwt op jQuery en jQuery Mobile.
Origineel werd het in 2012 ontwikkeld door M-Way Solutions maar nu behoort het tot Panacoda,  een duitse ontwikkelaar voor software \term{tools} en mobiele web applicaties.
Panacoda bezit ook Espresso,  een krachtige \term{tool} om applicaties te bouwen en ontwikkelen met The M-Project.
Het laat ook toe applicaties om te vormen tot \term{native} applicaties. 
The M-Project is \term{open-source},  vrijgegeven onder een MIT licentie en op GitHub te vinden.
Dit raamwerk is volledig JavaScript gedreven en steunt op de MVC architectuur.
Ook ondersteunt het HTML5 en CSS3 kenmerken zoals offline  beschikbaarheid en lokale opslag.
Op het moment van schrijven is The M-Project aan versie 1.4..  
Het is belangrijk op te merken dat in de zomer van 2013 versie 2.0 wordt verwacht.  
The M-Project zal van nul worden opgebouwd omdat enkel de code aanpassen niet meer voldoende bleek.  
De voornaamste werkpunten zijn performantie en platform-onafhankelijkheid~\cite{Panacoda,Laubach2013}.

\subsection{Moobile} % TIM

\subsection{DaVinci}% SANDER
DaVinci bestaat uit twee \term{tools}:  DaVinci Studio en DaVinci Animator.
De nadruk bij dit raamwerk ligt voornamelijk bij de generatie van code in een WYSIWYG omgeving.
De DaVinci Studio is een Eclipse plugin die HTML,  JavaScript en CSS code genereert.
De gebruiker kan UI elementen via \term{drag-and-drop} aan de applicatie toevoegen.
Het binden van data kan door op een visuele manier de mapping tussen UI en data weer te geven.
Het testen van de applicatie kan op een bijgeleverde emulator in een \term{N-screen} omgeving die de applicatie op verschillende layouts kan weergeven.
Het raamwerk gebruikt een open architectuur dat compatibel is met andere \term{open-source} raamwerken zoals jQuery, KnockOut of Backbone.
DaVinci Animator kan gebruikt worden om animaties op basis van HTML5 en CSS3 te maken in een grafische omgeving.
In SNU Research Park te Seoul worden deze \term{tools} ontwikkeld.
Op het moment van schrijven is DaVinci toe aan versie 2.0.  
Alle documentatie is momenteel nog niet vertaald van het Koreaans naar het Engels~\cite{Incross}.


\subsection{jQTouch}% TIM

%%%%%%%%%%%%%%%%%%%%%%%%%%%%%%%%%%%%%%%%%%%%%%%%%%%%%%%%%%%%%%%%%%
%%%%%%%%%%%%%%%%%%%%%%%%%%%%%%%%%%%%%%%%%%%%%%%%%%%%%%%%%%%%%%%%%%

% SANDER: bekijken aan de hand van paper
% hier alles dat we bekeken hebben, maar zonder onze eigen inbreng
\section{Vergelijken van raamwerken} 
\label{sec:vergelijken-raamwerken}
Om de verschillende mobiele HTML5 raamwerken te kunnen vergelijken hebben we een consistente manier nodig om dit te doen.  
Op deze manier worden alle raamwerken op dezelfde manier getest.

\subsection{ISO 25010}
HTML5 raamwerken zijn software en om software te vergelijken bestaat er de ISO 25010 standaard~\cite{Standard2010}.  
Hieronder vallen twee modellen:  de productkwaliteit en de kwaliteit van het product in gebruik.  
Beide modellen beschrijven de kwaliteit van software op basis van een aantal categorieën met specifieke kwaliteitseigenschappen. 
Het beoordelen van de categorieën kan gebeuren op basis van een checklist. 
 
\subsubsection{Productkwaliteit}
De acht karakteristieken die horen bij dit model zijn: functionele geschiktheid,  betrouwbaarheid,  performantie, efficiëntie, uitwisselbaarheid,  bruikbaarheid,  betrouwbaarheid, beveiligbaarheid,  onderhoudbaarheid en overdraagbaarheid.   
Vanzelfsprekend zijn niet alle categorieën even toepasbaar op HTML5 raamwerken.  
Beveiligbaarheid is bijvoorbeeld niet zo belangrijk bij mobiele HTML5 raamwerken.  
Performantie en overdraagbaarheid dan weer wel.

\subsubsection{Kwaliteit in gebruik}
De vijf karakteristieken voor dit model zijn: effectiviteit,  efficiëntie,  voldoening,  vrijheid van risico en context dekking. 
Elke karakteristiek kan toegewezen worden aan verschillende activiteiten van belanghebbenden. 
Weer zijn alle categorieën niet even toepasbaar.  
De risico die een mobiele webapplicatie meebrengt is niet van belang,  het moet vooral efficient zijn en voldoening scheppen.

De kwaliteit voor een systeem in gebruik wordt bepaald door de kwaliteit van de software,  de hardware en het besturingssysteem samen met de gebruikers, hun taken en de sociale omgeving.  
De belanghebbenden worden opgedeeld in primaire en secundaire gebruikers.  
De eerste zijn de personen die het systeem gebruiken. 
De laatste zijn diegene die zorgen voor het onderhoud.

\subsection{Bestaande use cases}
Op het web en in de literatuur kunnen we ook \term{use cases} terugvinden waar de proef op de som wordt genomen en twee of meer raamwerken met elkaar worden vergeleken.  
Deze werkwijze verschilt van \term{use case} tot \term{use case}

\subsubsection{Codefessions}
Op een blogpost van Codefessions wordt een vergelijking gemaakt tussen jQuery Mobile, Sencha Touch, jQTouch en Kendo UI~\cite{Sarrafi2012}.  
Als referentiesysteem gebruiken ze zeven criteria.  De eerste drie zijn de native \term{look-and-feel}, performantie en platform-onafhankelijke capaciteiten.  
Deze worden gequoteerd met een cijfer van 0 tot 5 waarbij 5 staat voor de maximale score. 
Kenmerken worden gequoteerd door de raamwerk met elkaar te vergelijken.  
Het raamwerk met de meeste kenmerken krijgt een 5, het tweede beste een 4, enzovoort.  
Op een analoge manier wordt code efficiëntie en gebruiksgemak gequoteerd.  
Het raamwerk dat de minste lijnen code vereist, krijgt de perfecte score. 
Hierbij moeten wel alle bestanden gerekend worden die het raamwerk nodig heeft om functioneel te zijn. 
Licenties krijgen een score van 0 tot 5 waarbij 0 betekent dat het niet beschikbaar is voor een individuele ontwikkelaar en 5 dat het raamwerk \term{open-source} en gratis te gebruiken is. 
Andere factoren zoals omkadering en uitbreidbaarheid worden niet in de vergelijkingstabel opgenomen omdat ze afhangen van de interesse van de gebruiker.  
Ze worden echter wel bekeken.

% \subsubsection{jQuery UI vs Kendo UI}
% Een andere, meer grondige vergelijking is te vinden op \url{www.jqueryuivskendoui.com}~\cite{Bristowe}.  Deze webpagina bestaat uit één grote tabel die meer specifieke kenmerken tussen beide raamwerkenen vergelijkt.  Er worden geen scores uitgedeeld. In de tabel vinden we onder andere een vergelijking van beschikbare thema's,  browser compatibiliteit,  form validatie,  ondersteuning van het product etc.
%TODO meer relevante use cases toevoegen

\subsection{Vergelijkingstabellen}
Naast ISO standaarden of al bestaande use cases, kunnen we ook tabellen raadplegen die zoveel mogelijk raamwerken en zoveel mogelijk kenmerken naast elkaar zetten.  
Op Wikipedia creëerde men bijvoorbeeld zo'n tabel voor JavaScript raamwerken~\cite{Wikipedia}.  

Specifiek voor mobiele HTML5 raamwerken bestaat er ook zo'n tabel,  te vinden op \url{www.markus-falk.com/mobile-frameworks-comparison-chart}~\cite{Falk2011}.  
We zien er een matrix met als rijen de verschillende raamwerken en in de kolommen de vergelijkingscriteria.  
Deze laatste worden opgedeeld in compatibiliteit met het besturingssysteem,  doel van de applicatie,  taal voor ontwikkeling,  hardware interactie,  UI,  licenties en andere.  
Deze laatste categorie bevat de criteria of er al-dan-niet een SDK beschikbaar is, encryptie ondersteund wordt en of advertenties worden ondersteund.  
Handig hierbij is dat de webpagina een stappenplan voorziet waarin je per categorie al je vereisten moet invullen.  
De resultaten zijn dan de raamwerken die compatibel zijn met je vereisten.

%%% Local Variables: 
%%% mode: latex
%%% TeX-master: "masterproef"
%%% End: 

\chapter{Mobiele HTML5 raamwerken}
\label{chap:raamwerken}

% TODO Tim: Waar zetten we WAAROM we deze raamwerken gekozen hebben en niet de andere?
% TODO Tim: user interface of gebruikersinterface?

In dit hoofdstuk wordt ingezoomd op de mobiele HTML5 raamwerken die dit werk vergelijkt, namelijk \jqm{}~(\ref{sec:raamwerk-jqm}), \st{}~(\ref{sec:raamwerk-st}), \kendo{}~(\ref{sec:raamwerk-kendo}) en \lungo{}~(\ref{sec:raamwerk-lungo}).
In de laatste sectie (\ref{sec:raamwerken-tabel}) wordt een tabel weergegeven, waarin deze gegevens worden vergeleken.

\section{\jqm}
\label{sec:raamwerk-jqm}
\jqm{} is een mobiel HTML5 \term{user interface} (UI) raamwerk dat werd aangekondigd in 2010~\cite{Resig2010}. 
In november 2011 werd versie~1.0 uitgebracht~\cite{Parker2011} en een jaar later werd in oktober versie~1.2 uitgebracht~\cite{Parker2012}. 
Op het moment van schrijven kwam versie~1.3 uit~\cite{Parker2013a}. 
Het raamwerk wordt beheerd door het jQuery Project dat onder andere jQuery Core beheert en waar \jqm{} afhankelijk van is~\cite{JQuery2012}. 
\jqm{} wordt door onder andere Adobe, BlackBerry en Mozilla gesponsord~\cite{JQuery2012a}.

\subsection{Omkadering}
\paragraph{Programmeertaal}
Om met \jqm{} aan de slag te kunnen, is niets meer nodig dan kennis over HTML, CSS en JavaScript. 
Alle UI-elementen worden geschreven in HTML en aangeduid met \code{data-}* attributen.

\paragraph{Tools}
Een standaard teksteditor voldoet om met \jqm{} aan de slag te kunnen. 
Natuurlijk kan het gemakkelijk zijn om van \term{integrated development environments}~(IDE's) zoals Aptana Studio~\cite{Aptana2012} of WebStorm~\cite{JetBrains2012} gebruik te maken, waardoor handige kenmerken zoals automatische code-aanvulling beschikbaar zijn.

Het is ook mogelijk om gebruik te maken van Codiqua om de UI-elementen op het scherm te slepen en neer te zetten. 
Codiqua zal automatisch op de achtergrond de HTML-code voorzien~\cite{Sperry2012}.

\paragraph{Documentatie}
Documentatie is te vinden op \url{www.jquerymobile.com/demos/1.2.0} voor versie~1.2. 
Hierop is een catalogus te vinden van alle mogelijke elementen waarover \jqm{} beschikt. 
Door de broncode van een voorbeeld te bekijken, kan worden gekeken welke code moet worden geschreven om tot dat resultaat te komen.

Naast de UI-elementen is er ook documentatie over de API. 
Deze gaat over initiële configuraties, \term{events} en methodes die kunnen worden gebruikt.

\paragraph{Marktadoptatie}
Op de website van \jqm{} wordt een reeks applicaties getoond die gemaakt zijn met hun raamwerk. 
Enkele voorbeelden zijn webapplicaties voor Ikea, Disney World, Stanford University en Moulin Rouge~\cite{JQuery2012a}. 

\paragraph{Licenties}
Vanaf september 2012 is het enkel nog mogelijk om \jqm{} onder de Massachusetts Institute of Technology (MIT) licentie te verkrijgen~\cite{Dmethvin2012}. 
Dit betekent dat de code wordt vrijgegeven als \term{open-source} en dat deze tegelijkertijd kan worden gebruikt in propriëtaire projecten en applicaties~\cite{PhilDutson2012}.

\subsection{Code en ontwikkeling}
Zoals werd aangehaald, schrijft men voornamelijk HTML5-code voorzien van \code{data-}* attributen. 
Daarna zal het raamwerk door middel van \term{progressive enhancement} allerhande code toevoegen om de beoogde UI-elementen correct te tonen in de browser. 
Dit wordt verder uitgelegd in de sectie browserondersteuning (zie \ref{sec:jqm-browser-support}).

Er zijn drie strategieën om webapplicaties te maken in \jqm{}~\cite{Broulik2012}. 
Een eerste is om de volledige applicatie in één webpagina te schrijven. 
De vele schermen van de webapplicatie zijn dan allemaal samengebracht op eenzelfde webpagina. 
Het voordeel bij deze aanpak is dat er initieel minder verzoeken zijn naar de server omdat alles in één bestand wordt opgehaald. 
Dit geldt ook zo voor de geïmporteerde CSS- en JavaScript-bestanden. 

Een tweede strategie is om voor ieder scherm een aparte webpagina aan te maken. 
Het voordeel hierbij is dat de eerste pagina waar de gebruiker op terecht komt, sneller wordt gedownload. 
Bij iedere navigatie naar een ander scherm, moet dit scherm via AJAX worden opgehaald, waardoor dit vertragend kan werken. 

Een laatste strategie is om een mix tussen beide te maken. 
Men kan bijvoorbeeld alle schermen die de gebruiker vaak nodig heeft op één webpagina plaatsen. 
De schermen die de gebruiker zelden nodig heeft, plaats men dan op aparte webpagina's.  

\subsection{Functionele kenmerken}
\jqm{} is een raamwerk dat voornamelijk UI-elementen aanbiedt, met name pagina's en dialoogvensters, werkbalken, knoppen, inhoud vormgeven, elementen voor formulieren en lijsten~\cite{JQuery2012b}.
Deze kenmerken zijn gebaseerd op versie~1.2.

\paragraph{Pagina's en dialoogvensters}
De basisstructuur van een pagina bestaat uit een koptekst, inhoud en voettekst. 
Bij het overgaan naar een andere pagina kan men kiezen uit tien overgangseffecten. 
Voordat deze overgang gebeurt, zal \jqm{} altijd eerst die pagina ophalen via AJAX en inladen in het DOM. 
Zo kan een soepel overgangseffect worden getoond aan de gebruiker. 
Daarnaast is het ook mogelijk om gelinkte pagina's op voorhand op te halen. 
Als laatste biedt \jqm{} ook dialoogvensters en pop-ups aan. 

\paragraph{Werkbalken}
Het is mogelijk om zowel knoppen bij de koptekst als bij de voettekst te plaatsen. 
Bij deze laatste kunnen typisch meer knoppen geplaatst worden, bij de koptekst slechts twee. 
Daarnaast is het ook mogelijk om navigatiebalken te maken. 
Aan zowel de werk- als navigatiebalken kunnen iconen worden toegevoegd.

\paragraph{Knoppen}
Het is ook mogelijk om knoppen te plaatsen in het inhoud gedeelde. 
Ook hier is er terug een variëteit aan mogelijkheden: grote of kleine, met iconen of zonder, gegroepeerd of niet. 

\paragraph{Inhoud vormgeven}
De inhoud van de pagina kan worden vormgegeven door gebruik te maken van een rooster. 
\jqm{} laat roosters tot vijf kolommen toe. 
Daarnaast zijn er ook nog opklapbare blokken ter beschikking. 
Als laatste kunnen deze blokken ook samengevoegd worden tot een accordeon. 

\paragraph{Elementen voor formulieren}
\jqm{} biedt alle gangbare elementen voor formulieren aan zoals tekstinvoer, een selectie uit een lijst, een zoekveld, een \term{slider} en een \term{switch}. 
Het raamwerk verplicht zelfs om de \code{<label>}-tag te gebruiken. 
Zo wordt de applicatie toegankelijker gemaakt voor bijvoorbeeld mensen met een \term{e-reader}.

\paragraph{Lijsten}
Een laatste categorie UI-elementen die \jqm{} aanbiedt, zijn lijsten. 
Deze gaan van standaard ongeordende lijsten tot lijsten met alle soorten decoraties als iconen, afbeeldingen, telbubbels en verdelers. 
Ook is het mogelijk om in deze lijsten te zoeken. 
Hiervoor dient de gebruiker enkel één data attribuut toe te voegen, waarna het raamwerk de implementatie voorziet. 

\subsection{Niet-functionele kenmerken}
\paragraph{Performantie}
Zoals gezegd schrijft de ontwikkelaar HTML5-code met specifieke data attributen en zal het raamwerk daarna de code verder aanvullen. 
Dit gebeurt enkel op de pagina die de gebruiker op dat moment bekijkt. 
Dit gaat dus ook op voor een webapplicatie waarbij alle schermen op één webpagina zijn geschreven. 
Deze webpagina bevat allemaal \code{<div>}-verpakkingen voor ieder scherm. 
\jqm{} zal enkel die \code{<div>} verder aanvullen die op dat moment getoond wordt aan de gebruiker. 

\paragraph{Aanpasbaarheid}
Als \jqm{} \term{out-of-the-box} wordt gebruikt, zit alles al goed qua kleur en design. 
Er is keuze uit vijf kleurenthema's die kunnen worden toegepast op de gehele applicatie of enkel op bepaalde elementen. 
Om een applicatie echt te laten onderscheiden van de andere, is een eigen kleurthema noodzakelijk. 
Hier is \jqm{} op voorzien door hun \term{stylesheet} op te delen in twee delen: thema's en structuur. 
Een ontwikkelaar kan ook enkel de structuur downloaden en zelf het thema in CSS schrijven. 
Doordat dit laatste heel wat inspanning vraagt, hebben de ontwikkelaars van \jqm{} ook een tool ter beschikking gesteld, namelijk ThemeRoller~\cite{JQuery2012c}. 
Hiermee worden de kleuren naar een voorbeeldapplicatie gesleept, waarna de overeenkomstige \term{stylesheet} kan worden gedownload.

\paragraph{Programmeerbaarheid}
Bij het programmeren in \jqm{} wordt geen enkel ontwerppatroon afgedwongen. 
De code voor de UI-elementen wordt tenslotte als HTML5-code geschreven. 
Voor de echte functionaliteit wordt beroep gedaan op JavaScript en meer bepaald op de jQuery Core bibliotheek. 
Ook deze dwingt geen ontwerppatroon af.

\paragraph{Browserondersteuning}
\label{sec:jqm-browser-support}

% TODO Tim: verder uitwerken

\jqm{} deelt browsers op in drie verschillende klassen: A, B en C~\cite{JQuery2012d}. 
Hierbij ondersteunt een klasse A browser alles, terwijl een klasse C browser enkel de basis HTML ondersteunt (en dus bijvoorbeeld geen hippe CCS3 overgangen).
\jqm{} maakt gebruikt van \emph{progressive enhancement} (zie \ref{par:progressive-enhancement}).

%%%%%%%%%%%%%%%%%%%%%%%%%%%%%%%%%%%%%%%%%%%%%%%%%%%%%%%%%%%%%%%%%%
%%%%%%%%%%%%%%%%%%%%%%%%%%%%%%%%%%%%%%%%%%%%%%%%%%%%%%%%%%%%%%%%%%

\section{\st}
\label{sec:raamwerk-st}

\st{} is een relatief verschillend raamwerk in vergelijking met \jqm{}.  
Het wordt ontwikkeld door Sencha,  een bedrijf dat in 2010 is ontstaan als een samensmelting van Ext JS,  jQuery Touch en Raphaël.  
Ext JS is een JavaScript raamwerk voor de ontwikkeling van web applicaties. 
jQuery Touch is een jQuery plugin voor mobiele web ontwikkeling.  
Het steunt op WebKit en voegt \term{touch events} toe aan jQuery.  
Raphaël,  ten slotte,  is een JavaScript bibliotheek voor vector tekeningen. 
Op het moment van schrijven is \st{} aan versie 2.1.1~\cite{Inc.}.  

\subsection{Omkadering}
\paragraph{Programmeertaal}
\st{} is JavaScript gedreven dus all functionaliteiten worden in JavaScript geïmplementeerd. 
Het aanroepen van het raamwerk gebeurt door het invoeren van de \st{} bibliotheek binnen \code{<script>}-elementen.  
Alle HTML code wordt bij het bekijken van de pagina gegenereerd.  

\paragraph{Tools}
Naast \st{} levert Sencha nog producten die \st{} uitbreiden of het leven van de ontwikkelaar makkelijker maken.  
Deze worden hieronder opgelijst~\cite{Inc.}.  

\subparagraph{Sencha Animator}
Dit is een desktop applicatie om CSS3 animaties te ontwerpen.  
Deze animaties worden enkel in WebKit browsers ondersteund.

\subparagraph{Sencha Architect}
Dit is een andere desktop applicatie waarmee je makkelijk een UI kan ontwikkelen met behulp van \term{drag-and-drop} commando's.  

\subparagraph{Sencha GXT}
Sencha GXT is een uitbreiding op Google Web Toolkit (GWT).  
De compiler van GWT laat toe applicaties in Java te schrijven en ze te compileren naar geoptimaliseerde,  \term{cross-browser} HTML5 en JavaScript.  Sencha GXT voegt grafieken,  widgets, etc. toe aan GWT.

\subparagraph{Sencha.IO}
Deze uitbreiding zorgt voor \term{cloud} services binnen mobiele applicaties.  

\paragraph{Documentatie}
Alle documentatie voor \st{} 2.1.1 is te vinden op \url{docs.sencha.com/touch/2-0}.  
Een zoekfunctie voor objecten,  eigenschappen en methoden is aanwezig om snel zaken op te zoeken.  
De meeste functionaliteiten zijn voorzien van codevoorbeelden samen met het resultaat hoe de browser de code rendert.  
Verder biedt de Sencha website ook een groot aanbod om Sencha te leren gebruiken \url{www.sencha.com/learn/touch/}.  
Hier staan handleidingen,  introductie video etc..

%Door de snelle ontwikkeling van Sencha blijft de documentatie niet altijd up-to-date.  Zo zijn vele methoden verouderd maar staat er geen alternatief vermeld. 
%TODO dit is misschien eerer subjectief?
Een ander handig raadslagwerk is de ‘Kitchen Sink'~\cite{Inc.2013}.  
Dit is een webapplicatie,  geschreven in \st{},  die de belangrijkste functionaliteiten bevat samen met de bijhorende code.  

\paragraph{Marktadoptatie}
Volgens de Sencha website is 50\% van de Fortune 100 - een lijst van de grootste Amerikaanse bedrijven gerangschikt op jaaromzet - een Sencha klant~\cite{Inc.}.  
Enkele van hun grootste klanten zijn CNN,  Samsung,  Cisco en  Visa.

\paragraph{Licenties}
\st{} is gratis binnen een commerciële context waarbij het bedrijf in kwestie de broncode niet deelt voor zijn gebruikers.  
Wanneer je dit wel wil doen bestaat er ook een gratis \term{open-source} versie van \st{}.  
Deze komt met een GNU GPL v3 \term{open-source} licentie wat wil zeggen dat je de vrijheid hebt om aanpassingen aan de broncode te maken en te verspreiden,  zolang je zelf je code maar gratis verspreid voor alle gebruikers.
  
Voor de ontwikkeling van eigen raamwerken of SDKs betaal je een \term{original equipment manufacturer} (OEM) licentie.  
Dit wil zeggen dat bedrijven hun producten gaan verkopen onder hun eigen merk en naam, maar gebruik maken van Sencha.  
Omdat het gebruik hiervan per gebruiker verschilt,  worden OEM licenties op maat gemaakt~\cite{Inc.}.

\subsection{Code en ontwikkeling}
Zoals reeds vermeld moet alle code in JavaScript worden geschreven en dient één HTML bestand slechts als container om de bestanden in te laden.  Sencha valt dus onder JavaScript gebaseerde raamwerken.  
De keuze voor deze aanpak heeft twee belangrijke motivaties.  
Enerzijds is \st{} gebouwd op Ext JS,  wat op zich een JavaScript raamwerk is.  
Anderzijds zorgt het voor een betere ondersteuning voor toestellen met verschillende resoluties.  
Samen met SASS en Compass kan Sencha lay-outs definiëren per device (zie sectie \ref{sec:sencha-aanpasbaarheid}).  
De \code{Ext.env.Browser} en \code{Ext.env.OS} eigenschappen en \code{Ext.Viewport.getOrientation} en \code{Ext.feature.has} methoden kunnen de vereisten bepalen en de juiste lay-out kiezen~\cite{JohnEClark2012}.

Om het de ontwikkelaars makkelijker te maken biedt Sencha ook SDK tools aan.  
Momenteel bevinden deze zich nog in bèta.  
Concreet zijn deze tools commando's voor de terminal die onder andere nieuwe projecten kunnen aanmaken, JavaScript bestanden kunnen optimaliseren maar vooral de webapplicatie kunnen omzetten naar native applicaties voor iOS en Android.

\paragraph{Debugging}
Het debuggen van je code gebeurt voornamelijk in de browser zelf.  
Tools als de Safari Web Inspector,  Chrome Developer Tools of Firebug moeten de fouten kunnen opsporen.  
De broncode van \st{} kan ook ingeladen worden met \code{sencha-touch-debug.js} als bibliotheek.  
Deze versie is niet gecomprimeerd en bevat commentaar en documentatie om makkelijker te zoeken waar in de code de fout zich juist bevond.

\subsection{Functionele kenmerken}
Net zoals \jqm{} heeft \st{} ook een hele hoop functionaliteiten om eenvoudig UI elementen te genereren.  
\st{} bevat alle elementen van de UI als JavaScript objecten.  
Net zoals alle objectgerichte programmeertalen maken deze objecten gebruik van een klassesysteem,  iets wat slechts vanaf \st{} 2 werd ingevoerd.  
Op die manier kunnen klassen worden gedefinieerd (\code{Ext.define}) en aangemaakt (\code{Ext.create}).  
Hierbij is ook overerving mogelijk.  
De basisklasse van alle objecten is \code{Ext.Component}.  
Componenten kunnen gerenderd worden, zichzelf tonen of verbergen,  centreren op het scherm en zichzelf aan- of uitzetten.   
Het aanmaken van componenten kan compacter door het gewenste component als \code{xtype} te definiëren.  

Een andere belangrijke component is \code{Ext.Container}.  
Containers kunnen subcomponenten bevatten en een lay-out specificeren.  
Alle componenten krijgen een naam die verwijst naar een namespace.  
Dit is handig om conflicten te vermijden tussen je eigen objecten en de standaard objecten van het raamwerk.  

Voor een opsomming van alle raamwerk componenten verwijzen we naar de documentatie~\cite{Inc.2013a}.

%jQuery subsecties:
%Pagina's en dialoogvensters
%werkbalken
%knoppen
%inhoud vormgeven
%elementen voor formulieren
%lijsten

\paragraph{Model}
Data kan intern worden voorgesteld met models.  
Dit is iets wat hoort bij het MVC patroon (zie sectie \ref{sec:sencha-programeerbaarheid}).  
Een model specificeert een lijst van velden die bij het model horen waarbij een veld een naam en een type heeft.  
Optioneel kunnen validaties bij de velden worden toegevoegd om data consistent te houden.  

\paragraph{Store}
\code{Ext.data.Store} is de klasse om instanties van een model op te slaan.  
Een \term{store} wordt voorzien van een \term{proxy}.  
Deze kan data aan de client of server zijde opslaan.  
Een \term{proxy} voor opslag aan client zijde kan zowel in het RAM geheugen als in de \term{local storage} van de browser opslaan.  
Een \term{proxy} voor server opslag kan data verzenden via AJAX (zelfde domein) of JSONP (verschillende domeinen).  
Een \term{proxy} kan ook nog voorzien worden van een \term{reader} die aangeeft hoe de ontvangen data gelezen moet worden.

\paragraph{View}
Een \term{view} is de benaming voor objecten die aan de gebruiker kunnen getoond worden.  
Een voorbeeld hiervan zijn lijsten,  waar vaak de data van een \term{store} wordt in weergegeven.  
Zo'n lijst kan makkelijk gefilterd of gesorteerd worden op basis van velden uit het model.  
Hiervoor moeten we \term{filters} of \term{sorters} aan de \term{store} toevoegen. 
De lay-out van één lijstitem bepalen kan via een \code{XTemplate}.  
Het sjabloon bepaalt de HTML structuur van elk item.  
Alle gedefinieerde velden van het model kunnen in de template worden opgeroepen of gemanipuleerd.

%TODO controllers?

\subsection{Niet-functionele kenmerken}
\paragraph{Performantie}
In vergelijking met versie 1.1 van \st{} is de performantie gestegen om wille van verschillende factoren.  
De introductie van het klasse systeem,  zoals besproken in de vorige sectie,  laat toe objecten dynamisch te laden. 
Het grote verschil tussen \code{Ext.define} en \code{Ext.create} is dat objecten enkel in het geheugen worden geladen na creatie.  
Het is dus de taak van de programmeur om objecten enkel te construeren wanneer ze nodig zijn.

Verder kwam versie 2.0 met een nieuwe lay-out \term{engine} die vooral het verwisselen van oriëntatie van het toestel versnelde.  
Ook een verbetering in performantie op Android toestellen,  voornamelijk bij scrollen en animaties,  werd ingevoerd~\cite{Inc.}.

Een benchmark voor deze verbeteringen zijn de opstarttijden van de Kitchen Sink applicatie.  
Het opstarten gebeurde met de verschillende \st{} versies en op verschillende toestellen.  
De resultaten zijn terug te vinden op figuur \ref{fig:sencha_performance}.  
Op bijna elk toestel blijkt \st{} 2.0 ongeveer één seconde sneller te werken~\cite{SenchaInc.2013}.

\begin{figure}
  \centering
  \includegraphics[width=0.8\textwidth]{figuren/sencha-touch-startup-times.png}
  \caption{\st{} Kitchen Sink opstarttijden~\cite{SenchaInc.2013}.}
  \label{fig:sencha_performance}
\end{figure}

\paragraph{Aanpasbaarheid}
\label{sec:sencha-aanpasbaarheid}
Elke component binnen het raamwerk moet overerven van \code{Ext.Component}.  
Deze voorziet een attribuut \code{ui}.  De waarde hiervan is een CSS klasse die bepaald hoe de component er zal uitzien.  
encha heeft al twee CSS klassen voorzien:  \code{light} en \code{dark}.  
Andere componenten kunnen deze lijst uitbreiden.  
Een knop kan bijvoorbeeld \code{normal},  \code{back},  \code{round},  \code{small},  \code{action} of \code{forward} als \code{ui} waarde hebben.

Het is ook mogelijk om eigen waarden voor \code{ui} te definiëren of de standaarden van Sencha aan te passen.  
Hiervoor moet je gebruik maken van SASS en Compass om je CSS bestanden aan te maken.  
SASS staat voor Syntactically Awesome Stylesheets en breidt CSS uit met variabelen,  geneste structuren,  mixins en overerving~\cite{Eppstein2013}.  
Mixins groeperen enkele CSS eigenschappen en kunnen worden herbruikt.  
Compass is een raamwerk bovenop SASS en CSS.  Het compileert SCSS (Sassy CSS) naar CSS bestanden~\cite{Eppstein2013a}.        

Sencha thema's bestaan allemaal uit een set van \term{mixins}.  
Door zelf \term{mixins} te creëren of reeds bestaande te manipuleren kunnen we eigen thema's creëren en ze aan de \code{ui}-waarde van een component toekennen.

\paragraph{Programmeerbaarheid}
\label{sec:sencha-programeerbaarheid}
Zoals reeds aangehaald ondersteund \st{} het MVC (Model-View-Controller) patroon.  
Dit patroon vermijdt lange JavaScript bestanden door ze logisch op te delen.  
Modellen groeperen velden tot een beschrijving van data-objecten,  views definiëren de weergave van componenten en controllers verbinden beide op basis van \term{events}.

In theorie zou het verschil tussen mobiele websites en applicaties enkel in de views terug te vinden zijn.  
Echter,  dit wordt nog niet volledig ondersteund en raadt men dus aan om hiervoor aparte projecten te voorzien.

\paragraph{Ondersteuning browser}
\st{} steunt op de WebKit browser \term{engine} dus moet de browser deze bevatten.  
Hoewel dit bij de meeste browsers geen probleem meer vormt vallen toch enkele populaire browsers uit de boot.  
\st{} is bijvoorbeeld niet compatibel met FireFox Mobile en Opera Mobile~\cite{JohnEClark2012}.

Zoals reeds vermeld zijn er ook methoden voorzien om informatie op te vragen over de context die gehanteerd wordt (browser, OS, toestel, etc.).  Verder kan \st{} ook vragen naar de ondersteuning van specifieke kenmerken (audio,  canvas,  CSS3, …),  analoog als Modernizr.  

Op de Secha website zijn voor sommige browsers en bijhorend besturingssystemen scorecards voorzien om hun compatibiliteit met HTLM5 en \st{} te bespreken~\cite{Inc.}.


\section{\kendo}
\label{sec:raamwerk-kendo}

\subsection{Omkadering}
\subsection{Code en ontwikkeling}
\subsection{Functionele kenmerken}
\subsection{Niet-functionele kenmerken}

\section{\lungo}
\label{sec:raamwerk-lungo}

\subsection{Omkadering}
\subsection{Code en ontwikkeling}
\subsection{Functionele kenmerken}
\subsection{Niet-functionele kenmerken}

\section{Tabel}
\label{sec:raamwerken-tabel}

\chapter{Vergelijkingscriteria}
\label{chap:vergelijkingscriteria}

Dit hoofdstuk bekijkt hoe de mobiele HTML5 raamwerken actief zullen worden vergeleken.
Hoofdzakelijk zal dit gebeuren aan de hand van een \term{proof of concept}~(POC).
Deze wordt geïntroduceerd in sectie \ref{sec:vergelijking-poc} en zal hoofdzakelijk de gekozen vergelijkingscriteria in sectie \ref{sec:vergelijking-criteria} drijven.
De criteria die worden voorgesteld, zullen voortaan actieve criteria worden genoemd.


\section{POC}
\label{sec:vergelijking-poc}
In samenspraak met Capgemini werd gekozen om een \term{proof of concept}~(POC) op te stellen.
%TODO hier refereren en reflecteren over POC in literatuur (tim)
Dit is een idee waarbij de uitvoerbaarheid in de verschillende raamwerken kan worden nagegaan.
Verschillende vergaderingen werden georganiseerd om tot een idee te komen dat vooral in de bedrijfswereld van toepassing is.
Het uiteindelijke idee is een applicatie die het mogelijk maakt voor werknemers om hun onkosten via hun mobiel apparaat door te sturen.

Het idee werd uitgewerkt door Capgemini en geleverd aan de auteurs als \term{mockup}.
Dit is een voorstelling van de applicatie als een reeks schermen zoals deze er zullen uitzien op een apparaat. 
Een voorbeeld van zo een scherm is te vinden op figuur~\ref{fig:poc}. 
Naast de schermen staan de functionele vereisten die op het scherm van toepassing zijn.
De bedoeling is dat deze POC wordt uitgewerkt zowel voor smartphone als tablet, zowel voor Android als iOS, zowel voor staande als liggende apparaten en zowel voor online als offline gebruik.

\begin{figure}
  \centering
  \includegraphics[trim=0cm 4.6cm 0cm 1.55cm,clip=true,width=\textwidth]{figuren/poc.pdf}
  \caption{POC bij het toevoegen van een nieuwe onkost met aan de linkerkant de weergave op een tablet en aan de rechterkant deze op een smartphone.}
  \label{fig:poc}
\end{figure}

\subsection{Aspecten}
\label{sec:vergelijking-poc-detail}

Een werknemer meldt zich eerst aan op de applicatie en kan daarna ofwel een nieuw onkostenformulier aanmaken of zijn doorgestuurde onkostenformulieren bekijken.
De term onkostenformulier is een groepering van meerdere onkosten met bijhorende bewijsstukken en de handtekening van de werknemer. 
Het aanmaken van een nieuw onkostenformulier verloopt in vier stappen.
Indien de werknemer al eerder begonnen was met het aanmaken van een formulier, zal hij worden gevraagd of hij verder wil gaan met dat formulier of met een nieuw formulier wil starten.

\begin{enumerate}
\item De eerste stap is het bekijken en/of aanpassen van de persoonlijke informatie van de werknemer.
Bij het aanpassen van deze gegevens, zullen deze worden gevalideerd.
Indien deze validatie faalt, krijg de werknemer een dialoogvenster te zien met de reden tot falen.
Ook worden de foute velden rood gemarkeerd.

\item In de tweede stap kan de werknemer zijn toegevoegde onkosten aan het formulier bekijken.
In het begin is deze lijst leeg, tenzij hij eerder een formulier aan het invullen was (zie infra).
Indien deze lijst onkosten bevat, is het mogelijk om hierop te klikken en deze te bekijken.
Aanpassen is niet mogelijk.

\item In stap drie kan een nieuwe onkost worden toegevoegd.
Dit kan ofwel een binnenlandse ofwel buitenlandse onkost zijn.
Voor beide dient een datum en projectcode te worden opgegeven.
De eerstgenoemde is een \term{datepicker} die teruggaat tot twee maanden in de tijd.
De laatstgenoemde bevat automatische aanvulling, maar de werknemer is niet verplicht om een projectcode uit de aanvulling te selecteren.
Daarnaast dient het type en bedrag van de onkost, alsook een bewijsstuk te worden opgegeven.
Bij een buitenlandse onkost moet de munteenheid worden opgegeven, waarna de applicatie deze automatisch omvormt naar euro.
Het scherm voor het toevoegen van een buitenlandse onkost wordt getoond op figuur \ref{fig:poc}. 
Net zoals bij stap één geldt ook hier validatie op de formuliervelden.

\item In deze laatste stap dient een handtekening te worden geplaatst waarna het formulier kan worden doorgestuurd.
Indien de gebruiker offline werkt, zal deze worden opgeslagen op het toestel.
De werknemer kan het formulier opnieuw doorsturen zodra hij terug online is.

\end{enumerate}

Bij het bekijken van de doorgestuurde formulieren is het mogelijk om per formulier de bijhorende PDF te downloaden. 
Deze bevat een overzicht van de onkosten met bijhorende bewijsstukken, alsook de handtekening van de werknemer.

\section{Criteria}
\label{sec:vergelijking-criteria}

In deze sectie zullen de actieve criteria toegelicht worden die zullen worden toegepast om de raamwerken te vergelijken.
In sectie \ref{sec:vergelijken-raamwerken} werden reeds technieken besproken die in de literatuur worden toegepast.
Elementen van deze technieken zullen terugkomen in de voorgestelde methode om de raamwerken te evalueren.
%TODO in elke sectie van een criteria een referentie naar literatuur (zie drive document) + reflecteren met ISO (sander)
Sectie \ref{sec:raamwerken-tabel} bevatte de passieve vergelijkingscriteria die raamwerken vergeleken met informatie over het raamwerk zelf.

Vijf criteria zullen worden gebruikt: populariteit (\ref{sec:vergelijking-populariteit}), productiviteit (\ref{sec:vergelijking-productiviteit}), gebruik (\ref{sec:vergelijking-gebruik}), ondersteuning (\ref{sec:vergelijking-ondersteuning}) en performantie (\ref{sec:vergelijking-performantie}). 
%TODO refereren naar puntensysteem literatuur(tim)
Elk raamwerk krijgt voor elk criterium een score afgeleid uit een formule. 
Deze scores zullen in een spinnenweb worden ondergebracht (zie sectie \ref{sec:vergelijking-spinnenweb}).
Zoals hierboven vermeld zal een POC gebruikt worden bij de vergelijking.
De implementatie van deze POC zal het gebruiks- en ondersteuningscriterium drijven.  
Dit komt omdat Capgemini de POC zo heeft opgesteld dat het de verschillende functionaliteiten bevat die van een normale applicatie verwacht worden.

%TODO populariteit in visionmobile gebruiken bij onze criteria:  er staat alleen percentages van het aantal developers dat welke tools gebruiken..
\subsection{Populariteit}
\label{sec:vergelijking-populariteit}
De populariteit van een raamwerk is een belangrijke factor want het bepaalt de gemeenschap en levendigheid van het raamwerk.
De definitie van gemeenschap op de blogpost van Codefessions~\cite{Sarrafi2012a} zegt ook dat dit een belangrijke factor is omdat het de toekomstige ontwikkeling en de hulp bij het gebruik van het raamwerk aantoont. 
De populariteit kan in cijfers worden uitgedrukt door gebruik te maken van sociale netwerken. 
Een tabel zal voorzien worden met in de rijen het aantal volgers op Twitter, sterren en \term{forkers} van \gh{},  vragen op \so{} en aantal vind-ik-leuks van \fb{}~\cite{Hales2012,Ayuso2012}.
%TODO wat is het beste? refs of drie extra zinnen?
De eerste drie termen worden ook in HTML5 and JavaScript Web Apps van Hales bekeken wanneer HTML5-raamwerken worden voorgesteld~\cite{Hales2012}.
\so{} vragen worden als criterium op een blogpost van Monocaffe gebruikt om mobiele raamwerken te vergelijken~\cite{Ayuso2012}.  
Het aantal vind-ik-leuks van \fb{} werd zelf geïntroduceerd.

GitHub kan worden gezien als een sociaal netwerk voor programmeurs~\cite{Catone2008} en bepaalt dus de actieve gemeenschap rond het raamwerk.
Raamwerken die niet op GitHub te vinden zijn krijgen nul voor zowel het aantal sterren en \term{forkers}.
Een alternatief hield de interpolatie van de GitHub data van de overige raamwerken in.
Omdat deze aanpak het raamwerk onterecht zou bevoordelen, is hier niet voor gekozen.

De som van Twitter volgers ($T_r$), \gh{} sterren ($S_r$), \gh{} \term{forkers} ($F_r$), \so{} vragen ($SO_r$) en \fb{} vind-ik-leuks ($FB_r$) vormt de score voor het populariteitscriterium:
\begin{equation}
  \text{Populariteit}_r=T_r+S_r+F_r+SO_r+FB_r
  \label{eq:populariteit}
\end{equation}
voor een raamwerk $r$.

Omdat deze gegevens zeer dynamisch zijn, zullen verschillende metingen in de tijd de evolutie van de data weergeven.
Ook zullen de uitkomsten van dit criterium worden vergeleken met data geleverd door Google Trends~\cite{Google2012a}.
Deze webapplicatie toont de evolutie van zoektermen op Google op een schaal van 100, waarbij 100 overeenkomt met de grootste zoekinteresse.
Voor elk raamwerk zal het aantal zoekopdrachten op Google in functie van de tijd worden uitgezet.

Er bestaat geen exacte formule om populariteit uit te drukken.
De formule die werd gekozen om de score voor dit criterium te quoteren is onderheven aan subjectiviteit.
Twee opmerkingen moeten hierbij worden gemaakt.
Enerzijds zijn de auteurs zich ervan bewust dat de doorsnede tussen sociale netwerken niet leeg is.
Zo kan éénzelfde persoon zowel een volger op Twitter zijn als een vind-ik-leuk op \fb{} plaatsen.
Verschillende individuen zullen dus dubbel geteld worden in de totale score voor populariteit van het raamwerk.
De score zal dus slechts een indicatie geven over de populariteit,  het is geen exacte weergave.
Ten tweede zijn de auteurs er zicht van bewust dat de inclusie van \so{} op twee manieren kan worden bekeken.
Enerzijds kunnen veel vragen duiden op veel onduidelijkheden over het raamwerk.
Anderzijds kan dit een maat zijn voor de populariteit van dit onderwerp.
De auteurs zijn van mening dat de tweede zienswijze correcter is dan de eerste en het dus valide is \so{} in de formule op te nemen.

%%%%%%%%%%%%%%%%%%%%%%%%%%%%%%%%%%%%%%%%%%%%%%%%%%%%%%%%%%%%%%%%%%
%%%%%%%%%%%%%%%%%%%%%%%%%%%%%%%%%%%%%%%%%%%%%%%%%%%%%%%%%%%%%%%%%%

\subsection{Productiviteit}
\label{sec:vergelijking-productiviteit}
%TODO geen eenduidige manier + perceptie
De productiviteit moet berekend hoe lang het duurt om met het raamwerk vertrouwd te raken en iets nuttig te kunnen bouwen.
Dit is belangrijk want bedrijven willen zo min mogelijk tijd verliezen om met nieuwe technolgieën aan de slag te kunnen.
In de ISO 25010-standaard zijn de categorieën bruikbaarheid en efficiëntie vergelijkbaar met dit criterium.

De auteurs zullen elk de POC in twee verschillende raamwerken maken en daarnaast ook een extra loginapplicatie in twee andere raamwerken.
De ene auteur maakt de POC in \jqm{} en \lungo{} en de loginapplicatie in \st{} en \kendo{}.
De andere zal dan de POC in \st{} en \kendo{} maken en de loginapplicatie in \jqm{} en \lungo{}.
De tijd die nodig is om de volledige POC te implementeren is een indicatie voor de productiviteit. 
Er wordt verondersteld dat de auteurs over een gemeenschappelijke technische achtergrond beschikken.
Toch kunnen beide onderling verschillen in efficiëntie,  waardoor de productiviteit verschilt.
Dit probleem is inherent aan dit criterium.
Toch is het belangijk om een schatting op deze manier te kunnen maken.

% Omdat de POC twee keer moet worden geïmplementeerd, wordt verwacht dat de tweede implementatie sneller zal verlopen.
% Dit probleem is onafwendbaar en zal bij de evaluatie van de data aangehaald worden.
% De uren voor de implementatie van de loginapplicatie zal de score correcter maken.
% 
% De som van de uren voor het implementeren van de POC ($t_{r,POC}$) en de loginapplicatie ($t_{r,login}$) vormt de score voor de productiviteit:
% \begin{equation}
%   \text{Productiviteit}_r = {t_{r,POC} + t_{r,login}}
%   \label{eq:productiviteit}
% \end{equation}
% voor een raamwerk $r$.

Er zijn echter vijf redenen waarom de implementaties van de POC geen goede indicatie zijn voor de productivteit.
Deze werden door de auteurs ervaren wanneer de implementatie in het tweede raamwerk werd uitgevoerd:
\begin{enumerate}
\item Betere ervaring met de POC versnelt bij de tweede implementatie het overzicht van vereisten die moeten worden geïmplementeerd. 
\item Een verbeterde ervaring met HTML5-raamwerken had een positieve invloed op de verdere implementaties.
Dit weerspiegelde zich vooral tussen \jqm{} en \lungo{}.
Hoewel ze beide op een verschillende \js{}-bibliotheek steunen - respectievelijk jQuery en QuoJS - zijn de gelijkenissen tussen deze twee raamwerken groot.
Ook leggen ze beide geen ontwerppatroon op.
\item Er kon code,  zoals van de implementatie in \jqm{},  overgenomen worden bij de implementatie van de POC met \lungo{} en \kendo{}.
\item Er kwamen bij de eerste implementatie problemen met de \term{backend} naar boven.
Deze waren bij de tweede implementatie reeds opgelost.
Door het onnauwkeurig opmeten van de tijd kan er geen schatting worden gemaakt van de tijd die aan de problemen van de \term{backend} werden besteed.
\item Niet de volledige POC kon met \lungo{} en \st{} worden ontwikkeld.
\end{enumerate}

De implementatie van de loginapplicatie is een alternatieve test van de productiviteit.
Deze applicatie bevat GI-elementen, validaties,  \term{backend} integratie en een lijst.
De implementatie van de loginapplicatie kan dus als voldoende steekproef beschouwd worden om ervaring met een raamwerk te testen.
Na het aanmelden met deze applicatie zal de gebruiker een lijst van $850$ elementen te zien krijgen.
Deze lijst is bedoeld als stresstest om de performantie te testen (zie sectie \ref{sec:vergelijking-performantie}).
De elementen in de lijst zullen voorzien worden van een afbeelding en tekst.
De lijst kan als een potentiële muziekapplicatie gezien worden waarbij de afbeelding en tekst naar liedjes verwijzen.
Het aantal elementen in de lijst - $850$ - is een schatting van het maximum aantal liedjes dat ooit is opgenomen~\cite{Zimmy2011}.
Het kan dus als bovengrens voor dit soort applicaties worden beschouwd.
De implementatie van deze applicatie zal een indicatie geven hoe snel,  zonder al te veel voorkennis van het raamwerk,  één eenvoudige applicatie opgeleverd kan worden.

De werkuren van de loginapplicatie bleken niet onderheven aan de vijf zonet opgenoemde redenen:
\begin{enumerate}
\item Bij elke implementatie werd met dezelfde achtergrondkennis gestart.  
De implementatie van de loginapplicatie is triviaal en eenduidig.
Er geldt dus voor alle raamwerken dat de ervaring met de applicatie reeds hoog was.
\item Eerst werd de implementatie van de POC gemaakt voordat aan de loginapplicatie werd begonnen.
Hierdoor was de algemene ervaring met HTML5-raamwerken reeds groot.
\item Er werd geen code gekopieerd. 
\item Er waren geen problemen met de \term{backend}.
\item Alle functionaliteit van de loginapplicatie kon met alle vier raamwerken worden gebouwd.
\end{enumerate}
Om al deze redenen werd beslist de score voor productiviteit te bepalen door enkel de uren van de login applicatie ($t_{r,login}$) te beschouwen.
Ook in de vergelijking van Burris werd voor een loginapplicatie gekozen~\cite{Burris}.
Deze vergelijkt enkel \st{} en \jqm{}.
De formule voor productiviteit is dan:
\begin{equation}
  \text{Productiviteit}_r = t_{r,login}
  \label{eq:productiviteit-enhanced}
\end{equation}
voor een raamwerk $r$.

De uitkomsten van dit criterium zullen gestaafd worden door het aantal lijnen code te presenteren die nodig waren voor zowel de POC als de loginapplicatie te bouwen.
Ook zullen de factoren die de leercurve bepalen, worden bekeken. 
Dit zijn ten eerste de tools die de programmeur kan gebruiken om eenvoudiger te ontwikkelen.
Vervolgens zal de kwaliteit en kwantiteit van de documentatie van elk raamwerk worden bekeken.
De mogelijkheden voor debuggen bepalen ook de leercurve en zullen worden onderzocht.
Tot slot zal gekeken worden naar de aanwezige literatuur van het raamwerk en waar ontwikkelaars met vragen terecht kunnen.

%%%%%%%%%%%%%%%%%%%%%%%%%%%%%%%%%%%%%%%%%%%%%%%%%%%%%%%%%%%%%%%%%%
%%%%%%%%%%%%%%%%%%%%%%%%%%%%%%%%%%%%%%%%%%%%%%%%%%%%%%%%%%%%%%%%%%

\subsection{Gebruik}
\label{sec:vergelijking-gebruik}
Dit criterium moet weergeven welke functionaliteit of plug-ins het raamwerk kan bieden.
Hier meer functionaliteiten het raamwerken te bieden heeft,  hoe minder de programmeur zelf moet schrijven en hoe bruikbaarder het raamwerk wordt.
Ook de ISO 25010-standaard probeert het gebruik met de categorie functionele geschiktheid te testen.

Uit de \term{mockup} schermen en de bijhorende functionele vereisten werden $13$ uitdagingen met in totaal $38$ deeluitdagingen geëxtraheerd.
Alle functionaliteit die potentieel door een raamwerk kan worden geleverd en in de POC wordt gebruikt, zit in een uitdaging vervat.  
Echter, een voorbeeld van functionaliteit van de POC die niet in een uitdaging zit, is de omzetting van \term{identifiers} naar een tekstuele vorm.
Dit is geen interessante functionaliteit omdat het eigen is aan de POC zelf.
De implementatie hiervan zal uitsluitend uit \js{}-code bestaan.
Alle uitdagingen en deeluitdagingen zijn in tabel~\ref{tabel:uitdagingen} te vinden.

\pgfplotstabletypeset[
  begin table=\begin{longtable}{l},
  end table=\caption{$13$ uitdagingen onderverdeeld in $38$ deeluitdagingen voor gebruik.}\label{tabel:uitdagingen}\end{longtable},
  skip coltypes=true,
  col sep=comma,
  string type,
  header=true,
  columns={Uitdaging},
  columns/Uitdaging/.style={column name=\textbf{Uitdagingen}, column type={l}},  
  every head row/.style={
    before row=\toprule,
    after row=\midrule},
  every last row/.style={
    after row=\bottomrule}
]{tabellen/uitdagingen.csv}

De wijze waarop het raamwerk de uitdaging aangaat zal de score bepalen.
Er onderscheiden zich drie gevallen.
De hoogste score ($2$) wordt toegekend wanneer de functionaliteit aangeboden wordt door het raamwerk. 
Een lagere score ($1$) betekent dat een plug-in moet worden gezocht.
Omdat de raamwerken bouwen op HTML5, zal een kenmerk van HTML5 ook als plug-in beschouwd worden.
Voor een oplijsting van de HTML5 kenmerken wordt naar sectie \ref{sec:html5-css3-js} verwezen.
Wanneer de implementatie zelf moet worden geschreven of een hack noodzakelijk is, zal de laagste score ($0$) worden toegekend.
Ook is het mogelijk  dat de uitdaging helemaal niet wordt geïmplementeerd.
Dit is mogelijk wanneer het raamwerk de functionaliteit niet ondersteund,  geen plug-in werd gevonden en niet aan een eigen implementatie wordt begonnen.
Dit zal leiden tot een $0$ score.
Wanneer CSS-code wordt gebruikt om de uitdaging te implementeren, zal de laagste score worden toegekend.
Het gebruik van CSS3 wordt echter als kenmerk van HTML5 gezien en vervolgens met $1$ gequoteerd.

Tabel \ref{tabel:scores-uitdagingen} toont de mogelijke scores $U_{r,i}$ van raamwerk $r$ en voor uitdaging $i$.
\begin{table}	
  \centering
  \begin{tabular}{ll}
    \toprule
    \textbf{Score} & \textbf{Verklaring}\\
    \midrule
    $U_{r,i} = 2$ & Ondersteund door het raamwerk\\
    $U_{r,i} = 1$ & Een plug-in of kenmerk van HTML5 is nodig\\
    $U_{r,i} = 0$ & Eigen implementatie of hack of niet geïmplementeerd\\ 
    \bottomrule
  \end{tabular}
  \caption{Beoordeling uitdagingen gebruikscriterium}
  \label{tabel:scores-uitdagingen}
\end{table}

De potentiële score van een uitdaging is discreet en ligt tussen $0$ en $2$.
Er zijn dus slechts $3$ scores waaruit gekozen kan worden om de implementatie te beoordelen.
De verklaringen bij de scores omvatten alle gevallen op een eenduidige manier.
Een alternatief bestaat uit $4$ scores waarbij HTML5-kenmerken een lagere score krijgen ten opzichte van plug-ins.
Omdat de raamwerken afhankelijk zijn van HTML5 werd hiervoor niet gekozen.

De formule voor gebruik is de volgende:
\begin{equation}
  \text{Gebruik}_r = \sum_{i=1}^{38}{\left(U_{r,i}\right)}
  \label{eq:gebruik}
\end{equation}
voor een raamwerk $r$ en een deeluitdaging $i$.
Omdat er $38$ deeluitdagingen zijn, kan een raamwerk voor dit criterium maximaal $76$ behalen.

%%%%%%%%%%%%%%%%%%%%%%%%%%%%%%%%%%%%%%%%%%%%%%%%%%%%%%%%%%%%%%%%%%
%%%%%%%%%%%%%%%%%%%%%%%%%%%%%%%%%%%%%%%%%%%%%%%%%%%%%%%%%%%%%%%%%%

\subsection{Ondersteuning}
\label{sec:vergelijking-ondersteuning}
Dit criterium moet weergeven hoe goed het raamwerk verschillende toestellen en verschillende besturingssystemen ondersteund.
Het is belangrijk dat een zo breed mogelijk publiek met éénzelfde applicatie kan worden bereikt.
%TODO deze reden geldig?
Het ondersteunen van verschillende platformen werd door meer dan $75\%$ van de ondervraagde ontwikkelaars in het Vision Mobile rapport aangehaald als hoofdreden om \term{cross-platform} tools te gebruiken.~\cite{Mobile2012} 
De ISO 25010-standaard beschrijft ook de overdraagbaarheid naar verschillende platformen.


Enkel de standaard browser van het besturingssysteem zal beschouwd worden.
Voor Android toestellen is dit de Android browser of Chrome.  
Vanaf Android~4.0 wordt Chrome als standaard browser beschouwd~\cite{Wimberly2008}.
Voor iOS is Mobile Safari de standaard browser.

Een context wordt gedefinieerd als één bepaalde configuratie van toestel, besturingssysteem en browser.
In elke context zal de functionaliteit van de POC op ondersteuning worden getest.
Uitdagingen die gebruikt zijn om het gebruikscriterium te testen, kunnen hier worden hergebruikt.
Aangezien sommige uitdagingen triviaal gelden voor elk apparaat zal er slechts een subset van deze uitdagingen getest worden.
De overgebleven uitdagingen zijn:
\begin{itemize}
 \item \uit{toestel}
 \item \uit{formulieren}
 \item \uit{autoaanvullen}
 \item \uit{afbeelding}
 \item \uit{validatie}
 \item \uit{handtekening}
 \item \uit{pdf}
 \item \uit{offline}
\end{itemize}
Voor dit criterium worden alle deeluitdagingen verwaarloosd behalve bij \uit{formulieren} en \uit{offline}.
Een uitdaging zal enkel slagen als alle deeluitdagingen ondersteund worden.
Zo kan bijvoorbeeld op een apparaat getest worden of auto-aanvullen werkt.
Uitdaging \uit{autoaanvullen} bevat als deeluitdagingen het ophalen van suggesties en het tonen van een dropdownmenu.
De werking van de uitdaging is een combinatie van beide en zal dus enkel slagen als beide worden ondersteund.
De deeluitdagingen van \uit{formulieren} en \uit{offline} kunnen wel op ondersteuning worden getest.
De twee deeluitdagingen van \term{datepicker} die bij \uit{formulieren} horen, zullen echter worden samengenomen zodat enkel een \term{datepicker} op zich en niet een aanpasbare \term{datepicker} op ondersteuning wordt gecontroleerd.


Het is belangrijk dat het raamwerk en niet een eigen implementatie op ondersteuning wordt getest.
Wanneer een uitdaging in het vorige criterium een $0$ behaalde, wil dit zeggen dat het raamwerk de uitdaging al niet ondersteunde.
In dit geval moet de uitdaging niet worden gecontroleerd.
Hierdoor is het aantal uitdagingen of deeluitdagingen die getest worden afhankelijk van het raamwerk.


De score van een uitdaging of deeluitdaging kan $1$ of $0$ zijn, respectievelijk een correcte of foutieve uitvoering.
De Cross Platform Capabilities zoals beschreven in op de Codefessions blogpost~\cite{Sarrafi2012a} geven een score aan raamwerken op een gelijkaardige manier.
In totaal zullen acht contexten worden gebruikt.
Deze worden in tabel \ref{tabel:toestellen-hci} weergegeven.

 \begin{table}
 \centering
 \resizebox{\textwidth}{!} {
 \pgfplotstabletypeset[
   begin table=\begin{tabular}{l l l l l},
   end table=\end{tabular},
   col sep=comma,
   header=true,
   string type,
   skip coltypes=true,
   columns={Apparaat,Soort,Lancering,BS,Browser},
   columns/Apparaat/.style={column name=\textbf{Apparaat}},  
   columns/Soort/.style={column name=\textbf{Soort}},
   columns/Lancering/.style={column name=\textbf{Lancering}},
   columns/BS/.style={column name=\textbf{BS}},
   columns/Browser/.style={column name=\textbf{Browser}},
   every head row/.style={
     before row=\toprule,
     after row=\midrule},
   every last row/.style={
     after row=\bottomrule}
 ]{tabellen/apparaten.csv}
 }
 \caption{Acht contexten: apparaten met hun soort, lancering, besturingssysteem~(BS) en browser.}
 \label{tabel:toestellen-hci}
 \end{table}
 
De keuze van de acht contexten waarop ondersteuning wordt getest, is voornamelijk bepaald door de beschikbaarheid van de apparaten op het Departement Computerwetenschappen van de KU Leuven.
Er werd een evenwichtige keuze gemaakt tussen besturingssysteem,  browser en type apparaat.
Er werd gekozen voor vier Android en vier iOS apparaten.
Bij de vier Android apparaten zijn er twee met een Android browser en twee met Chrome browser.
Ook werd er gekozen voor vier smartphones en vier tablets.

De som van de scores van de verschillende contexten bepaalt de score van het ondersteuningscriterium:
\begin{equation}
  \text{Ondersteuning}_r = \sum_{c=1}^{8}{\left(\sum_{i=1}^{N_r}U_{r,c,i}\right)}
  \label{eq:ondersteuning}
\end{equation}
voor  een raamwerk $r$, een context $c$, $N_r$ het maximum aantal geïmplementeerde deeluitdagingen voor een raamwerk $r$ en een uitdaging $i$. 


Indien het raamwerk een implementatie bevat voor alle uitdagingen en deeluitdagingen kan er per context maximaal $13$ gescoord worden.
Indien de acht contexten alle uitdagingen en deeluitdagingen correct weergeven zal de maximale score van $104$ behaald worden.

%%%%%%%%%%%%%%%%%%%%%%%%%%%%%%%%%%%%%%%%%%%%%%%%%%%%%%%%%%%%%%%%%%
%%%%%%%%%%%%%%%%%%%%%%%%%%%%%%%%%%%%%%%%%%%%%%%%%%%%%%%%%%%%%%%%%%

\subsection{Performantie}
\label{sec:vergelijking-performantie}
Performantie wordt opgesplitst in twee verschillende factoren: downloadtijd en gebruikerservaring.
Het is noodzakelijk dat een applicatie zowel snel wordt gedownload als vlot is in gebruikerservaring.
De ISO 25010-standaard gebruikt voor de eerstgenoemde de categorie prestatie-efficiëntie om de snelheid en gebruikte middelen te beoordelen.

De downloadtijd meet hoelang het duurt om de webapplicatie te downloaden.
De gebruikerservaring meet hoe vlot het gaat om door een lange lijst van 850 elementen te scrollen.
Zoals in sectie~\ref{sec:vergelijking-productiviteit} werd verteld, wordt de loginapplicatie als stresstest gebruikt om de performantie te testen.
De volledige POC kon niet in ieder raamwerk worden geïmplementeerd omdat de raamwerken niet alle kenmerken konden aanbieden.
Dit is in tegenstelling tot de loginapplicatie die wel in de vier raamwerken kan worden geïmplementeerd.
Hierdoor zal de POC niet worden gebruikt in dit criterium.
De downloadtijden en gebruikerservaring zullen op acht verschillende apparaten worden opgemeten.
Dit zijn dezelfde apparaten als bij het ondersteuningscriterium (zie tabel \ref{tabel:toestellen-hci}).

\subsubsection{Gemiddelde downloadtijd}
Bij de downloadtijden onderscheiden zich twee gevallen die samen de totale downloadtijd bepalen.
Eerste zal de  downloadtijd van de loginapplicatie worden bekeken~($\widehat{l}_{r,c,login}$). 
Vervolgens zal de tijd worden opgemeten om de loginapplicatie uit het cachegeheugen te downloaden~($\widehat{l}_{r,c,login_{cache}}$).
Deze downloadtijden zullen voldoende keren per apparaat moeten worden uitgevoerd om een betrouwbare meting te bekomen.

Het opmeten van de downloadtijden zal met TCPdump~\cite{Tcpdump2010} gebeuren, zoals werd voorgesteld door Thair~\cite{Thair2011}.
Hiervoor wordt een laptop als hotspot ingesteld en zullen de acht apparaten op deze hotspot connecteren via WiFi.
Wanneer de meting wordt gestart, zal op het apparaat naar de applicatie gesurft worden.
Nadat alle bestanden zijn ingeladen wordt de meting beëindigd. 
De uitvoer van TCPdump is een PCAP-bestand die de HTTP-trafiek bevat.
Deze zal via PCAP Web Performance Analyzer~\cite{SongL.bmcquadeMdsteele2010} worden omgezet naar een HAR-file, waarna een HTTP-waterval zal worden getoond.
Hieruit kan de totale downloadtijd worden gehaald van de gedownloade bestanden voor die applicatie.

De gemiddelde downloadtijd voor een raamwerk wordt bepaald door de som van de gemiddelde downloadtijden per apparaat:
\begin{equation}
  \text{Gemiddelde downloadtijd}_r= \frac{\sum\limits_{c=1}^{8}{\left(\widehat{l}_{r,c,login}+\widehat{l}_{r,c,login_{cache}}\right)}}{8}
    \label{eq:totale-downloadtijd}
\end{equation}
voor een raamwerk $r$ en context $c$.

\subsubsection{Gebruikerservaring}
Eerst werd voorgesteld om de rendertijd ($\left(\widehat{l}_{r,c,lijst}\right)$) te bepalen in plaats van de gebruikerservaring.
De rendertijd is de tijd die het raamwerk nodig heeft om de GI-elementen te renderen.
Hiervoor wordt een lijst van $850$ elementen gebruikt die getoond wordt na aanmelden op de loginapplicatie.
De tijd die het raamwerk nodig heeft om de lijst de renderen kan gemeten worden met \js-code.
Net zoals de downloadtijden zullen deze voldoende keren per apparaat moeten worden uitgevoerd.

De gemiddelde rendertijd is:
\begin{equation}
 \text{Gemiddelde rendertijd}_r= \frac{\sum\limits_{c=1}^{8}{\left(\widehat{l}_{r,c,lijst}\right)}}{8}
 \label{eq:totale-gebruikerservaring}
\end{equation}
voor een raamwerk $r$ en context $c$.

De rendertijd kon via \js{} enkel worden opgemeten in \jqm{} en \kendo{}.
Bij de twee andere raamwerken werden de betreffende gebeurtenissen niet gevonden om correct de tijd op te meten.
Doordat er maar data voor twee raamwerken voor handen was, werd de rendertijd vervangen door de gebruikerservaring van een lijst.
Deze bestaat eruit de vlotheid van het scrollen door de lijst van 850 lijstelementen voor de vier raamwerken op de acht apparaten te vergelijken.
Per apparaat wordt een score van 1, 2, 3 of 4 uitgedeeld aan de raamwerken.
Hierbij is 4 de beste score wat overeenkomt met het vlotste scrollen door de lijst relatief ten opzichte van de drie andere raamwerken.
Deze test werd uitgevoerd door twee personen.

Om de score voor gebruikerservaring van een raamwerk te bepalen worden de scores voor dat raamwerk op ieder apparaat opgeteld. De formule voor gebruikerservaring voor een raamwerk $r$ wordt:
\begin{equation}
  \text{Gebruikerservaring}_r = \sum_{c=1}^{8}{\text{ervaring}_{r,c}}
  \label{eq:performantie-gebruikservaring}
\end{equation}

In het bekomen eindklassement komt de hoogste totaalscore overeen met het raamwerk dat de vlotste scrolervaring aanbiedt. 

\subsubsection{Totaal}
De performantie wordt bepaald door de gemiddelde downloadtijd en de gebruikerservaring.
De opzet van de formule is om een raamwerk dat slecht scoort op de gemiddelde downloadtijd, maar sterk scoort op gebruikerservaring, een middelmatige score te geven.
Aangezien deze laatste geen eenheid heeft en de eerstgenoemde uitgedrukt wordt in seconden, wordt de gemiddelde downloadtijd gedeeld door de gebruikerservaring. De nieuwe formule voor de score voor de performantie wordt:
\begin{equation}
  \text{Performantie}_r = \frac{\text{Gemiddelde downloadtijd}_r}{\text{Gebruikerservaring}_r}
  \label{eq:performantie-enhanced}
\end{equation}
van een raamwerk $r$. 

%De formule voor het performantiecriterium wordt dan:
%\begin{equation}
%  \text{Performantie}_r= \text{Gemiddelde downloadtijd}_r + \text{Gemiddelde rendertijd}_r
%  \label{eq:performantie}
%\end{equation}
%voor een raamwerk $r$.

De maximale responsetijd is wanneer de loginapplicatie niet uit cache wordt geladen:
\begin{equation}
  \text{Maximale reponsetijd}_r= \frac{\sum\limits_{c=1}^{8}\left(\widehat{l}_{r,c,login} + \widehat{l}_{r,c,lijst}\right)}{8}
  \label{eq:performantie-max}
\end{equation}

Deze maximale responsetijd kan gecategoriseerd worden met limieten uitgedrukt in seconden zoals opgelegd door Jakob Nielsen~\cite{Nielsen1993}:  
\begin{itemize}
\item $\text{Maximale responsetijd}_r < 0.1\unit{s}$: de gebruiker heeft het gevoel dat het systeem direct reageert.
\item $\text{Maximale responsetijd}_r < 1\unit{s}$: de gedachtengang van de gebruiker zal niet worden onderbroken, maar hij zal toch een vertraging waarnemen.
\item $\text{Maximale responsetijd}_r < 10\unit{s}$: de limiet om de aandacht van de gebruiker te behouden.
\end{itemize}

De maximale responsetijd kan echter niet worden berekend omdat er geen rendertijden bij \st{} en \lungo{} kunnen worden berekend.
Indien de rendertijd uit de formule wordt weggelaten,  blijft de maximale responsetijd een schatting voor de limiet van Nielsen.

Om de scores van het performantiecriterium te staven zal de downloadgrootte van de loginapplicatie worden bekeken.
Daarnaast zal ook de gemiddelde downloadtijd van de POC en de loginapplicatie met elkaar worden vergeleken.
Ook zullen de resultaten gecontroleerd worden met Google Page Speed~\cite{Morgan2011}. 
Deze tool kan de code van een webpagina analyseren en de performantie testen specifiek voor mobiele apparaten.
Het resultaat is een score op 100 en een lijst van werkpunten om de performantie van de applicatie te verbeteren.
Een hoge score duidt op weinig plaats voor verbetering,  een lagere score duidt op meer plaats voor verbetering.
Google Page Speed meet niet de tijd om een pagina te laden.

%%%%%%%%%%%%%%%%%%%%%%%%%%%%%%%%%%%%%%%%%%%%%%%%%%%%%%%%%%%%%%%%%%
%%%%%%%%%%%%%%%%%%%%%%%%%%%%%%%%%%%%%%%%%%%%%%%%%%%%%%%%%%%%%%%%%%

\section{Vergelijkingsoverzicht}
\label{sec:vergelijking-spinnenweb}

Om de scores van de vijf criteria samen te vatten zal een spinnenweb worden gebruikt.
Hierdoor moet elke score op dezelfde schaal worden gebracht om duidelijk de verschillen te kunnen waarnemen.
De Matlab-extensie om spinnenwebben te genereren, vereist dit ook~\cite{Martti2007}.
Er werd gekozen om alle scores te relativeren.
Hiervoor moet elke score van een criterium gedeeld worden door het maximaal behaalde resultaat van dat criterium.
Alle scores zullen vervolgens tussen $0$ en $1$ liggen.
Deze methode zal ervoor zorgen dat het raamwerk met de beste score een $1$ behaalt.

Om verwarring te voorkomen, moeten ook de scores voor het productiviteitscriterium en performantiecriterium geïnverteerd worden.
Dit komt omdat voor deze criteria geldt:  hoe lager de score,  hoe beter het raamwerk.

De formules om de relatieve scores te bereken worden hieronder weergegeven.
De relatieve scores zullen gebruikt worden om het spinnenweb op te stellen.

\begin{equation}
  \text{Populariteit}_r^{\pentagon}=\frac{\text{Populariteit}_r}{\underset{m}{\max}\{\text{Populariteit}_m\}}
  \label{eq:rel-populariteit}
\end{equation}

\begin{equation}
  \text{Productiviteit}_r^{\pentagon} = \frac{\text{Productiviteit}_r^{-1}}{\underset{m}{\max}\{\text{Productiviteit}_m^{-1}\}}
  \label{eq:rel-productiviteit}
\end{equation}

\begin{equation}
  \text{Gebruik}_r^{\pentagon} = \frac{\text{Gebruik}_r}{\underset{m}{\max}\{\text{Gebruik}_m\}}
  \label{eq:rel-gebruik}
\end{equation}

\begin{equation}
  \text{Ondersteuning}_r^{\pentagon} = \frac{\text{Ondersteuning}_r}{\underset{m}{\max}\{\text{Ondersteuning}_m\}}
  \label{eq:rel-ondersteuning}
\end{equation}

\begin{equation}
  \text{Performantie}_r^{\pentagon}= \frac{\text{Performantie}_r^{-1}}{\underset{m}{\max}\{\text{Performantie}_m^{-1}\}}
  \label{eq:rel-performantie}
\end{equation}

\begin{equation}
\begin{split}
  \text{Score}_r &= \frac{1}{5} \left( \text{Populariteit}_r^{\pentagon}
  + \text{Productiviteit}_r^{\pentagon} 
  + \text{Gebruik}_r^{\pentagon} \right. \\
  &+ \left. \text{Ondersteuning}_r^{\pentagon}
  + \text{Performantie}_r^{\pentagon} \right)
  \end{split}
  \label{eq:rel-totaal}
\end{equation}

\chapter{Evaluatie}
\label{chap:evaluatie}

In dit hoofdstuk wordt de vergelijking uitgevoerd op basis van de vijf vergelijkingscriteria uit hoofdstuk \ref{chap:vergelijkingscriteria}, namelijk populariteit~(\ref{sec:evaluatie-populariteit}), productiviteit~(\ref{sec:evaluatie-productiviteit}), gebruik~(\ref{sec:evaluatie-gebruik}), ondersteuning~(\ref{sec:evaluatie-ondersteuning}) en performantie~(\ref{sec:evaluatie-performantie}). 
Daarna zullen deze vijf vergelijkingscriteria in sectie~\ref{sec:evaluatie-spinnenweb} worden samengevat.

%%%%%%%%%%%%%%%%%%%%%%%%%%%%%%%%%%%%%%%%%%%%%%%%%%%%%%%%%%%%%%%%%%%%%%%%

\section{Populariteit} % 2 blz inclusief google trends
\label{sec:evaluatie-populariteit}

De populariteit van de vier raamwerken op 8 mei 2013 wordt samengevat in tabel~\ref{tabel:evaluatie-populariteit}. 

\begin{table}[H]
\centering
\pgfplotstabletypeset[
  begin table=\begin{tabular}{p{8cm} p{1cm} p{1cm} p{1cm} p{1cm}},
  end table=\end{tabular},
  skip coltypes=true,
  col sep=comma,
  string type,
  header=true,
  columns={Populariteit,jQM,ST,Kendo,Lungo},
  columns/Populariteit/.style={column name=\textbf{Populariteit}, column type={l}},  
  columns/jQM/.style={column name=\textbf{\jqma}, column type={c}},
  columns/ST/.style={column name=\textbf{\sta}, column type={c}},
  columns/Lungo/.style={column name=\textbf{\lungoa}, column type={c}},
  columns/Kendo/.style={column name=\textbf{\kendoa}, column type={c}},
  every head row/.style={
    before row=\toprule,
    after row=\midrule},
  every last row/.style={
  	before row=\midrule,
    after row=\bottomrule}
]{tabellen/populariteit.csv}
\caption{Overzicht van populariteit op 8 mei 2013 voor \st{}~(\sta), \kendo{}~(\kendoa), \jqm{}~(\jqma) en \lungo{}~(\lungoa).}
\label{tabel:evaluatie-populariteit}
\end{table}

\kendo{} neemt de eerste plaats voor zich dankzij het zeer groot aantal vind-ik-leuks op \fb.
\jqm{} en \st{} slepen respectievelijk een tweede en derde plaats in de wacht, ondanks het feit dat ze in de literatuur de meest aangehaalde raamwerken zijn~\cite{David2011,Firtman2013,Hales2012,Oeflman2011}. 
Als laatste eindigt \lungo{} met een opmerkelijke lage populariteit op \so{} en \fb.
Bij het kijken naar de totaalscore kunnen twee groepen worden waargenomen, enerzijds de groep bestaande uit \kendo{} en \jqm{} en anderzijds de groep bestaande uit \st{} en \lungo{}.

Op Twitter heeft \jqm{} de meeste volgers, gevolgd door \kendo.
Op de voorlaatste plaats komt \lungo{}, maar als het aantal \term{tweets} wordt uitgezet ten opzichte van het aantal volgers, kan er gesteld worden dat \lungo{} het meest actief is.
\jqm{} en \kendo{} hebben een vergelijkbare activiteit bij het sturen van \term{tweets}.
\st{} heeft het minst aantal volgers en het aantal verstuurde \term{tweets} is slechts 1.

%TODO: als het open source is, zal het populairder zijn ??! verschil jQM/lungo <-> ST/Kendo door Github

In tegenstelling tot \jqm{} en \lungo{} bevinden \kendo{} en \st{} zich niet op \gh{}.
Zelfs indien \gh{} wordt weggelaten, blijft de rangschikking ongewijzigd.

\kendo{} verwijst op zijn website voor ondersteuning rechtstreeks naar de fora op \so{}. 
Toch blijft de populariteit van \kendo{} op \so{} lager dan die van \jqm{}.
\st{} behaalt de voorlaatste plaats, maar verbazender is \lungo{} die slechts een dertigtal vragen op \so{} heeft en dus op de laatste plaats eindigt.

\kendo{} en \jqm{} hebben beide een fanpgina op \fb{} opgericht in respectievelijk november 2011 en augustus 2010.
De fanpgina van \kendo{} heeft dus in een kortere tijd veel meer vind-ik-leuks opgeleverd dat de eerder opgerichte fanpgina van \jqm{}.
De verschillende producten van \kendo{} worden geaggregeerd op één fanpagina. 
\st{} en \lungo{} hebben enkel een interessepagina op \fb.
Dit verklaart het grote verschil in vind-ik-leuks op \fb.

Deze populariteit werd ook iedere week bijgehouden over een periode van een kleine twee maand en kan gevonden worden op figuur~\ref{fig:populariteit-evolutie}.
Opvallend is de sterke opmars van \kendo{} in deze korte periode.
Dit komt grotendeels door het enorm stijgend aantal \fb{} vind-ik-leuks.
De andere drie raamwerken stijgen gestaag.

%TODO vectorieel maken
\begin{figure}
  \centering
  \includegraphics[width=\textwidth]{figuren/populariteit.png}
  \caption{Populariteit waargenomen over de periode van 15 april tot 22 mei 2013.}
  \label{fig:populariteit-evolutie}
\end{figure}

Als laatste wordt de populariteit aan de hand van Google Trends bekeken op figuur~\ref{fig:google-trends}.
Duidelijk is dat hier \jqm{} de grote winnaar is.
Sinds 2011 maakt het raamwerk een grote opmars door het uitbrengen van de eerste stabiele versie~1.0.
Eind 2012 kende \jqm{} echter een serieuze daling, maar deze werd terug een stijging omgezet door het uitbrengen van versie~1.3. 
\st{} kende een piek in maart 2012 bij het uitbrengen van \st{}~2.0.
Sinds begin 2012 maakt \kendo{} een opmars en als de trend zich verder zet, zal het \st{} inhalen.
Dit komt overeen met de waargenomen opmars van \kendo{} op figuur~\ref{fig:populariteit-evolutie}.
\lungo{} is nauwelijks op de grafiek waarneembaar.

\begin{figure}[H]
  \centering
  \includegraphics[width=\textwidth]{figuren/google-trends.pdf}
  \caption{Populariteit op Google Trends waargenomen van januari 2008 tot heden waarbij een resultaat van 100 overeenkomt met de grootste zoekinteresse~\cite{Google2012a}}
  \label{fig:google-trends}
\end{figure}
\section{Productiviteit}
\label{sec:evaluatie-productiviteit}
% De productiviteit van de vier raamwerken wordt samengevat in tabel
% Daarna zullen de geïmplementeerde POC's~(zie \ref{sec:evaluatie-productiviteit-poc}) en loginschermen~(zie \ref{sec:evaluatie-productiviteit-login}) per raamwerk uitvoerig worden besproken.

In deze sectie zal de productiviteit van de vier raamwerken worden onderzocht.
Voor de score van productiviteit wordt naar formule \ref{eq:productiviteit} verwezen.
Tabel~\ref{tabel:evaluatie-productiviteit} bevat een overzicht van de behaalde scores.

\begin{table}[H]
\centering
\pgfplotstabletypeset[
  begin table=\begin{tabular}{p{8cm} p{1cm} p{1cm} p{1cm} p{1cm}},
  end table=\end{tabular},
  skip coltypes=true,
  col sep=comma,
  string type,
  header=true,
  columns={Productiviteit,ST,Kendo,jQM,Lungo},
  columns/Productiviteit/.style={column name=\textbf{Productiviteit}, column type={l}},  
  columns/ST/.style={column name=\textbf{\sta}, column type={c}},
  columns/ST/.style={column name=\textbf{\sta}, column type={c}},
  columns/jQM/.style={column name=\textbf{\jqma}, column type={c}},
  columns/Lungo/.style={column name=\textbf{\lungoa}, column type={c}},
  columns/Kendo/.style={column name=\textbf{\kendoa}, column type={c}},
  every head row/.style={
    before row=\toprule,
    after row=\midrule},
  every last row/.style={
  	before row=\midrule,
    after row=\bottomrule}
]{tabellen/productiviteit.csv}
\caption{Overzicht van productiviteit voor \st{}~(\sta), \kendo{}~(\kendoa), \jqm{}~(\jqma) en \lungo{}~(\lungoa).}
\label{tabel:evaluatie-productiviteit}
\end{table}

\lungo{} is de afgetekende winnaar,  gevold door \kendo{} en \jqm{}. 
\st{} blijkt het minst productief te zijn.

%TODO: gewoon zeggen wat er fout is aan de gegevens
% 5 redenen waarom de tijden van POC onbruikbaar zijn
% - lungo niet alles geïmplementeerd
% - backend problemen, niet goed bijgehouden op Toggl
% - ervaring met de POC
% - kopiëren van code uit vorige implementatie (werd niet gedaan bij de loginschermen)
% - ervaring van programmeren met HTML5 raamwerken in het algemeen 

% indien enkel het loginscherm wordt gebruikt als productiviteit, dan treden deze problemen niet op. (de 5 vorige redeneren ontkrachten)
% nieuwe formule introduceren met nummering 

De belangrijkste factor waarom \lungo{} en \kendo{} beter scoren is omdat beide raamwerken als tweede werden behandeld.
De POC werd eerst in \jqm{} en \st{} geïmplementeerd.
Hierdoor waren de programmeurs beter vertrouwd met de POC en HTML5-raamwerken in het algemeen.
Bij de eerste implementatie kwamen er ook problemen met de \term{backend} naar boven.
Deze waren bij de tweede implementatie reeds opgelost.
Een schatting van de tijd die aan de \term{backend} werd besteed, situeert zich tussen $10$ en $25$ uur. %TODO Sander: dit vermelden? Tim: zeker, maar ik vind die schatting wel te breed van interval
De vertrouwdheid met HTML5-raamwerken weerspiegeld zich vooral tussen \jqm{} en \lungo{}.
Hoewel ze beide op een verschillende \js{}-bibliotheek steunen - respectievelijk jQuery en QuoJS - zijn de gelijkenissen tussen deze twee raamwerken niet te ontkennen.
%TODO Tim: het aantal uren daalt, maar de productiviteit stijgt wel hé ;-)
De productiviteit van \lungo{} daalde met $71\%$ ten opzichte van \jqm{}.
Belangrijk is ook dat niet de volledige POC met \lungo{} kon worden ontwikkeld.
Hierop zal in sectie \ref{sec:evaluatie-gebruik} verder worden ingegaan.
De verschillen tussen \st{} en \kendo{} zijn groter want ze bieden een ander type - respectievelijk \js-gedreven en zowel \js- als opmaakgedreven - en hanteren een andere architectuur - respectievelijk MVC en MVVM.
De productiviteit van \kendo{} daalde met $44\%$ ten opzichte van \st{}.

De werkuren van de loginapplicatie bevestigen voorgaande resulaten niet.
Alle functionaliteit van de loginapplictie kon met de vier raamwerken worden gebouwd.
Uit deze werkuren blijkt \jqm{} productiever dan \lungo{}. 
De volgorde van \kendo{} en \st{} blijft echter wel behouden.
De relatieve verschillen tussen \jqm{} en \lungo{} en tussen \st{} en \kendo{} zijn ook verkleind.  
Respectievelijk een stijging van $11.7\%$ en een daling met $31.8\%$.
Toch is er een trend vast te stellen:  opmaakgedreven raamwerken blijken productiever dan \js-gedreven raamwerken.

In wat volgt zullen andere factoren worden besproken die de verschillen in productiviteit kunnen verklaren.

\subsection{Tools}
%TODO Tim: onderstaande zin is wat ongelukkig, natuurlijk kan ik geen jQM app maken met Sencha Architect ;-)
Enkel bij de ontwikkeling met \st{} kon beroep worden gedaan op Sencha Architect om het ontwikkelingsproces te vergemakkelijken.
%TODO Tim: misschien de ondersteuning voor welke OS'es
Dit is een desktopapplicatie die het ontwikkelingsproces vergemakkelijkt met een GGI en \term{drag-and-drop} commando's.  
Deze tool kan 30 dagen gratis gebruikt worden of kan worden aangekocht voor $\$399$.
Bij de ontwikkeling van de POC werd Sencha Architect versie 2.1 gebruikt voor een tijdsduur van $21$ dagen.
Het grootste voordeel van Sencha Architect kon bij de ontwikkeling van \code{Views} worden gevonden.
De \code{Views} kunnen specifiek voor mobiele schermen worden geoptimaliseerd.
Dit zowel voor staande als liggende apparaten.
Na $21$ dagen werd de PhpStorm als IDE gebruikt bij de ontwikkeling van \st{}.
Dezelfde IDE werd ook gebruikt bij de ontwikkeling met andere raamwerken en biedt ondersteuning voor zowel Windows, Mac als Linux.
Een voordeel bij het gebruik van deze IDE is onder andere de automatische codeaanvulling.
Daarnaast toont de IDE hints voor het optimaliseren van \js{}-code die specifiek gebruikt maakt van de jQuery-bibliotheek.

\subsection{Boilerplate code}
Het initialiseren van een nieuwe applicatie van het raamwerk beïnvloed ook sterk de productiviteit.
\st{} biedt hiervoor terug een tool,  Sencha Cmd,  die deze functionaliteit mogelijk maakt~\cite{Sencha2012}.
Deze tool kan de initiële applicaties opzetten,  bestanden toevoegen en de applicatie bouwen en uitrollen.
Ondanks dat de initialisatie van een nieuwe project kan worden geautomatiseerd, zijn de bekomen bestanden van het project niet zo duidelijk.
Een nieuwe \code{Controller},  \code{Store},  \code{Model} of \code{View} genereren kan automatisch met Sencha Cmd of manueel.
Bij de manuele aanpak moet het nieuwe \js-bestand in de juiste folder worden ondergebracht.
\st{} legt een strenge structuur van folders op en de applicatie zal enkel werken als aan deze structuur wordt vastgehouden.
De ervaringen van de auteurs met \st{} zegt dat deze structuur tot veel verwarring leidt.

\jqm{} en \kendo{} beschrijven boilerplate code bij hun documentatie~\cite{JQuery2012b,Telerikd}.
Ook is bij de documentatie van beide raamwerken een expliciete sectie aanwezig die de programmeur helpt om een applicatie op te zetten.
Bij \lungo{} is dit niet het geval.  
Om een \lungo{} applicatie op te zetten moest naar de broncode van de voorbeelden op de documentatie gekeken worden.
% leercurve: 
% -ST lag hoger door MVC en pure JS (zie loginscherm steekt er met 7 uur met de kop bovenuit)
% -jQM en Kendo hadden boiler plate code, ST had tool project te maken maar dan… allemaal files en folders, je weet zo niet direct waar gestart, vanaf je iets verkeerd veranderd wordt er niets meer gerenderd. 
% gestart met jQM en ST, waardoor productiviteit van de andere 2 hoger ligt (incl problemen met backend, zeker 20u)
% -Lungo had geen boilerplate code, naar eerste example de broncode kopiëren. quojs zeer gelijkaardig aan jquery syntax. door de opgedane kennis van jquery verlaagt het de curve sterk, anders is het praktisch ondoenbaar (documentatie quojs is 1 bladzijde met code en kernwoorden)

\subsection{Documentatie}
Ook de kwaliteit en kwantiteit van de documentatie kunnen de scores van de productiviteit verklaren.
De \st{} documentatie is het grootst in vergelijking met de andere raamwerken.
De grootte van de documentatie maakt het moeilijk om zelf onduidelijkheden gericht te zoeken.
De zoekfunctie met auto-aanvulling is noodzakelijk om de juiste documentatie terug te vinden.
De documentatie van \kendo{} is overzichtelijker en nog steeds volledig.
De combinatie van de secties API en Getting Started waren de voornaamste drijfveer.
Ook worden alle kenmerken die het raamwerk aanbiedt met demo's en codevoorbeelden getoond.
De zoekfunctie van de documentatie is niet optimaal en hierdoor ook maar weinig gebruikt.
De documentatie van \jqm{}~1.2 bevat geen zoekfunctie en codevoorbeelden.
Hierdoor moet naar de broncode worden gekeken om de code van een kenmerk te begrijpen.
Een voordeel van de \jqm{} documentatie is dat de documentatie zelf met \jqm{} is gebouwd.
Een belangrijke opmerking is dat de \jqm{}~1.3 wel een zoekfunctie en codevoorbeelden in zijn documentatie heeft opgenomen.
De documentatie van \lungo{} is zeer beknopt.
Dezelfde opmerking kan worden gemaakt bij de documentatie van \quo.
Alle kenmerken van \lungo{} worden met codevoorbeelden verduidelijkt.
Sommige voorbeelden zijn echter incorrect.
De documentatie is opgedeeld in Prototype en \js-API.
De eerste was voornamelijk handig om de loginapplicatie te ontwikkelen. %TODO Sander:  geldt deze opmerking ook voor u Tim?

% documentatie: 
% -ST: demo (Kitchen snik app), zoekfunctie is goed (met autocomplete), maar wel veel meer documentatie dan de andere, waardoor zonder zoekfunctionaliteit moeilijk is
% -Kendo: interactief te leren, heel overzichtelijk, veel demo's (music store),, onoverzichtelijke zoekfunctie
% -jQM: 1.2 geen zoekfunctie, geen echte code, je moet kijken naar de broncode, documentatie zelf is geschreven met het framework. vanaf 1.3 wel zoekfunctie en codevoorbeelden
% -Lungo: prototype heel goed, maar dan de rest zeer zeer beknopt. geen zoekfunctie. soms zijn is de voorbeeldcode niet correct. 

\subsection{Debugging}
Het zoeken naar bugs in de code verliep bij elk raamwerk gelijkaardig.
De applicatie werd lokaal uitgerold en in de Web Inspector van Chrome gedebugd.
Debuggen op de apparaten kon door deze te connecteren via USB met de computer.
Android toestellen met een Chrome browser kunnen debugt worden met de Web Inspector.
Debuggen op iOS toestellen kan op een Mac met dezelfde Web Inspector maar in de Safari browser. %TODO Sander: wat met xcode?
% debugging:
% - ST: beige auteurs, dezelfde aanpak, alles in commentaar zetten tot iets werkende en dan beetje bij beetje uit commentaar halen
% - ST: debug voor extra info (niet hetzelfde als jqm.min en jqm)
% - altijd debuggen in chrome met console, vanaf het daar volledig werkt, pas debuggen op de devices zelf: chrome: connecteren via usb debugging, safari via xcode

\subsection{Literatuur}
Zowel \jqm{} als \st{} werden vaak in de literatuur aangehaald.
Op Safari Books Online kunnen 13 boeken teruggevonden worden die volledig op \jqm{} zijn toegespitst.
Het boek van Dutson in de Sam Teach Yourself serie werd gebruikt om met \jqm{} vertrouwd te geraken~\cite{PhilDutson2012}.
%TODO: hadden wij dat ander ook niet doorgelezen van Pro jQM ouzo?
Vier boeken over \st{} kunnen op Safari Books Online worden teruggevonden waarbij het werk van Clark werd gebruikt~\cite{JohnEClark2012}.
\kendo{} komt slecht in een boek van voor~\cite{Bhandari2013},  \lungo{} helemaal niet.
Het boek rond \kendo{} is slechts een proefdruk en wordt pas in augustus 2013 officieel geplubliceerd.
Dit werk werd niet gebruikt.
%TODO: Lungo heeft nog een literatuur


\subsection{Vragen}
%TODO Sander: activiteit van de fora van de raamwerken bespreken?  Kendo:  issues op fora met fiddle verklaard
Wanneer de programmeur met problemen van het raamwerk werd geconfronteerd, moest op het web naar oplossingen gezocht worden.
In sectie \ref{sec:evaluatie-populariteit} werden reeds het aantal vragen van het raamwerk op Stack Overflow bekeken.
Hoe groter dit aantal,  hoe groter de kans dat een probleem reeds op Stack Overflow is aangehaald.
Bij \lungo{} is dit aantal minimaal en was de programmeur vaak aangewezen om zelf oplossingen van zijn problemen te zoeken.
% vragen:
% - ST, jQM, Kendo vaak antwoord op stack overflow (zie populariteit)
% - Lungo: je bent op jezelf gewezen
%Sander












% Uit de cijfers blijft Lungo  meest productief, maar door de voorgaande kennis van jquery, POC en backend + het feit dat er bepaalde dingen niet geïmplementeerd konden worden in lungo, moet dit met een korreltje zout worden genomen (het loginscherm is hiervan een indicatie)


%%%%%%%%%%%%%%%%%
% 
% \subsection{POC}
% \label{sec:evaluatie-productiviteit-poc}
% 
% \paragraph{\jqm}
% 
% \paragraph{\st}
% 
% \paragraph{\kendo}
% 
% \paragraph{\lungo}
% 
% %%%%%%%%%%%%%%%%%
% 
% \subsection{Loginscherm}
% \label{sec:evaluatie-productiviteit-login}
% 
% \paragraph{\jqm}
% 
% \paragraph{\st}
% Eerst werden Getting Started en Building Your First App gevolgd, wat tutorials van \st{} zelf zijn.
% Hiermee worden de initiële configuraties bekomen voor een werkende applicatie.
% Nadelen aan deze twee tutorials waren vele inconsistenties en soms foute links.
% 
% Daarna werd er gericht gezocht hoe een koptekst en formulier konden worden gemaakt.
% Hiervoor werd terug gebruik gemaakt van de documentatiesite van \st.
% Ook hierop werd beroep gedaan om een model voor de gebruiker te maken met de nodige validaties.
% De verzendknop voor het formulier werd gevonden in een antwoord op een StackOverflow-vraag.
% 
% \paragraph{\kendo}
% Er werd gestart met een inleidende pagina op de documentatiesite van \kendo{} zelf.
% Hiermee werd de minimale code bekomen voor een werkende applicatie.

\section{Gebruik}
\label{sec:evaluatie-gebruik}
Het gebruik van de vier raamwerken wordt samengevat voor de 13 uitdagingen in tabel \ref{tabel:evaluatie-gebruik}.
Per sectie zal iedere uitdaging per raamwerk worden besproken.

\begin{table}[H]
\centering
\pgfplotstabletypeset[
  begin table=\begin{tabular}{p{8cm} p{0.8cm} p{0.8cm} p{0.8cm} p{0.8cm} p{0.3cm}},
  end table=\end{tabular},
  skip coltypes=true,
  col sep=comma,
  string type,
  header=true,
  columns={Uitdaging,Max,ST(abs),Kendo(abs),jQM(abs),Lungo(abs)},
  columns/Uitdaging/.style={column name=\textbf{Uitdaging}, column type={l}},
  columns/Max/.style={column name=\textbf{Max}, column type={l}},    
  columns/jQM(abs)/.style={column name=\textbf{\jqma}, column type={c}},
  columns/ST(abs)/.style={column name=\textbf{\sta}, column type={c}},
  columns/Lungo(abs)/.style={column name=\textbf{\lungoa}, column type={c}},
  columns/Kendo(abs)/.style={column name=\textbf{\kendoa}, column type={c}},
  every head row/.style={
    before row=\toprule,
    after row=\midrule},
  every last row/.style={
  	before row=\midrule,
    after row=\bottomrule}
]{tabellen/gebruik.csv}
\caption{Overzicht van gebruik voor \st{}~(\sta), \kendo{}~(\kendoa), \jqm{}~(\jqma) en \lungo{}~(\lungoa).}
\label{tabel:evaluatie-gebruik}
\end{table}

Het raamwerk dat het beste scoort bij gebruik is \kendo{}, kort gevolgd door \st{}.
Dit kan grotendeels door de aanwezigheid van een architectuur, respectievelijk MVVM en MVC, worden verklaard.
Het ontbreken van een architectuur bij de andere twee raamwerken resulteert in een omslachtige aanpak, waardoor punten worden verloren.
Daarnaast worden beide raamwerken beheerd door een bedrijf en aangezien de POC een typische bedrijfsapplicatie is, kan mede de goede score worden verklaard.
Opmerkelijk is de perfecte score van \kendo{} voor formulieren, maar de mindere ondersteuning voor offline dan \st{}.
\jqm{} behaalt de helft met wat overschot, maar goed is dat zeker niet.
Vooral op het vlak om data automatisch in te laden in velden of lijsten, scoort het nul door het ontbreken van een architectuur. 
Als laatste komt \lungo{} dat een onvoldoende behaalt.
Dezelfde pijnpunten van \jqm{} zijn ook geldig voor \lungo{}.
Daarenboven is er bij \lungo{} een totaal gebrek aan formuliervalidatie en aan de meer geavanceerdere formulierelementen.
Voor deze laatste moet er het geluk zijn om plug-in te vinden of zal de functionaliteit zelf moeten worden geïmplementeerd. 

\kendo{} en \st{} behalen elk zes perfecte uitdagingen.
\kendo{} scoort twee maal onvoldoende, terwijl \st{} dat slechts één keer doet.
\jqm{} heeft drie perfecte uitdagingen en vier onvoldoendes.
Als laatste komt \lungo{} met slechts één perfecte uitdaging en zeven onvoldoendes.

Een algemene trend bij iedere raamwerk is de volle ondersteuning van AJAX-oproepen, behalve dan bij één geval voor \lungo{}.
Ook laadschermen en dialoogvensters zijn bij ieder raamwerk volledig aanwezig.


%%%%%%%%%%%%%

\subsection{\uit{anatomie}}
\label{sec:evaluatie-gebruik-anatomie}

In tabel \ref{tabel:evaluatie-gebruik-anatomie} worden de resultaten getoond van de drie deeluitdagingen van \uit{anatomie}.
Onder de tabel wordt per raamwerk verklaard waarom dat resultaat werd behaald.

\begin{table}[H]
\centering
\pgfplotstabletypeset[
  begin table=\begin{tabular}{p{8cm} p{0.8cm} p{0.8cm} p{0.8cm} p{0.8cm} p{0.3cm}},
  end table=\end{tabular},
  skip coltypes=true,
  col sep=comma,
  string type,
  header=true,
  columns={Uitdaging,Max,ST(abs),Kendo(abs),jQM(abs),Lungo(abs)},
  columns/Uitdaging/.style={column name=\textbf{Uitdaging}, column type={l}},
  columns/Max/.style={column name=\textbf{Max}, column type={l}},    
  columns/jQM(abs)/.style={column name=\textbf{\jqma}, column type={c}},
  columns/ST(abs)/.style={column name=\textbf{\sta}, column type={c}},
  columns/Lungo(abs)/.style={column name=\textbf{\lungoa}, column type={c}},
  columns/Kendo(abs)/.style={column name=\textbf{\kendoa}, column type={c}},
  every head row/.style={
    before row=\toprule,
    after row=\midrule},
  every last row/.style={
  	before row=\midrule,
    after row=\bottomrule}
]{tabellen/gebruik/anatomie.csv}
\caption{Gebruik van \uit{anatomie} voor \st{}~(\sta), \kendo{}~(\kendoa), \jqm{}~(\jqma) en \lungo{}~(\lungoa).}
\label{tabel:evaluatie-gebruik-anatomie}
\end{table}

\paragraph{\st}
Het opbouwen van een \code{view} in \st{} gebeurt hiërarchisch met containers.
Een component die aan een container kan worden toegevoegd is de \code{toolbar}.
Een \code{toolbar} kan zowel bovenaan als onderaan een container worden vastgezet.
Dit kan dan dienstdoen als hoofdtekst en voettekst.
Een cascade van hoofdteksten is door de hiërarchische opbouw van containers mogelijk en laat zo onderkopteksten toe.
Een \code{toolbar} kan voorzien worden van een titel door de \code{title} eigenschap in te vullen.
Knoppen aan een \code{toolbar} toevoegen kan door een lijst van \code{buttons} aan de \code{items} eigenschap toe te voegen.

Het toevoegen van een tabbar verloopt analoog door een \code{tabpanel} toe te voegen.
De \code{items} eigenschap bevat dan een lijst van \code{views} met een titel waarbij de \code{view} zichtbaar wordt als op de titel wordt gedrukt.

Knoppen van kleur veranderen kan door de \code{style} van een knop te zetten.
Deze eigenschap laat toe om CSS-eigenschappen als tekst of object aan de knop toe te voegen.

\paragraph{\kendo}
De anatomie van een \code{view} wordt met het data-attribuut \code{data-layout} bepaald.
Hier wordt de id van een HTML-fragment opgegeven waar het skelet van een \code{view} staat.
Hoofd- en/of voetteksten toevoegen kan door in het fragment de \code{header} of \code{footer} tags te gebruiken.
Een hoofdtekst kan een \code{navbar} hebben waarin knoppen kunnen worden gedefinieerd.
\code{Header}-tags kunnen niet genest worden, dus zijn onderhoofdteksten niet mogelijk in \kendo{}.
%TODO referentie

Een tabbar moet met een \code{buttongroup} worden gemaakt.
Dit is een lijst van knopppen.
Na deze lijst moeten de bijhorende \code{views} worden geschreven.
Het initialiseren van de lijst van knoppen moet via \code{\$("\#id").kendoMobileButtonGroup()}.
Het bijhorende \js-object bevat een \code{select} gebeurtenis die zich afspeeld wanneer een knop is geslecteerd.
Aan deze gebeurtenis moet een functie worden gekoppeld die alle \code{views} verbergt behalve diegene die hoort bij de geselecteerde knop.
Deze operaties steunen op de CSS-manipulatie zoals aangeboden door jQuery.

De kleur van een knop wijzigen moet met CSS gebeuren.
Omdat knoppen met HTML-tags worden aangemaakt kan de CSS-klasse rechtstreeks aan de tags worden toegevoegd.

\paragraph{\jqm}
Het toevoegen van een kop- en voettekst gebeurt door gebruik te maken van \code{data-role="header"} en \code{data-role="footer"}. 
Wel moest dezelfde code op ieder scherm worden herhaald. 
Dit kan worden vermeden door gebruik te maken van eenzelfde \code{data-id} attribuut. 
Daarnaast werd de voettekst gefixeerd aan de onderkant van het scherm en de bijhorende logo's links en rechts uitgelijnd. 
Voor dit laatste werd gebruik gemaakt van de zogenaamde \term{grid} die \jqm{}~1.3 zelf aanbiedt. 
De voettekst wordt niet getoond op een smartphone, wat wordt bekomen door gebruik te maken van de CSS3 media queries.

Voor de onderkoptekst werd eerst geprobeerd om bovenaan een lijst met enkel één lijstdeler te plaatsen, maar dan schoof de inhoud van de pagina niet mee naar onder. 
De uiteindelijke oplossing kwam vanuit de documentatie om dit met behulp van de CSS-klasse \code{ui-bar} te implementeren~\cite{JQuery2013b}. 
Deze extra titel wordt ook gebruikt op de smartphone om naar de speciale navigatie te gaan (zie \ref{sec:evaluatie-gebruik-toestel}).

Standaard is er een tabbalk aanwezig in jQuery Mobile, maar de POC impliceerde een tabbalk die niet de volledige breedte innam.
Daarom werd gekozen voor \code{fieldset} met twee opties.

De kleur van de knoppen aanpassen kan op twee manieren. 
Ofwel wordt dit gedaan met CSS-code ofwel wordt gebruik gemaakt van ThemeRoller~\cite{JQuery2012c}. 
Deze laatste manier werd gebruikt om de knoppen groen te maken. 
Men sleept dan eenvoudigweg in die webinterface de groene kleur op de knop en daarna kan de bijhorende CSS-code worden gedownload. 
Door daarna de knop te annoteren met het \code{data-theme}-attribuut wordt het betreffende thema geactiveerd. 
Om de knop blauw te maken was er geen nood aan een aanpassing, doordat blauw al één van de standaard thema's (\term{swatch} \code{b}) was en kon deze direct worden gebruikt.

\paragraph{\lungo}
Het tonen van kop- en voettekst gebeurt door de verschillende schermen van de applicatie te omvatten door \code{<article>}-tags en daarbinnen de \code{<header>}- en \code{<footer>}-tag te gebruiken.
Deze drie tags zijn nieuw in HTML5.
In de voettekst kunnen door CSS-regels de twee logo's links en rechts uitgelijnd worden.
De onderkoptekst werd met een omweg bekomen door een lijst te maken met slechts één lijstitem.

Het maken van een tabbar is standaard aanwezig in \lungo{}.
Deze wordt bekomen door een \code{<nav>}-tag te annoteren met de \code{groupbar}-klasse.
In de \code{<nav>}-tag worden de links naar de verschillende tabbladen gemaakt door middel van \code{<a>}-tags.
De tabbar wordt getoond over de volledige breedte en komt onmiddellijk onder de koptekst en dus boven de onderkoptekst.
Dit in tegenstelling tot het gevraagde in de POC waar de tabbar onder de onderkoptekst diende te komen.

Het veranderen van de kleur van knoppen gebeurt in CSS waarbij de achtergrondkleur van de knop eenvoudig kan worden aangepast.

%%%%%%%%%%%%%

\subsection{\uit{toestel}}
\label{sec:evaluatie-gebruik-toestel}

In tabel \ref{tabel:evaluatie-gebruik-toestel} worden de resultaten getoond van de drie deeluitdagingen van \uit{toestel}.
Onder de tabel wordt per raamwerk verklaard waarom dat resultaat werd behaald.

\begin{table}[H]
\centering
\pgfplotstabletypeset[
  begin table=\begin{tabular}{p{8cm} p{0.8cm} p{0.8cm} p{0.8cm} p{0.8cm} p{0.3cm}},
  end table=\end{tabular},
  skip coltypes=true,
  col sep=comma,
  string type,
  header=true,
  columns={Uitdaging,Max,ST(abs),Kendo(abs),jQM(abs),Lungo(abs)},
  columns/Uitdaging/.style={column name=\textbf{Uitdaging}, column type={l}},
  columns/Max/.style={column name=\textbf{Max}, column type={l}},    
  columns/jQM(abs)/.style={column name=\textbf{\jqma}, column type={c}},
  columns/ST(abs)/.style={column name=\textbf{\sta}, column type={c}},
  columns/Lungo(abs)/.style={column name=\textbf{\lungoa}, column type={c}},
  columns/Kendo(abs)/.style={column name=\textbf{\kendoa}, column type={c}},
  every head row/.style={
    before row=\toprule,
    after row=\midrule},
  every last row/.style={
  	before row=\midrule,
    after row=\bottomrule}
]{tabellen/gebruik/toestel.csv}
\caption{Gebruik van \uit{toestel} voor \st{}~(\sta), \kendo{}~(\kendoa), \jqm{}~(\jqma) en \lungo{}~(\lungoa).}
\label{tabel:evaluatie-gebruik-toestel}
\end{table}

\paragraph{\st}
\st{} ondersteunt het herkennen van de context waarin de applicatie wordt gebruikt.
\st{} kan zowel besturingssysteem, browser als ondersteunde (HTML5-)kenmerken opvragen en herkennen.
Het besturingssysteem kan bevraagd worden via \code{Ext.os.name}.
Deze methode herkent onder andere Android, iOS, Windows en BlackBerry.
Er kan ook gebruik worden gemaakt van de \term{singleton} klasse \code{Ext.os.is}.
Zo geeft \code{Ext.os.is.Android} terug of Android het gebruikte besturingssysteem is of niet.
Het opvragen en herkennen van browser en (HTML5-)kenmerken gebeurt op een analoge manier.


\st{} voorziet vijf lay-outs die aan een component kunnen worden toegekend:
\begin{description}
 \item [\code{HBox}] plaatst de componenten horizontaal naast elkaar.
 \item [\code{VBox}] plaatst de componenten verticaal onder elkaar.
 \item [\code{Card}] plaatst de componenten boven elkaar.
 \item [\code{Fit}] maakt de component passend voor zijn ouder container.
 \item [\code{Docking}] maakt het plaatsen van extra componenten mogelijk in de top-, rechter-, bodem- of linkerrand van zijn ouder container.
\end{description}
De creatie van de tablet lay-out steunt op de \code{HBox} lay-out.
De \code{flex} eigenschap van deze lay-out definieert de ratio van de groottes van beide componenten.
De creatie van de smartphone lay-out maakt het menu in de linkse component van de lay-out onzichtbaar.
Om naar het menu terug te keren moet een extra knop in de hoofdtekst worden toegevoegd die naar het menu navigeert.
\st{} ondersteund geen klikbare hoofdteksten die deze functionaliteit toelaten.

\paragraph{\kendo}
De methodes die \code{Kendo.support} aanbiedt, kunnen de context waarin de applicatie wordt uitgevoerd, opvragen.
Een onderscheid maken tussen smartphone of tablet kan met \code{kendo.support.tablet}.

De lay-out van de tablet is mogelijk met een \code{splitview}.
Deze \code{view} is specifiek voor tablets en kan het scherm horizontaal of verticaal opdelen.
Een \code{splitview} moet als waarde bij het data-attribuut \code{data-role} worden toegekend.
De verschillende schermen binnen een \code{splitview} moeten als \code{pane} worden geannoteerd.
Het scherm van de \code{splitview} dat moet wijzigen bij het aanklikken van een knop, moet met het \code{data-target} attribuut bij de kop worden gedefinieerd.

De tablet lay-out wordt als standaard gebruikt.
Om de smartphone lay-out te verkrijgen moeten drie aanpassingen gebeuren.
Beide \code{panes} van de \code{splitview} moeten als apparte \code{view} worden toegevoegd ter vervanging van de \code{splitview}.
Ook moeten de \code{data-target} attributen worden verwijderd.
Ten slotte moet de hoofdtekst linken naar het menu in de schermen voor het toevoegen van een uitgave.

%TODO layout als data-role bespreken?

% Dit zijn de mogelijkheden:
% \begin{description}
%   \item [\code{touch}] geeft terug of de browser \term{touch} gebeurtenissen ondersteund.
%   \item [\code{pointers}] geeft terug of de browser \code{pointer} gebeurtenissen ondersteund.
%   \item [\code{scrollbar}] geeft de breedte van de \code{scrollbar} terug in pixels.
%   \item [\code{hasHW3D}] geeft terug of de browser 3D-ondersteuning biedt voor transities en transformaties.
%   \item [\code{hasNativeScrolling}] geeft terug of de browser deze CSS-eigenschap ondersteund.
%   \item [\code{devicePixelRatio}] geeft het huidige pixel ratio terug (Android).
%   \item [\code{placeHolder}] geeft terug of de browser invoer placeholders ondersteund.
%   \item [\code{zoomLevel}] geeft het huidige zoom niveau terug van de browser.
% \end{description}

\paragraph{\jqm}
Het raamwerk biedt zelf geen functies aan om te herkennen of het toestel een smartphone of tablet is.
In \jqm{} is er ook geen functionaliteit aanwezig om een menu te tonen voor tablets, maar niet voor smartphones. 
Eerst werd gezocht naar plug-ins aan de hand van een blogpost~\cite{Deering2012}, wat leidde tot: Splitview~\cite{Rahman2013}, SimpleSplitView~\cite{Yared2013} en Multiview~\cite{Franck2012}. 
Deze drie mogelijke kanshebbers hadden elk hun tekorten. 
Zo was de eerste destructief ten opzichte van het raamwerk. 
Dit betekent dat de bestanden van het raamwerk zelf werden aangepast, wat het moeilijker maakt als er moeten worden geüpdatet naar een nieuwe versie. 
De tweede plug-in werkte enkel tot versie 1.0.1 van \jqm{}. 
De laatste plug-in had moeite met het zich aanpassen aan veranderende afmetingen van de browser. 

Uiteindelijk werd van een plug-in afgestapt doordat werd aangetoond hoe via CSS3 media queries hetzelfde kon worden bereikt~\cite{Hadlock2012}. 
Ook uit de documentatie van versie 1.3 \cite{JQuery2013e} blijkt dat dit de correcte manier is om hiermee om te gaan.
Daarnaast wordt op de documentatiesite van \jqm{}~1.2 al een gelijkaardige lay-out gebruikt~\cite{JQuery2012b}. 
De uiteindelijke oplossing voor het probleem kwam voort uit het idee van CSS3 media queries en de documentatiesite van \jqm{}~1.2.
Bij CSS3 media queries moet zelf een breekpunt (uitgedrukt in pixels) worden opgegeven wanneer dient geschakeld te worden tussen smartphonelay-out of tabletlay-out.

Het smartphonemenu is altijd geactiveerd en kan ook worden gebruikt als de applicatie op een tablet wordt gebruikt.

\paragraph{\lungo}
Het raamwerk zelf biedt de functie \code{Lungo.Core.isMobile()} aan om te weten of het huidige apparaat een mobiel apparaat is.
QuoJS biedt daarnaast ook nog \code{browser}, \code{os.name}, \code{os.name}, \code{env.os.version} en \code{screen} aan.
Deze laatste geeft de breedte en hoogte terug.
Dit betekent dat de ontwikkelaar nog altijd zelf instaat voor de bepaling of het een smartphone of tablet is.
Als oplossing werd gebruik gemaakt van CSS3 media queries, met dezelfde aanpak als \jqm{}.
Het smartphonemenu is altijd geactiveerd en kan ook worden gebruikt als de applicatie op een tablet wordt gebruikt.

%%%%%%%%%%%%%

\subsection{\uit{laadscherm}} 
\label{sec:evaluatie-gebruik-laadscherm}

In tabel \ref{tabel:evaluatie-gebruik-laadscherm} worden de resultaten getoond van de twee deeluitdagingen van \uit{laadscherm}.
Onder de tabel wordt per raamwerk verklaard waarom dat resultaat werd behaald.

\begin{table}[H]
\centering
\pgfplotstabletypeset[
  begin table=\begin{tabular}{p{8cm} p{0.8cm} p{0.8cm} p{0.8cm} p{0.8cm} p{0.3cm}},
  end table=\end{tabular},
  skip coltypes=true,
  col sep=comma,
  string type,
  header=true,
  columns={Uitdaging,Max,ST(abs),Kendo(abs),jQM(abs),Lungo(abs)},
  columns/Uitdaging/.style={column name=\textbf{Uitdaging}, column type={l}},
  columns/Max/.style={column name=\textbf{Max}, column type={l}},    
  columns/jQM(abs)/.style={column name=\textbf{\jqma}, column type={c}},
  columns/ST(abs)/.style={column name=\textbf{\sta}, column type={c}},
  columns/Lungo(abs)/.style={column name=\textbf{\lungoa}, column type={c}},
  columns/Kendo(abs)/.style={column name=\textbf{\kendoa}, column type={c}},
  every head row/.style={
    before row=\toprule,
    after row=\midrule},
  every last row/.style={
  	before row=\midrule,
    after row=\bottomrule}
]{tabellen/gebruik/laadscherm.csv}
\caption{Gebruik van \uit{laadscherm} voor \st{}~(\sta), \kendo{}~(\kendoa), \jqm{}~(\jqma) en \lungo{}~(\lungoa).}
\label{tabel:evaluatie-gebruik-laadscherm}
\end{table}

\paragraph{\st}
Een laadscherm tonen kan door een masker met xtype \code{loadmask} op de huidige \code{view} te plaatsen.
Een object van dit xtype kan een bericht bevatten wat de laadtekst voorstelt.
Het toten van een masker kan door de \code{setMasked} methode op een \code{view} op te roepen.
Een courante aanpak is te werken met de \term{singleton} klasse \code{Ext.Viewport} die de huidige zichtbare \code{view} voorsteld.
Hierop de \code{setMasked} methode oproepen verzekert dat het laadscherm bovenop alle andere schermen wordt geplaatst.

De \code{Ext.Msg} \term{singleton} klasse bevat alle methoden om dialogen weer te geven.
\st{} biedt drie standaarden van dialogen aan: \code{alert}, \code{promt} en \code{confirm}.
De eerste laat de gebruiker een bericht zien,  de tweede vraagt de gebruiker om invoer en de laatste vraagt bevestiging aan de gebruiker.
Deze drie standaarden zijn als methode in \code{Ext.Msg} beschikbaar.
Parameters van de methoden kunnen de titel, tekst en functie van de knoppen bepalen.
Een meer generieke aanpak is het oproepen van de \code{show} methode van \code{Ext.Msg}.
Intern roepen de standaard dialogen deze methode ook op.

\paragraph{\kendo}
%laadscherm
Een \code{Application} object wordt na initialisatie van een \kendo{} applicatie aangemaakt.
Hierop kunnen de \code{showLoading} en \code{hideLoading} methoden worden opgeroepen om de animatie van het laadscherm te tonen of te verbergen.

%dialoogvenster
Een dialoogvenster wordt in \kendo{} \code{ModelView} genoemd (niet te verwarren met het \code{viewModel} van de MVVM-architectuur).
Het attribuut \code{data-role} moet \code{modelview} als waarde hebben om een HTML-fragment als dialoogvenster te definiëren.
Dit fragment kan hoofd- en/of voetteksten bevatten.
Het is mogelijk het \code{ModelView} in \js{} te selecteren met \code{\$("\#id").data("kendoMobileModalView")}.
%todo waarbij id voor de id van het modelview staat?
Openen van het venster kan in \js{} met de \code{open} methode of in HTML door te linken naar het \code{modelview} en het \code{data-rel} attribuut aan \code{modelview} gelijk te stellen.

\paragraph{\jqm}
Het standaard laadscherm is enkel een \term{spinner} die ronddraait, die niet opvallend aanwezig is en zonder tekst eronder.
Door de opties in de API te gebruiken, komt de \term{spinner} duidelijk naar voor door een zwarte achtergrond en kan er ook een tekst worden ondergezet.

De laadschermen werden in combinatie met de AJAX-oproepen gebruikt.
Bij de AJAX-oproep wordt de functie \code{beforeSend} voor de oproep opgeroepen en de functie \code{complete} na de oproep.
Dit zijn dus de ideale plaatsen om respectievelijk de laadschermen te tonen en te verbergen.
Indien de AJAX-oproepen in een ketting worden geplaatst, lukt deze aanpak niet meer.
De \code{complete}-functie van de eerste AJAX-oproep zal pas opgeroepen worden als de laatste AJAX-oproep klaar is.
Hierdoor zal gedurende de ketting van AJAX-oproepen, alleen het eerste laadscherm zichtbaar zijn en zullen de andere nooit worden getoond.
Een oplossing hiervoor is om het verbergen van het laadscherm uit de \code{complete}-oproep te halen en deze in zowel in de \code{succes}-functie als de \code{error}-functie te plaatsen.

Om gemakkelijk dialoogvensters te tonen, werd eerst gebruik gemaakt van DateBox~\cite{Sage2013} als plug-in.
Uiteindelijk bleek de plug-in niet zo gemakkelijk aanpasbaar en daarenboven zijn dialoogvensters standaard in \jqm{} aanwezig.
Het is dan ook helemaal niet nodig om hiervoor een plug-in te gebruiken.
Door zelf het dialoogvenster met \jqm{} aan te maken, kon de lay-out gemakkelijker worden aangepast.

\paragraph{\lungo}
Een laadscherm of dialoogvenster tonen gebeurt met eenzelfde functie die wordt aangeboden door het raamwerk, namelijk: \code{Lungo.Notification.show()}.
Indien er geen parameters worden meegegeven, zal een laadscherm getoond worden.

Het laadscherm wordt in samenwerking met de AJAX-oproepen gebruikt.
Net voor de AJAX-oproep zal het laadscherm worden getoond en in de \term{callback} van de AJAX-oproep zal het worden verborgen.
Er treedt zich echter een probleem op wanneer deze manier gebruikt wordt in een ketting van AJAX-oproepen.
Volgens de documentatie verbergt de functie \code{Lungo.Notification.hide()} het huidige laadscherm of dialoogvenster. 
Uit ervaring blijkt dat wanneer eerst een laadscherm wordt getoond, daarna wordt verborgen en daarna een dialoogvenster wordt getoond, het dialoogvenster niet verschijnt verborgen.
Dit betekent dus dat de verbergfunctie ervoor zorgt dat het dialoogvenster niet wordt getoond.
Een oplossing hiervoor is om bij een ketting van vensters, nooit de vensters tussenin te sluiten.
Wanneer zowel een laadscherm als dialoogvenster worden geopend, zal enkel dat laatste getoond worden.

Om een dialoogvenster te tonen, worden de volgende parameters meegegeven: titel, omschrijving, tijd op het scherm en de \term{callback} functie.
Indien er beslist wordt om geen tijd op te geven, kan het venster worden gesloten met \code{Lungo.Notification.hide()}.
Daarnaast worden er ook specifieke dialoogvensters aangeboden om een succes- of foutmelding te tonen.
Deze zullen respectievelijk een groene en rode kleur hebben.


%%%%%%%%%%%%%

%TODO: controleren of iedereen iets over dat optieveld en die schakelaar heeft geschreven
\subsection{\uit{formulieren}}
\label{sec:evaluatie-gebruik-formulieren}

In tabel \ref{tabel:evaluatie-gebruik-formulieren} worden de resultaten getoond van de zeven deeluitdaging van \uit{formulieren}.
Onder de tabel wordt per raamwerk verklaard waarom dat resultaat werd behaald.

\begin{table}[H]
\centering
\pgfplotstabletypeset[
  begin table=\begin{tabular}{p{8cm} p{0.8cm} p{0.8cm} p{0.8cm} p{0.8cm} p{0.3cm}},
  end table=\end{tabular},
  skip coltypes=true,
  col sep=comma,
  string type,
  header=true,
  columns={Uitdaging,Max,ST(abs),Kendo(abs),jQM(abs),Lungo(abs)},
  columns/Uitdaging/.style={column name=\textbf{Uitdaging}, column type={l}},
  columns/Max/.style={column name=\textbf{Max}, column type={l}},    
  columns/jQM(abs)/.style={column name=\textbf{\jqma}, column type={c}},
  columns/ST(abs)/.style={column name=\textbf{\sta}, column type={c}},
  columns/Lungo(abs)/.style={column name=\textbf{\lungoa}, column type={c}},
  columns/Kendo(abs)/.style={column name=\textbf{\kendoa}, column type={c}},
  every head row/.style={
    before row=\toprule,
    after row=\midrule},
  every last row/.style={
  	before row=\midrule,
    after row=\bottomrule}
]{tabellen/gebruik/formulieren.csv}
\caption{Gebruik van \uit{formulieren} voor \st{}~(\sta), \kendo{}~(\kendoa), \jqm{}~(\jqma) en \lungo{}~(\lungoa).}
\label{tabel:evaluatie-gebruik-formulieren}
\end{table}

\paragraph{\st} 
% Placeholders, text, email and number fields are supported by the framework and can be easily created.  
% Labels can be avoided by not defining them.  
% Creating custom datepickers is not supported.  
% It is impossible to ignore the days field and only years can be delimited.  
% Clearing the form after it was send, has to be programmed manually.
Een formulier wordt in \st{} \code{fielset} genoemd.
Een \code{view} van een formulier voorzien kan door in de rij van elementen een object met \code{xtype} \code{fieldset} te maken.
Dit object kan op zijn beurt voorzien worden van een rij van elementen.
Volgende velden worden aangeboden in \st{}:
\begin{itemize}
  \item \code{textfield}        Ext.field.Text
  \item \code{numberfield}      Ext.field.Number
  \item \code{emailfield}	 Ext.field.Email			
  \item \code{textareafield}    Ext.field.TextArea
  \item \code{hiddenfield}      Ext.field.Hidden
  \item \code{radiofield}       Ext.field.Radio
  \item \code{checkboxfield}    Ext.field.Checkbox
  \item \code{selectfield}      Ext.field.Select	
  \item \code{togglefield}      Ext.field.Toggle
  \item \code{fieldset}         Ext.form.FieldSet
\end{itemize}

Tekst-, email en nummervelden worden bij het renderen tot HTML5-invoertypes omgevormd en bijgevolg worden op mobiele toestellen bijhorende virtuele toetsenborden weergegeven.
Een placeholder toevoegen kan door een veld met \code{placeholder} eigenschap te voorzien en de waarde aan de gewenste placeholder gelijk te stellen. 
Een label toevoegen verloopt analoog,  deze weglaten zal geen label renderen.

Een aangepaste \term{datepicker} maken is niet standaard voorzien.
Een standaard \term{datepicker} is echter wel voorzien met \code{datepicker} als xtype.
Deze kan enkel geconfigureerd worden door een begin- en eindjaar in te stellen.
Een \term{datepicker} maken waarbij het bereik kleiner is dan een jaar, is niet mogelijk.
Ook is het onmogelijk om enkel een maand- en jaarveld te tonen.

%TODO challenge bekijken (enkel reset oproepen normaal ok )
Het leegmaken van een formulier gebeurt niet automatisch wanneer het verzonden wordt.
Hiervoor moet de \code{reset} methode op het bijhorende \code{formpanel} worden opgeroepen.

\paragraph{\kendo}
 Formulierelementen definiëren kan via data-attributen door gebruik te maken van de opmaakgedreven aanpak van \kendo.
 Deze methodologie volgt dus sterk de HTML5-standaard.
 Het placeholder attribuut kan een placeholder definiëren,  het vermijden van een label zal geen labels genereren.
 Het type van het formulierelement moet met het type attribuut worden weergegeven.
 Volgende types worden door \kendo{} ondersteund:
 \begin{itemize}
  \item \code{text}
  \item \code{password}
  \item \code{search}
  \item \code{url}
  \item \code{email}
  \item \code{number}
  \item \code{tel}
  \item \code{file} (niet in iOS)
  \item \code{date}
  \item \code{timemonth} 
  \item \code{datetime}
 \end{itemize}

 \term{Datepickers} worden als widget aangeboden in het Web luik van \kendo{}.
 Het raamwerk zal een invoerelement omvormen naar een \code{kendoDatePicker}.
 Het invoerelement ziet er als volgt uit: \code{<input id=\"datepicker\"\/>}.  
 Vervolgens moet het element worden geïnitialiseerd met \code{\$("\#datePicker").kendoDatePicker()}.
 De datepicker is aanpasbaar zoals gevraagd in de POC.
 Het bereik van de data selectie kan worden ingeperkt door de \code{min} en \code{max} eigenschap van de \code{kendoDatePicker} te zetten.
 Deze eigenschappen worden bij initialisatie van het object meegegeven.
 Enkel maand- en jaarvelden tonen kan door de diepte van de \term{datepicker} in te stellen.
 Hiervoor moet de eigenschap \code{depth} aan \code{year} worden gelijkgesteld.
 
 Het wissen van formulieren steunt op de MVVM-architectuur.
 Een fomulier is gebonden aan een \code{(view)model}:  de inhoud van elk formulierelement komt overeen met de waarde van een eigenschap van een \code{(view)model} met dezelfde naam.
 Wanneer een uitgave wordt verzonden, zal de huidige waarde van het \code{(view)model} in een \js-object worden opgeslagen en wordt het \code{(view)model} gereset.
 Door de dubbele binding tussen formulier en \code{(view)model} zal ook de inhoud van de formulierelementen worden gewist.
 
\paragraph{\jqm} 
Voor het toevoegen van \term{placeholders} in de formuliervelden kon beroep worden gedaan op het \code{placeholder}-attribuut in HTML5. 
Labels zijn verplicht in \jqm{}, maar kunnen onzichtbaar worden gemaakt met de CSS-klasse \code{ui-hide-label}~\cite{JQuery2013}. 
Wat wel opmerkelijk is wanneer een formulier wordt ingevuld, daarna verstuurd en dan wordt teruggegaan, het formulier nog alle waarden bevat. 
Men moet na het formulier te hebben verstuurd, zelf het formulier altijd leegmaken. 
Dit kan met behulp van \js{} via de \code{reset()}-functie op het formulier.
 
Voor de types van de formuliervelden werd beroep gedaan op de volgende types: \code{text}, \code{number} en \code{email}. 
%Deze zorgen ervoor dat op de mobiele apparaten aangepaste toetsenborden te voorschijn komen. 
Het \code{date} type werd echter niet gebruikt om wille van twee redenen.
Ten eerste was hiervoor een slechte ondersteuning naar mobiele browsers toe~\cite{Deveria2013b}.
Android~2.3 ondersteunt dit niet en de \term{placeholder}-tekst in het veld ontbrak op iOS~6 en Android~4.2.
Hierdoor weet de gebruiker in eerste instantie niet wat hij hier moet invullen. 
Zelf een \term{placeholder} instellen is onmogelijk voor een \code{date}-type~\cite{Berjon2012}. 
Een tweede probleem was het opleggen van het bereik van datums, wat met het \code{date}-type onmogelijk is. 
Beide problemen werden opgelost door gebruik te maken van de Date \& Time Picker van Mobiscroll~\cite{Mobiscroll2013} die ook aangepaste lay-out heeft conform met die van \jqm{}. 
Het veld heeft dan wel het type \code{text}.
Het is dus in principe mogelijk om iets anders dan een datum in te geven. 
Dit wordt belet door ook nog eens een datumvalidatie (zie \ref{sec:evaluatie-gebruik-validatie}) te doen op dit tekstveld mocht de plug-in het niet hebben afgedwongen.
 
Het was ook nodig om enkel de maand en jaar in te geven als datum, dus zonder dag.
Ook hier kon niet het \code{date}-type gebruikt worden, omdat daar ook een dag voor nodig is. 
Daardoor werden de maanden handmatig geprogrammeerd als vaste lijstitems. 
De jaren zijn dynamisch en zijn telkens dit jaar, het volgende en het vorige jaar. 
Deze functionaliteit kon ook met de plug-in van Mobiscroll worden verwezenlijkt.

Als laatste werd zowel het optieveld als de schakelaar door \jqm{} zelf aangeboden en konden direct gebruikt worden. 
 
\paragraph{\lungo} 
Het toevoegen van \term{placeholders} in de formuliervelden gebeurt met het HTML5-atttribuut \code{placeholder}.
In \lungo{} zijn labels niet verplicht.
Indien deze niet gewenst zijn, kunnen deze gewoon uit de HMTL5-code weggelaten worden.

De types \code{text}, \code{number} en \code{email} voor formuliervelden worden verwezenlijkt door deze als type voor de \code{input}-tags mee te geven in het formulier.
Gelijkaardig met de twee aangehaalde problemen voor \jqm{}, werd niet gekozen voor het \code{date}-type, maar een plug-in om de functionaliteit met datums op te lossen.
De \code{date-picker} werd gebruikt van de plug-in pagina van Lungo zelf~\cite{TapQuo2013b}.
Bij deze plug-in is al voorbeeldcode aanwezig die nodig is om automatisch een \term{datepicker} te openen en de aangeklikte datum in het formulierveld te zetten.
De plug-in laat echter niet toe om een bereik op te geven.
De datum met enkel een maand en jaar diende handmatig geprogrammeerd te worden omdat de aangeboden plug-in hiervoor geen ondersteuning bood.

Een optieveld werd niet aangeboden door \lungo{} en werd vervangen door een \term{dropdown}menu. 
Een schakelaar daarentegen werd dan weer wel aangeboden.
Het legen van een formulier gebeurt in \lungo{} door de \code{reset}-functie in \js{} op te roepen op dat formulier.

%%%%%%%%%%%%%

\subsection{\uit{vullen}}
\label{sec:evaluatie-gebruik-vullen}

In tabel \ref{tabel:evaluatie-gebruik-vullen} worden de resultaten getoond van de twee deeluitdaging van \uit{vullen}.
Onder de tabel wordt per raamwerk verklaard waarom dat resultaat werd behaald.

\begin{table}[H]
\centering
\pgfplotstabletypeset[
  begin table=\begin{tabular}{p{8cm} p{0.8cm} p{0.8cm} p{0.8cm} p{0.8cm} p{0.3cm}},
  end table=\end{tabular},
  skip coltypes=true,
  col sep=comma,
  string type,
  header=true,
  columns={Uitdaging,Max,ST(abs),Kendo(abs),jQM(abs),Lungo(abs)},
  columns/Uitdaging/.style={column name=\textbf{Uitdaging}, column type={l}},
  columns/Max/.style={column name=\textbf{Max}, column type={l}},    
  columns/jQM(abs)/.style={column name=\textbf{\jqma}, column type={c}},
  columns/ST(abs)/.style={column name=\textbf{\sta}, column type={c}},
  columns/Lungo(abs)/.style={column name=\textbf{\lungoa}, column type={c}},
  columns/Kendo(abs)/.style={column name=\textbf{\kendoa}, column type={c}},
  every head row/.style={
    before row=\toprule,
    after row=\midrule},
  every last row/.style={
  	before row=\midrule,
    after row=\bottomrule}
]{tabellen/gebruik/vullen.csv}
\caption{Gebruik van \uit{vullen} voor \st{}~(\sta), \kendo{}~(\kendoa), \jqm{}~(\jqma) en \lungo{}~(\lungoa).}
\label{tabel:evaluatie-gebruik-vullen}
\end{table}

\paragraph{\st}
Het invullen van een formulier wordt ondersteund door de MVC-architectuur.
Twee verschillende methoden worden in de POC gebruikt.
Een eerste maakt gebruik van de \code{setRecord} methode van een \code{formpanel}.
De modelinstantie die het formulier zal invullen als parameter worden meegeven.
\st{} zal automatisch de velden invullen waarbij de naam gelijk is aan de eigenschap van het model.
Zo kan tekst op tekstvelden worden gemapt,  nummers op numerieke velden en \code{booleans} op \code{togglefields}.
Een opmerking over het invullen van een \code{radiofield} moet worden gemaakt.
Een model kan worden voorzien met eigenschappen met volgende types:
\begin{itemize}
  \item auto (Default, implies no conversion)
  \item string
  \item int
  \item float
  \item boolean
  \item date
\end{itemize}
Er bestaat dus geen vlekkeloze mapping tussen een eigenschap van een model en een \code{radiofield}.
Hetzelfde geldt voor een \code{checkboxfield}.
Om deze in te vullen moet de \code{setGroupValue} van het veld worden aangesproken.


%TODO u13 lijsten en click invullen van formulier,  hier ook...
De tweede methode voor het invullen van formulieren maakt gebruikt van een \code{navigationview} en wordt in \uit{lijsten}.

Velden read-only maken kan door objecten te voozien van de \code{readOnly} eigenschap en de waarde op \code{true} te zetten.
Bij \code{radiofields} en \code{checkboxfields} heet deze eigenschap \code{disabled}.


\paragraph{\kendo}
Het invullen van een formulier steunt ook op de MVVM-architectuur en is gelijkaardig aan het resetten van een formulier.
De dubbele binding tussen een formulier en \code{(view)model} wordt met de HTML-tag \code{data-model} aangegeven.
Een \code{viewmodel} wordt in \kendo{} \code{ObservableObject} genoemd.
Een \code{model} heet \code{Model} en erft over van een \code{ObservableObject}.
Deze laatste breidt een \code{ObservableObject} uit met de mogelijkheid om schema's,  velden en methoden te definiëren.  
In wat volgt zal aangenomen worden dat een \code{ObservableObject} wordt gebruikt in plaats van een \code{Model},  tenzij anders vermeld.
Om een formulier met data te vullen is het de taak van de programmeur de velden van het \code{ObservableObject} van de correct waarden te voorzien.
De gebonden formulierelementen zullen vervolgens automatisch worden ingevuld.


Read-only velden moeten in het data-attribuut van het formulierelement worden gespecificeerd.
Alle invoer types buiten radio- en selectknoppen gebruiken hiervoor het \code{readOnly} sleutelwoord.
Radio- en selectknoppen worden onbeschikbaar met het \code{disabled} sleutelwoord.

\paragraph{\jqm}
Om een formulierveld in te vullen met data, dient eerst het formulierveld gezocht te worden en daarna zijn waarde gezet te worden.
Dit gebeurt typisch voor velden van het type \code{input} en \code{textarea} volgens de volgende code: \code{\$("\#form-veld").val("waarde")}.
Bij het \code{select}-type voor een veld kan deze code niet worden gebruik.
Hier moet de waarde worden gezocht in de lijst en dan aan de gevonden waarde het \code{selected}-attribuut worden toegevoegd.
Een gelijkaardige manier dient gevolgd te worden voor het \code{radio}-type voor een veld.
Ook hier moet eerst de waarde worden gezocht, waarna aan de gevonden waarde het \code{checked}-attribuut wordt toegevoegd.
Het vullen van formuliervelden wordt niet door \jqm{} geautomatiseerd, wat betekent dat dit dus voor iedere formulierveld dient te gebeuren.

Het \term{read-only} maken van velden gebeurt via het HTML-attribuut \code{readonly}.
Dit geld voor alle types van velden, behalve voor \code{radio}-elementen en \code{select}-items waar \code{disabled} wordt gebruikt.
Voor \code{select}-items moeten de andere niet-benodigde lijstitems verwijderd worden, want deze kunnen nog steeds aangeklikt worden.

\paragraph{\lungo}
Velden vullen met data dient handmatig te gebeuren door eerst het formulierveld op te zoeken en daarna de waarde te zetten.
Dit gebeurt typisch volgens de volgende code: \code{\$\$("\#form-veld").val("waarde")}.
Deze functie kan ook gebruikt worden voor \code{select}-types.
Geoptimaliseerde mobiele lay-out voor \code{radio}-types is niet aanwezig in \lungo.
Een opmerking dient wel gemaakt te worden dat de waarde altijd een \code{string} moet zijn.
Dit betekent dus voor getallen dat deze altijd eerst moeten worden omgevormd met de \js{}-functie \code{toString()}. 

Het \term{read-only} maken van velden gebeurt via het HTML-attribuut \code{readonly}.
Dit gaat voor alle types van velden, behalve voor \code{select}-types.
Daar worden de niet-benodigde lijstitems verwijderd en wordt de \code{select} zelf \code{disabled} gemaakt.

%%%%%%%%%%%%%

\subsection{\uit{autoaanvullen}}
\label{sec:evaluatie-gebruik-autoaanvullen}

In tabel \ref{tabel:evaluatie-gebruik-autoaanvullen} worden de resultaten getoond van de twee deeluitdagingen van \uit{autoaanvullen}.
Onder de tabel wordt per raamwerk verklaard waarom dat resultaat werd behaald.

\begin{table}[H]
\centering
\pgfplotstabletypeset[
  begin table=\begin{tabular}{p{8cm} p{0.8cm} p{0.8cm} p{0.8cm} p{0.8cm} p{0.3cm}},
  end table=\end{tabular},
  skip coltypes=true,
  col sep=comma,
  string type,
  header=true,
  columns={Uitdaging,Max,ST(abs),Kendo(abs),jQM(abs),Lungo(abs)},
  columns/Uitdaging/.style={column name=\textbf{Uitdaging}, column type={l}},
  columns/Max/.style={column name=\textbf{Max}, column type={l}},    
  columns/jQM(abs)/.style={column name=\textbf{\jqma}, column type={c}},
  columns/ST(abs)/.style={column name=\textbf{\sta}, column type={c}},
  columns/Lungo(abs)/.style={column name=\textbf{\lungoa}, column type={c}},
  columns/Kendo(abs)/.style={column name=\textbf{\kendoa}, column type={c}},
  every head row/.style={
    before row=\toprule,
    after row=\midrule},
  every last row/.style={
  	before row=\midrule,
    after row=\bottomrule}
]{tabellen/gebruik/autoaanvullen.csv}
\caption{Gebruik van \uit{autoaanvullen} voor \st{}~(\sta), \kendo{}~(\kendoa), \jqm{}~(\jqma) en \lungo{}~(\lungoa).}
\label{tabel:evaluatie-gebruik-autoaanvullen}
\end{table}

\paragraph{\st}
Het automatisch aanvullen van een formulierelement steunt op een plug-in van Tajur~\cite{Tajur2012}.
Deze plug-in is niet op de Sencha Market terug te vinden.
Door het \js-bestand toe te voegen wordt het xtype \code{autocompletefield} beschikbaar.
Een object met dit xtype kan een \code{proxy} definiëren die de server kan aanspreken om suggesties asynchroon op te halen.
Ook is het mogelijk het maximaal aantal suggesties vast te leggen.

De \term{backend} server die bij de POC hoort geeft bij een bepaald sleutelwoord suggesties in een JSON-rij terug.
De rij is voorzien van een sleutel maar alle elementen van de rij hebben geen sleutel.
\st{} voorziet vier methoden om de resultaten van een \code{proxy} te parsen naar modelinstanties:
\begin{description}
 \item [\code{JsonReader}] parst JSON-sleutels naar model velden.
 \item [\code{XmlReader}] parst XML-tags naar model velden.
 \item [\code{ArrayReader}] mapt elementen van een rij op velden van een model.
\end{description}
Geen van voorgaande methoden was in staat de rij met suggesties te parsen van rij-element naar modelinstantie.
Hierdoor kon geen klikbare dropdownmenu worden getoond.

\paragraph{\kendo}
Het automatisch aanvullen van een formulierelement wordt als widget door \kendo{} Web aangeboden.
Een HTML-invoerelement met id met worden aangemaakt en geinitialiseerd met \code{\$("\#id").kendoAutoComplete()}.
Elementen die automatisch aanvullen kunnen zowel van een lokale als externe bron worden aangeleverd.
Externe suggesties moeten via een \code{dataSource} worden ingeladen.
Een \code{DataSource} ondersteunt alle CRUD (\term{Create, Read, Update en Delete}) operaties en het sorteren, pagineren, filteren, groeperen en aggregeren van data.
Deze moet geconfigureerd worden om suggesties van de \term{backend} op te halen.
Buiten een \code{DataSource} kan het minimale aantal suggesties en een filter worden opgegeven.
De filter bepaalt de methode om suggesties op te halen:  en kan \code{startswith}, \code{endswith} of \code{contains} zijn.

\paragraph{\jqm}
Indien er wordt gebruik gemaakt van versie~1.2 is een plug-in nodig om auto-aanvulling te bekomen.
Hiervoor kan de plug-in van Andy Matthews worden gebruikt~\cite{Matthews2013}. 
Dit is een zeer gemakkelijk te integreren plug-in die zowel met lokale data als data op afstand kan werken.
Sinds versie~1.3 voorziet \jqm{} deze functionaliteit zelf~\cite{JQuery2013c}.
Er wordt aangehaakt op het \code{listviewbeforefilter}-\term{event} waarbij een eigen filterfunctie geschreven kan worden.
De voorbeeldcode op de site kon integraal worden gebruikt. 

\paragraph{\lungo}
Standaard biedt \lungo{} geen auto-aanvullig aan, maar wel op zijn site van plug-ins~\cite{TapQuo2013b}.
Daar werd de plug-in AutoComplete gebruikt.
De voorbeeldcode maakt het gemakkelijk om onmiddellijk een werkend voorbeeld van de plug-in te hebben.
Veel code kon dus gewoon worden overgenomen om een werkende auto-aanvulling te bekomen.
Er diende nog één extra CSS-regel te worden toegevoegd om de bollen voor de suggesties te verwijderen.

%%%%%%%%%%%%%

\subsection{\uit{afbeelding}}
\label{sec:evaluatie-gebruik-afbeelding}

In tabel \ref{tabel:evaluatie-gebruik-afbeelding} worden de resultaten getoond van de drie deeluitdagingen van \uit{afbeelding}.
Onder de tabel wordt per raamwerk verklaard waarom dat resultaat werd behaald.

\begin{table}[H]
\centering
\pgfplotstabletypeset[
  begin table=\begin{tabular}{p{8cm} p{0.8cm} p{0.8cm} p{0.8cm} p{0.8cm} p{0.3cm}},
  end table=\end{tabular},
  skip coltypes=true,
  col sep=comma,
  string type,
  header=true,
  columns={Uitdaging,Max,ST(abs),Kendo(abs),jQM(abs),Lungo(abs)},
  columns/Uitdaging/.style={column name=\textbf{Uitdaging}, column type={l}},
  columns/Max/.style={column name=\textbf{Max}, column type={l}},    
  columns/jQM(abs)/.style={column name=\textbf{\jqma}, column type={c}},
  columns/ST(abs)/.style={column name=\textbf{\sta}, column type={c}},
  columns/Lungo(abs)/.style={column name=\textbf{\lungoa}, column type={c}},
  columns/Kendo(abs)/.style={column name=\textbf{\kendoa}, column type={c}},
  every head row/.style={
    before row=\toprule,
    after row=\midrule},
  every last row/.style={
  	before row=\midrule,
    after row=\bottomrule}
]{tabellen/gebruik/afbeelding.csv}
\caption{Gebruik van \uit{afbeelding} voor \st{}~(\sta), \kendo{}~(\kendoa), \jqm{}~(\jqma) en \lungo{}~(\lungoa).}
\label{tabel:evaluatie-gebruik-afbeelding}
\end{table}

\paragraph{\st}
Het opladen van een afbeelding steunt op een plug-in van Smirnov~\cite{Smirnov2012} en kan in de Sencha Market gevonden worden op \exturl{market.sencha.com/extensions/file-uploading-component-for-sencha-touch}.
De plug-in is generiek voor het opladen van elk type bestand,  niet uitsluitend afbeeldingen.
Het \js-bestand moet in de touch/src/ux folder worden geplaatst en de \code{Ext.ux.Fileup} klasse moet worden geïnitialiseerd.
Het xtype \code{img} wordt dan beschikbaar voor \st{} componenten.

De plug-in voorziet twee modes voor het opladen van bestanden: lokale als base64 of extern naar een server.
De eerste laat toe afbeeldingen in het DOM of \term{local storage} te laden.
Dit laatste is een aspect van de POC.

Nadat een bestand is opgeladen kunnen twee gebeurtenissen zich voordoen:  \code{loadsuccess} of \code{loadfailure}.
Het is de taak van een \code{controller} om deze gebeurtenissen op te vangen en een bijhorende methode te definiëren.
De succes functie krijgt de base64 text mee en kan een voorbeeld van de afbeelding laten weergeven.

\paragraph{\kendo}
\kendo{} Web biedt een widget aan die het opladen van bestanden toelaat.
Deze widget kan in twee modes worden gebruikt: syncroon of asynchroon.
Een HTML-invoerelement met \code{id} moet worden toegevoegd en geïnitialiseerd met \code{\$("\#id").kendoUpload()}.
In synchrone wordt het formulier met het oplaadelement verzonden naar de \term{backend} als een uitgave wordt toegevoegd.
In asynchrone mode gebeurt het opladen meteen na het selecteren van de afbeelding.
Deze methode voor het opladen van bestanden steunt op de HTML5 File API.
Het verzoek naar de \term{backend} is een POST-verzoek met \code{Content Type} \code{multipart/form-data}.
Omdat een voorbeeld van afbeelding werd gevraagd, is de asynchrone oplossing gekozen.	
\kendo{} voorziet \term{backend} integratie met ASP.NET MVC,  JSP en PHP.
Voor deze drie technologieën is een implementatie beschikbaar om het opladen van bestanden aan serverzijde af te handelen.
Er werd gekozen om de PHP-implementatie te gebruiken.
Wanneer een afbeelding succesvol is opgeladen, wordt de \code{success callback} opgeroepen met het bestand als parameter.
Het bestand wordt met een \code{FileReader} gelezen,  aan een \code{canvas} toegevoegd en naar base64 omgezet met de \code{toDataURL} methode.
Deze werkwijze is aanloog aan de \jqm{} implementatie.

\paragraph{\jqm}
Het toevoegen van een afbeelding gebeurt door \code{file} als invoertype van het formulierveld te gebruiken. 
In versie~1.2 wordt dit veld nog niet opgemaakt met lay-out, maar dit gebeurt wel in versie 1.3~\cite{JQuery2013d}. 
Het omvormen van de afbeelding naar base64 werd geïmplementeerd met de FileReaderAPI en het canvas, wat beide HTML5-specificaties zijn. 
De aangeklikte afbeelding wordt gelezen door middel van de FileReaderAPI, waarna het tijdelijk als afbeelding wordt opgeslagen en daarna geïmporteerd wordt op het canvas. 
Eenmaal geïmporteerd, kan de functie \code{.toDataURL()} opgeroepen worden op het canvas om de geïmporteerde afbeelding om te vormen naar base64. 

Het voorvertonen van het geüploade afbeelding hangt af van het mobiele besturingssysteem.
Zo wordt op iOS~6 een miniatuurafbeelding getoond, terwijl op Android de bestandsnaam wordt getoond.
Het is natuurlijk ook mogelijk om de preview na conversie zelf te tonen op het scherm.
Bij iOS zouden er dan twee voorvertoningen te zien zijn op hetzelfde scherm.

\paragraph{\lungo}
Een afbeelding kiezen gebeurt door het formulierveld met het type \code{file} toe te voegen.
De methode om een afbeelding om te vormen naar base64 is volledig analoog met die van \jqm{}.

%%%%%%%%%%%%%

\subsection{\uit{validatie}}
\label{sec:evaluatie-gebruik-validatie}

In tabel \ref{tabel:evaluatie-gebruik-validatie} worden de resultaten getoond van de vier deeluitdaging van \uit{validatie}.
Onder de tabel wordt per raamwerk verklaard waarom dat resultaat werd behaald.


\begin{table}[H]
\centering
\pgfplotstabletypeset[
  begin table=\begin{tabular}{p{8cm} p{0.8cm} p{0.8cm} p{0.8cm} p{0.8cm} p{0.3cm}},
  end table=\end{tabular},
  skip coltypes=true,
  col sep=comma,
  string type,
  header=true,
  columns={Uitdaging,Max,ST(abs),Kendo(abs),jQM(abs),Lungo(abs)},
  columns/Uitdaging/.style={column name=\textbf{Uitdaging}, column type={l}},
  columns/Max/.style={column name=\textbf{Max}, column type={l}},    
  columns/jQM(abs)/.style={column name=\textbf{\jqma}, column type={c}},
  columns/ST(abs)/.style={column name=\textbf{\sta}, column type={c}},
  columns/Lungo(abs)/.style={column name=\textbf{\lungoa}, column type={c}},
  columns/Kendo(abs)/.style={column name=\textbf{\kendoa}, column type={c}},
  every head row/.style={
    before row=\toprule,
    after row=\midrule},
  every last row/.style={
  	before row=\midrule,
    after row=\bottomrule}
]{tabellen/gebruik/validatie.csv}
\caption{Gebruik van \uit{validatie} voor \st{}~(\sta), \kendo{}~(\kendoa), \jqm{}~(\jqma) en \lungo{}~(\lungoa).}
\label{tabel:evaluatie-gebruik-validatie}
\end{table}

\paragraph{\st}
Een model kan worden voorzien van validatieregels.
Deze regels worden als objecten in een rij aan de \code{validations} eigenschap van een model toegekend.
Volgende validatieregels zijn ingebouwd:
\begin{description}
  \item [presence] verzekert dat het het veld een waarde heeft waarbij nul als geldig wordt beschouwd,  lege tekst niet.
  \item [length] verzekert dat een text een minimale en/of maximale waarde heeft.
  \item [format] verzekert dat een text voldoet aan een opgegeven reguliere expressie.
  \item [inclusion] verzekert dat de waarde van een veld gelijk is aan een element van een gespecifieerde set.
  \item [exclusion] verzekert dat de waarde van een veld zeker niet gelijk is aan een element van een gespecifieerde set.
\end{description}
De controle of een opgegeven waarde een nummer is kan met de \code{format} regel en de \code{/\d+/} reguliere expressie.
Om een bepaalde modelinstantie te valideren moet de \code{validate} methode op de instantie worden opgeroepen.
Om eigen validatieregels toe te laten moet de implementatie van deze methode worden overschreven. \footnote{Informatie gevonden op \exturl{www.sencha.com/forum/showthread.php?122680-Conditional-fields-validations}}
Deze functionaliteit zit dus niet standaard in \st{}.
Met de nieuwe \code{validate} methode kan een \code{validator} aan een validatieregel worden toegevoegd.
Dit is een functie die de programmeur zelf bepaalt en \code{true} of \code{false} teruggeeft bij het al dan niet slagen van een conditie.

Het opbouwen van een foutenboodschap kan door te itereren over de fouten die na validtie werden teruggevonden.
Een specifieke foutenboodschap kan aan elke validatieregel worden toegekend.

Invalide formulierelementen aanduiden met een rode rand wordt niet door \st{} ondersteund.
Hiervoor moet CSS worden gebruikt.
Foutief ingevulde formulierelementen moeten na valiatie met een CSS-klasse worden aangevuld.

\paragraph{\kendo}
%TODO verwijs naar niet support voor HTML5 validatie..
Het \kendo{} raamwerk ondersteunt de validaties zoals aangeboden binnen HTML5:
\begin{itemize}
  \item [required] verzekert dat een het veld een waarde heeft.
  \item [pattern] verzekert dat de waarde van een veld voldoet aan een opgegeven reguliere expressie.
  \item [min/max] verzekert dat de waarde van een veld groter en/of kleiner is dan een opgegeven waarde.
  \item [data types] verzekert dat de waarde van een veld gelijk is aan het opgegeven type (e-mail, url, number, enz.)
\end{itemize}
Deze validaties moeten binnen het data-attribuut van het formulierelement worden aangebracht.

De \kendo{} \code{validator} is compatibel met deze HTML5-validaties.
Een \code{validator} moet in \js worden aangemaakt met volgend commando:  \code{\$("\#myform").kendoValidator().data("kendoValidator")}.
Hierbij kan de jQuery selector eender welk element uit het DOM aanduiden.
De \code{validator} zal geselecteerde invoerelementen controleren op validatieregels.

Eigen condities kunnen als validatieregels worden geformuleerd als \js-functie die \code{true} teruggeeft als de validatie slaagt.
Deze functie kan aan de set van regels van een \code{validator} worden toegevoegd.
De validatiecontrole starten kan door de \code{validate} methode op de \code{validator} op te roepen.
Bij het controleren van invoerelementen worden altijd eerst de standaard validatieregels gecontoleerd,  daarna de eigen validatieregels.
De volgorde waarin de controles worden uitgevoerd ligt vast en zal stoppen zodra één controle mislukt.

%TODO mijn tricky implementatie van other -> remkars validatie vermelden?

Validatieberichten kunnen voor standaard validaties door het raamwerk zelf worden opgebouwd.
Deze kunnen worden overschreven door zelf een data-attribuut \code{validationMessage} aan het invoerelement toe te kennen.
Bij eigen validatieregels kan een validatiebericht per regel worden gespecificeerd.

Invalide formulierelementen aanduiden met een rode rand wordt niet door \kendo{} ondersteund.
De standaard implementatie voorziet een \code{tooltip} per invalied veld.
Deze zal het validatiebericht naast het invalide formulierelement plaatsen. 

Een venster tonen waarbij alle foutenboodschappen zijn samengevat is ook niet standaard aanwezig.
De \code{errors} methode van een \code{validator} geeft een rij van foutenboodschappen terug.
De volledige foutenboodschap moet vervolgens worden geconstrueerd door alle foutenboodschappen te concateneren.
Hoe een dialoogvenster gemaakt wordt zal in \uit{dialoog} worden besproken.

\paragraph{\jqm}
Validatie is niet standaard aanwezig in jQuery Mobile. 
Eerst werd geprobeerd om de verplichte velden te voorzien van het \code{required}-attribuut in HTML5. 
Dit werd niet gedaan om twee redenen.
Enerzijds is hiervoor geen ondersteuning voor mobiele browsers~\cite{Deveria2013}. 
Anderzijds is het ook nodig om de velden te valideren op hun waarde.
Als oplossing werd de plug-in van Jörn Zaefferer gebruikt~\cite{Zaefferer2013}. 
Deze lost beide problemen op.
De plug-in zelf kan op twee manieren gebruikt worden: enerzijds annoteren van de formuliervelden met speciale CSS-klassen in de HTML-code ofwel anderzijds door programmatie met \js{}. 
Beide aanpakken werden getest en slaagden. 

De plug-in bevat de volgende ingebakken validatieregels die nodig waren: \code{required}, \code{number}, \code{email} en \code{date}.
Daarnaast was het nodig dat een veld verplicht was enkel indien een bepaalde optie aangevinkt was.
Zo een afhankelijkheidsrelatie is standaard aanwezig in de plug-in.

De plug-in toont standaard een foutboodschap onder het foute formulierveld.
Door de uitgebreide API van de plug-in die ook uitvoerig gedocumenteerd is, konden alle foutboodschappen samen in een dialoogvenster worden weergegeven.

Een specifiek mobiel probleem was dat bij het tonen van het dialoogvenster, de plug-in op de achtergrond de cursor op het eerste veld zette. 
Hierdoor verscheen het toetsenbord op het scherm van het mobiele apparaat wanneer het dialoogvenster tevoorschijn kwam, wat niet de bedoeling is. 
Dit werd opgelost door \code{focusInvalid:false} in te stellen in de plug-in.

De plug-in annoteert de foute velden met de CSS-klasse \code{error}.
Hierdoor kon de rode rond in CSS worden geprogrammeerd. 
Dit ging voor \code{input} en \code{textarea}, maar gaf problemen voor \code{select} en \code{fieldset}.
Door de extra code die \jqm{} genereert rond deze velden, moest via de DOM de omringende code geannoteerd worden om de rode rand te bekomen. 
Deze functie kon aangehaakt worden op de \code{highlight} en \code{unhighlight} functies van de plug-in.

\paragraph{\lungo}
Validatie is niet aanwezig in \lungo{} en er kon ook geen plug-in voor QuoJS gevonden worden~\cite{Ameye2013}.
Er werd geprobeerd om bestaande plug-ins voor andere \js{}-bibliotheken om te vormen en deze te laten werken met QuoJS.
Aangezien deze manier niet direct een oplossing bracht en er geen beroep kon worden gedaan op HTML5-validatie (zie \jqm{}), werd alle validatie manueel geprogrammeerd. 
Bij wijze van voorbeeld werd enkel validatie op het loginscherm geprogrammeerd.
Validatie op de rest van de formulieren doorheen de POC is volgens dezelfde werkwijze mogelijk, maar werd niet geïmplementeerd.
Dit verandert namelijk niets aan de score.

Het tonen van foutboodschappen alsook het tonen van een rode rand rond de foute velden werd ook zelf geprogrammeerd.
Bij een foutief gevalideerd veld werd een CSS-klasse toegevoegd die zorgde voor een rode rand.

%%%%%%%%%%%%%

\subsection{\uit{handtekening}}
\label{sec:evaluatie-gebruik-handtekening}

In tabel \ref{tabel:evaluatie-gebruik-handtekening} worden de resultaten getoond van de deeluitdaging van \uit{handtekening}.
Onder de tabel wordt per raamwerk verklaard waarom dat resultaat werd behaald.

\begin{table}[H]
\centering
\pgfplotstabletypeset[
  begin table=\begin{tabular}{p{8cm} p{0.8cm} p{0.8cm} p{0.8cm} p{0.8cm} p{0.3cm}},
  end table=\end{tabular},
  skip coltypes=true,
  col sep=comma,
  string type,
  header=true,
  columns={Uitdaging,Max,ST(abs),Kendo(abs),jQM(abs),Lungo(abs)},
  columns/Uitdaging/.style={column name=\textbf{Uitdaging}, column type={l}},
  columns/Max/.style={column name=\textbf{Max}, column type={l}},    
  columns/jQM(abs)/.style={column name=\textbf{\jqma}, column type={c}},
  columns/ST(abs)/.style={column name=\textbf{\sta}, column type={c}},
  columns/Lungo(abs)/.style={column name=\textbf{\lungoa}, column type={c}},
  columns/Kendo(abs)/.style={column name=\textbf{\kendoa}, column type={c}},
  every head row/.style={
    before row=\toprule,
    after row=\midrule},
  every last row/.style={
  	before row=\midrule,
    after row=\bottomrule}
]{tabellen/gebruik/handtekening.csv}
\caption{Gebruik van \uit{handtekening} voor \st{}~(\sta), \kendo{}~(\kendoa), \jqm{}~(\jqma) en \lungo{}~(\lungoa).}
\label{tabel:evaluatie-gebruik-handtekening}
\end{table}

\paragraph{\st}
Het tekenen van een handtekening steunt op een plug-in van SimFla~\cite{SimFla2011} en is in de Sencha Market te vinden op \exturl{market.sencha.com/extensions/signature-pad-field}.
Een plug-in aan het raamwerk toevoegen kan door het \js-bestand in de touch/src/ux folder te plaatsen.
Vervolgens moet de plug-in worden geladen bij het initialiseren van de applicatie.

De plug-in maakt een nieuw xtype \code{signaturefield} beschikbaar dat als veld in een formulier kan worden gebruikt.

De plug-in maakt gebruikt van het HTML5-canvas en retourneert de handtekening als geëncodeerde base64-tekst.

\paragraph{\kendo}
Aangezien \kendo{} steunt op de jQuery bibliotheek is \kendo{} ook perfect compatibel met jQuery plug-ins.
Net zoals bij \jqm{} werd ook de jSignature handtekening van Willow Systems~\cite{Systems2013} geïmplementeerd.

\paragraph{\jqm}
Er werd gezocht naar een plug-in om deze functionaliteit te bekomen, doordat \jqm{} dit niet standaard aanbiedt. 
Eerst werd gewerkt met Signature Pad van Thomas Bradley~\cite{Bradley2013}. 
Door de lange tijd die werd besteed aan het aanpassen van de lay-out, werd overgestapt naar jSignature van Willow Systems~\cite{Systems2013}. 
Deze laatste gaf ook het voordeel dat de breedte van het gebied om te handtekening in te zetten, zich automatisch naar 100\% schaalde. 
De plug-in maakt gebruik van het HTML5 \code{canvas}-element en de \code{.toDataURL()} methode.
Hierdoor kan de base64-string bekomen worden die nodig is om door te sturen naar de server.

\paragraph{\lungo}
Het maken van een handtekening is niet standaard aanwezig en daarenboven kon ook geen plug-in worden gevonden.
Er kan echter worden gebruik gemaakt van plug-ins die op andere \js{}-bibliotheken dan QuoJS steunen, maar de auteurs besloten om deze niet te beschouwen.
Daarenboven zouden er dan ook twee \js{}-bibliotheken aanwezig zijn in de applicatie.


%%%%%%%%%%%%%

\subsection{\uit{ajax}}
\label{sec:evaluatie-gebruik-ajax}

In tabel \ref{tabel:evaluatie-gebruik-ajax} worden de resultaten getoond van de vier deeluitdagingen van \uit{ajax}.
Onder de tabel wordt per raamwerk verklaard waarom dat resultaat werd behaald.

\begin{table}[H]
\centering
\pgfplotstabletypeset[
  begin table=\begin{tabular}{p{8cm} p{0.8cm} p{0.8cm} p{0.8cm} p{0.8cm} p{0.3cm}},
  end table=\end{tabular},
  skip coltypes=true,
  col sep=comma,
  string type,
  header=true,
  columns={Uitdaging,Max,ST(abs),Kendo(abs),jQM(abs),Lungo(abs)},
  columns/Uitdaging/.style={column name=\textbf{Uitdaging}, column type={l}},
  columns/Max/.style={column name=\textbf{Max}, column type={l}},    
  columns/jQM(abs)/.style={column name=\textbf{\jqma}, column type={c}},
  columns/ST(abs)/.style={column name=\textbf{\sta}, column type={c}},
  columns/Lungo(abs)/.style={column name=\textbf{\lungoa}, column type={c}},
  columns/Kendo(abs)/.style={column name=\textbf{\kendoa}, column type={c}},
  every head row/.style={
    before row=\toprule,
    after row=\midrule},
  every last row/.style={
  	before row=\midrule,
    after row=\bottomrule}
]{tabellen/gebruik/ajax.csv}
\caption{Gebruik van \uit{ajax} voor \st{}~(\sta), \kendo{}~(\kendoa), \jqm{}~(\jqma) en \lungo{}~(\lungoa).}
\label{tabel:evaluatie-gebruik-ajax}
\end{table}

\paragraph{\st}
AJAX-verzoeken kunnen zowel expliciet via een directe oproep met \code{Ext.Ajax.request} als impliciet via \code{stores} worden uitgevoerd.
De expliciete oproep is gelijkaardig aan de \code{\$.ajax} methode van jQuery.
Een enige uitzondering is te vinden bij kruis-domein AJAX-verzoeken.
Om aan de CORS-standaarden (Cross-Origin Resource Sharing) te voldoen moet de eigenschap \code{useDefaultXhrHeader} op \code{false} worden gezet.
%TODO referentie cors + opzoeken options request

De tweede,  impliciete,  methode voor AJAX-verzoeken is via \code{stores}.
Een \code{store} wordt voorzien van een \code{proxy}.  
Deze kan data aan de klant of server zijde opslaan.  
Een \code{proxy} voor opslag aan client zijde kan zowel in het RAM-geheugen als in de \term{local storage} en \term{session storage} van de browser opslaan.  
Een \code{proxy} voor server opslag kan data verzenden via AJAX (zelfde domein) of JSONP (verschillende domeinen).  
Een \code{proxy} kan ook geconfigureerd worden met \code{readers} en \code{writers} om data van de server te lezen of naar de server te schrijven.

Het verzenden van een JSON-\term{payload} moet via een expliciet AJAX-verzoek gebeuren.
Data kan via \code{Ext.encode} naar JSON worden geëncodeerd en via de \code{jsonData} eigenschap aan het verzoek worden gekoppeld.

\paragraph{\kendo}
Om asynchrone verzoeken naar de \term{backend} te implementeren, moet een \code{DataSource} worden gebruikt.
Zoals reeds besproken in de vorige uitdaging biedt dit object CRUD operaties.
Dit kan zowel op lokale (\js-objecten en \js-rijen) als externe data (XML, JSON, JSONP).
De \code{transport} eigenschap kan de configuraties bevatten om data te creëren (\code{create} eigenschap),  lezen (\code{read} eigenschap),  verwijderen (\code{destroy} eigenschap) en op te waarderen (\code{update} eigenschap).
Deze vier eigenschappen moeten geconfigureerd worden zoals de \code{\$.ajax} methode van jQuery.

Hoe data moet worden geparset, staat gedefinieerd in de \code{schema} eigenschap van de \code{DataSource}.
Zowel JSON als XML wordt ondersteund.
Aan een \code{schema} kan een \code{model} worden toegekend.
Er onderscheiden zich twee gevallen:  een bestaand \code{viewModel} kan met data worden geladen of nieuwe instanties van een \code{model} kunnen worden aangemaakt.
Het eerste geval zal de velden van één \code{viewModel} wijzigen als CRUD-operaties worden uitgevoerd.
De eigenschap moet dan aan het \code{viewModel} worden gelijkgesteld.
Het tweede geval zal het aantal instanties van een \code{model} wijzigen als CRUD-operaties worden uitgevoerd.
De eigenschap kan aan een reeds gedefinieerd \code{Model} worden gelijkgesteld of een model kan lokaal worden gedefinieerd.

\paragraph{\jqm}
Het maken van oproepen via AJAX gebeurt door jQuery. 
Dit gebeurt met de functie \code{\$.ajax} waar onder andere kan ingesteld worden wat het te verwachten antwoord is (zoals tekst, JSON of XML). 
Bij het succesvol uitvoeren van de oproep wordt de \code{succes}-functie opgeroepen, bij faling de \code{error}-functie waarna een relevante foutboodschap wordt getoond.

In jQuery is de functie \code{parseJSON} aanwezig, maar aangezien in de AJAX-oproep ingesteld wordt dat JSON wordt verwacht, parst jQuery al automatisch het antwoord. 
Hierdoor is de functie \code{parseJSON} niet nodig en kunnen direct worden omgaan met het antwoord.

Net zoals bij JSON het geval was, is het ook niet nodig om expliciet de \code{parseXML}-functie te gebruiken. 
Het doorlopen en opvragen van gegevens uit het XML-bestand vraagt meer werk. 
Waar er bij JSON direct kon worden omgegaan met de data, moet bij XML dat gebeuren aan de hand van selectoren.

Het versturen van JSON is gelijkaardig met het versturen van andere data.
Eerst zal de JSON-data moeten worden omgezet naar een string, wat gebeurt door \code{JSON.stringify}.
Daarna zal in de AJAX-oproep moeten worden aangegeven  dat de inhoud JSON is.
Dit gebeurt door \code{contentType:"{}application/json"} te schrijven.

\paragraph{\lungo}
Standaard biedt \lungo{} functies aan voor het ophalen en versturen van data via AJAX.
Deze functies zullen intern de functies van QuoJS oproepen.
De URL, de data, de callback en het type dienen hierbij te worden opgegeven.
De laatstgenoemde kan \code{text}, \code{json}, \code{xml} of \code{html} zijn.
Met \code{text}, \code{json} kan onmiddellijk worden omgegeven.
Voor \code{xml} dient er gebruik te worden gemaakt van de selectoren in QuoJS om de gevraagde data op te zoeken.

Bij het versturen van JSON-data konden de functies van \lungo{} zelf niet worden gebruikt.
Deze hadden te weinig opties om aan te geven dat de verstuurde data JSON was.
Hierdoor werden de functies van QuoJS gebruikt, die meer opties hadden.
Toch bleef er een probleem bij het versturen van JSON-data.
Na lang zoekwerk hoe QuoJS met deze oproep omging, bleek uiteindelijk dat de bibliotheek altijd de parameters wilde serialiseren.
Dit is uiteraard niet nodig als ruwe data, zoals JSON, wordt meegegeven.
Aangezien dit een fout was in de bibliotheek zelf werd om het probleem zo snel mogelijk te verhelpen het \js-bestand zelf aangepast.

%%%%%%%%%%%%%

\subsection{\uit{lijsten}}
\label{sec:evaluatie-gebruik-lijsten}

In tabel \ref{tabel:evaluatie-gebruik-lijsten} worden de resultaten getoond van de drie deeluitdagingen van \uit{lijsten}.
Onder de tabel wordt per raamwerk verklaard waarom dat resultaat werd behaald.

\begin{table}[H]
\centering
\pgfplotstabletypeset[
  begin table=\begin{tabular}{p{8cm} p{0.8cm} p{0.8cm} p{0.8cm} p{0.8cm} p{0.3cm}},
  end table=\end{tabular},
  skip coltypes=true,
  col sep=comma,
  string type,
  header=true,
  columns={Uitdaging,Max,ST(abs),Kendo(abs),jQM(abs),Lungo(abs)},
  columns/Uitdaging/.style={column name=\textbf{Uitdaging}, column type={l}},
  columns/Max/.style={column name=\textbf{Max}, column type={l}},    
  columns/jQM(abs)/.style={column name=\textbf{\jqma}, column type={c}},
  columns/ST(abs)/.style={column name=\textbf{\sta}, column type={c}},
  columns/Lungo(abs)/.style={column name=\textbf{\lungoa}, column type={c}},
  columns/Kendo(abs)/.style={column name=\textbf{\kendoa}, column type={c}},
  every head row/.style={
    before row=\toprule,
    after row=\midrule},
  every last row/.style={
  	before row=\midrule,
    after row=\bottomrule}
]{tabellen/gebruik/lijsten.csv}
\caption{Gebruik van \uit{lijsten} voor \st{}~(\sta), \kendo{}~(\kendoa), \jqm{}~(\jqma) en \lungo{}~(\lungoa).}
\label{tabel:evaluatie-gebruik-lijsten}
\end{table}

\paragraph{\st}
Zoals besproken in de vorige uitdaging kan een lijst voorzien worden van een sjabloon.
Dit sjabloon kan HTML-code of een instantie van de \code{Ext.XTemplate} klasse zijn.
De eerste definieert met HTML-tags de lay-out van de lijstelementen,  de tweede is geavanceerder.
De functionaliteiten van \code{Ext.XTemplate} zijn:
\begin{itemize}
  \item Doorlopen van een rij.
  \item Conditionele processen met de basis operatoren.
  \item Ondersteuning voor basis wiskundige operaties.
  \item Uitvoeren van willekeurige \js-code.
  \item Eigen functies in het sjabloon oproepen.
\end{itemize}
Het formateren van een datum kan door de \code{date} methode van \code{Ext.util.Format} op te roepen in het sjabloon.

Zoals reeds besproken bij \uit{vullen}, kan een formulier worden ingevuld met een \code{navigationview}.
Deze \code{view} heeft een \code{push} en \code{pop} methode om een \code{view} op een \code{stack} te plaatsen of af te halen.
Om een \code{view} te tonen die hoort bij een lijstelement wordt gebruik gemaakt van deze \code{navigationview} en de \code{push} methode.
De lijst wordt in een \code{navigationview} ingesloten.
Het aanklikken van een lijstelement veroorzaakt een \code{disclosure} gebeurtenis.
Een \code{controller} kan deze gebeurtenis opvangen en de \code{push} methode op de \code{navigationview} oproepen.
De methode kan geparameteriseerd worden met een modelinstantie, analoog als de \code{setRecord} methode van een formulierpaneel.

Het sorteren van een lijst kan automatisch met een \code{store}.
Een \code{store} kan aan een lijst worden gekoppeld zodat alle modelinstanties van de \code{store} in de lijst worden weergegeven.
De \code{store} moet dan voorzien worden van een \code{sorter}.
Deze kan modelinstanties van een \code{store} sorteren op basis van eigenschappen van het bijhorende model.
Ook kan de richting van sorteren worden geconfigureerd.
Meerdere \code{sorters} definiëren is mogelijk voor het geval er gelijkheden op vorige niveau's optreden.
Een \code{store} sorteren kan ook expliciet door de \code{sort} methode op de \code{store} op te roepen.

\paragraph{\kendo}
Lijsten worden met een \code{listview} als \code{data-role} weergegeven.
Wanneer lijsten met een \code{DataSource} gebonden worden zullen alle instanties van de \code{DataSource} als element in de lijst verschijnen.
%todo binding tussen js lijst en listview
De opmaak van lijsten kan met \kendo{} \code{Templates} worden uitgedrukt.
Deze sjablonen zijn apparte scripts van het type \code{text/x-kendo-template} en moeten in een HTML-bestand worden geschreven.
De sjablonen hebben toegang tot de velden van de modelinstanties die aan de lijst zijn toegekend.
Hiervoor moet de veldnaam tussen \term{hashtags} worden gebruikt.
Binnen de scripts is ook mogelijk om \js-functies op te roepen door de functie ook binnen \term{hashtags} te schrijven.

De link van elk lijstelement moet in het sjabloon worden gedefinieerd.
Om de elementen uit het uitgavenoverzicht te linken, werd gebruik gaakt van een geparameteriseerde \code{view}.
Dit laat toe om parameters in de link naar de \code{view} op te geven,  analoog aan HTTP GET-verzoeken.
De parameter die wordt doorgegeven is de id van de uitgave.
Wanneer naar de \code{view} wordt genavigeerd, zal een functie met opgegeven parameter worden uitgevoerd.
De functie zal een \code{ObservableObject} laden met data van de uitgave die hoort bij de meegekregen id.
Het \code{ObservableObject} is gekoppeld aan een formulier dat automatisch zal worden ingevuld zodra het object wordt geïnitialiseerd.

Het sorteren van een lijst kan door de \code{DataSource} die aan de lijst is gekoppeld van een sorteereigenschap te voorzien.
Deze sorteereigenschap bepaalt het veld waarop gesorteerd moet worden en eventueel een sorteerrichting.

\paragraph{\jqm}
Het laden van data in een lijst dient zelf geprogrammeerd te worden.
Hiervoor wordt gebruik gemaakt van de \code{.each()}-functie van jQuery om ieder data-item te overlopen.
Per item moet een \code{append} gebeuren van een lijstitem op de lijst.
In dit item wordt de template van de lijst geschreven.
Na alle elementen te hebben overlopen, moet de lijst ververst worden zodat \jqm{} de correcte lay-out toepast op de volledige lijst

Het klikbaar maken van de gegenereerde lijstitems gebeurt bij het genereren van de items zelf.
Er kunnen hiervoor twee manieren gekozen worden.
Enerzijds kan worden gebruik gemaakt van de \code{<a>}-tags in HTML.
Anderzijds kan ieder lijstitem een \code{id} krijgen, waarna een \code{click}-\term{event}  wordt gebonden aan al deze elementen.
Op basis van de \code{id} kan de uit te voeren actie bepaald worden.

Het sorteren van data werd geïmplementeerd door eerst in \js{} een vergelijkingsfunctie te schrijven.
Daarna wordt deze functie meegegeven aan de sorteerfunctie die ook in \js{} aanwezig is.
Er komt dus geen functionaliteit van het raamwerk om data te sorteren.

\paragraph{\lungo}
Het laden van data in een lijst dient zelf geprogrammeerd te worden.
Hiervoor wordt gebruik gemaakt van de functie \code{\$\$.each} van QuoJS om ieder element van de array te overlopen.
Per element gebeurt een \code{append} op de lijst waar de data dient geladen te worden.
Het toegevoegde lijstitem bepaalt de template van de lijst.

Het klikbaar maken van de gegenereerde lijstitems gebeurt bij het genereren van de items zelf.
Er kunnen hiervoor twee manieren gekozen worden (zie ook \jqm{}).
Enerzijds kan worden gebruik gemaakt van de \code{<a>}-tags in HTML.
Anderzijds kan ieder lijstitem een \code{id} krijgen, waarna een \code{click}-\term{event}  wordt gebonden aan al deze elementen.
Op basis van de \code{id} kan de uit te voeren actie bepaald worden.

Het sorteren van data gebeurt door de aangeboden functies van het raamwerk zelf.
De functie \code{Lungo.Core.orderByProperty} maakt het mogelijk om te sorteren volgens een bepaalde eigenschap, zowel oplopend als aflopend.


%%%%%%%%%%%%%

\subsection{\uit{pdf}}
\label{sec:evaluatie-gebruik-pdf}

In tabel \ref{tabel:evaluatie-gebruik-pdf} worden de resultaten getoond van de twee deeluitdagingen van \uit{pdf}.
Onder de tabel wordt per raamwerk verklaard waarom dat resultaat werd behaald.

\begin{table}[H]
\centering
\pgfplotstabletypeset[
  begin table=\begin{tabular}{p{8cm} p{0.8cm} p{0.8cm} p{0.8cm} p{0.8cm} p{0.3cm}},
  end table=\end{tabular},
  skip coltypes=true,
  col sep=comma,
  string type,
  header=true,
  columns={Uitdaging,Max,ST(abs),Kendo(abs),jQM(abs),Lungo(abs)},
  columns/Uitdaging/.style={column name=\textbf{Uitdaging}, column type={l}},
  columns/Max/.style={column name=\textbf{Max}, column type={l}},    
  columns/jQM(abs)/.style={column name=\textbf{\jqma}, column type={c}},
  columns/ST(abs)/.style={column name=\textbf{\sta}, column type={c}},
  columns/Lungo(abs)/.style={column name=\textbf{\lungoa}, column type={c}},
  columns/Kendo(abs)/.style={column name=\textbf{\kendoa}, column type={c}},
  every head row/.style={
    before row=\toprule,
    after row=\midrule},
  every last row/.style={
  	before row=\midrule,
    after row=\bottomrule}
]{tabellen/gebruik/pdf.csv}
\caption{Gebruik van \uit{pdf} voor \st{}~(\sta), \kendo{}~(\kendoa), \jqm{}~(\jqma) en \lungo{}~(\lungoa).}
\label{tabel:evaluatie-gebruik-pdf}
\end{table}

\paragraph{\st}
Het tonen van een PDF steunt op een plug-in van Fiedler~\cite{Fiedler2012} en kan op de Sencha Market gevonden worden op \exturl{market.sencha.com/extensions/pdf-viewer-panel}.
Het tonen van een PDF-bestand kan door de huidige \code{view} te wijzigen naar een \code{Ext.ux.PDF view}.
Deze \code{view} bestaat uit een paneel met een hoofdtekst.
Het paneel toont één pagina van het PDF-bestand,  de hoofdtekst bevat de navigatie naar andere pagina's.

Om de plug-in in de POC in te passen waren echter twee aanpassingen noodzakelijk.
Het PDF-bestand moet via een POST verzoek worden opgehaald waarbij parameters het exacte PDF-bestand aanduiden.
Ook moest er een terugknop in de hoofdtekst van het paneel worden aangebracht om terug naar het overzicht van doorgestuurde formulieren te gaan.
Beide aanpassingen moesten in het \js-bestand van de plug-in worden aangebracht.

\paragraph{\kendo}
De implementatie voor het tonen van het PDF-bestand is analoog als de implementatie met \jqm{}.
%TODO terug aanhalen? Aan elk lijstelement in het overzicht van uitgavenformulieren wordt een functie gebonden.

\paragraph{\jqm}
AJAX is bedoeld om tekst op te halen, maar geen ruwe data zoals een PDF~\cite{Scott2009}. 
Hierdoor werd gebruik gemaakt van een verborgen formulier met de nodige parameters die de PDF ophaalt bij de backend. 
Bij het klikken op een lijstitem in het overzicht, wordt dit verborgen formulier opgestuurd naar de \term{backend} die dan een PDF teruggeeft in de browser. 
Het weergeven van de PDF wordt overgelaten aan het mobiel apparaat dat de correcte applicatie hiervoor opstart.

\paragraph{\lungo}
Om dezelfde reden als \jqm{} werd ook hier geen gebruik gemaakt van AJAX om de PDF op te vragen.
Een verborgen formulier, net zoals dat voor \jqm{} het geval was, werd gebruikt om de PDF op te halen.
Het weergeven van de PDF wordt overgelaten aan het mobiel apparaat dat de correcte applicatie hiervoor opstart.

%%%%%%%%%%%%%

\subsection{\uit{offline}}
\label{sec:evaluatie-gebruik-offline}

In tabel \ref{tabel:evaluatie-gebruik-offline} worden de resultaten getoond van de twee deeluitdagingen van \uit{offline}.
Onder de tabel wordt per raamwerk verklaard waarom dat resultaat werd behaald.

\begin{table}[H]
\centering
\pgfplotstabletypeset[
  begin table=\begin{tabular}{p{8cm} p{0.8cm} p{0.8cm} p{0.8cm} p{0.8cm} p{0.3cm}},
  end table=\end{tabular},
  skip coltypes=true,
  col sep=comma,
  string type,
  header=true,
  columns={Uitdaging,Max,ST(abs),Kendo(abs),jQM(abs),Lungo(abs)},
  columns/Uitdaging/.style={column name=\textbf{Uitdaging}, column type={l}},
  columns/Max/.style={column name=\textbf{Max}, column type={l}},    
  columns/jQM(abs)/.style={column name=\textbf{\jqma}, column type={c}},
  columns/ST(abs)/.style={column name=\textbf{\sta}, column type={c}},
  columns/Lungo(abs)/.style={column name=\textbf{\lungoa}, column type={c}},
  columns/Kendo(abs)/.style={column name=\textbf{\kendoa}, column type={c}},
  every head row/.style={
    before row=\toprule,
    after row=\midrule},
  every last row/.style={
  	before row=\midrule,
    after row=\bottomrule}
]{tabellen/gebruik/offline.csv}
\caption{Gebruik van \uit{offline} voor \st{}~(\sta), \kendo{}~(\kendoa), \jqm{}~(\jqma) en \lungo{}~(\lungoa).}
\label{tabel:evaluatie-gebruik-offline}
\end{table}

\paragraph{\st}
Zoals besproken bij \uit{ajax} kan een \code{store} voozien worden van een \code{proxy} die data opslaat aan klantzijde.
Deze \code{store} maakt gebruik van de HTML5-localStorage API.
Om inlog gegevens en onverzonden onkosten lokaal te bewaren moeten twee \code{stores} met deze \code{proxy} worden gedefinieerd.
Een belangrijke opmerking is dat geen twee \code{proxies} aan een \code{store} kunnen worden toegevoegd.
Gegevens van een gebruiker moeten van de server worden opgehaald - met een AJAX-\code{proxy} - en lokaal worden opgeslagen - met een \term{local storage} \code{proxy}.
Hiervoor zijn twee verschillende \code{store} instanties nodig die gesynchronizeerd moeten worden!
Bij het laden van de applicatie zal de \code{store} die gebruikers lokaal opslaat, op data worden gecontroleerd.
Indien er data wordt gevonden,  was de gebruiker reeds ingelogd.
De applicatie zal dan meteen naar het startscherm navigeren.

Om onverzonden uitgaven lokaal op te slaan, moeten de uitgaven ook aan een \code{store} met lokale \code{proxy} worden toegevoegd.
Deze worden dan automatisch naar de \term{local storage} weggeschreven.
De controle op onverzonde uitgaven wordt herleid tot het controleren van data in de \code{store}.
Het verwijderen van onverzonden lokale uitgaven kan door de data in de \code{store} te wissen.

De applicatie offline beschikbaar maken wordt ondersteund door Sencha Cmd~\cite{Sencha2012}.

Een applicatie,  zoals geinitialiseerd door Sencha Cmd,  moet alle benodigde \js-, en CSS-bestanden in een JSON-bestand onderbrengen.
Een \code{microloader} zal de afhankelijke bestanden automatisch laden bij het opstarten van de applicatie.

Het bouwen en uitrollen van een applicatie kan op vier niveaus:
\begin{description}
  \item [testing] maakt een testapplicatie om de kwaliteit te testen.  \js- en CSS-bestanden worden samengevoegd maar niet verkleind om makkelijk te debuggen
  \item [package] maakt een zelfstandige applicatie die verspreidbaar is en vanop een bestandensyteem,  zonder web server,  kan lopen.
  \item [production] maakt een applicatie die op een webserver beschikbaar wordt gemaakt waarvan de \js-, en CSS- bestanden zijn samengevoegd en verkleind.  Het maakt de applicatie ook offline beschikbaar door gebruik te maken van de HTML5-applicatie cache.  Ook is het mogelijk de applicatie op te waarderen naar een nieuwe versie.
  \item [native] maakt een \term{native} applicatie die op het Android of iOS besturingssysteem kan lopen.
\end{description}
Om de applicatie offline beschikbaar te maken moet de applicatie gebouwd worden voor productie.
De tool zal automatisch een manifest bestand aanmaken die alle vereiste bestanden bevat.

\paragraph{\kendo}
%user opslaan => jqm (check bij employeedatasource)
Vanuit het raamwerk komt er standaard geen ondersteuning om gegevens offline te bewaren.
Hiervoor werd gebruik gemaakt van \code{localStorage}, wat gespecificeerd is in HTML5.
Het controleren of \code{localStorage} al dan niet wordt ondersteund, gebeurd door Modernizr~\cite{Modernizr2012}.

Na het aanmelden zullen alle gegevens van de werknemer geserialiseerd worden opgeslagen.
Deze gegevens zitten in een \code{OservableObject} en kunnen met de \code{toJSON} methode als JSON-object worden verkregen.
Het serialiseren zelf kan via de \code{JSON.stringify} methode.
Wanneer het aanmeldscherm wordt getoond zal gecontroleerd worden of er gegevens van een werknemer lokaal beschikbaar zijn.
Indien dit het geval is, wordt automatisch naar naar het startscherm genavigeerd.

Wanneer een uitgave wordt toegevoegd zal deze aan een \code{DataSource} en in de \code{localStorage} worden toegevoegd.
De controle op reeds bestaande uitgaven zoekt naar reeds bestaande uitgaven in de \code{localStorage}.
Nadat een uitgaveformulier is verzonden worden de uitgaven uit de \code{localStorage} verwijderd.

    
%expense opslaan => add Expense toevoegen aan localstorage

Het offline beschikbaar maken van de applicatie wordt vanuit \kendo{} zelf niet ondersteund.
De HTML5 Application Cache kan zelf worden aangemaakt.
Om het proces te vergemakkelijken werd het \code{.appcache}-bestand dat een lijst is van alle offline-bestanden, gegenereerd aan de hand van Yeoman~\cite{Yeoman2013}.

\paragraph{\jqm}
Vanuit het raamwerk komt er standaard geen ondersteuning om gegevens offline te bewaren.
Hiervoor werd gebruik gemaakt van \code{localStorage}, wat gespecificeerd is in HTML5.
Omdat het kan voorkomen dat \code{localStorage} niet wordt ondersteund, werd zelf een fallback geschreven die de data bijhoudt als \js{}-variabelen.
Natuurlijk wordt bij het vernieuwen van de applicatie deze data gewist, maar kan de applicatie wel in eenzelfde browsersessie volledig worden gebruikt.
Het controleren of \code{localStorage} al dan niet wordt ondersteund, gebeurd door Modernizr~\cite{Modernizr2012}.


Het startscherm werd als eerste scherm gekozen, omdat \jqm{} altijd het eerste scherm in de HTML-code in inlaadt.
Indien gemerkt wordt dat de gebruiker niet aangemeld was, dan wordt hij doorverwezen naar het inlogscherm.
Hierdoor zal, als de applicatie offline is, de gebruiker kunnen navigeren doorheen de applicatie doordat zijn gebruikersgegevens werden bewaard.

Het offline beschikbaar maken van de applicatie wordt vanuit \jqm{} zelf niet ondersteund, omdat van de HTML5 Application Cache kan worden gebruik gemaakt.
Om het proces te vergemakkelijken werd het \code{.appcache}-bestand dat een lijst is van alle offline-bestanden, gegenereerd aan de hand van Yeoman~\cite{Yeoman2013}.

\paragraph{\lungo}
Het raamwerk biedt zelf functies aan om data op te slaan.
Intern zal het gebruik maken van \code{localStorage}, wat gespecificeerd is in HTML5.
Indien het niet ondersteund wordt op het apparaat, zal het raamwerk zelf een fallback voorzien.

Het offline beschikbaar maken van de applicatie wordt vanuit \lungo{} zelf niet ondersteund, omdat van de HTML5 Application Cache kan worden gebruik gemaakt.
Om het proces te vergemakkelijken werd het \code{.appcache}-bestand dat een lijst is van alle offline-bestanden, gegenereerd aan de hand van Yeoman~\cite{Yeoman2013}.
\section{Ondersteuning}
\label{sec:evaluatie-ondersteuning}

\begin{table}[H]
\centering
\pgfplotstabletypeset[
  col sep=comma,
  string type,
  header=true,
  columns={Apparaat,jQM,ST,Kendo,Lungo},
  columns/Apparaat/.style={column name=\textbf{Apparaat}, column type={l}},  
  columns/jQM/.style={column name=\textbf{\jqma}, column type={c}},
  columns/ST/.style={column name=\textbf{\sta}, column type={c}},
  columns/Kendo/.style={column name=\textbf{\kendoa}, column type={c}},
  columns/Lungo/.style={column name=\textbf{\lungoa}, column type={c}},
  every head row/.style={
    before row=\toprule,
    after row=\midrule},
  every last row/.style={
  	before row=\toprule,
 	after row=\bottomrule}
]{tabellen/ondersteuning.csv}
\caption{Samenvattende tabel voor ondersteuningscriterium}
\label{tabel:evaluatie-ondersteuning}
\end{table}
\section{Performantie}
\label{sec:evaluatie-performantie}

In sectie~\ref{sec:evaluatie-downloadtijd} zal eerst de downloadtijd worden besproken.
Daarna zal in sectie~\ref{sec:evaluatie-gebruikerservaring} de gebruikerservaring worden besproken.

%%%%%%%%%%%%%%%%%%

\subsection{Downloadtijd}
\label{sec:evaluatie-downloadtijd}

Op figuur \ref{fig:performantie} wordt de gemiddelde downloadtijd van de POC en login, zowel gewoon als uit cache, voor de vier raamwerken getoond.

\begin{figure}[H]
  \centering
  \includegraphics[width=\textwidth]{figuren/performance.pdf}
  \caption{Gemiddelde downloadtijd van POC,  POC uit cache,  login en login uit cache voor elk raamwerk. Minder is beter.}
  \label{fig:performantie}
\end{figure}

\lungo{} behaalt de eerste plaats.
Als er gekeken wordt naar de POC heeft \lungo{} maar een derde van de tijd nodig ten opzichte van het traagste raamwerk, \st.
\jqm{} en \kendo{} behalen respectievelijk een tweede en derde plaats.
Aangezien niet alles werd geïmplementeerd in de POC voor \lungo{}, zou men kunnen stellen dat dat de reden is.
Als er echter wordt gekeken naar de loginapplicatie waar alles wel vergelijkbaar is, dan blijft \lungo{} het snelste raamwerk.
Het is zelfs meer dan de helft sneller dan \jqm{}, \kendo{} of \st{}.
Deze drie raamwerken behalen quasi dezelfde opstarttijd.
\kendo{} en \st{} nemen respectievelijk voorlaatste en laatste plaats in.
Dit wordt verklaard door het gebruik van een architectuur waardoor een grotere \js{}-code dient te worden gedownload ten opzichte van raamwerken die geen architectuur afdwingen (zie ook tabel \ref{tabel:raamwerken-tabel}).

Als naar de versie uit cache wordt gekeken voor zowel POC als loginapplicatie, scoren \kendo{}, \jqm{} en \lungo{} hetzelfde.
Daarentegen behaalt \st{} telkens een vele tragere tijd.
Enerzijds komt dit doordat de drie eerstgenoemde raamwerken enkel gebruik maken van HTML5 Application Cache.
\st{} gebruikt daarnaast ook nog een eigen mechanisme waardoor de grotere laadtijd wordt verklaard (zie \ref{sec:performantie-st}).
Anderzijds gebruiken de drie eerstgenoemde raamwerken Yeoman om de applicatie te bouwen.
De webapplicaties gemaakt \st{} gebruiken daarentegen Sencha Cmd.

%%%% De cache factor van ST is constant (gemiddeld 1,8)
%%%% De andere raamwerken hebben een beduidend grotere cache factor 

Indien \st{} buiten beschouwen wordt gelaten, duurt het eerste keer laden van de POC gemiddeld 5,73 seconden. 
Het laden van de versie uit cache duurt slechts gemiddeld 400 milliseconden.
De eerste keer laden van de loginapplicatie duurt gemiddeld 3,32 seconden.
Indien deze uit cache komt, duurt dit nog slechts gemiddeld 420 milliseconden.
Dit zijn aanvaardbare tijden volgens Jakob Nielsen~\cite{Nielsen1993} doordat de tijden uit cache onder de seconde blijven.
Volgens Nielsen dient er wel feedback over het laden te worden gegeven aan de gebruiker.
Dit neemt de browser voor zich door bij het laden een laadbalk of spinner te tonen.

%%%%%%%%%%%%%%%%%%

\subsection{Gebruikerservaring}
\label{sec:evaluatie-gebruikerservaring}
De gebruikerservaring via \js{} kon enkel worden opgemeten in \jqm{} en \kendo{}.
Bij de twee andere raamwerken werden de betreffende \term{events} niet gevonden om correct de tijd op te meten.
% TODO: eventueel referenties naar vragen op so
Doordat er maar data voor twee raamwerken voor handen was, werd deze methode vervangen door een subjectieve lijsttest.
Deze bestaat eruit de vlotheid van het scrollen door de lijst van 850 lijstelementen voor de vier raamwerken op de 8 apparaten te vergelijken.
Per apparaat word een score van 1, 2, 3 of 4 uitgedeeld aan de raamwerken.
Hierbij is 4 de beste score wat overeenkomt met het vlotste scrollen door de lijst relatief ten opzichte van de drie andere raamwerken.
Deze test werd uitgevoerd door twee personen.
% TODO: is dat nu goed of slecht als we dat maar met 2 personen hebben gedaan?
%Er dient echter opgemerkt te worden dat het tijdsbudget het niet toeliet om deze test te laten uitvoeren door minstens vijf personen te laten uitvoeren, wat voorgesteld wordt door Nielsen.

Om de totaalscore van een raamwerk te bepalen worden de scores voor dat raamwerk op ieder apparaat opgeteld.
In het bekomen eindklassement komt de hoogste totaalscore overeen met het raamwerk dat de vlotste scrolervaring aanbiedt. In tabel \ref{tabel:evaluatie-performantie-gebruikerservaring} wordt de totaalscore voor de gebruikerservaring getoond.

\begin{table}[H]
\centering
\pgfplotstabletypeset[
  begin table=\begin{tabular}{p{8cm} p{0.8cm} p{0.8cm} p{0.8cm} p{0.8cm} p{0.3cm}},
  end table=\end{tabular},
  skip coltypes=true,
  col sep=comma,
  string type,
  header=true,
  columns={Apparaat,ST,Kendo,jQM,Lungo},
  columns/Apparaat/.style={column name=\textbf{Apparaat}, column type={l}},
  columns/ST/.style={column name=\textbf{\sta}, column type={l}},  
  columns/jQM/.style={column name=\textbf{\jqma}, column type={l}},    
  columns/Kendo/.style={column name=\textbf{\kendoa}, column type={l}},   
  columns/Lungo/.style={column name=\textbf{\lungoa}, column type={l}},   
  every head row/.style={
    before row=\toprule,
    after row=\midrule},
  every last row/.style={
  	before row=\midrule,
    after row=\bottomrule}
]{tabellen/performantie-gebruikerservaring.csv}
\caption{Gebruikerservaring voor \st{}~(\sta), \kendo{}~(\kendoa), \jqm{}~(\jqma) en \lungo{}~(\lungoa).}
\label{tabel:evaluatie-performantie-gebruikerservaring}
\end{table}

\st{} behaalde de maximale score.
Dit wil zeggen dat op alle toestellen het scrollen door de lijst van \st{} het vlotst ging.
\jqm{} werd zes keer als tweede beste beoordeeld. 
Op de \htc{} liep \kendo{} vlotter,  op de \ipadi{} was \lungo{} nummer twee.
De lijst genereren met \kendo{} op iOS-toestellen was onmogelijk omdat de applicatie de browser liet crashen.
De reden alsook de grens waarom \kendo{} niet crasht op iOS-toestellen werd door tijdsbudget niet gecontroleerd.
Een mogelijke denkpiste is dat \kendo{} een overhead genereerd die het maximale toegelaten geheugen voor het iOS-besturingssysteem overschrijdt.
Op Android toestellen kon de \kendo{} lijst echter wel worden getoond.
De score van \kendo{} is dus slechts voor vier apparaten.


%%%%%%%%%%%%%%%%%%

\subsection{Performantie}
Aangezien de gebruikerservaring werd vervangen door de subjectieve lijsttest dient de formule om de totale performantie te berekenen~(zie \ref{eq:performantie}), te worden herschreven.
De opzet van de nieuwe formule is om een raamwerk dat slecht scoort op downloadtijd, maar sterk scoort op de subjectieve lijsttest, een middelmatige score te geven.
Aangezien deze laatste geen eenheid heeft en de eerstgenoemde uitgedrukt wordt in seconden, wordt de downloadtijd gedeeld door de gebruikerservaring. De score voor de performantie van een raamwerk $r$ wordt:
\begin{equation}
  \text{Performantie}'_r = \frac{\sum_{c=1}^{8}{\left(l_{r,c,\text{POC}}+l_{r,c,\text{POC}_{cache}}+l_{r,c,\text{login}}+l_{r,c,\text{login}_{cache}}\right)}}{\text{Gebruikerservaring}_r}
  \label{eq:performantie-enhanced}
\end{equation}

%TODO: invoegen van nieuwe formule 
% hierdoor wordt de formule aangepast, aangezien lijsttest (uitgedrukt in eenheid) en downloadtijd (uitgebrukt in seconden) niet kunnen worden opgeteld, daarom als fator gebruiken
% van de totale downloaded wordt gedeeld met die factor 

% nieuwe formule
% nieuwe tabel: 3 rijen (totale download, factor, nieuw abs totaal)

\begin{table}[H]
\centering
\pgfplotstabletypeset[
  begin table=\begin{tabular}{p{8cm} p{0.8cm} p{0.8cm} p{0.8cm} p{0.8cm} p{0.3cm}},
  end table=\end{tabular},
  skip coltypes=true,
  col sep=comma,
  string type,
  header=true,
  columns={Performantie,ST,Kendo,jQM,Lungo},
  columns/Performantie/.style={column name=\textbf{Performantie}, column type={l}},
  columns/ST/.style={column name=\textbf{\sta}, column type={l}},  
  columns/jQM/.style={column name=\textbf{\jqma}, column type={l}},    
  columns/Kendo/.style={column name=\textbf{\kendoa}, column type={l}},   
  columns/Lungo/.style={column name=\textbf{\lungoa}, column type={l}}, 
  every head row/.style={
    before row=\toprule,
    after row=\midrule},
  every last row/.style={
  	before row=\midrule,
    after row=\bottomrule}
]{tabellen/performantie.csv}
\caption{Overzicht van performantie voor \st{}~(\sta), \kendo{}~(\kendoa), \jqm{}~(\jqma) en \lungo{}~(\lungoa). Minder is beter.}
\label{tabel:evaluatie-gebruik}
\end{table}

%%%%% TOT HIER HIERSCHRIJVEN %%%%%%

In wat volgt zullen metrieken worden besproken die de score van de performantie zullen duiden.
De data van de metrieken is weergegeven in tabel~\ref{tabel:performantie-verklaring}.

\paragraph{Google Page Speed}
De score op 100 die Google Page Speed~\cite{Morgan2011} aan de applicatie toekent kan in tabel~\ref{tabel:performantie-verklaring} worden teruggevonden voor zowel de POC als de loginapplicatie.
\st{} scoort het best ($96$),  gevolgd door \lungo{} ($88$),  \jqm{}($71$) en \kendo{}($66$).
Dezelfde trend kan bij de loginapplicatie teruggevonden worden.
Het enige verschil is dat bij de POC van \kendo{} de afbeeldingen niet optimaal gecomprimeerd zijn.
Bij de loginapplicatie van \kendo{} is dit wel gebeurd.
Door deze uitbreiding krijgt de loginapplicatie van \kendo{} en beter score dan \jqm{}.

Er kan geconcludeerd worden dat Sencha Cmd de applicatie optimaal weet te bouwen.
Hoewel \st{} de meeste tijd vraagt om te laden, zal het na het laden sneller werken.
Dit wordt bevestigd in de test over gebruikservaring.

\paragraph{Downloadgrootte en HTML-code}
De HAR-bestanden die werden gebruikt om de laadtijd op te meten bevatten ook de grootte van de pakketten die moeten worden opgehaald.
Omdat pakketten verloren gaan zullen ontvangen bestanden incompleet zijn en moeten ze worden herverzonden. %TODO Sander: klopt dat?
Hierdoor zal het aantal ontvangen bytes variëren van meting tot meting.
De performantietesten werden op acht toestellen uitgevoerd en elke test werd drie keer uitgevoerd.
Het gemiddelde van alle downloadgroottes bepaalt de grootte zoals deze kan worden teruggevonden in tabel~\ref{tabel:performantie-verklaring}.
Bij de POC moet \st{} de meeste data ophalen ($1126.4$kB),  gevolgd door \kendo{} ($194.65kB$kB), \lungo{} ($249.08$kB) en \jqm{} ($194.65$kB).
Opmerkelijk is dat \lungo{} meer data moet ophalen maar toch een snellere laadtijd behaald.
Ook het verschil in downloadgrootte tussen de POC en loginapplicatie is opmerkelijk.
Bij \jqm{} kan slechts een daling van $4.74\%$ worden waargenomen.
\st{} en \lungo{} waren gelijkaardig en de downloadgrootte daalde respectievelijk met $78.02\%$ en $76.01\%$.
De reductie in data bij \kendo{} was $45.30\%$.

Bij alle applicaties werd de \js{}- en CSS-code verkleind en samengevoegd.
De HTML-code werd niet gewijzigd.
Tabel~\ref{tabel:performantie-verklaring} bevat het aantal lijnen HTML-code dat moest worden opgehaald.
Hieruit is duidelijk te zien welke raamwerken opmaakgedreven zijn.
%TODO Sander: moet hier nog meer vermeld worden?

%TODO geen komma's in de tabellen als dat niet nodig is
\begin{table}[H]
\centering
\pgfplotstabletypeset[
  begin table=\begin{tabular}{p{8cm} p{1cm} p{1cm} p{1cm} p{1cm}},
  end table=\end{tabular},
  skip coltypes=true,
  col sep=comma,
  string type,
  header=true,
  columns={Performantie Metrieken,jQM,ST,Kendo,Lungo},
  columns/Performantie Metrieken/.style={column name=\textbf{Performantie}, column type={l}},  
  columns/ST/.style={column name=\textbf{\sta}, column type={c}},
  columns/jQM/.style={column name=\textbf{\jqma}, column type={c}},
  columns/Kendo/.style={column name=\textbf{\kendoa}, column type={c}},
  columns/Lungo/.style={column name=\textbf{\lungoa}, column type={c}},
  every head row/.style={
    before row=\toprule,
    after row=\midrule},
  every last row/.style={
    after row=\bottomrule}
]{tabellen/performantie/performantie-verklaring.csv}
\caption{Metrieken gebruikt bij de verklaring van performantiecriterium voor \st{}~(\sta), \kendo{}~(\kendoa), \jqm{}~(\jqma) en \lungo{}~(\lungoa).}
\label{tabel:performantie-verklaring}
\end{table}


%%%%%% TODO alles naar appendix %%%%%%%


\subsection{\st}
\label{sec:performantie-st}
Op figuur~\ref{fig:performantie-st} worden de gemiddelde downloadtijd van \st{} getoond op elk apparaat.

Voor de POC is een dalende downloadtijd waarneembaar wanneer het Android-apparaat recenter wordt.
De downloadtijd van de POC op de \gs{} duurde gemiddeld $31.43$s!
Gemiddeld moeten Android toestellen $5$ seconden langer laden in vergelijking met iOS toestellen.
Dit gemiddelde wordt sterk beinvloed door de trage laadtijd van de \gs{}.

\st{} heeft een andere aanpak voor het cachen van een applicatie door de introductie van een \term{Delta-update} mechanisme.
Dit mechanisme wil voorkomen dat bij een kleine aanpassing in de code,  alle bestanden opnieuw moeten worden opgehaald die in het \term{manifest} bestand staan opgelijst.
De \term{Micro-loader} is verantwoordelijk voor het asynchroon ophalen van all benodigde \js{}- en CSS-bestanden.
Na het bouwen van een applicatie met Sencha Cmd,  zullen de gewijzigde bestanden gearchiveerd worden en worden de veranderingen tussen elke versie opgeslagen.
Na het laden van de applicatie, zal de \code{Micro-loader} met een GET-verzoek controleren op wijzigingen.
Dit GET-verzoek zal de grootste tijd voor zijn rekening nemen bij de laadtijden bij applicaties uit de cache.
\st{} heeft er dus voor gekozen om aan performantie in te boeten ten voordele van het update mechanisme.
%TODO referentie http://www.sencha.com/blog/behind-sencha-command-and-the-build-process


Een laatste opmerking die bij \st{} moet worden gemaakt, is dat AJAX-verzoeken van een \code{proxy} naar een ander domein altijd vooraf worden gegaan met een OPTIONS-verzoek.
Dit is een verzoek om informatie over de beschikbare opties van het communicatiekanaal op te vragen.
Standaard zet \st{} de \code{X-Requested-With} op XMLHttpRequest en hierdoor zal de browser een OPTIONS-verzoek als \term{preflight} sturen.
%Setting custom headers on XHR requests triggers a preflight request. %http://remysharp.com/2011/04/21/getting-cors-working/
%http://stackoverflow.com/questions/10236056/when-loading-a-store-in-sencha-touch-2-how-can-i-stop-the-additional-options-ht
% POST /resources/userService/login?_dc=1368367749599 HTTP/1.1
% Host: kulcapexpenseapp.appspot.com
% Connection: keep-alive
% Content-Length: 54
% Origin: http://sandervanloock.github.io
% User-Agent: Mozilla/5.0 (X11; Linux i686) AppleWebKit/537.11 (KHTML, like Gecko) Chrome/23.0.1271.64 Safari/537.11
% Content-Type: application/x-www-form-urlencoded; charset=UTF-8
% Accept: */*
% Referer: http://sandervanloock.github.io/HTMobieL/Sencha/build/ExpenseApp/production/index.html
% Accept-Encoding: gzip,deflate,sdch
% Accept-Language: nl-NL,nl;q=0.8,en-US;q=0.6,en;q=0.4
% Accept-Charset: ISO-8859-1,utf-8;q=0.7,*;q=0.3
% 
% Accept:*/*
% Accept-Charset:ISO-8859-1,utf-8;q=0.7,*;q=0.3
% Accept-Encoding:gzip,deflate,sdch
% Accept-Language:nl-NL,nl;q=0.8,en-US;q=0.6,en;q=0.4
% Connection:keep-alive
% Content-Length:23
% Content-Type:application/x-www-form-urlencoded; charset=UTF-8
% Host:kulcapexpenseapp.appspot.com
% Origin:http://sandervanloock.github.io
% Referer:http://sandervanloock.github.io/HTMobieL/Sencha/build/ExpenseApp/production/index.html
% User-Agent:Mozilla/5.0 (X11; Linux i686) AppleWebKit/537.11 (KHTML, like Gecko) Chrome/23.0.1271.64 Safari/537.11
% X-Requested-With:XMLHttpRequest

\begin{figure}[H]
  \centering
  \includegraphics[width=\textwidth]{figuren/performance-st.pdf}
  \caption{Gemiddelde downloadtijden van \st{} voor POC,  POC uit cache,  Login en Login uit cache voor elk apparaat.}
  \label{fig:performantie-st}
\end{figure}

\subsection{\kendo}
Op figuur~\ref{fig:performantie-kendo} worden de gemiddelde downloadtijd van \kendo{} getoond op elk apparaat.

De \gtab{} vertoond de hoogste laadtijd,  gevolgd door de \iphoneiii{} en \htc.
Opmerkelijk is dat de gecachete versie van de loginapplicatie op de \nexus{} $10$ keer trager laadt dan de gecachete versie van de POC.
Het ophalen van een gecachete applicatie werkt bij de \gs{} het traagst.

Bij \kendo{} is er geen opmerkelijk verschil waarneembaar tussen Android en iOS toestellen (Android gemiddeld slechts $60$ms trager).

\begin{figure}[H]
  \centering
  \includegraphics[width=\textwidth]{figuren/performance-kendo.pdf}
  \caption{Gemiddelde downloadtijden van \kendo{} voor POC,  POC uit cache,  login en login uit cache voor elk apparaat.}
  \label{fig:performantie-kendo}
\end{figure}

\subsection{\jqm}
Op figuur~\ref{fig:performantie-jqm} worden de gemiddelde downloadtijd van \jqm{} getoond op elk apparaat.

%TODO Sander: fout (\gs is minst recent...
Voor de POC is een dalende downloadtijd waarneembaar wanneer het Android-apparaat recenter wordt.
Dit is echter ook zo voor de iPads van Apple, maar voor de iPhones stijgt de downloadtijd.
Er zijn minimale verschillen bij de gecachete versie van de POC, waarbij het het langste duurt op de \ipadi{}.

Als de loginapplicatie wordt bekeken, wordt hetzelfde waargenomen als voor de POC.
Enkel bij de Android-apparaten wordt de downloadtijd trager, naarmate het toestel recenter wordt.
Dit is in tegenstelling tot de POC.

Een opmerkelijke waarneming is dat het langer duurt om de gecachete versie van het loginscherm te laden dan de volledige POC.

\begin{figure}[H]
  \centering
  \includegraphics[width=\textwidth]{figuren/performance-jquery.pdf}
  \caption{Gemiddelde downloadtijd van \jqm{} voor POC,  POC uit cache, login en login uit cache voor elk apparaat.}
  \label{fig:performantie-jqm}
\end{figure}

\subsection{\lungo}

\begin{figure}[H]
  \centering
  \includegraphics[width=\textwidth]{figuren/performance-lungo.pdf}
  \caption{Gemiddelde downloadtijd van \lungo{} voor POC,  POC uit cache,  Login en Login uit cache voor elk apparaat.}
  \label{fig:performantie-lungo}
\end{figure}

\section{Vergelijkingsoverzicht}
\label{sec:evaluatie-spinnenweb}

Figuur \ref{fig:spinnenweb-final} toont een overzicht van de scores van de vier raamwerken op de vijf criteria in de vorm van een spinnenweb.

\begin{figure}[H]
  \centering
  \includegraphics[width=\textwidth]{figuren/spidergraph-final-nl.pdf}
  \caption{Vergelijkingsoverzicht met de vijf vergelijkingscriteria voor \st{},  \kendo{},  \jqm{} en \lungo{}.}
  \label{fig:spinnenweb-final}
\end{figure}

De formules van de vergelijkingscriteria voor het plotten van het spinnenweb kunnen teruggevonden worden in sectie \ref{sec:vergelijking-spinnenweb}.
Daar werden de gerelativeerde formules voorgesteld om een score tussen $0$ en $1$ te bekomen.
Ook werden de formules voor productiviteit en performantie geinverteerd omdat voor deze criteria geldt:  hoe lager de score, hoe beter het raamwerk.

De formules van productiviteit en performantie zijn echter gewijzigd na de evaluatie en werden formule \ref{eq:productiviteit-enhanced} en \ref{eq:performantie-enhanced}.
Hun nieuwe relatieve formules zijn dan respectievelijk

\begin{equation}
  REL(\text{Productiviteit'}_r) = \frac{\left(\text{Productiviteit}'_R\right)^{-1}}{\max_{m}\{\left(\text{Productiviteit}'_m\right)^{-1}\}}
  \label{eq:rel-productiviteit-final}
\end{equation}

en

\begin{equation}
  REL(\text{Performantie'}_r)= \frac{\left(\text{Performantie}'_r\right)^{-1}}{\max_{m}\{\left(\text{Performantie}'_m\right)^{-1}\}}
  \label{eq:rel-performantie-final}
\end{equation}.

De oppervlakten van de vier vijfhoeken is de volgende:
\begin{description}
 \item [\jqm{}] $1.87$
 \item [\kendo{}] $1.32$
 \item [\lungo{}] $0.88$
 \item [\st{}] $0.73$
\end{description}

De eenheden komen overeen met de eenheid zoals weergegeven op de assen.
In wat volgt zullen de oppervlaktes van de raamwerken verklaart worden door conclusies te trekken over de resultaten van alle criteria.

% but...

\jqm{} heeft de grootste oppervlakte ($1.87$) en is de winnaar.
Een van de belangrijkste factoren die \jqm{} tot winnaar maakt is het feit dat er geen architectuur wordt afgedwongen.
Dit maakt dat de leercurve veel lager ligt ten voordele van de productiviteit.
Ook een betere documentatie en omkadering maakt \jqm{} productiever.
De grotere populariteit in vergelijking met \lungo{} maakt \jqm{} aantrekkelijker als er geen architectuur noodzakelijk is.
De afwezigheid van een architectuur brengt echter twee nadelen met zich mee.
Ten eerst zal het raamwerk minder functionaliteit kunnen aanbieden.
Door plug-ins en HTML5-kenmerken zal dit gebrek aan functionaliteit toch beperkt blijven.
Een tweede nadeel is de code die moet geschreven worden bij gebruik van het \jqm{} raamwerk.
\jqm{} is opmaakgedreven waarbij veel HTML-code moet worden geschreven.
Deze verbositeit maakt de code zeer gevoelig voor fouten. 
%TODO: link HTML <-> JS leggen is error prone ...

%TODO slagzin voor elk raamwerk? :Wanneer er voor een productief raamwerk,  zonder architectuur,  moet worden gekozen dat zeer populair is,  kies dan voor \jqm{}!

% jQM volgt % Waarom kiezen voor jqm:
% +/- geen architectuur (
    % minder gebruik
    % lagere leercurve (veel HTML, link met js is het lastigst)
    % meer code schrijven (error prone)
% + zeer populair (jQuery core) zie stackoverlow => vragen worden door een hele community opgelost (geen payed support)
% als een easy (geen architectuur) raamwerk + populair (belangrijke klanten,  gekend op social media)

\kendo{} bekleedt de tweede plaats met een opervlakte van $1.32$.
In tegenstelling tot \jqm{} heeft het wel een architectuur en is het dus meer bruikbaar.
De score van de productiviteit bevindt zich onder de helft maar in vergelijking met \st{} scoort het beter.
Het grote nadeel van \kendo{},  wat niet in het spinnenweb kan worden gezien,  is de hoge kost voor een licentie.
Het feit dat \kendo{} niet \term{open-source} is reflecteerd zich echter niet in de populariteit.
Van alle vier raamwerken is \kendo{} het minst performant.
De laadtijden waren lager in vergelijking met \st{} maar de crashes op iOS-toestellen zorgden voor een lage gebruikservaring.
Het feit dat \kendo{} een \term{native look-and-feel} aanbiedt is een belangrijk kenmerk maar dit kan ook niet op het spinnenweb worden teruggevonden.
De iOS-lay-out is sterk gelijkend is op de echte iOS-lay-out.
Dit was bij Android minder geslaagd.

% kendo is winnaar (uitstekend op pop, gebruik en ondersteuning % Waarom kiezen voor kendo:
% - architectuur => minder productief maar meer bruikbaar
% - hoge licentiekost => professionele support. Ondanks niet open source toch populiar (meest populair)
% + native look-and-feel => laadtijden stijgen?  meer op native applicatie (positieve impact niet geevalueerd) (iOS lay-out sterk gelijkend op
%echte iOS,  Android minder gelijkend, Blackberry en windows niet bekeken)

\lungo{} heeft de derde grootste oppervlakte ($0.88$).
Op drie van de vijf criteria scoort \lungo{} het slechtst,  voor de overige criteria is \lungo{} tweede.
Er is geen architectuur opgelegd en weer kan dit voor- en nadeel zowel in productiviteit en gebruik opgemerkt worden.
De voordelen bij productiviteit zijn echter minder dan \jqm{} omdat de ondersteuning en documentatie minimaal is.
De nadelen bij gebruik zijn dan weer uitvergroot omdat er niet zoveel functionaliteit en plug-ins als bij \jqm{} konden werden teruggevonden.
Positief is dan weer dat \lungo{} de beste laadtijden behaalden omdat \quo{}, de \js{}-bibliotheek van \lungo{},  geoptimaliseerd is voor mobiele apparaten.
De testen op gebruikservaring zwakte de performantie van \lungo{} af.
\lungo{} is veruit het minst populair.
Dit kan zowel met de literatuur als met sociale netwerken bevestigd worden.

% tweede is lungo % Waarom kiezen voor lungo:
% - productiviteit => verkeerde indruk (zonder voorkennis / sumire documentatie maakt het moeilijk)
% - support => niet bij oudere android OS (2.3) X)(marktwaarde android 2.3 tov alle devices)X% van de devices vallen dan uit de boot
% + performatie zeer positief => quo js geoptimaliseerd (bedrijf gespecialiseerd in mobile user experience)! (weinig support help hier ook)
% nauwelijks bekend (literatuur, social networks (zie populariteit), ...)

\st{} is volgens de gekozen criteria het slechtste raamwerk ($0.73$).
De combinatie van de MVC-architectuur en \js-gedreven opmaak maken \st{} zowel het minst productief als minst performant in vergelijking met de andere raamwerken.
Alle HTML-code wordt door het raamwerk zelf aangemaakt waardoor de \js-bestanden zeer groot zijn.
De downloadtijden waren dan ook opmerkelijk langer door het Delta Update mechanisme
De gebruikservaringtesten gaven echter het tegenovergestelde resultaat want na de initiële wachttijd had \st{} de laagste responstijden.
De tools die Sencha aanbiedt om ontwikkelaars te helpen (Sencha Architect en Sencha Cmd) blijken niet voldoende om het verschil met andere raamwerken te dichten.
De tools laten werken is een leerproces op zich.
Hoewel de ondersteuning van \st{} hoog is, moet er opgemerkt worden dat het afhankelijk is van de WebKit \term{engine}.
Toestellen met een standaard browser die deze \term{engine} niet bevatten, zoals Windows Phone,  kunnen \st{} niet gebruiken.

% Sencha (nergens numero uno) % Waar niet kiezen voor st:
% + / - Tools voorhanden (Sencha Cmd / Architect ) om programmeren makkelijker te maken maar tools niet even handig (of betalend)
% - Javascript driven maakt het lastig (alle simple HTML moet in js worden geschreven): meeste lijnen JS + grote bibliotheek => minder performant
% - gebaseerd op webkit => geteste toestellen hadden dit wel (windows phone geen webkit)


%%%%%%%%%%%%%%%%% NIET BESPROKEN

% besluiten  (de getallen zijn geen procenten)
% 1) Kendo: 1,51 
% 2) Lungo: 1,14
% 3) jQM: 1,03
% 4) ST: 0,62

%%%%%%%%%%%%%%%%%%%%%%%%%%


% daarna de bindparagraaf door onze 2 improvents (productiviteit en performantie)
% referen naar de nieuwe formules in de respectievelijke secties
% nieuwe formules voor het bereken van het relatieve waarden voor de respectievelijke formules
% tonen van de finale grafiek


%nieuwe zaken finaal versus initieel:
%Productiviteit
  % jqm naar eerste plaats,  andere boeten in (raamwerk met architectuur komen achteraan,  stemt overeen met verwachtingen)
%Performantie:
  %Kendo: iOS crash van lange lijst: performantie daalt van 40% naar 25%
  %Sencha : Performantie verhoogt want gebruikservaring maximaal

%jQM naar eerst plaats, rest schuift een plaats door (buiten st)
% snelle ontwikkeling die performant is en overal ondersteund wordt geen payed support nodig hebt en geen geavanceerde features wil implementeren





\chapter{Besluit}
\label{chap:besluit}

\section{Geleerde lessen}
% beter en vollediger uitwerken van criteria op voorhand! niet enkel bekijken als je effectie moet gaan evalueren
% het aantal ongewenste verrassingen blijft zo beperkt

\section{Verbeteringen}
% POC updaten (pull-to-refresh,  meer items laden,  ...) HTML5 features meer toevoegen (GPS, audio,  drag and drop (herorden lijst, lang duwen), carousel met swipe, push eventes 
% toggle entries beter bijhouden


\section{Toekomstig werk}
% ook het criterium uitbreidbaarheid erbij betrekken, want nu komen ST en Kendo niet helemaal tot uiting in onze spidergraph
% onderzoeken ophalen icons in cache 
% onderzoeken crash kendo op ios
% nieuwe frameworks toevoegen + updates van huidige frameworks blijven controleren (resultaten ook updaten)
% methodologie blijven verder toetsen
% subjectieve gebruikservaringstesten met > 5 mensen
% het finale resultaat van de framework bekijken (look-and-feel van kendo, nice dialogs van lungo,...)
% is battery use an issue for web applications?
% vergelijking web / hybrid / native
% windows en blackberry ondersteunen?

\section{Conclusie}
%

%%% Local Variables: 
%%% mode: latex
%%% TeX-master: "masterproef"
%%% End: 


%% Bijlagen
\appendixpage*          
\appendix
%TODO poster aanpassen
\chapter{Poster}
\includepdf[pages={1}]{../Poster/htmobiel.pdf}
\chapter{Wetenschappelijk artikel}
\includepdf[pages={1-11}]{../Artikel/artikel.pdf}
%TODO zoveel heb ik niet aangepast, er zal dus veel dubbel staan. hoe lossen we dat op? geen technische appendix?
\input{app-technisch}
\chapter{Ondersteuning}
\label{app:ondersteuning}

%%% Local Variables: 
%%% mode: latex
%%% TeX-master: "masterproef"
%%% End: 

\chapter{Performantie}
\label{app:performantie}

In deze appendix wordt een gedetailleerd overzicht gegeven van de performantie voor de vier raamwerken op de acht apparaten: 
\st{} (zie \ref{app:performantie-st}),
\kendo{} (zie \ref{app:performantie-kendo}),
\jqm{} (zie \ref{app:performantie-jqm}),
\lungo{} (zie \ref{app:performantie-lungo}).

%%%%%%%%

\section{\st}
\label{app:performantie-st}
Op figuur~\ref{fig:performantie-st} worden de gemiddelde downloadtijd van \st{} getoond op elk apparaat.0.75

\begin{figure}
  \centering
  \includegraphics[width=0.75\textwidth]{figuren/performance-st.pdf}
  \caption{Gemiddelde downloadtijden van \st{} voor POC,  POC uit cache,  loginapplicatie en loginapplicatie uit cache voor elk apparaat.}
  \label{fig:performantie-st}
\end{figure}

Voor de POC is een dalende downloadtijd waarneembaar wanneer het Android-apparaat recenter wordt.
De downloadtijd van de POC op de \gs{} duurde gemiddeld $31.43$s.
Gemiddeld moeten Android toestellen $5$ seconden langer laden in vergelijking met iOS toestellen.
Dit gemiddelde wordt sterk beinvloed door de trage downloadtijd van de \gs{}.

\st{} heeft een andere aanpak voor het cachen van een applicatie door de introductie van een \term{Delta-update} mechanisme.
Dit mechanisme wil voorkomen dat bij een kleine aanpassing in de code,  alle bestanden opnieuw moeten worden opgehaald die in het \term{manifest} bestand staan opgelijst.
De \term{Micro-loader} is verantwoordelijk voor het asynchroon ophalen van all benodigde \js{}- en CSS-bestanden.
Na het bouwen van een applicatie met Sencha Cmd,  zullen de gewijzigde bestanden gearchiveerd worden en worden de veranderingen tussen elke versie opgeslagen.
Na het laden van de applicatie, zal de \code{Micro-loader} met een GET-verzoek controleren op wijzigingen.
Dit GET-verzoek zal de grootste tijd voor zijn rekening nemen bij de downloadtijden bij applicaties uit de cache.
\st{} heeft er dus voor gekozen om aan performantie in te boeten ten voordele van het update mechanisme.
%TODO referentie http://www.sencha.com/blog/behind-sencha-command-and-the-build-process


Een laatste opmerking die bij \st{} moet worden gemaakt, is dat AJAX-verzoeken van een \code{proxy} naar een ander domein altijd vooraf worden gegaan met een OPTIONS-verzoek.
Dit is een verzoek om informatie over de beschikbare opties van het communicatiekanaal op te vragen.
Standaard zet \st{} de \code{X-Requested-With} op XMLHttpRequest en hierdoor zal de browser een OPTIONS-verzoek als \term{preflight} sturen.
%Setting custom headers on XHR requests triggers a preflight request. %http://remysharp.com/2011/04/21/getting-cors-working/
%http://stackoverflow.com/questions/10236056/when-loading-a-store-in-sencha-touch-2-how-can-i-stop-the-additional-options-ht
% POST /resources/userService/login?_dc=1368367749599 HTTP/1.1
% Host: kulcapexpenseapp.appspot.com
% Connection: keep-alive
% Content-Length: 54
% Origin: http://sandervanloock.github.io
% User-Agent: Mozilla/5.0 (X11; Linux i686) AppleWebKit/537.11 (KHTML, like Gecko) Chrome/23.0.1271.64 Safari/537.11
% Content-Type: application/x-www-form-urlencoded; charset=UTF-8
% Accept: */*
% Referer: http://sandervanloock.github.io/HTMobieL/Sencha/build/ExpenseApp/production/index.html
% Accept-Encoding: gzip,deflate,sdch
% Accept-Language: nl-NL,nl;q=0.8,en-US;q=0.6,en;q=0.4
% Accept-Charset: ISO-8859-1,utf-8;q=0.7,*;q=0.3
% 
% Accept:*/*
% Accept-Charset:ISO-8859-1,utf-8;q=0.7,*;q=0.3
% Accept-Encoding:gzip,deflate,sdch
% Accept-Language:nl-NL,nl;q=0.8,en-US;q=0.6,en;q=0.4
% Connection:keep-alive
% Content-Length:23
% Content-Type:application/x-www-form-urlencoded; charset=UTF-8
% Host:kulcapexpenseapp.appspot.com
% Origin:http://sandervanloock.github.io
% Referer:http://sandervanloock.github.io/HTMobieL/Sencha/build/ExpenseApp/production/index.html
% User-Agent:Mozilla/5.0 (X11; Linux i686) AppleWebKit/537.11 (KHTML, like Gecko) Chrome/23.0.1271.64 Safari/537.11
% X-Requested-With:XMLHttpRequest

%%%%%%%%

\section{\kendo}
\label{app:performantie-kendo}
Op figuur~\ref{fig:performantie-kendo} worden de gemiddelde downloadtijd van \kendo{} getoond op elk apparaat.

\begin{figure}
  \centering
  \includegraphics[width=0.75\textwidth]{figuren/performance-kendo.pdf}
  \caption{Gemiddelde downloadtijden van \kendo{} voor POC,  POC uit cache,  loginapplicatie en loginapplicatie uit cache voor elk apparaat.}
  \label{fig:performantie-kendo}
\end{figure}

De \gtab{} vertoond de hoogste downloadtijd,  gevolgd door de \iphoneiii{} en \htc.
Opmerkelijk is dat de gecachete versie van de loginapplicatie op de \nexus{} $10$ keer trager laadt dan de gecachete versie van de POC.
Het ophalen van een gecachete applicatie werkt bij de \gs{} het traagst.

Bij \kendo{} is er geen opmerkelijk verschil waarneembaar tussen Android en iOS toestellen (Android gemiddeld slechts $60$ms trager).

%%%%%%%%

\section{\jqm}
\label{app:performantie-jqm}
Op figuur~\ref{fig:performantie-jqm} worden de gemiddelde downloadtijd van \jqm{} getoond op elk apparaat.

\begin{figure}
  \centering
  \includegraphics[width=0.75\textwidth]{figuren/performance-jquery.pdf}
  \caption{Gemiddelde downloadtijd van \jqm{} voor POC,  POC uit cache, loginapplicatie en loginapplicatie uit cache voor elk apparaat.}
  \label{fig:performantie-jqm}
\end{figure}

Voor de POC is een dalende downloadtijd waarneembaar wanneer de Android-versies op het apparaat recenter worden.
iPads dalen in de downloadtijd als het apparaat recenter word, daarentegen stijgt de downloadtijd bij iPhones.
Er zijn minimale verschillen bij de POC uit cache, waarbij het het langste duurt op de \iphoneiv{}.

Als de loginapplicatie wordt bekeken, wordt hetzelfde waargenomen als voor de POC.
Enkel bij de Android-apparaten wordt de downloadtijd trager, naarmate het toestel recenter wordt. 
Dit is in tegenstelling tot de POC.
Enkel de \nexus{} volgt deze trend niet.
Op de \nexus{} wordt de loginappliactie zelfs even snel gedownload als de POC.
Een opmerkelijke waarneming is dat het op de \nexus{} langer duurt om de loginapplicatie uit cache te laden dan de volledige POC uit cache.

%%%%%%%%

\section{\lungo}
\label{app:performantie-lungo}

\begin{figure}
  \centering
  \includegraphics[width=0.75\textwidth]{figuren/performance-lungo.pdf}
  \caption{Gemiddelde downloadtijd van \lungo{} voor POC,  POC uit cache,  loginapplicatie en loginapplicatie uit cache voor elk apparaat.}
  \label{fig:performantie-lungo}
\end{figure}

Bij \lungo{} is er geen begrijpbare trend waarneembaar voor Android.
Wel is er opmerkelijke waarneming op de \gs{}, waarbij de downloadtijden uit cache langer duren dan 1 seconde, terwijl op alle apparaten deze tijden rond de 0,25 seconden liggen.

Bij iOS daalt de downloadtijd als het toestel recenter is.
Zo is de downloadtijd op \ipadiii{} sneller dan \ipadi{}.
Het geldt ook zo dat \iphoneiv{} sneller download dan \iphoneiii{}.

%%% Local Variables: 
%%% mode: latex
%%% TeX-master: "masterproef"
%%% End: 


\backmatter
% Na de bijlagen plaatst men nog de bibliografie.
% Je kan de  standaard "abbrv" bibliografiestijl vervangen door een andere.
\bibliographystyle{abbrv}
\bibliography{../Referenties/alles-nl}

\end{document}

%%% Local Variables: 	
%%% mode: latex
%%% TeX-master: t
%%% End: 
