\documentclass[master=cws,dutch]{kulemt}
\setup{title={Vergelijkende studie van raamwerken voor de ontwikkeling van mobiele HTML5 applicaties},
  author={Tim Ameye\and Sander Van Loock},
  promotor={Prof.\,dr.\,ir.\ E. Duval},
  assessor={Ir.\,W. Eetveel\and W. Eetrest},
  assistant={Ir.\ A.~Assistent \and D.~Vriend}}
% De volgende \setup mag verwijderd worden als geen fiche gewenst is.
\setup{filingcard,
  translatedtitle={The best master thesis ever},
  udc=621.3,
  shortabstract={Hier komt een heel bondig abstract van hooguit 500
    woorden. \LaTeX\ commando's mogen hier gebruikt worden. Blanco lijnen
    (of het commando \texttt{\string\pa r}) zijn wel niet toegelaten!
    \endgraf \lipsum[2]}}
% Verwijder de "%" op de volgende lijn als je de kaft wil afdrukken
%\setup{coverpageonly}
% Verwijder de "%" op de volgende lijn als je enkel de eerste pagina's wil
% afdrukken en de rest bv. via Word aanmaken.
%\setup{frontpagesonly}

% Kies de fonts voor de gewone tekst, bv. Latin Modern
\setup{font=lm}

% Hier kun je dan nog andere pakketten laden of eigen definities voorzien
\usepackage[utf8]{inputenc}
\usepackage{kulemtx}
\headstyles{kulemtman}
\kulemtmanToC

%TODO:  werkt niet -> nieuwe paragraaf niet indenteren maar wel lijn tussen laten.
%\usepackage{parskip} 

% engelse term die we niet vertalen naar het nederlands
\newcommand{\term}[1]{\emph{#1}}	

% code commande
\newcommand{\code}[1]{\texttt{#1}}

% Tenslotte wordt hyperref gebruikt voor pdf bestanden.
% Dit mag verwijderd worden voor de af te drukken versie.
\usepackage[pdfusetitle,colorlinks,plainpages=false]{hyperref}

% zodat we niet http:// staan hebben in onze tekst, maar de link wel werkt
\renewcommand{\url}[1]{\href{http://#1}{#1}}

%%%%%%%
% Om wat tekst te genereren wordt hier het lipsum pakket gebruikt.
% Bij een echte masterproef heb je dit natuurlijk nooit nodig!
\IfFileExists{lipsum.sty}%
 {\usepackage{lipsum}\setlipsumdefault{11-13}}%
 {\newcommand{\lipsum}[1][11-13]{\par Hier komt wat tekst: lipsum ##1.\par}}
%%%%%%%

%\includeonly{hfdst-n}
\begin{document}

\begin{preface}
  Dit is mijn dankwoord om iedereen te danken die mij bezig gehouden heeft.
  Hierbij dank ik mijn promotor, mijn begeleider en de voltallige jury.
  Ook mijn familie heeft mij erg gesteund natuurlijk.
\end{preface}

\tableofcontents*
%TODO hoe kunnen we derde niveau van structuur voorzien?

\begin{abstract}
  In dit \texttt{abstract} environment wordt een al dan niet uitgebreide
  samenvatting van het werk gegeven. De bedoeling is wel dat dit tot
  1~bladzijde beperkt blijft.
\end{abstract}

% Een lijst van figuren en tabellen is optioneel
\listoffigures
\listoftables
% Bij een beperkt aantal figuren en tabellen gebruik je liever het volgende:
%\listoffiguresandtables
% De lijst van symbolen is eveneens optioneel.
% Deze lijst moet wel manueel aangemaakt worden, bv. als volgt:
\chapter{Lijst van afkortingen}
\section*{Afkortingen}
\begin{flushleft}
  \renewcommand{\arraystretch}{1.1}
  \begin{tabularx}{\textwidth}{@{}p{12mm}X@{}}
     AJAX & Asynchronous JavaScript And XML \\
     API & Application Programming Interface \\
     CSS & Cascading Style Sheets \\
     DOM & Document Object Model \\
     GWT & Google Web Toolkit \\
     GPU & Graphics Processing Unit \\
     (G)UI & (Graphical) User Interface \\
     HTML & HyperText Markup Language \\
     IDE & Integrated Development Environment \\
     JSON & JavaScript Object Notation \\
     PDF & Portable Document Format \\
     PPI & Pixels Per Inch \\
     RIA & Rich Internet Application \\
     SASS & Syntactically Awesome Stylesheets \\
     SDK & Software Development Kit \\
     SEO & Search Engine Optimization \\
     XML & Extensible Markup Language
  \end{tabularx}
\end{flushleft}
%\section*{Symbolen}
%\begin{flushleft}
%  \renewcommand{\arraystretch}{1.1}
%  \begin{tabularx}{\textwidth}{@{}p{12mm}X@{}}
%    42    & ``The Answer to the Ultimate Question of Life, the Universe,
%            and Everything'' volgens de \cite{h2g2} \\
%    $c$   & Lichtsnelheid \\
%    $E$   & Energie \\
%    $m$   & Massa \\
%    $\pi$ & Het getal pi \\
%  \end{tabularx}
%\end{flushleft}

% Nu begint de eigenlijke tekst
\mainmatter

\chapter{Inleiding} 
\label{inleiding}
%In dit hoofdstuk wordt het werk ingeleid. Het doel wordt gedefinieerd en er wordt uitgelegd wat de te volgen weg is (beter bekend als de rode draad).

\section{Achtergrondinformatie}
% - web apps (cross platform)
% - HTML5/js/css3
% - wat doet framework?
% - om de ontwikkeling vergemakkelijk worden frameworks aangeboden... 

Het gebruik van smartphones en tablets stijgt ontzettend snel in onze samenleving.
Voorheen was er de \term{feature phone} waarop enkel de voorgeïnstalleerde applicaties kon worden gebruikt.
Nu kunnen smartphones en tablets ook extra applicaties vanuit een winkel downloaden en installeren.
Ontwikkelaars van deze mobiele applicaties worden geconfronteerd met de variëteit aan mobiele besturingssystemen die op deze apparaten aanwezig zijn.
Dit komt doordat een applicatie dient te worden geprogrammeerd aan de hand van een SDK (Software Development Kit) die specifiek is voor het besturingssysteem.
Ontwikkelaars zullen dus eenzelfde applicatie in verschillende programmeertalen dienen te programmeren om een zo groot mogelijk publiek te bereiken.
Niet enkel het programmeren, maar ook het onderhoud van de applicaties in verschillende programmeertalen brengt een grote kost met zich mee.

Een oplossing hiervoor is het maken van een mobiele webapplicatie, gebruikmakend van HTML5.
Ten eerste wordt deze rechtstreeks in een webbrowser geopend en dus niet langer vanuit een winkel geïnstalleerd.
Dit betekent dus dat ieder mobiel apparaat dat een webbrowser heeft, de webapplicatie kan openen ongeacht zijn mobiel besturingssysteem.
Ten tweede wordt de applicatie slechts in één programmeertaal geschreven, wat de kost verlaagd.
Om het ontwikkelingsproces van deze mobiele HTML5-applicaties te versnellen worden raamwerken aangeboden die helpen bij de functionaliteit van de applicatie en de elementen voor de gebruikersinterface. 

\section{Probleembeschrijving}
% - heel veel frameworks, nog geen literatuur die vergelijkt (alleen blogs e.d.)
% - eerder voorstelling ipv objectieve verglijking.
% - beste framework?
% - hoe vergelijken?

Mobiele HTML5-raamwerken zijn er in overvloed en ook de verschillende versies van eenzelfde raamwerk volgen elkaar in snel tempo op.
In de huidige literatuur worden er vaak raamwerken aangehaald en besproken, maar niet vergeleken.
Indien deze toch worden vergeleken, gebeurt dit vaak subjectief of worden punten gegeven zonder een gestaafde methode te gebruiken.
Ook bestaat er geen literatuur die vergelijkingen van mobiele HTML5-raamwerken aggregeert.

\section{Doelstellingen}
% Is er een beste framework?
% contribuite:  methodologie uitwerken om OBJECTIEF en VISUEEL raamwerken te vergelijken

Deze thesistekst bestaat uit twee doelstellingen.
Een eerste doel is het definiëren van een methodologie om HTML5-raamwerken met elkaar te vergelijken.
Deze methodologie moet alle belangrijke aspecten van de raamwerken tegen het licht houden.
Ook moet er geprobeerd worden de werkwijze zo objectief mogelijk te laten verlopen en het resultaat van de studie op een eenvoudige,  visuele manier aan de lezer te presenteren.
Het tweede doel omvat de effectieve vergelijking van de raamwerken zelf.
Door de grote verscheidenheid van HTML5-raamwerken moeten de bestudeerde raamwerken zo worden gekozen dat ze zoveel mogelijk aspecten bevatten.
Hier komt ook de afweging tussen het aantal bestudeerde raamwerken en de diepte van de vergelijkende studie de kop op steken.
De raamwerken die worden gekozen moeten vervolgens worden vergeleken met de vooropgestelde methodologie.
Het resultaat moet alle positieve en negatieve aspecten van de raamwerken bevatten.
Vervolgens moet er gekeken worden of er één raamwerk het beste is of er verschillende raamwerken in verschillende situaties als beste kunnen worden bestempeld.

\section{Toepassingsgebied}
% mobiele wereld (mobile = booming)
% web (web = booming)
% kruising tussen web en mobile = super booming!
% bedrijfswereld (capgemini) HTML5 iets nieuws,  bedrijven kunnen nu een met een gerust hart een goede keuze maken (mss beter doelstellingen)

Mobiele HTML5-raamwerken vergemakkelijken de ontwikkeling van mobiele HTML5-applicaties.
Deze applicaties zijn toegankelijk via het web en geoptimaliseerd om op mobiele apparaten te kunnen werken.
Het aanspreken van mobiele applicaties via het web heeft zowel voor- als nadelen.
Zeker wanneer er wordt vergeleken met \term{native} of hybride applicaties.
De focus van deze studie ligt echter niet op het onderzoeken van deze voor- of nadelen.
Wel zullen de verschillende technologieën besproken worden om mobiele applicaties te maken.

Omdat Capgemini deze thesis mee ondersteunt zullen de applicaties gemaakt met HTML5-raamwerken vanuit een bedrijfscontext worden benaderd.
Dit zal vooral naar boven komen in de keuze van vergelijkingscriteria en de methode om deze criteria te testen.

\section{Overzicht}
Eerst wordt in hoofdstuk~\ref{chap:literatuurstudie} de basis van dit werk uitgelegd.
Vervolgens worden in hoofdstuk~\ref{chap:raamwerken} de vier gekozen raamwerken uitvoerig besproken.
Daarna zullen in hoofdstuk~\ref{chap:vergelijkingscriteria} de gekozen vergelijkingscriteria aan bod komen en verantwoord worden.
Hieropvolgend wordt in hoofdstuk~\ref{chap:evaluatie} deze vergelijking uitgevoerd op de gekozen raamwerken aan de hand van de gekozen criteria.
Als laatste wordt in hoofdstuk~\ref{chap:besluit} het besluit geformuleerd.

%%% Local Variables: 
%%% mode: latex
%%% TeX-master: "masterproef"
%%% End: 

%Gebruik van stylesheets term in het nederlands: http://www.bol.com/nl/p/websites-opmaken-met-css/1001004010718921/
	%T: check

%OPMERKING:  het gebruik van 'CSS-stylesheets' slaat op niets (kijk naar de betekenis van CSS :-))
	%T: idd, maar we gebruiken toch nergens te term CSS-stylesheets?
	
%OPMERKING: is het UI-elementen of UI elementen idem voor HTML5-code of HTML5 code?
	%T: het is met een streepje

\chapter{Literatuurstudie}
\label{chap:literatuurstudie}
In sectie \ref{sec:mobiele-apparaten} wordt bekeken welke mobiele apparaten er allemaal bestaan. 
Vervolgens wordt er gekeken wat er onder de motorkap van deze apparaten zit, namelijk welke mobiele besturingssystemen (\ref{sec:mobiele-besturingssystemen}), welke mobiele applicaties (\ref{sec:mobiele-applicaties}) en welke mobiele webbrowsers (\ref{sec:mobiele-webbrowsers}) er bestaan. 
Daarna komen de drie bouwblokken van het web aan bod (\ref{sec:html5-css3-js}), namelijk HTML, CSS en JavaScript.
Hierna wordt ingegaan op vele mobiele HTML5 raamwerken (\ref{sec:mobiele-html5-raamwerken}).  
Ten slotte worden verschillende, reeds bestaande manieren om raamwerken te vergelijken, bekeken (\ref{sec:vergelijken-raamwerken}).

%%%%%%%%%%%%%%%%%%%%%%%%%%%%%%%%%%%%%%%%%%%%%%%%%%%%%%%%%%%%%%%%%%
%%%%%%%%%%%%%%%%%%%%%%%%%%%%%%%%%%%%%%%%%%%%%%%%%%%%%%%%%%%%%%%%%%

% TODO Tim: verwerken pie charts

\section{Mobiele apparaten}
\label{sec:mobiele-apparaten}
Mobiele apparaten vind je in alle soorten en maten, met weinig of veel opties, voor weinig of veel geld. 
Het verdient daarom de aandacht om deze diversiteit onder de loep te nemen. 
Eerst zullen we de soorten mobiele apparaten bekijken volgens \cite{GCF2013} en daarna zullen we ingaan op de kenmerken volgens \cite{PhilDutson2012}.

\subsection{Soorten}
Sinds de voorstelling van de Apple iPhone in 2007~\cite{David2011}, stijgt het gebruik van de \term{smartphone} ontzettend snel in onze samenleving.  
Momenteel zijn er al meer dan 1 miljard \term{smartphones} in gebruik~\cite{Yang2012}. 
Dit zal tegen 2015 verdubbeld zijn~\cite{Gillett2012}.
Foto's of video's nemen, navigeren naar het dichtstbijzijnde restaurant of nog snel het weer voor de komende dagen opzoeken, het is allemaal mogelijk. 
Hoewel Apple de lat hoog heeft gelegd met het uitbrengen van de iPhone, zijn er ook nog andere spelers op de markt. 
Zo hebben we bijvoorbeeld ook de op Google's Android gebaseerde \term{smartphones} zoals de Nexus 4 en de op Windows Phone gebaseerde \term{smartphones} zoals de Nokia Lumia 800.

Niet enkel de \term{smartphone} behoort tot de categorie van mobiele apparaten, maar ook de \term{tablet}. 
Tegen 2016 zulen er 760 miljoen \term{tablets} in gebruik zijn~\cite{Gillett2012}.
Ook hier kan terug gedacht worden aan één van Apple's succesvolle producten, namelijk de in 2010 uitgebrachte iPad~\cite{Apple2010}. 
Er dient echter wel opgemerkt te worden dat tien jaar voordien, Microsoft al eerder een \term{tablet} uitbracht met veel minder succes~\cite{Microsoft2000}.

De \term{e-reader} behoort tot de laatste categorie van mobiele apparaten. 
Deze wordt hoofdzakelijk gebruikt om digitale boeken te lezen, maar betere modellen laten bijvoorbeeld ook toe om te surfen op het Internet. 
Ook hier bestaat er een variëteit aan modellen zoals de Kindle van Amazon en de Reader van Sony.

\subsection{Kenmerken}
Door de vele verschillende soorten en modellen aan mobiele apparaten, is het nodig om op een hoog niveau te bekijken over welke kenmerken deze allemaal (kunnen) beschikken. 
Bij deze bespreking zullen we ingaan op de voornaamste kenmerken van \term{smartphones} en \term{tablets}. De kenmerken en tekst zijn gebaseerd op~\cite{PhilDutson2012}.

\subsubsection{Resolutie en PPI}
Een eerste kenmerk, waar vooral Apple met haar Retina graag mee uitpakt, is de resolutie. 
Dit is het aantal pixels getoond op het beeldscherm en wordt uitgedrukt in breedte $\times$ hoogte. 
Hoe kleiner, hoe minder er op het scherm kan worden getoond. 
Dit is vooral belangrijk wanneer veel informatie op het scherm wordt getoond. 
Indien men maar over een kleine resolutie beschikt, zal men moet scrollen om te rest van de informatie te kunnen zien.
Een overzicht van resoluties van bekende mobiele apparaten wordt getoond op de figuur \ref{fig:resoluties}.

Als men naast de resolutie ook nog eens gaat rekening houden met de fysieke grootte van het scherm, dan kunnen we spreken van over pixels per inch~(PPI). 
De eerste iPhone had een resolutie van 320$\times$480 en een 3,5” scherm, wat neerkomt op 163 PPI. 
De iPhone4 (Retina) daarentegen heeft een resolutie van 640$\times$960 en een 3,5” scherm, wat neerkomt op 326 PPI. 
Met andere woorden zijn er meer pixels op dezelfde fysieke grootte geplaatst, wat een scherper beeld tot resultaat heeft. 

% TODO afbeelding misschien vectorieel maken:  
\begin{figure}
  \centering
  \includegraphics[height=0.8\textwidth]{figuren/mobile-devices-resolutions.png}
  \caption{Resoluties van bekende mobiele apparaten~\cite{Wolfermann2012}.}
  \label{fig:resoluties}
\end{figure}

\subsubsection{Aanraakscherm}
De populaire soorten schermen zijn resistieve en capacitieve aanraakschermen. De eerstgenoemde soort maakt gebruik van twee lagen die gescheiden worden door een tussenruimte. Door druk ontstaat er contact tussen de twee lagen. Meestal wordt bij deze soort schermen een stylus meegeleverd. 

De laatstgenoemde soort maakt gebruik van veranderingen in frequentie. Door het scherm aan te raken met je vinger, dat een geleider is, ontstaat er een kleine verandering in frequentie die gedetecteerd wordt. Niet-geleidende materialen zullen geen frequentieverandering veroorzaken, wat verklaart dat zo'n scherm niet reageert als het wordt aangeraakt met een handschoen.

\subsubsection{GPS}
Met het \term{global positioning system} (GPS) kan de gebruiker zijn locatie opvragen en doorgeven aan een applicatie om zo bijvoorbeeld het dichtstbijzijnde restaurant te vinden. 
Doordat het wat kan duren vooraleer de locatie is vastgesteld via GPS, kan het mobiel apparaat ook gebruik maken van mobiele masten of het Internet om zo, hetzij minder nauwkeurig, sneller de locatie te bepalen.

\subsubsection{Camera}
Praktisch ieder recent mobiele apparaat is uitgerust met een camera. 
Sommige bevatten zelfs twee camera's. 
De camera vooraan is veelal van mindere kwaliteit en wordt gebruikt om videogesprekken te voeren. 
Achteraan het apparaat zit dan een camera met hogere resolutie om mooie foto's te kunnen maken.

Twee andere toepassingen van de camera zijn toegevoegde realiteit en het inscannen van barcodes.
Bij het eerstgenoemde wordt informatie toegevoegd aan het beeld dat door de camera wordt geregistreerd.
Het laatstgenoemde wordt gebruikt om de populaire QR-code in te scannen en te zien wat ze betekent.
Zo'n code kan tekst bevatten, een link naar een website, een telefoonnummer, enzovoort. 

\subsubsection{Verbinding}
%TODO: Wifi, 3G, EDGE

In deze periode wil iedereen met elkaar verbonden zijn, dus ook op mobiele apparaten.
We bespreken kort Wifi, 3G, Bluetooth en infrarood.
Het mobiel apparaat kan meerdere mogelijkheden voorzien om verbinding te maken. 
Enerzijds kan men verbinden via Wi-Fi. Daarnaast zijn er ook nog andere technologieën zoals 3G mogelijk.

% \subsubsection{Oriëntatie}
% Een handig kenmerk is dat vele mobiele apparaten kunnen detecteren hoe ze gehouden worden door de gebruiker. Dit maakt het mogelijk om de informatie zo optimaal mogelijk op het scherm te tonen. Men kan bijvoorbeeld een verschillende lay-out voorzien voor een staand en liggend scherm.
% 
% \subsubsection{Versnellingsmeter}
% Als het mobiele apparaat een versnellingsmeter bevat, is het mogelijk om hiervan in spelletjes e.d. gebruik van te maken.
% 
% \subsubsection{Afstandssensor}
% Niet veel mobiele apparaten beschikken over dit kenmerk, maar dit kan bijvoorbeeld gebruikt worden wanneer een gebruiker met zijn smartphone aan het bellen is, waarbij het scherm zichzelf automatisch uitschakelt als het tegen de wang wordt gehouden.

% \subsubsection{Fysiek toetsenbord}
% Sommige apparaten beschikken ook nog over een fysiek toetsenbord, soms ook in combinatie met een aanraakscherm. 

% \subsubsection{Barometer}
% Sommige mobiele apparaten zijn uitgerust met een barometer. Naast het meten van de druk die kan helpen bij het bepalen van het weer, helpt deze de GPS bij het bepalen van de locatie.

%TODO Sander:  Als er in de tekst naar kenmerken van een device wordt verwezen krijg je altijd het voorbeeld GPS of Camera.  Dat lijken mij ook de twee belangrijkste voor POC.  Deze twee zijn mss voldoende om dan te bespreken?

%TODO Tim: Inderdaad, of we kunnen deze kleine paragrafen allemaal in 1 grote steken.

%%%%%%%%%%%%%%%%%%%%%%%%%%%%%%%%%%%%%%%%%%%%%%%%%%%%%%%%%%%%%%%%%%
%%%%%%%%%%%%%%%%%%%%%%%%%%%%%%%%%%%%%%%%%%%%%%%%%%%%%%%%%%%%%%%%%%

\section{Mobiele besturingssystemen}
\label{sec:mobiele-besturingssystemen}
Net zoals er brede waaier bestaat aan besturingssystemen voor computers, geldt dit ook zo voor mobiele apparaten. We geven hier een overzicht van mobiele besturingssystemen met een significant marktaandeel~\cite{David2011, Hales2012} zoals iOS en Android, maar ook een nieuwkomer op de markt, namelijk Windows Phone.

\subsection{iOS}
Het iPhone besturingssysteem is voor het eerst uitgekomen in juni 2007 tezamen met de iPhone. Later werd het hernoemd naar iPhone OS en uiteindelijk werd het iOS. Het is duidelijk dat iOS gebonden is aan de hardware van Apple. Verschillende versies volgden elkaar op: iOS 2 (juli 2008), iOS 3 (juni 2009), iOS 4 (juni 2010) en iOS 5 (oktober 2011)~\cite{Deitel2012, PhilDutson2012}. 

De nieuwste versie, iOS 6, werd uitgegeven in september 2012. Nieuwigheden zijn onder andere hun eigen Maps (in plaats van Google Maps) en een Pass Kit (de vervanging van het traditionele trein-, cinematicket, enz.). Daarnaast zijn er ook ander andere verbeteringen uitgevoerd met betrekking tot sociale media en spraakcommando's~\cite{Deitel2012}.

Op figuur \ref{fig:marketshare-ios} is te zien dat bijna twee derde van de iOS-gebruikers al iOS 6 gebruikt.

\begin{figure}
  \centering
  \includegraphics[width=0.5\textwidth]{figuren/marketshare-ios-2012-11-14.png}
  \caption{Marktaandeel iOS-besturingssystemen op 14 november 2012~\cite{Sylvain2012}.}
  \label{fig:marketshare-ios}
\end{figure}

Browsen op het web gebeurt met de geïnstalleerde Mobile Safari webbrowser (zie \ref{sec:mobile-safari}). Applicaties kunnen gedownload worden in de App Store, die sinds iOS 2 aanwezig is~\cite{Deitel2012}. 

\subsection{Android}
Android Inc. werd opgericht in 2003 en werd in 2005 overgekocht door Google Inc~\cite{Satyesh2012}. Het is net zoals iOS een mobiel besturingssysteem, maar in tegenstelling tot iOS is het open~\cite{David2011}. De eerste stabiele versie, Android 1.0, kwam uit in september 2008. Ook hier volgden verschillende versies elkaar op: Android 2.0 (oktober 2009), Android 3.0 (februari 2011) en Android 4.0 (oktober 2011)~\cite{Satyesh2012}. Hun nieuwste versie, Android 4.2, werd aangekondigd in oktober 2012~\cite{Sawers2012}. 

Op figuur \ref{fig:marketshare-android} is het marktaandeel te zien van de verschillende Android besturingssystemen, waargenomen over een periode van 14 dagen. Het is duidelijk dat Gingerbread (Android 2.3) meer dan de helft van het marktaandeel inneemt.
Applicaties worden gedownload in Google Play. Android bevat ook een standaard browser (zie \ref{sec:android-browser}).

\begin{figure}
  \centering
  \includegraphics[width=0.7\textwidth]{figuren/marketshare-android-2012-11-01.png}
  \caption{Marktaandeel Android besturingssystemen op 1 november 2012~\cite{Android2012}.}
  \label{fig:marketshare-android}
\end{figure}

\subsection{Windows Phone}
Windows Phone van Microsoft werd aangekondigd in oktober 2010 als vervanging voor Windows Mobile~\cite{Seitz2010,Lieberman2010}. Dit is duidelijk te zien als we kijken naar de versies: de laatste versie was Windows Mobile 6.5.3 en de eerste versie is Windows Phone 7. In 2011 ging Microsoft een partnerovereenkomst aan met Nokia om zo snel de markt te kunnen overwinnen~\cite{Microsoft2011}. De nieuwste versie, Windows Phone 8, werd aangekondigd in oktober 2012~\cite{Reed2012}. 

%%%%%%%%%%%%%%%%%%%%%%%%%%%%%%%%%%%%%%%%%%%%%%%%%%%%%%%%%%%%%%%%%%
%%%%%%%%%%%%%%%%%%%%%%%%%%%%%%%%%%%%%%%%%%%%%%%%%%%%%%%%%%%%%%%%%%

\section{Mobiele applicaties}
\label{sec:mobiele-applicaties}
%TODO Sander: eventueel eerst voorstellen,  dan een paragraafje minivergelijking..
Er zijn drie mogelijkheden om mobiele applicaties te maken~\cite{Accenture2012,Hales2012}. Eén aanpak is het maken van een webapplicatie.
Zo'n applicatie wordt geopend vanuit de webbrowser. Een andere aanpak is een \term{native} applicatie. Hierbij zal de gebruiker de applicatie installeren op zijn apparaat. Als laatste kan een mix van de vorige gemaakt worden en dat wordt een hybride applicatie genoemd.

\subsection{Webapplicaties}
In het rapport 'The (Not So) Future Web'~\cite{Phifer2011} uit juni 2011 wordt gesteld dat tegen 2015 60\% van alle mobiele bedrijfsapplicaties en 40\% van alle mobiele consumentenapplicaties, webapplicaties zullen zijn. Er zijn namelijk veel voordelen~\cite{Accenture2012} verbonden aan webapplicaties.

Ten eerste heeft iedereen die een webbrowser heeft op zijn mobiel apparaat, toegang tot de applicatie.  Dit voordeel gaat niet op voor een native applicatie dat enkel voor een specifiek platform is geschreven. 

Ten tweede, aansluitend bij het bovenstaande voordeel, moet de code slechts eenmaal worden geschreven. Een vaak voorkomende term die dit samenvat is WORA: \term{write once, run anywhere}~\cite{Hales2012}. Dit is in tegenstelling tot een native applicatie die specifiek geschreven is voor bijvoorbeeld iOS, Android en Windows Phone. Daar dient de code driemaal te worden geschreven \'en te worden onderhouden.

Ten derde moeten webapplicaties niet worden geverifieerd vooraleer ze worden uitgebracht. Dit is wel zo bij native applicaties. Hierdoor kan in een webapplicatie een belangrijke update snel doorgevoerd worden, terwijl de native applicatie nogmaals het verificatieproces moet doorlopen.

\subsection{Native applicaties}
Een andere mogelijk is om een native applicatie te schrijven. Voordelen~\cite{Accenture2012} hier zijn onder meer de snelheidswinst doordat de applicatie rechtstreeks met het besturingssysteem kan werken. Aansluitend bij het vorige kan ook worden geargumenteerd dat het over het algemeen een native applicatie gemakkelijker de kenmerken van het mobiel apparaat, zoals de camera of GPS, aan kan spreken. Ten derde blijft beveiliging nog altijd een knelpunt bij webapplicaties. Een native applicatie heeft hier minder problemen. Als laatste kan opgemerkt worden dat het gebruik van een winkel (\term{store}) voor het aanbieden van een applicatie als voordeel kan gezien worden, afgezien van het verificatieproces. De applicatiewinkel zorgt namelijk voor reclame en correcte uitbetaling bij gebruik van de applicatie.

\subsection{Hybride applicaties}
Er bestaat een mix tussen de twee voorgaande soorten van mobiele applicaties, namelijk een hybride applicatie~\cite{Accenture2012}. Hierbij wordt de webapplicatie verpakt in een native applicatie. Hierdoor kan men specifieke kenmerken van het mobiel apparaat benaderen die men vanuit een pure webapplicatie niet kon benaderen.

%%%%%%%%%%%%%%%%%%%%%%%%%%%%%%%%%%%%%%%%%%%%%%%%%%%%%%%%%%%%%%%%%%
%%%%%%%%%%%%%%%%%%%%%%%%%%%%%%%%%%%%%%%%%%%%%%%%%%%%%%%%%%%%%%%%%%

\section{Mobiele webbrowsers}
\label{sec:mobiele-webbrowsers}
Sinds 2008 spreken we van het mobiele web~\cite{Hales2012}. 
Vanuit mobiele webbrowsers op tablets en smartphones wordt het web meer en meer aangesproken. 
Deze mobiele webbrowsers vormen als het ware kleine besturingssystemen, waardoor de browser zelf een platform wordt~\cite{Hales2012}. 
Ze geven namelijk toegang tot allerlei kenmerken van het mobiele apparaat zoals camera en GPS. 
Denk maar aan het heel concreet voorbeeld van Google die het besturingssysteem Chrome OS maakte op basis van de Chrome webbrowser~\cite{Hales2012}.

Vanuit het standpunt om webapplicaties te maken, is het dan ook zeer belangrijk om deze evolutie op te volgen. 
Een webbrowser haalt namelijk webpagina's op die geschreven zijn in HTML en andere technologieën. 
Doordat deze technologieën evolueren (zie \ref{sec:html5-css3-js}), zullen de webbrowsers zelf ook (moeten) mee evolueren. 
Niet iedere browser zal dit op dezelfde manier doen, waardoor er verschillen zullen ontstaan waar men rekening mee zal moeten houden. 
Het is namelijk ongewenst dat een webapplicatie enkel op Mobile Safari werkt als men een zo breed mogelijk publiek wenst te bereiken. 

Hieronder bespreken we enkele mobiele webbrowsers. 
Eerst halen we de twee meest populaire browsers aan, namelijk Mobile Safari en de native Android browser~\cite{Hales2012}. 
Ze zijn beide op de WebKit browser \term{engine} gebaseerd~\cite{Oeflman2011}. 
Zo'n \term{engine} zorgt ervoor dat de code van de opgehaalde webpagina wordt omgezet naar de webpagina die de gebruiker te zien krijgt. 
We bekijken ook kort Internet Explorer Mobile en Opera Mobile. 
Het marktaandeel van de genoemde browsers kunt u zien in tabel \ref{tbl:marktaandeel-browsers}.
% TODO beter beschrijven browser engine

% TODO toevoegen data over IE mobile
\begin{table}
\begin{center}
\begin{tabular}{ll}
\hline
\textbf{Mobiele webbrowser} & \textbf{Marktaandeel} (\%) \\
\hline
\hline
Mobile Safari				& 61.50 \\
Android browser				& 26.09 \\
Opera Mini				& 7.02 	\\
Chrome					& 1.14 	\\
Opera Mobile				& 0.53 	\\
Internet Explorer Mobile & 		\\
Andere					& 		\\
\hline
\end{tabular}
\caption{Marktaandeel mobiele webbrowsers op november 2012~\cite{NetApplications2012}.}
\label{tbl:marktaandeel-browsers}
\end{center}
\end{table}

\subsection{Mobile Safari}
\label{sec:mobile-safari}
Deze webbrowser van Apple zit standaard bij iOS en kan ook enkel op dit besturingssysteem worden gebruikt. 
Apple heeft veel moeite gedaan om telkens de laatste nieuwe specificaties van HTML5 in zijn webbrowsers te implementeren~\cite{Hales2012}. Natuurlijk zal dit ook te maken hebben met het feit dat ze geen Flash meer ondersteunen op hun iPods, iPhones en iPads~\cite{Jobs2010}.

\subsection{Android browser}
\label{sec:android-browser}
Android biedt de native Android browser aan. 
Implementatie van de HTML5 specificaties hebben wat aangesleept, maar vanaf Android 4.0 gaat dit een stuk beter~\cite{Hales2012}. 
Daarnaast is het nu ook mogelijk om de Chrome webbrowser op mobiele apparaten te installeren.

\subsection{Internet Explorer Mobile}
Net zoals je bij Windows ook Internet Explorer krijgt, geldt dit ook voor hun mobiel  besturingssysteem. 
Bij de nieuwe Windows Phone 8 zal Internet Explorer Mobile 10 worden meegeleverd. Deze gebruikt dezelfde \term{engine} als Internet Explorer 10. \term{WebSockets}, \term{Web Workers}, \term{Application Cache} en \term{IndexedDB} worden hierin ondersteund~\cite{Hales2012}, meer daarover in~\ref{sec:html5-css3-js}.

\subsection{Opera Mobile/Mini}
Op het moment van schrijven is Opera Mobile 12.10 de beste mobiele HTML5 browser~\cite{Sights2012}. 
Opera heeft eigenlijk twee aparte browsers, namelijk Opera Mobile en Opera Mini. 
Bij deze laatste staat de browser \term{engine} op servers van Opera, waardoor het niet het mobiel apparaat is die de webpagina verwerkt. 
De server zal, na verwerking, deze webpagina op een gecomprimeerde manier doorsturen naar de browser op het apparaat~\cite{PhilDutson2012}.

%\subsection{Mobile Firefox / Fennec}
%Mobile Firefox 16 sleept op dit moment nog net een podiumplaats in de wacht en eindigt derde, voor Mobile Safari. Mozilla staat bekend voor zijn drijvende community. 

%%%%%%%%%%%%%%%%%%%%%%%%%%%%%%%%%%%%%%%%%%%%%%%%%%%%%%%%%%%%%%%%%%
%%%%%%%%%%%%%%%%%%%%%%%%%%%%%%%%%%%%%%%%%%%%%%%%%%%%%%%%%%%%%%%%%%

\section{HTML5, CSS3 en JavaScript}
\label{sec:html5-css3-js}
De drie bouwstenen voor webontwikkeling zijn HTML5, CSS3 en JavaScript. 
HTML5 is verantwoordelijk voor de inhoud, CSS3 voor de presentatie en JavaScript voor de functionaliteit~\cite{PhilDutson2012}. 
Hieronder zullen we dan ook deze bouwstenen toelichten.

\subsection{HTML5}
Zoals uitgelegd in \cite{MacDonald2011} stopte in 1998 het W3C (World Web Consortium) met het werken aan de HTML standaard en alle energie ging uit naar zijn opvolger: XHTML 1.0, een verbeterde HTML versie die XML-gedreven is. 
XHTML kwam in grote mate overeen met HTML, maar de syntax was veel strikter. 
In het begin kon het zijn naam waarmaken en webontwerpers helpen betere resultaten te boeken doordat ze slechte gewoontes moesten opgeven. 
Jammer genoeg bleven de beloofde voordelen uit. 
Wat veel erger was voor de nieuwe standaard, was dat geen enkele browser klaagde indien deze strikte syntax niet werd gevolgd.

In ~\cite{MacDonald2011} staat ook de reactie die hierop kwam van het W3C.  
Ze brachten een nieuwe versie uit, namelijk XHTML 2.
De manier waarop webpagina's werden geschreven veranderde doordat vele tags waren veranderd of verwijderd. 
Daarenboven sleepte deze nieuwe standaard maar aan en aan, wat ook niet in hun voordeel was. 

In plaats van te onderzoeken wat er mis was met HTML, wat XHTML probeerde te doen, werd in 2004 onderzocht wat er ontbrak. 
Opera Software, Mozilla Foundation en Apple vormden de WHATWG (Web Hypertext Application Technology Working Group). 
Ze wilden HTML niet vervangen, maar uitbreiden en die manier moest achterwaarts compatibel zijn. Na reflectie geloofde ook het W3C in deze aanpak, weliswaar op hun eigen manier.  
Zo werd HTML5 geboren, waarbij versie 5 refereert naar waar de vorige versie, HTML 4.01, gestopt was.

HTML5 is volgens ~\cite{MacDonald2011} nog altijd in ontwerp. 
Hierdoor kunnen nieuwe kenmerken op ieder momenten worden toegevoegd.  
Er is ook nog steeds onduidelijkheid waar HTML5 ons zal brengen.  
Het W3C focust op een unieke HTML5 standaard (verwacht rond 2014) terwijl WHATWG de nieuwe markup-taal ziet als levende taal waarbij voortdurend  nieuwe dingen kunnen worden toegevoegd. 
Een belangrijke opmerking hierbij is dat het laatste woord altijd bij de webbrowserfabrikant ligt, net zoals dat het geval was met de strikte syntax in XHTML. 
Als een kenmerk niet in de browser wordt ondersteund, heeft het ook geen kans op overleven.

\subsubsection{Drie basisprincipes}
Achter HTML5 zit een filosofie die in drie basisprincipes kan worden samengevat~\cite{MacDonald2011}.  
De eerste is achterwaartse comptabiliteit. De standaard mag geen veranderingen invoeren die oudere pagina's zou doen breken. 
Ten tweede moet de standaard geen nieuwe specificaties afdwingen die door de meerderheid op een andere manier worden gedaan. 
Als laatste moeten de specificaties ook een praktisch nut hebben. 
Dit betekent dat daar waar veel vraag naar is, ook het beste opweegt om in de specificaties op te nemen.

\subsubsection{Acht technologieklassen}
HTML5 kan ook bekeken worden als de volgende acht technologieklassen~\cite{W3C2012}. 
Iedere klasse wordt met enkele concrete voorbeelden aangehaald.

\begin{description}
\item [Multimedia] De nieuwe video- en audiotags maken het mogelijk om video- en geluidsfragmenten toe te voegen zonder gebruik te maken van plug-ins van derden zoals Adobe Flash en Microsoft Silverlight.

\item [Offline en opslag]  Mobiele apparaten zijn onstabiel in hun verbinding met het Internet. HTML5 voorziet het offline werken in de \term{cache}, lokale opslag (vroeger kon men enkel gebruik maken van de zogenaamde cookies) en een API om bestanden te manipuleren.

\item [Performantie en integratie]  \term{Web Workers} maken het mogelijk om langdurige JavaScript taken in de achtergrond uit te voeren zodat webapplicaties dynamisch en snel blijven.

\item [Semantiek]  Een hele hoop nieuwe tags zorgen voor meer semantiek binnen webpagina's. Waar voorheen de webpagina bestond uit en verzameling \code{<div>}-elementen, kan nu veel concreter worden aangegeven wat er precies binnen die tags staat. Dit kan voor \term{search engine optimization} (SEO) een grote impact hebben. Daarnaast biedt dit ook mogelijkheden voor \term{e-readers} die nu beter de pagina kunnen analyseren.

\item [CSS3]  Hand in hand met HTML5 gaat CCS3 (zie \ref{ref:css3}). Het laat toe om webpagina's op te maken afhankelijk van het formaat van het mobiele apparaat. Ook kunnen webpagina's met effecten worden uitgebreid. 

\item [3D, grafieken en effecten]  De nieuwe \code{<canvas>}-tag in samenwerking met enkele lijnen JavaScript zijn enorm krachtig om eenvoudig tekeningen en animaties zelf te programmeren.

\item [Verbinding]  \term{Events} aan server zijde kunnen data naar \term{WebSockets} pushen. Hierdoor moet de webpagina niet meer voortdurend de server raadplegen, wat veel efficiënter is.
%TODO:  WebSockets is een naam voor een specifieke technologie.  Web workers is geen eigennaam maar duidt een JavaScript script aan (defined by W3C)  Wat is \term en wat niet?

\item [Toegang tot het apparaat] Webapplicaties kunnen meer en meer kenmerken zoals camera en GPS aanspreken net zoals native applicaties dat kunnen. 
\end{description}

Er dient opgemerkt te worden dat aangehaalde klassen zoals CSS3 en geolocatie niet tot de specificaties van HTML5 behoren. 
Toch worden ze onder de koepel van HTML5 gezien~\cite{MacDonald2011}.

\subsubsection{Kenmerken detecteren en opvullen}
\paragraph{Kenmerken detecteren}
Door enerzijds de levendigheid van HTML5 en anderzijds het verdeelde landschap van browsers en besturingssystemen, worden niet alle kenmerken van HTML5 overal ondersteund. 
Een eerste mogelijkheid is om zelf op te zoeken welke kenmerken op welke apparaten werken. 
Dat kan je bijvoorbeeld controleren op \url{www.caniuse.com} en \url{www.mobilehtml5.org}~\cite{MacDonald2011}. 

%tool is volgens woordenlijst.org een aanvaarde nederlandse term!
Wat nog handiger is, is om op het apparaat zelf te detecteren of het gewenste kenmerk beschikbaar is. 
Een erg handige tool hiervoor is Modernizr~\cite{Modernizr2012}. 
Het toevoegen van dit JavaScript-bestand creëert een JavaScript-object dat voor elk kenmerk teruggeeft of het al dan niet in de gebruikte browser wordt ondersteund.

\paragraph{Kenmerken opvullen}
Wanneer eenmaal gedetecteerd is dat een kenmerk niet aanwezig is, zijn er twee mogelijkheden: ofwel terugvallen op een alternatief of simuleren van dat kenmerk. 
Een voorbeeld van dit eerste kan gebeuren bij het gebruiken van de \code{<video>}-tag. 
Indien dit niet wordt ondersteund, kan men terugvallen op de Adobe Flash plug-in. 
Voor het simuleren van een kenmerk maakt men gebruik van \term{polyfills}. 
Dit zijn alternatieven op basis van JavaScript waarbij de native functionaliteit die normaal moet aanwezig zijn, geëmuleerd wordt~\cite{MacDonald2011,Weyl2011}.

\subsubsection{HTML5e}
Een bedrijf wil enerzijds een stabiele webapplicatie en wil anderzijds ook van deze nieuwe kenmerken zoveel mogelijk gebruik gaan maken. 
De term HTML5e~\cite{Hales2012} omvat de vijf meest ondersteunde HTML5 kenmerken in browsers. 
Op figuur \ref{fig:html5e} vind je een tabel die voor mobiele webbrowsers van toepassing is.

\begin{figure}
  \centering
  \includegraphics[width=0.8\textwidth]{figuren/html5e}
  \caption{HTML5e mobiele ondersteuning~\cite{Hales2012}}
  \label{fig:html5e}
\end{figure}

\subsection{CSS3}
\label{ref:css3}
Hand in hand met HTML5 gaat CSS3, dat zorgt voor de presentatie. 
Het is namelijk het hart van webdesign. 
CSS3 heeft hetzelfde probleem zoals HTML5 als het aankomt op de ondersteuning bij browsers~\cite{MacDonald2011}. 
Ook hier is er dus een brede waaier aan kenmerken die nog niet overal worden ondersteund. 
Kenmerken die enkel in een bepaalde browser ondersteund worden, worden voorafgaan door een browserprefix (zoals \code{-webkit-} voor WebKit gebaseerde browsers en \code{-o-} voor Opera).

In deze sectie zullen we kort belangrijke eigenschappen bespreken zoals \term{media queries}, effecten en lettertypes aan de hand van~\cite{MacDonald2011}.

\subsubsection{Media queries}
Zoals al aangehaald, hebben we verschillende apparaten met verschillende schermen en resoluties. 
Een goeie webpagina bestaat erin deze elementen zo goed mogelijk te benutten. 
Dit kan vanaf nu door gebruik te maken van \term{media queries} in CSS3. 
De website kan zich hiermee aanpassen aan het apparaat waarop het wordt getoond. 
Dit wordt in het Engels omschreven als \term{responsive design}.
% TODO nederlands woord voor responsive design

Ook CSS3 volgt het principe van achterwaartse compatibiliteit. 
Browsers die deze \term{media queries} niet ondersteunen, zullen deze negeren en enkel de gewone lay-out toepassen ongeacht het toestel.

\subsubsection{Effecten}
Transparantie, afgeronde hoeken, schaduw en kleurenverloop zijn maar enkele van de nieuwe kenmerken in CSS3. 
Voorheen moest de webdesigner deze dingen vaak met afbeeldingen oplossen, maar nu kan dit allemaal gebeuren met CSS3. 
Daarnaast hebben we ook effecten als transformaties en transities. 
Zo is het mogelijk wanneer men over een afbeelding gaat, deze ingezoomd en geroteerd kan worden. 

Dit is zeer vooruitstrevend om wille van twee zaken. 
Enerzijds schrijf men dingen makkelijker in CSS dan met JavaScript-code. 
Anderzijds komt er ook meer en meer ondersteuning vanuit de hardware. 
Zo worden 3D transformaties in CSS3 versneld door de \term{graphics processing unit} (GPU)~\cite{Hales2012,Kool2012}.

\subsubsection{Lettertypes}
Een laatste kenmerk in CSS3 is de betere ondersteuning van lettertypes. 
Waar vroeger enkel gewerkt kon worden met veilige lettertypes voor het web, is het nu mogelijk om eigen lettertypes op te laden en te gebruiken op je website.

\subsection{JavaScript}
\label{ref:javascript}
JavaScript gaat terug tot in 1995, toen LiveScript~\cite{McFarland2011}. 
Het heeft een lange weg afgelegd tot nu en is niet altijd even ernstig genomen. 
Dit kwam omdat men niet inzag wat er allemaal mee kon worden gedaan. 

Op dit moment is het maar al te duidelijk waar JavaScript in uitblinkt: het aanpassen van het \term{document object model} of kortweg DOM~\cite{PhilDutson2012}. 
Dit is een API voor HTML-documenten~\cite{Hegaret2004}. 
Hierdoor kunnen dynamische interfaces gecreëerd worden, kan op gebeurtenissen - zoals ergens op klikken - onmiddellijk gereageerd worden en is de website dan ook meer bruikbaar geworden door deze directe feedback~\cite{McFarland2011}.

% TODO referentie verschillende manieren intrepeteren
Het schrijven van JavaScript is niet gemakkelijk om twee redenen~\cite{McFarland2011}. 
Ten eerste, vergelijkbaar met HTML5 en CSS3, kunnen browsers JavaScript op verschillende manieren interpreteren. 
Gelukkig is er de laatste tijd veel gestandaardiseerd, maar toch blijven er nog verschillen. %TODO Sander: is dit niet wat vaag?  `veel gestandardiseerd'.  Concreet maken met referentie?
De ontwikkelaar dient dus tijdens het programmeren met deze verschillen rekening te houden. 
Ten tweede vergt het schrijven van simpele, veel voorkomende taken soms veel code.

Een oplossing voor de bovenstaande pijnpunten is gebruik maken van een bibliotheek. 
Een voorbeeld hiervan is de populaire jQuery Core bibliotheek. 
Het is ook mogelijk om jQuery uit te breiden met verscheidene plug-ins om de functionaliteit nog te vergroten~\cite{McFarland2011}.

%%%%%%%%%%%%%%%%%%%%%%%%%%%%%%%%%%%%%%%%%%%%%%%%%%%%%%%%%%%%%%%%%%
%%%%%%%%%%%%%%%%%%%%%%%%%%%%%%%%%%%%%%%%%%%%%%%%%%%%%%%%%%%%%%%%%%

%TODO Sander:  deze sectie slechts alle frameworks aanhalen (ook diegene dat we niet gaan vergelijken).  In een ander hoofdstuk (analyse?) de gebruikte frameworks grondiger bestuderen en argumenteren waarom deze gekozen hebben.  Of waarom we de andere niet gekozen hebben.

\section{Mobiele HTML5 raamwerken}
\label{sec:mobiele-html5-raamwerken}

% paragraaf per framework
% - welk framework (markup / javascript)
% - korte geschiedenis en versie
% - bedrijf, licentie

\subsection{jQuery Mobile} % TIM

\subsection{Sencha Touch}% SANDER
Sencha Touch wordt ontwikkeld door Sencha,  een bedrijf dat in 2010 is ontstaan als een samensmelting van Ext JS,  jQuery Touch en Raphaël.
Ext JS kan beschouwd worden als de voorganger van Sencha Touch. 
Sencha Touch is net als Ext JS een JavaScript gedreven raamwerk met een MVC architectuur.
All functionaliteiten worden dus in JavaScript geschreven en MVC bepaalt de implementatie.
Het aanroepen van het raamwerk gebeurt door het invoeren van de Sencha Touch bibliotheek binnen \code{<script>}-elementen.
Sencha Touch is gratis binnen een commerciële context waarbij het bedrijf in kwestie de broncode niet deelt voor zijn gebruikers.  
Wanneer je dit wel wil doen bestaat er ook een gratis \term{open-source} versie van Sencha Touch.
Op het moment van schrijven is Sencha Touch aan versie 2.1.1~\cite{Inc.}. 

\subsection{Kendo UI} % SANDER
Kendo UI is een HTML5 raamwerk van de hand van Telerik.
Buiten Kendo UI is Telerik voornamelijk gericht op \term{tools} voor de ontwikkelaar.
Zo ontwikkelen ze DevTools dat een grafische gebruikersinterface aan .NET ontwikkelaar aanbiedt.
Ze voorzien ook Icenium voor de ontwikkeling van hybride applicaties en kan men zien als de tegenhanger van Kendo UI dat webapplicaties aanbiedt.
Kendo UI bestaat uit drie luiken:  Web, Mobile en DataViz.  
Het eerste is gericht op de ontwikkeling van desktop en mobiele applicaties,  het tweede voegt een \term{native look-and-feel} toe aan mobiele applicaties en het laatste zorgt voor data visualisatie met HTML5 en JavaScript technologie.
Kendo UI is een JavaScript gedreven raamwerk met een MVVM architectuur dat steunt op de jQuery bibliotheek.
Verder heeft de ontwikkelaar ook de mogelijkheid om eenvoudig de \term{backend} in integreren aan de klantzijde.
.NET,  PHP en JSP zij momenteel de ondersteunde \term{server side wrappers}.
Een licentie voor Kendo UI waarbij één van voornoemde \term{wrappers} mogelijk is kost $\$999$.
Zonder \term{backend} integratie betaal je $\$300$ minder.
Op het moment van schrijven is Kendo UI aan versie 2013 Q1~\cite{Telerik}. 


\subsection{Lungo} % TIM

\subsection{The M-Project} % SANDER
The M-Project is een JavaScript/HTML5 raamwerk dat bouwt op jQuery en jQuery Mobile.
Origineel werd het in 2012 ontwikkeld door M-Way Solutions maar nu behoort het tot Panacoda,  een duitse ontwikkelaar voor software \term{tools} en mobiele web applicaties.
Panacoda bezit ook Espresso,  een krachtige \term{tool} om applicaties te bouwen en ontwikkelen met The M-Project.
Het laat ook toe applicaties om te vormen tot \term{native} applicaties. 
The M-Project is \term{open-source},  vrijgegeven onder een MIT licentie en op GitHub te vinden.
Dit raamwerk is volledig JavaScript gedreven en steunt op de MVC architectuur.
Ook ondersteunt het HTML5 en CSS3 kenmerken zoals offline  beschikbaarheid en lokale opslag.
Op het moment van schrijven is The M-Project aan versie 1.4..  
Het is belangrijk op te merken dat in de zomer van 2013 versie 2.0 wordt verwacht.  
The M-Project zal van nul worden opgebouwd omdat enkel de code aanpassen niet meer voldoende bleek.  
De voornaamste werkpunten zijn performantie en platform-onafhankelijkheid~\cite{Panacoda,Laubach2013}.

\subsection{Moobile} % TIM

\subsection{DaVinci}% SANDER
DaVinci bestaat uit twee \term{tools}:  DaVinci Studio en DaVinci Animator.
De nadruk bij dit raamwerk ligt voornamelijk bij de generatie van code in een WYSIWYG omgeving.
De DaVinci Studio is een Eclipse plugin die HTML,  JavaScript en CSS code genereert.
De gebruiker kan UI elementen via \term{drag-and-drop} aan de applicatie toevoegen.
Het binden van data kan door op een visuele manier de mapping tussen UI en data weer te geven.
Het testen van de applicatie kan op een bijgeleverde emulator in een \term{N-screen} omgeving die de applicatie op verschillende layouts kan weergeven.
Het raamwerk gebruikt een open architectuur dat compatibel is met andere \term{open-source} raamwerken zoals jQuery, KnockOut of Backbone.
DaVinci Animator kan gebruikt worden om animaties op basis van HTML5 en CSS3 te maken in een grafische omgeving.
In SNU Research Park te Seoul worden deze \term{tools} ontwikkeld.
Op het moment van schrijven is DaVinci toe aan versie 2.0.  
Alle documentatie is momenteel nog niet vertaald van het Koreaans naar het Engels~\cite{Incross}.


\subsection{jQTouch}% TIM

%%%%%%%%%%%%%%%%%%%%%%%%%%%%%%%%%%%%%%%%%%%%%%%%%%%%%%%%%%%%%%%%%%
%%%%%%%%%%%%%%%%%%%%%%%%%%%%%%%%%%%%%%%%%%%%%%%%%%%%%%%%%%%%%%%%%%

% SANDER: bekijken aan de hand van paper
% hier alles dat we bekeken hebben, maar zonder onze eigen inbreng
\section{Vergelijken van raamwerken} 
\label{sec:vergelijken-raamwerken}
Om de verschillende mobiele HTML5 raamwerken te kunnen vergelijken hebben we een consistente manier nodig om dit te doen.  
Op deze manier worden alle raamwerken op dezelfde manier getest.

\subsection{ISO 25010}
HTML5 raamwerken zijn software en om software te vergelijken bestaat er de ISO 25010 standaard~\cite{Standard2010}.  
Hieronder vallen twee modellen:  de productkwaliteit en de kwaliteit van het product in gebruik.  
Beide modellen beschrijven de kwaliteit van software op basis van een aantal categorieën met specifieke kwaliteitseigenschappen. 
Het beoordelen van de categorieën kan gebeuren op basis van een checklist. 
 
\subsubsection{Productkwaliteit}
De acht karakteristieken die horen bij dit model zijn: functionele geschiktheid,  betrouwbaarheid,  performantie, efficiëntie, uitwisselbaarheid,  bruikbaarheid,  betrouwbaarheid, beveiligbaarheid,  onderhoudbaarheid en overdraagbaarheid.   
Vanzelfsprekend zijn niet alle categorieën even toepasbaar op HTML5 raamwerken.  
Beveiligbaarheid is bijvoorbeeld niet zo belangrijk bij mobiele HTML5 raamwerken.  
Performantie en overdraagbaarheid dan weer wel.

\subsubsection{Kwaliteit in gebruik}
De vijf karakteristieken voor dit model zijn: effectiviteit,  efficiëntie,  voldoening,  vrijheid van risico en context dekking. 
Elke karakteristiek kan toegewezen worden aan verschillende activiteiten van belanghebbenden. 
Weer zijn alle categorieën niet even toepasbaar.  
De risico die een mobiele webapplicatie meebrengt is niet van belang,  het moet vooral efficient zijn en voldoening scheppen.

De kwaliteit voor een systeem in gebruik wordt bepaald door de kwaliteit van de software,  de hardware en het besturingssysteem samen met de gebruikers, hun taken en de sociale omgeving.  
De belanghebbenden worden opgedeeld in primaire en secundaire gebruikers.  
De eerste zijn de personen die het systeem gebruiken. 
De laatste zijn diegene die zorgen voor het onderhoud.

\subsection{Bestaande use cases}
Op het web en in de literatuur kunnen we ook \term{use cases} terugvinden waar de proef op de som wordt genomen en twee of meer raamwerken met elkaar worden vergeleken.  
Deze werkwijze verschilt van \term{use case} tot \term{use case}

\subsubsection{Codefessions}
Op een blogpost van Codefessions wordt een vergelijking gemaakt tussen jQuery Mobile, Sencha Touch, jQTouch en Kendo UI~\cite{Sarrafi2012}.  
Als referentiesysteem gebruiken ze zeven criteria.  De eerste drie zijn de native \term{look-and-feel}, performantie en platform-onafhankelijke capaciteiten.  
Deze worden gequoteerd met een cijfer van 0 tot 5 waarbij 5 staat voor de maximale score. 
Kenmerken worden gequoteerd door de raamwerk met elkaar te vergelijken.  
Het raamwerk met de meeste kenmerken krijgt een 5, het tweede beste een 4, enzovoort.  
Op een analoge manier wordt code efficiëntie en gebruiksgemak gequoteerd.  
Het raamwerk dat de minste lijnen code vereist, krijgt de perfecte score. 
Hierbij moeten wel alle bestanden gerekend worden die het raamwerk nodig heeft om functioneel te zijn. 
Licenties krijgen een score van 0 tot 5 waarbij 0 betekent dat het niet beschikbaar is voor een individuele ontwikkelaar en 5 dat het raamwerk \term{open-source} en gratis te gebruiken is. 
Andere factoren zoals omkadering en uitbreidbaarheid worden niet in de vergelijkingstabel opgenomen omdat ze afhangen van de interesse van de gebruiker.  
Ze worden echter wel bekeken.

% \subsubsection{jQuery UI vs Kendo UI}
% Een andere, meer grondige vergelijking is te vinden op \url{www.jqueryuivskendoui.com}~\cite{Bristowe}.  Deze webpagina bestaat uit één grote tabel die meer specifieke kenmerken tussen beide raamwerkenen vergelijkt.  Er worden geen scores uitgedeeld. In de tabel vinden we onder andere een vergelijking van beschikbare thema's,  browser compatibiliteit,  form validatie,  ondersteuning van het product etc.
%TODO meer relevante use cases toevoegen

\subsection{Vergelijkingstabellen}
Naast ISO standaarden of al bestaande use cases, kunnen we ook tabellen raadplegen die zoveel mogelijk raamwerken en zoveel mogelijk kenmerken naast elkaar zetten.  
Op Wikipedia creëerde men bijvoorbeeld zo'n tabel voor JavaScript raamwerken~\cite{Wikipedia}.  

Specifiek voor mobiele HTML5 raamwerken bestaat er ook zo'n tabel,  te vinden op \url{www.markus-falk.com/mobile-frameworks-comparison-chart}~\cite{Falk2011}.  
We zien er een matrix met als rijen de verschillende raamwerken en in de kolommen de vergelijkingscriteria.  
Deze laatste worden opgedeeld in compatibiliteit met het besturingssysteem,  doel van de applicatie,  taal voor ontwikkeling,  hardware interactie,  UI,  licenties en andere.  
Deze laatste categorie bevat de criteria of er al-dan-niet een SDK beschikbaar is, encryptie ondersteund wordt en of advertenties worden ondersteund.  
Handig hierbij is dat de webpagina een stappenplan voorziet waarin je per categorie al je vereisten moet invullen.  
De resultaten zijn dan de raamwerken die compatibel zijn met je vereisten.

%%% Local Variables: 
%%% mode: latex
%%% TeX-master: "masterproef"
%%% End: 

\chapter{Vergelijkingscriteria}
\label{chap:vergelijkingscriteria}

Dit hoofdstuk bekijkt hoe de mobiele HTML5 raamwerken actief zullen worden vergeleken.
Hoofdzakelijk zal dit gebeuren aan de hand van een \term{proof of concept}~(POC).
Deze wordt geïntroduceerd in sectie \ref{sec:vergelijking-poc} en zal hoofdzakelijk de gekozen vergelijkingscriteria in sectie \ref{sec:vergelijking-criteria} drijven.
De criteria die worden voorgesteld, zullen voortaan actieve criteria worden genoemd.


\section{POC}
\label{sec:vergelijking-poc}
In samenspraak met Capgemini werd gekozen om een \term{proof of concept}~(POC) op te stellen.
%TODO hier refereren en reflecteren over POC in literatuur (tim)
Dit is een idee waarbij de uitvoerbaarheid in de verschillende raamwerken kan worden nagegaan.
Verschillende vergaderingen werden georganiseerd om tot een idee te komen dat vooral in de bedrijfswereld van toepassing is.
Het uiteindelijke idee is een applicatie die het mogelijk maakt voor werknemers om hun onkosten via hun mobiel apparaat door te sturen.

Het idee werd uitgewerkt door Capgemini en geleverd aan de auteurs als \term{mockup}.
Dit is een voorstelling van de applicatie als een reeks schermen zoals deze er zullen uitzien op een apparaat. 
Een voorbeeld van zo een scherm is te vinden op figuur~\ref{fig:poc}. 
Naast de schermen staan de functionele vereisten die op het scherm van toepassing zijn.
De bedoeling is dat deze POC wordt uitgewerkt zowel voor smartphone als tablet, zowel voor Android als iOS, zowel voor staande als liggende apparaten en zowel voor online als offline gebruik.

\begin{figure}
  \centering
  \includegraphics[trim=0cm 4.6cm 0cm 1.55cm,clip=true,width=\textwidth]{figuren/poc.pdf}
  \caption{POC bij het toevoegen van een nieuwe onkost met aan de linkerkant de weergave op een tablet en aan de rechterkant deze op een smartphone.}
  \label{fig:poc}
\end{figure}

\subsection{Aspecten}
\label{sec:vergelijking-poc-detail}

Een werknemer meldt zich eerst aan op de applicatie en kan daarna ofwel een nieuw onkostenformulier aanmaken of zijn doorgestuurde onkostenformulieren bekijken.
De term onkostenformulier is een groepering van meerdere onkosten met bijhorende bewijsstukken en de handtekening van de werknemer. 
Het aanmaken van een nieuw onkostenformulier verloopt in vier stappen.
Indien de werknemer al eerder begonnen was met het aanmaken van een formulier, zal hij worden gevraagd of hij verder wil gaan met dat formulier of met een nieuw formulier wil starten.

\begin{enumerate}
\item De eerste stap is het bekijken en/of aanpassen van de persoonlijke informatie van de werknemer.
Bij het aanpassen van deze gegevens, zullen deze worden gevalideerd.
Indien deze validatie faalt, krijg de werknemer een dialoogvenster te zien met de reden tot falen.
Ook worden de foute velden rood gemarkeerd.

\item In de tweede stap kan de werknemer zijn toegevoegde onkosten aan het formulier bekijken.
In het begin is deze lijst leeg, tenzij hij eerder een formulier aan het invullen was (zie infra).
Indien deze lijst onkosten bevat, is het mogelijk om hierop te klikken en deze te bekijken.
Aanpassen is niet mogelijk.

\item In stap drie kan een nieuwe onkost worden toegevoegd.
Dit kan ofwel een binnenlandse ofwel buitenlandse onkost zijn.
Voor beide dient een datum en projectcode te worden opgegeven.
De eerstgenoemde is een \term{datepicker} die teruggaat tot twee maanden in de tijd.
De laatstgenoemde bevat automatische aanvulling, maar de werknemer is niet verplicht om een projectcode uit de aanvulling te selecteren.
Daarnaast dient het type en bedrag van de onkost, alsook een bewijsstuk te worden opgegeven.
Bij een buitenlandse onkost moet de munteenheid worden opgegeven, waarna de applicatie deze automatisch omvormt naar euro.
Het scherm voor het toevoegen van een buitenlandse onkost wordt getoond op figuur \ref{fig:poc}. 
Net zoals bij stap één geldt ook hier validatie op de formuliervelden.

\item In deze laatste stap dient een handtekening te worden geplaatst waarna het formulier kan worden doorgestuurd.
Indien de gebruiker offline werkt, zal deze worden opgeslagen op het toestel.
De werknemer kan het formulier opnieuw doorsturen zodra hij terug online is.

\end{enumerate}

Bij het bekijken van de doorgestuurde formulieren is het mogelijk om per formulier de bijhorende PDF te downloaden. 
Deze bevat een overzicht van de onkosten met bijhorende bewijsstukken, alsook de handtekening van de werknemer.

\section{Criteria}
\label{sec:vergelijking-criteria}

In deze sectie zullen de actieve criteria toegelicht worden die zullen worden toegepast om de raamwerken te vergelijken.
In sectie \ref{sec:vergelijken-raamwerken} werden reeds technieken besproken die in de literatuur worden toegepast.
Elementen van deze technieken zullen terugkomen in de voorgestelde methode om de raamwerken te evalueren.
%TODO in elke sectie van een criteria een referentie naar literatuur (zie drive document) + reflecteren met ISO (sander)
Sectie \ref{sec:raamwerken-tabel} bevatte de passieve vergelijkingscriteria die raamwerken vergeleken met informatie over het raamwerk zelf.

Vijf criteria zullen worden gebruikt: populariteit (\ref{sec:vergelijking-populariteit}), productiviteit (\ref{sec:vergelijking-productiviteit}), gebruik (\ref{sec:vergelijking-gebruik}), ondersteuning (\ref{sec:vergelijking-ondersteuning}) en performantie (\ref{sec:vergelijking-performantie}). 
%TODO refereren naar puntensysteem literatuur(tim)
Elk raamwerk krijgt voor elk criterium een score afgeleid uit een formule. 
Deze scores zullen in een spinnenweb worden ondergebracht (zie sectie \ref{sec:vergelijking-spinnenweb}).
Zoals hierboven vermeld zal een POC gebruikt worden bij de vergelijking.
De implementatie van deze POC zal het gebruiks- en ondersteuningscriterium drijven.  
Dit komt omdat Capgemini de POC zo heeft opgesteld dat het de verschillende functionaliteiten bevat die van een normale applicatie verwacht worden.

%TODO populariteit in visionmobile gebruiken bij onze criteria:  er staat alleen percentages van het aantal developers dat welke tools gebruiken..
\subsection{Populariteit}
\label{sec:vergelijking-populariteit}
De populariteit van een raamwerk is een belangrijke factor want het bepaalt de gemeenschap en levendigheid van het raamwerk.
De definitie van gemeenschap op de blogpost van Codefessions~\cite{Sarrafi2012a} zegt ook dat dit een belangrijke factor is omdat het de toekomstige ontwikkeling en de hulp bij het gebruik van het raamwerk aantoont. 
De populariteit kan in cijfers worden uitgedrukt door gebruik te maken van sociale netwerken. 
Een tabel zal voorzien worden met in de rijen het aantal volgers op Twitter, sterren en \term{forkers} van \gh{},  vragen op \so{} en aantal vind-ik-leuks van \fb{}~\cite{Hales2012,Ayuso2012}.
%TODO wat is het beste? refs of drie extra zinnen?
De eerste drie termen worden ook in HTML5 and JavaScript Web Apps van Hales bekeken wanneer HTML5-raamwerken worden voorgesteld~\cite{Hales2012}.
\so{} vragen worden als criterium op een blogpost van Monocaffe gebruikt om mobiele raamwerken te vergelijken~\cite{Ayuso2012}.  
Het aantal vind-ik-leuks van \fb{} werd zelf geïntroduceerd.

GitHub kan worden gezien als een sociaal netwerk voor programmeurs~\cite{Catone2008} en bepaalt dus de actieve gemeenschap rond het raamwerk.
Raamwerken die niet op GitHub te vinden zijn krijgen nul voor zowel het aantal sterren en \term{forkers}.
Een alternatief hield de interpolatie van de GitHub data van de overige raamwerken in.
Omdat deze aanpak het raamwerk onterecht zou bevoordelen, is hier niet voor gekozen.

De som van Twitter volgers ($T_r$), \gh{} sterren ($S_r$), \gh{} \term{forkers} ($F_r$), \so{} vragen ($SO_r$) en \fb{} vind-ik-leuks ($FB_r$) vormt de score voor het populariteitscriterium:
\begin{equation}
  \text{Populariteit}_r=T_r+S_r+F_r+SO_r+FB_r
  \label{eq:populariteit}
\end{equation}
voor een raamwerk $r$.

Omdat deze gegevens zeer dynamisch zijn, zullen verschillende metingen in de tijd de evolutie van de data weergeven.
Ook zullen de uitkomsten van dit criterium worden vergeleken met data geleverd door Google Trends~\cite{Google2012a}.
Deze webapplicatie toont de evolutie van zoektermen op Google op een schaal van 100, waarbij 100 overeenkomt met de grootste zoekinteresse.
Voor elk raamwerk zal het aantal zoekopdrachten op Google in functie van de tijd worden uitgezet.

Er bestaat geen exacte formule om populariteit uit te drukken.
De formule die werd gekozen om de score voor dit criterium te quoteren is onderheven aan subjectiviteit.
Twee opmerkingen moeten hierbij worden gemaakt.
Enerzijds zijn de auteurs zich ervan bewust dat de doorsnede tussen sociale netwerken niet leeg is.
Zo kan éénzelfde persoon zowel een volger op Twitter zijn als een vind-ik-leuk op \fb{} plaatsen.
Verschillende individuen zullen dus dubbel geteld worden in de totale score voor populariteit van het raamwerk.
De score zal dus slechts een indicatie geven over de populariteit,  het is geen exacte weergave.
Ten tweede zijn de auteurs er zicht van bewust dat de inclusie van \so{} op twee manieren kan worden bekeken.
Enerzijds kunnen veel vragen duiden op veel onduidelijkheden over het raamwerk.
Anderzijds kan dit een maat zijn voor de populariteit van dit onderwerp.
De auteurs zijn van mening dat de tweede zienswijze correcter is dan de eerste en het dus valide is \so{} in de formule op te nemen.

%%%%%%%%%%%%%%%%%%%%%%%%%%%%%%%%%%%%%%%%%%%%%%%%%%%%%%%%%%%%%%%%%%
%%%%%%%%%%%%%%%%%%%%%%%%%%%%%%%%%%%%%%%%%%%%%%%%%%%%%%%%%%%%%%%%%%

\subsection{Productiviteit}
\label{sec:vergelijking-productiviteit}
%TODO geen eenduidige manier + perceptie
De productiviteit moet berekend hoe lang het duurt om met het raamwerk vertrouwd te raken en iets nuttig te kunnen bouwen.
Dit is belangrijk want bedrijven willen zo min mogelijk tijd verliezen om met nieuwe technolgieën aan de slag te kunnen.
In de ISO 25010-standaard zijn de categorieën bruikbaarheid en efficiëntie vergelijkbaar met dit criterium.

De auteurs zullen elk de POC in twee verschillende raamwerken maken en daarnaast ook een extra loginapplicatie in twee andere raamwerken.
De ene auteur maakt de POC in \jqm{} en \lungo{} en de loginapplicatie in \st{} en \kendo{}.
De andere zal dan de POC in \st{} en \kendo{} maken en de loginapplicatie in \jqm{} en \lungo{}.
De tijd die nodig is om de volledige POC te implementeren is een indicatie voor de productiviteit. 
Er wordt verondersteld dat de auteurs over een gemeenschappelijke technische achtergrond beschikken.
Toch kunnen beide onderling verschillen in efficiëntie,  waardoor de productiviteit verschilt.
Dit probleem is inherent aan dit criterium.
Toch is het belangijk om een schatting op deze manier te kunnen maken.

% Omdat de POC twee keer moet worden geïmplementeerd, wordt verwacht dat de tweede implementatie sneller zal verlopen.
% Dit probleem is onafwendbaar en zal bij de evaluatie van de data aangehaald worden.
% De uren voor de implementatie van de loginapplicatie zal de score correcter maken.
% 
% De som van de uren voor het implementeren van de POC ($t_{r,POC}$) en de loginapplicatie ($t_{r,login}$) vormt de score voor de productiviteit:
% \begin{equation}
%   \text{Productiviteit}_r = {t_{r,POC} + t_{r,login}}
%   \label{eq:productiviteit}
% \end{equation}
% voor een raamwerk $r$.

Er zijn echter vijf redenen waarom de implementaties van de POC geen goede indicatie zijn voor de productivteit.
Deze werden door de auteurs ervaren wanneer de implementatie in het tweede raamwerk werd uitgevoerd:
\begin{enumerate}
\item Betere ervaring met de POC versnelt bij de tweede implementatie het overzicht van vereisten die moeten worden geïmplementeerd. 
\item Een verbeterde ervaring met HTML5-raamwerken had een positieve invloed op de verdere implementaties.
Dit weerspiegelde zich vooral tussen \jqm{} en \lungo{}.
Hoewel ze beide op een verschillende \js{}-bibliotheek steunen - respectievelijk jQuery en QuoJS - zijn de gelijkenissen tussen deze twee raamwerken groot.
Ook leggen ze beide geen ontwerppatroon op.
\item Er kon code,  zoals van de implementatie in \jqm{},  overgenomen worden bij de implementatie van de POC met \lungo{} en \kendo{}.
\item Er kwamen bij de eerste implementatie problemen met de \term{backend} naar boven.
Deze waren bij de tweede implementatie reeds opgelost.
Door het onnauwkeurig opmeten van de tijd kan er geen schatting worden gemaakt van de tijd die aan de problemen van de \term{backend} werden besteed.
\item Niet de volledige POC kon met \lungo{} en \st{} worden ontwikkeld.
\end{enumerate}

De implementatie van de loginapplicatie is een alternatieve test van de productiviteit.
Deze applicatie bevat GI-elementen, validaties,  \term{backend} integratie en een lijst.
De implementatie van de loginapplicatie kan dus als voldoende steekproef beschouwd worden om ervaring met een raamwerk te testen.
Na het aanmelden met deze applicatie zal de gebruiker een lijst van $850$ elementen te zien krijgen.
Deze lijst is bedoeld als stresstest om de performantie te testen (zie sectie \ref{sec:vergelijking-performantie}).
De elementen in de lijst zullen voorzien worden van een afbeelding en tekst.
De lijst kan als een potentiële muziekapplicatie gezien worden waarbij de afbeelding en tekst naar liedjes verwijzen.
Het aantal elementen in de lijst - $850$ - is een schatting van het maximum aantal liedjes dat ooit is opgenomen~\cite{Zimmy2011}.
Het kan dus als bovengrens voor dit soort applicaties worden beschouwd.
De implementatie van deze applicatie zal een indicatie geven hoe snel,  zonder al te veel voorkennis van het raamwerk,  één eenvoudige applicatie opgeleverd kan worden.

De werkuren van de loginapplicatie bleken niet onderheven aan de vijf zonet opgenoemde redenen:
\begin{enumerate}
\item Bij elke implementatie werd met dezelfde achtergrondkennis gestart.  
De implementatie van de loginapplicatie is triviaal en eenduidig.
Er geldt dus voor alle raamwerken dat de ervaring met de applicatie reeds hoog was.
\item Eerst werd de implementatie van de POC gemaakt voordat aan de loginapplicatie werd begonnen.
Hierdoor was de algemene ervaring met HTML5-raamwerken reeds groot.
\item Er werd geen code gekopieerd. 
\item Er waren geen problemen met de \term{backend}.
\item Alle functionaliteit van de loginapplicatie kon met alle vier raamwerken worden gebouwd.
\end{enumerate}
Om al deze redenen werd beslist de score voor productiviteit te bepalen door enkel de uren van de login applicatie ($t_{r,login}$) te beschouwen.
Ook in de vergelijking van Burris werd voor een loginapplicatie gekozen~\cite{Burris}.
Deze vergelijkt enkel \st{} en \jqm{}.
De formule voor productiviteit is dan:
\begin{equation}
  \text{Productiviteit}_r = t_{r,login}
  \label{eq:productiviteit-enhanced}
\end{equation}
voor een raamwerk $r$.

De uitkomsten van dit criterium zullen gestaafd worden door het aantal lijnen code te presenteren die nodig waren voor zowel de POC als de loginapplicatie te bouwen.
Ook zullen de factoren die de leercurve bepalen, worden bekeken. 
Dit zijn ten eerste de tools die de programmeur kan gebruiken om eenvoudiger te ontwikkelen.
Vervolgens zal de kwaliteit en kwantiteit van de documentatie van elk raamwerk worden bekeken.
De mogelijkheden voor debuggen bepalen ook de leercurve en zullen worden onderzocht.
Tot slot zal gekeken worden naar de aanwezige literatuur van het raamwerk en waar ontwikkelaars met vragen terecht kunnen.

%%%%%%%%%%%%%%%%%%%%%%%%%%%%%%%%%%%%%%%%%%%%%%%%%%%%%%%%%%%%%%%%%%
%%%%%%%%%%%%%%%%%%%%%%%%%%%%%%%%%%%%%%%%%%%%%%%%%%%%%%%%%%%%%%%%%%

\subsection{Gebruik}
\label{sec:vergelijking-gebruik}
Dit criterium moet weergeven welke functionaliteit of plug-ins het raamwerk kan bieden.
Hier meer functionaliteiten het raamwerken te bieden heeft,  hoe minder de programmeur zelf moet schrijven en hoe bruikbaarder het raamwerk wordt.
Ook de ISO 25010-standaard probeert het gebruik met de categorie functionele geschiktheid te testen.

Uit de \term{mockup} schermen en de bijhorende functionele vereisten werden $13$ uitdagingen met in totaal $38$ deeluitdagingen geëxtraheerd.
Alle functionaliteit die potentieel door een raamwerk kan worden geleverd en in de POC wordt gebruikt, zit in een uitdaging vervat.  
Echter, een voorbeeld van functionaliteit van de POC die niet in een uitdaging zit, is de omzetting van \term{identifiers} naar een tekstuele vorm.
Dit is geen interessante functionaliteit omdat het eigen is aan de POC zelf.
De implementatie hiervan zal uitsluitend uit \js{}-code bestaan.
Alle uitdagingen en deeluitdagingen zijn in tabel~\ref{tabel:uitdagingen} te vinden.

\pgfplotstabletypeset[
  begin table=\begin{longtable}{l},
  end table=\caption{$13$ uitdagingen onderverdeeld in $38$ deeluitdagingen voor gebruik.}\label{tabel:uitdagingen}\end{longtable},
  skip coltypes=true,
  col sep=comma,
  string type,
  header=true,
  columns={Uitdaging},
  columns/Uitdaging/.style={column name=\textbf{Uitdagingen}, column type={l}},  
  every head row/.style={
    before row=\toprule,
    after row=\midrule},
  every last row/.style={
    after row=\bottomrule}
]{tabellen/uitdagingen.csv}

De wijze waarop het raamwerk de uitdaging aangaat zal de score bepalen.
Er onderscheiden zich drie gevallen.
De hoogste score ($2$) wordt toegekend wanneer de functionaliteit aangeboden wordt door het raamwerk. 
Een lagere score ($1$) betekent dat een plug-in moet worden gezocht.
Omdat de raamwerken bouwen op HTML5, zal een kenmerk van HTML5 ook als plug-in beschouwd worden.
Voor een oplijsting van de HTML5 kenmerken wordt naar sectie \ref{sec:html5-css3-js} verwezen.
Wanneer de implementatie zelf moet worden geschreven of een hack noodzakelijk is, zal de laagste score ($0$) worden toegekend.
Ook is het mogelijk  dat de uitdaging helemaal niet wordt geïmplementeerd.
Dit is mogelijk wanneer het raamwerk de functionaliteit niet ondersteund,  geen plug-in werd gevonden en niet aan een eigen implementatie wordt begonnen.
Dit zal leiden tot een $0$ score.
Wanneer CSS-code wordt gebruikt om de uitdaging te implementeren, zal de laagste score worden toegekend.
Het gebruik van CSS3 wordt echter als kenmerk van HTML5 gezien en vervolgens met $1$ gequoteerd.

Tabel \ref{tabel:scores-uitdagingen} toont de mogelijke scores $U_{r,i}$ van raamwerk $r$ en voor uitdaging $i$.
\begin{table}	
  \centering
  \begin{tabular}{ll}
    \toprule
    \textbf{Score} & \textbf{Verklaring}\\
    \midrule
    $U_{r,i} = 2$ & Ondersteund door het raamwerk\\
    $U_{r,i} = 1$ & Een plug-in of kenmerk van HTML5 is nodig\\
    $U_{r,i} = 0$ & Eigen implementatie of hack of niet geïmplementeerd\\ 
    \bottomrule
  \end{tabular}
  \caption{Beoordeling uitdagingen gebruikscriterium}
  \label{tabel:scores-uitdagingen}
\end{table}

De potentiële score van een uitdaging is discreet en ligt tussen $0$ en $2$.
Er zijn dus slechts $3$ scores waaruit gekozen kan worden om de implementatie te beoordelen.
De verklaringen bij de scores omvatten alle gevallen op een eenduidige manier.
Een alternatief bestaat uit $4$ scores waarbij HTML5-kenmerken een lagere score krijgen ten opzichte van plug-ins.
Omdat de raamwerken afhankelijk zijn van HTML5 werd hiervoor niet gekozen.

De formule voor gebruik is de volgende:
\begin{equation}
  \text{Gebruik}_r = \sum_{i=1}^{38}{\left(U_{r,i}\right)}
  \label{eq:gebruik}
\end{equation}
voor een raamwerk $r$ en een deeluitdaging $i$.
Omdat er $38$ deeluitdagingen zijn, kan een raamwerk voor dit criterium maximaal $76$ behalen.

%%%%%%%%%%%%%%%%%%%%%%%%%%%%%%%%%%%%%%%%%%%%%%%%%%%%%%%%%%%%%%%%%%
%%%%%%%%%%%%%%%%%%%%%%%%%%%%%%%%%%%%%%%%%%%%%%%%%%%%%%%%%%%%%%%%%%

\subsection{Ondersteuning}
\label{sec:vergelijking-ondersteuning}
Dit criterium moet weergeven hoe goed het raamwerk verschillende toestellen en verschillende besturingssystemen ondersteund.
Het is belangrijk dat een zo breed mogelijk publiek met éénzelfde applicatie kan worden bereikt.
%TODO deze reden geldig?
Het ondersteunen van verschillende platformen werd door meer dan $75\%$ van de ondervraagde ontwikkelaars in het Vision Mobile rapport aangehaald als hoofdreden om \term{cross-platform} tools te gebruiken.~\cite{Mobile2012} 
De ISO 25010-standaard beschrijft ook de overdraagbaarheid naar verschillende platformen.


Enkel de standaard browser van het besturingssysteem zal beschouwd worden.
Voor Android toestellen is dit de Android browser of Chrome.  
Vanaf Android~4.0 wordt Chrome als standaard browser beschouwd~\cite{Wimberly2008}.
Voor iOS is Mobile Safari de standaard browser.

Een context wordt gedefinieerd als één bepaalde configuratie van toestel, besturingssysteem en browser.
In elke context zal de functionaliteit van de POC op ondersteuning worden getest.
Uitdagingen die gebruikt zijn om het gebruikscriterium te testen, kunnen hier worden hergebruikt.
Aangezien sommige uitdagingen triviaal gelden voor elk apparaat zal er slechts een subset van deze uitdagingen getest worden.
De overgebleven uitdagingen zijn:
\begin{itemize}
 \item \uit{toestel}
 \item \uit{formulieren}
 \item \uit{autoaanvullen}
 \item \uit{afbeelding}
 \item \uit{validatie}
 \item \uit{handtekening}
 \item \uit{pdf}
 \item \uit{offline}
\end{itemize}
Voor dit criterium worden alle deeluitdagingen verwaarloosd behalve bij \uit{formulieren} en \uit{offline}.
Een uitdaging zal enkel slagen als alle deeluitdagingen ondersteund worden.
Zo kan bijvoorbeeld op een apparaat getest worden of auto-aanvullen werkt.
Uitdaging \uit{autoaanvullen} bevat als deeluitdagingen het ophalen van suggesties en het tonen van een dropdownmenu.
De werking van de uitdaging is een combinatie van beide en zal dus enkel slagen als beide worden ondersteund.
De deeluitdagingen van \uit{formulieren} en \uit{offline} kunnen wel op ondersteuning worden getest.
De twee deeluitdagingen van \term{datepicker} die bij \uit{formulieren} horen, zullen echter worden samengenomen zodat enkel een \term{datepicker} op zich en niet een aanpasbare \term{datepicker} op ondersteuning wordt gecontroleerd.


Het is belangrijk dat het raamwerk en niet een eigen implementatie op ondersteuning wordt getest.
Wanneer een uitdaging in het vorige criterium een $0$ behaalde, wil dit zeggen dat het raamwerk de uitdaging al niet ondersteunde.
In dit geval moet de uitdaging niet worden gecontroleerd.
Hierdoor is het aantal uitdagingen of deeluitdagingen die getest worden afhankelijk van het raamwerk.


De score van een uitdaging of deeluitdaging kan $1$ of $0$ zijn, respectievelijk een correcte of foutieve uitvoering.
De Cross Platform Capabilities zoals beschreven in op de Codefessions blogpost~\cite{Sarrafi2012a} geven een score aan raamwerken op een gelijkaardige manier.
In totaal zullen acht contexten worden gebruikt.
Deze worden in tabel \ref{tabel:toestellen-hci} weergegeven.

 \begin{table}
 \centering
 \resizebox{\textwidth}{!} {
 \pgfplotstabletypeset[
   begin table=\begin{tabular}{l l l l l},
   end table=\end{tabular},
   col sep=comma,
   header=true,
   string type,
   skip coltypes=true,
   columns={Apparaat,Soort,Lancering,BS,Browser},
   columns/Apparaat/.style={column name=\textbf{Apparaat}},  
   columns/Soort/.style={column name=\textbf{Soort}},
   columns/Lancering/.style={column name=\textbf{Lancering}},
   columns/BS/.style={column name=\textbf{BS}},
   columns/Browser/.style={column name=\textbf{Browser}},
   every head row/.style={
     before row=\toprule,
     after row=\midrule},
   every last row/.style={
     after row=\bottomrule}
 ]{tabellen/apparaten.csv}
 }
 \caption{Acht contexten: apparaten met hun soort, lancering, besturingssysteem~(BS) en browser.}
 \label{tabel:toestellen-hci}
 \end{table}
 
De keuze van de acht contexten waarop ondersteuning wordt getest, is voornamelijk bepaald door de beschikbaarheid van de apparaten op het Departement Computerwetenschappen van de KU Leuven.
Er werd een evenwichtige keuze gemaakt tussen besturingssysteem,  browser en type apparaat.
Er werd gekozen voor vier Android en vier iOS apparaten.
Bij de vier Android apparaten zijn er twee met een Android browser en twee met Chrome browser.
Ook werd er gekozen voor vier smartphones en vier tablets.

De som van de scores van de verschillende contexten bepaalt de score van het ondersteuningscriterium:
\begin{equation}
  \text{Ondersteuning}_r = \sum_{c=1}^{8}{\left(\sum_{i=1}^{N_r}U_{r,c,i}\right)}
  \label{eq:ondersteuning}
\end{equation}
voor  een raamwerk $r$, een context $c$, $N_r$ het maximum aantal geïmplementeerde deeluitdagingen voor een raamwerk $r$ en een uitdaging $i$. 


Indien het raamwerk een implementatie bevat voor alle uitdagingen en deeluitdagingen kan er per context maximaal $13$ gescoord worden.
Indien de acht contexten alle uitdagingen en deeluitdagingen correct weergeven zal de maximale score van $104$ behaald worden.

%%%%%%%%%%%%%%%%%%%%%%%%%%%%%%%%%%%%%%%%%%%%%%%%%%%%%%%%%%%%%%%%%%
%%%%%%%%%%%%%%%%%%%%%%%%%%%%%%%%%%%%%%%%%%%%%%%%%%%%%%%%%%%%%%%%%%

\subsection{Performantie}
\label{sec:vergelijking-performantie}
Performantie wordt opgesplitst in twee verschillende factoren: downloadtijd en gebruikerservaring.
Het is noodzakelijk dat een applicatie zowel snel wordt gedownload als vlot is in gebruikerservaring.
De ISO 25010-standaard gebruikt voor de eerstgenoemde de categorie prestatie-efficiëntie om de snelheid en gebruikte middelen te beoordelen.

De downloadtijd meet hoelang het duurt om de webapplicatie te downloaden.
De gebruikerservaring meet hoe vlot het gaat om door een lange lijst van 850 elementen te scrollen.
Zoals in sectie~\ref{sec:vergelijking-productiviteit} werd verteld, wordt de loginapplicatie als stresstest gebruikt om de performantie te testen.
De volledige POC kon niet in ieder raamwerk worden geïmplementeerd omdat de raamwerken niet alle kenmerken konden aanbieden.
Dit is in tegenstelling tot de loginapplicatie die wel in de vier raamwerken kan worden geïmplementeerd.
Hierdoor zal de POC niet worden gebruikt in dit criterium.
De downloadtijden en gebruikerservaring zullen op acht verschillende apparaten worden opgemeten.
Dit zijn dezelfde apparaten als bij het ondersteuningscriterium (zie tabel \ref{tabel:toestellen-hci}).

\subsubsection{Gemiddelde downloadtijd}
Bij de downloadtijden onderscheiden zich twee gevallen die samen de totale downloadtijd bepalen.
Eerste zal de  downloadtijd van de loginapplicatie worden bekeken~($\widehat{l}_{r,c,login}$). 
Vervolgens zal de tijd worden opgemeten om de loginapplicatie uit het cachegeheugen te downloaden~($\widehat{l}_{r,c,login_{cache}}$).
Deze downloadtijden zullen voldoende keren per apparaat moeten worden uitgevoerd om een betrouwbare meting te bekomen.

Het opmeten van de downloadtijden zal met TCPdump~\cite{Tcpdump2010} gebeuren, zoals werd voorgesteld door Thair~\cite{Thair2011}.
Hiervoor wordt een laptop als hotspot ingesteld en zullen de acht apparaten op deze hotspot connecteren via WiFi.
Wanneer de meting wordt gestart, zal op het apparaat naar de applicatie gesurft worden.
Nadat alle bestanden zijn ingeladen wordt de meting beëindigd. 
De uitvoer van TCPdump is een PCAP-bestand die de HTTP-trafiek bevat.
Deze zal via PCAP Web Performance Analyzer~\cite{SongL.bmcquadeMdsteele2010} worden omgezet naar een HAR-file, waarna een HTTP-waterval zal worden getoond.
Hieruit kan de totale downloadtijd worden gehaald van de gedownloade bestanden voor die applicatie.

De gemiddelde downloadtijd voor een raamwerk wordt bepaald door de som van de gemiddelde downloadtijden per apparaat:
\begin{equation}
  \text{Gemiddelde downloadtijd}_r= \frac{\sum\limits_{c=1}^{8}{\left(\widehat{l}_{r,c,login}+\widehat{l}_{r,c,login_{cache}}\right)}}{8}
    \label{eq:totale-downloadtijd}
\end{equation}
voor een raamwerk $r$ en context $c$.

\subsubsection{Gebruikerservaring}
Eerst werd voorgesteld om de rendertijd ($\left(\widehat{l}_{r,c,lijst}\right)$) te bepalen in plaats van de gebruikerservaring.
De rendertijd is de tijd die het raamwerk nodig heeft om de GI-elementen te renderen.
Hiervoor wordt een lijst van $850$ elementen gebruikt die getoond wordt na aanmelden op de loginapplicatie.
De tijd die het raamwerk nodig heeft om de lijst de renderen kan gemeten worden met \js-code.
Net zoals de downloadtijden zullen deze voldoende keren per apparaat moeten worden uitgevoerd.

De gemiddelde rendertijd is:
\begin{equation}
 \text{Gemiddelde rendertijd}_r= \frac{\sum\limits_{c=1}^{8}{\left(\widehat{l}_{r,c,lijst}\right)}}{8}
 \label{eq:totale-gebruikerservaring}
\end{equation}
voor een raamwerk $r$ en context $c$.

De rendertijd kon via \js{} enkel worden opgemeten in \jqm{} en \kendo{}.
Bij de twee andere raamwerken werden de betreffende gebeurtenissen niet gevonden om correct de tijd op te meten.
Doordat er maar data voor twee raamwerken voor handen was, werd de rendertijd vervangen door de gebruikerservaring van een lijst.
Deze bestaat eruit de vlotheid van het scrollen door de lijst van 850 lijstelementen voor de vier raamwerken op de acht apparaten te vergelijken.
Per apparaat wordt een score van 1, 2, 3 of 4 uitgedeeld aan de raamwerken.
Hierbij is 4 de beste score wat overeenkomt met het vlotste scrollen door de lijst relatief ten opzichte van de drie andere raamwerken.
Deze test werd uitgevoerd door twee personen.

Om de score voor gebruikerservaring van een raamwerk te bepalen worden de scores voor dat raamwerk op ieder apparaat opgeteld. De formule voor gebruikerservaring voor een raamwerk $r$ wordt:
\begin{equation}
  \text{Gebruikerservaring}_r = \sum_{c=1}^{8}{\text{ervaring}_{r,c}}
  \label{eq:performantie-gebruikservaring}
\end{equation}

In het bekomen eindklassement komt de hoogste totaalscore overeen met het raamwerk dat de vlotste scrolervaring aanbiedt. 

\subsubsection{Totaal}
De performantie wordt bepaald door de gemiddelde downloadtijd en de gebruikerservaring.
De opzet van de formule is om een raamwerk dat slecht scoort op de gemiddelde downloadtijd, maar sterk scoort op gebruikerservaring, een middelmatige score te geven.
Aangezien deze laatste geen eenheid heeft en de eerstgenoemde uitgedrukt wordt in seconden, wordt de gemiddelde downloadtijd gedeeld door de gebruikerservaring. De nieuwe formule voor de score voor de performantie wordt:
\begin{equation}
  \text{Performantie}_r = \frac{\text{Gemiddelde downloadtijd}_r}{\text{Gebruikerservaring}_r}
  \label{eq:performantie-enhanced}
\end{equation}
van een raamwerk $r$. 

%De formule voor het performantiecriterium wordt dan:
%\begin{equation}
%  \text{Performantie}_r= \text{Gemiddelde downloadtijd}_r + \text{Gemiddelde rendertijd}_r
%  \label{eq:performantie}
%\end{equation}
%voor een raamwerk $r$.

De maximale responsetijd is wanneer de loginapplicatie niet uit cache wordt geladen:
\begin{equation}
  \text{Maximale reponsetijd}_r= \frac{\sum\limits_{c=1}^{8}\left(\widehat{l}_{r,c,login} + \widehat{l}_{r,c,lijst}\right)}{8}
  \label{eq:performantie-max}
\end{equation}

Deze maximale responsetijd kan gecategoriseerd worden met limieten uitgedrukt in seconden zoals opgelegd door Jakob Nielsen~\cite{Nielsen1993}:  
\begin{itemize}
\item $\text{Maximale responsetijd}_r < 0.1\unit{s}$: de gebruiker heeft het gevoel dat het systeem direct reageert.
\item $\text{Maximale responsetijd}_r < 1\unit{s}$: de gedachtengang van de gebruiker zal niet worden onderbroken, maar hij zal toch een vertraging waarnemen.
\item $\text{Maximale responsetijd}_r < 10\unit{s}$: de limiet om de aandacht van de gebruiker te behouden.
\end{itemize}

De maximale responsetijd kan echter niet worden berekend omdat er geen rendertijden bij \st{} en \lungo{} kunnen worden berekend.
Indien de rendertijd uit de formule wordt weggelaten,  blijft de maximale responsetijd een schatting voor de limiet van Nielsen.

Om de scores van het performantiecriterium te staven zal de downloadgrootte van de loginapplicatie worden bekeken.
Daarnaast zal ook de gemiddelde downloadtijd van de POC en de loginapplicatie met elkaar worden vergeleken.
Ook zullen de resultaten gecontroleerd worden met Google Page Speed~\cite{Morgan2011}. 
Deze tool kan de code van een webpagina analyseren en de performantie testen specifiek voor mobiele apparaten.
Het resultaat is een score op 100 en een lijst van werkpunten om de performantie van de applicatie te verbeteren.
Een hoge score duidt op weinig plaats voor verbetering,  een lagere score duidt op meer plaats voor verbetering.
Google Page Speed meet niet de tijd om een pagina te laden.

%%%%%%%%%%%%%%%%%%%%%%%%%%%%%%%%%%%%%%%%%%%%%%%%%%%%%%%%%%%%%%%%%%
%%%%%%%%%%%%%%%%%%%%%%%%%%%%%%%%%%%%%%%%%%%%%%%%%%%%%%%%%%%%%%%%%%

\section{Vergelijkingsoverzicht}
\label{sec:vergelijking-spinnenweb}

Om de scores van de vijf criteria samen te vatten zal een spinnenweb worden gebruikt.
Hierdoor moet elke score op dezelfde schaal worden gebracht om duidelijk de verschillen te kunnen waarnemen.
De Matlab-extensie om spinnenwebben te genereren, vereist dit ook~\cite{Martti2007}.
Er werd gekozen om alle scores te relativeren.
Hiervoor moet elke score van een criterium gedeeld worden door het maximaal behaalde resultaat van dat criterium.
Alle scores zullen vervolgens tussen $0$ en $1$ liggen.
Deze methode zal ervoor zorgen dat het raamwerk met de beste score een $1$ behaalt.

Om verwarring te voorkomen, moeten ook de scores voor het productiviteitscriterium en performantiecriterium geïnverteerd worden.
Dit komt omdat voor deze criteria geldt:  hoe lager de score,  hoe beter het raamwerk.

De formules om de relatieve scores te bereken worden hieronder weergegeven.
De relatieve scores zullen gebruikt worden om het spinnenweb op te stellen.

\begin{equation}
  \text{Populariteit}_r^{\pentagon}=\frac{\text{Populariteit}_r}{\underset{m}{\max}\{\text{Populariteit}_m\}}
  \label{eq:rel-populariteit}
\end{equation}

\begin{equation}
  \text{Productiviteit}_r^{\pentagon} = \frac{\text{Productiviteit}_r^{-1}}{\underset{m}{\max}\{\text{Productiviteit}_m^{-1}\}}
  \label{eq:rel-productiviteit}
\end{equation}

\begin{equation}
  \text{Gebruik}_r^{\pentagon} = \frac{\text{Gebruik}_r}{\underset{m}{\max}\{\text{Gebruik}_m\}}
  \label{eq:rel-gebruik}
\end{equation}

\begin{equation}
  \text{Ondersteuning}_r^{\pentagon} = \frac{\text{Ondersteuning}_r}{\underset{m}{\max}\{\text{Ondersteuning}_m\}}
  \label{eq:rel-ondersteuning}
\end{equation}

\begin{equation}
  \text{Performantie}_r^{\pentagon}= \frac{\text{Performantie}_r^{-1}}{\underset{m}{\max}\{\text{Performantie}_m^{-1}\}}
  \label{eq:rel-performantie}
\end{equation}

\begin{equation}
\begin{split}
  \text{Score}_r &= \frac{1}{5} \left( \text{Populariteit}_r^{\pentagon}
  + \text{Productiviteit}_r^{\pentagon} 
  + \text{Gebruik}_r^{\pentagon} \right. \\
  &+ \left. \text{Ondersteuning}_r^{\pentagon}
  + \text{Performantie}_r^{\pentagon} \right)
  \end{split}
  \label{eq:rel-totaal}
\end{equation}

\chapter{Mobiele HTML5 raamwerken}
\label{chap:raamwerken}

% TODO Tim: Waar zetten we WAAROM we deze raamwerken gekozen hebben en niet de andere?
% TODO Tim: user interface of gebruikersinterface?

In dit hoofdstuk wordt ingezoomd op de mobiele HTML5 raamwerken die dit werk vergelijkt, namelijk \jqm{}~(\ref{sec:raamwerk-jqm}), \st{}~(\ref{sec:raamwerk-st}), \kendo{}~(\ref{sec:raamwerk-kendo}) en \lungo{}~(\ref{sec:raamwerk-lungo}).
In de laatste sectie (\ref{sec:raamwerken-tabel}) wordt een tabel weergegeven, waarin deze gegevens worden vergeleken.

\section{\jqm}
\label{sec:raamwerk-jqm}
\jqm{} is een mobiel HTML5 \term{user interface} (UI) raamwerk dat werd aangekondigd in 2010~\cite{Resig2010}. 
In november 2011 werd versie~1.0 uitgebracht~\cite{Parker2011} en een jaar later werd in oktober versie~1.2 uitgebracht~\cite{Parker2012}. 
Op het moment van schrijven kwam versie~1.3 uit~\cite{Parker2013a}. 
Het raamwerk wordt beheerd door het jQuery Project dat onder andere jQuery Core beheert en waar \jqm{} afhankelijk van is~\cite{JQuery2012}. 
\jqm{} wordt door onder andere Adobe, BlackBerry en Mozilla gesponsord~\cite{JQuery2012a}.

\subsection{Omkadering}
\paragraph{Programmeertaal}
Om met \jqm{} aan de slag te kunnen, is niets meer nodig dan kennis over HTML, CSS en JavaScript. 
Alle UI-elementen worden geschreven in HTML en aangeduid met \code{data-}* attributen.

\paragraph{Tools}
Een standaard teksteditor voldoet om met \jqm{} aan de slag te kunnen. 
Natuurlijk kan het gemakkelijk zijn om van \term{integrated development environments}~(IDE's) zoals Aptana Studio~\cite{Aptana2012} of WebStorm~\cite{JetBrains2012} gebruik te maken, waardoor handige kenmerken zoals automatische code-aanvulling beschikbaar zijn.

Het is ook mogelijk om gebruik te maken van Codiqua om de UI-elementen op het scherm te slepen en neer te zetten. 
Codiqua zal automatisch op de achtergrond de HTML-code voorzien~\cite{Sperry2012}.

\paragraph{Documentatie}
Documentatie is te vinden op \url{www.jquerymobile.com/demos/1.2.0} voor versie~1.2. 
Hierop is een catalogus te vinden van alle mogelijke elementen waarover \jqm{} beschikt. 
Door de broncode van een voorbeeld te bekijken, kan worden gekeken welke code moet worden geschreven om tot dat resultaat te komen.

Naast de UI-elementen is er ook documentatie over de API. 
Deze gaat over initiële configuraties, \term{events} en methodes die kunnen worden gebruikt.

\paragraph{Marktadoptatie}
Op de website van \jqm{} wordt een reeks applicaties getoond die gemaakt zijn met hun raamwerk. 
Enkele voorbeelden zijn webapplicaties voor Ikea, Disney World, Stanford University en Moulin Rouge~\cite{JQuery2012a}. 

\paragraph{Licenties}
Vanaf september 2012 is het enkel nog mogelijk om \jqm{} onder de Massachusetts Institute of Technology (MIT) licentie te verkrijgen~\cite{Dmethvin2012}. 
Dit betekent dat de code wordt vrijgegeven als \term{open-source} en dat deze tegelijkertijd kan worden gebruikt in propriëtaire projecten en applicaties~\cite{PhilDutson2012}.

\subsection{Code en ontwikkeling}
Zoals werd aangehaald, schrijft men voornamelijk HTML5-code voorzien van \code{data-}* attributen. 
Daarna zal het raamwerk door middel van \term{progressive enhancement} allerhande code toevoegen om de beoogde UI-elementen correct te tonen in de browser. 
Dit wordt verder uitgelegd in de sectie browserondersteuning (zie \ref{sec:jqm-browser-support}).

Er zijn drie strategieën om webapplicaties te maken in \jqm{}~\cite{Broulik2012}. 
Een eerste is om de volledige applicatie in één webpagina te schrijven. 
De vele schermen van de webapplicatie zijn dan allemaal samengebracht op eenzelfde webpagina. 
Het voordeel bij deze aanpak is dat er initieel minder verzoeken zijn naar de server omdat alles in één bestand wordt opgehaald. 
Dit geldt ook zo voor de geïmporteerde CSS- en JavaScript-bestanden. 

Een tweede strategie is om voor ieder scherm een aparte webpagina aan te maken. 
Het voordeel hierbij is dat de eerste pagina waar de gebruiker op terecht komt, sneller wordt gedownload. 
Bij iedere navigatie naar een ander scherm, moet dit scherm via AJAX worden opgehaald, waardoor dit vertragend kan werken. 

Een laatste strategie is om een mix tussen beide te maken. 
Men kan bijvoorbeeld alle schermen die de gebruiker vaak nodig heeft op één webpagina plaatsen. 
De schermen die de gebruiker zelden nodig heeft, plaats men dan op aparte webpagina's.  

\subsection{Functionele kenmerken}
\jqm{} is een raamwerk dat voornamelijk UI-elementen aanbiedt, met name pagina's en dialoogvensters, werkbalken, knoppen, inhoud vormgeven, elementen voor formulieren en lijsten~\cite{JQuery2012b}.
Deze kenmerken zijn gebaseerd op versie~1.2.

\paragraph{Pagina's en dialoogvensters}
De basisstructuur van een pagina bestaat uit een koptekst, inhoud en voettekst. 
Bij het overgaan naar een andere pagina kan men kiezen uit tien overgangseffecten. 
Voordat deze overgang gebeurt, zal \jqm{} altijd eerst die pagina ophalen via AJAX en inladen in het DOM. 
Zo kan een soepel overgangseffect worden getoond aan de gebruiker. 
Daarnaast is het ook mogelijk om gelinkte pagina's op voorhand op te halen. 
Als laatste biedt \jqm{} ook dialoogvensters en pop-ups aan. 

\paragraph{Werkbalken}
Het is mogelijk om zowel knoppen bij de koptekst als bij de voettekst te plaatsen. 
Bij deze laatste kunnen typisch meer knoppen geplaatst worden, bij de koptekst slechts twee. 
Daarnaast is het ook mogelijk om navigatiebalken te maken. 
Aan zowel de werk- als navigatiebalken kunnen iconen worden toegevoegd.

\paragraph{Knoppen}
Het is ook mogelijk om knoppen te plaatsen in het inhoud gedeelde. 
Ook hier is er terug een variëteit aan mogelijkheden: grote of kleine, met iconen of zonder, gegroepeerd of niet. 

\paragraph{Inhoud vormgeven}
De inhoud van de pagina kan worden vormgegeven door gebruik te maken van een rooster. 
\jqm{} laat roosters tot vijf kolommen toe. 
Daarnaast zijn er ook nog opklapbare blokken ter beschikking. 
Als laatste kunnen deze blokken ook samengevoegd worden tot een accordeon. 

\paragraph{Elementen voor formulieren}
\jqm{} biedt alle gangbare elementen voor formulieren aan zoals tekstinvoer, een selectie uit een lijst, een zoekveld, een \term{slider} en een \term{switch}. 
Het raamwerk verplicht zelfs om de \code{<label>}-tag te gebruiken. 
Zo wordt de applicatie toegankelijker gemaakt voor bijvoorbeeld mensen met een \term{e-reader}.

\paragraph{Lijsten}
Een laatste categorie UI-elementen die \jqm{} aanbiedt, zijn lijsten. 
Deze gaan van standaard ongeordende lijsten tot lijsten met alle soorten decoraties als iconen, afbeeldingen, telbubbels en verdelers. 
Ook is het mogelijk om in deze lijsten te zoeken. 
Hiervoor dient de gebruiker enkel één data attribuut toe te voegen, waarna het raamwerk de implementatie voorziet. 

\subsection{Niet-functionele kenmerken}
\paragraph{Performantie}
Zoals gezegd schrijft de ontwikkelaar HTML5-code met specifieke data attributen en zal het raamwerk daarna de code verder aanvullen. 
Dit gebeurt enkel op de pagina die de gebruiker op dat moment bekijkt. 
Dit gaat dus ook op voor een webapplicatie waarbij alle schermen op één webpagina zijn geschreven. 
Deze webpagina bevat allemaal \code{<div>}-verpakkingen voor ieder scherm. 
\jqm{} zal enkel die \code{<div>} verder aanvullen die op dat moment getoond wordt aan de gebruiker. 

\paragraph{Aanpasbaarheid}
Als \jqm{} \term{out-of-the-box} wordt gebruikt, zit alles al goed qua kleur en design. 
Er is keuze uit vijf kleurenthema's die kunnen worden toegepast op de gehele applicatie of enkel op bepaalde elementen. 
Om een applicatie echt te laten onderscheiden van de andere, is een eigen kleurthema noodzakelijk. 
Hier is \jqm{} op voorzien door hun \term{stylesheet} op te delen in twee delen: thema's en structuur. 
Een ontwikkelaar kan ook enkel de structuur downloaden en zelf het thema in CSS schrijven. 
Doordat dit laatste heel wat inspanning vraagt, hebben de ontwikkelaars van \jqm{} ook een tool ter beschikking gesteld, namelijk ThemeRoller~\cite{JQuery2012c}. 
Hiermee worden de kleuren naar een voorbeeldapplicatie gesleept, waarna de overeenkomstige \term{stylesheet} kan worden gedownload.

\paragraph{Programmeerbaarheid}
Bij het programmeren in \jqm{} wordt geen enkel ontwerppatroon afgedwongen. 
De code voor de UI-elementen wordt tenslotte als HTML5-code geschreven. 
Voor de echte functionaliteit wordt beroep gedaan op JavaScript en meer bepaald op de jQuery Core bibliotheek. 
Ook deze dwingt geen ontwerppatroon af.

\paragraph{Browserondersteuning}
\label{sec:jqm-browser-support}

% TODO Tim: verder uitwerken

\jqm{} deelt browsers op in drie verschillende klassen: A, B en C~\cite{JQuery2012d}. 
Hierbij ondersteunt een klasse A browser alles, terwijl een klasse C browser enkel de basis HTML ondersteunt (en dus bijvoorbeeld geen hippe CCS3 overgangen).
\jqm{} maakt gebruikt van \emph{progressive enhancement} (zie \ref{par:progressive-enhancement}).

%%%%%%%%%%%%%%%%%%%%%%%%%%%%%%%%%%%%%%%%%%%%%%%%%%%%%%%%%%%%%%%%%%
%%%%%%%%%%%%%%%%%%%%%%%%%%%%%%%%%%%%%%%%%%%%%%%%%%%%%%%%%%%%%%%%%%

\section{\st}
\label{sec:raamwerk-st}

\st{} is een relatief verschillend raamwerk in vergelijking met \jqm{}.  
Het wordt ontwikkeld door Sencha,  een bedrijf dat in 2010 is ontstaan als een samensmelting van Ext JS,  jQuery Touch en Raphaël.  
Ext JS is een JavaScript raamwerk voor de ontwikkeling van web applicaties. 
jQuery Touch is een jQuery plugin voor mobiele web ontwikkeling.  
Het steunt op WebKit en voegt \term{touch events} toe aan jQuery.  
Raphaël,  ten slotte,  is een JavaScript bibliotheek voor vector tekeningen. 
Op het moment van schrijven is \st{} aan versie 2.1.1~\cite{Inc.}.  

\subsection{Omkadering}
\paragraph{Programmeertaal}
\st{} is JavaScript gedreven dus all functionaliteiten worden in JavaScript geïmplementeerd. 
Het aanroepen van het raamwerk gebeurt door het invoeren van de \st{} bibliotheek binnen \code{<script>}-elementen.  
Alle HTML code wordt bij het bekijken van de pagina gegenereerd.  

\paragraph{Tools}
Naast \st{} levert Sencha nog producten die \st{} uitbreiden of het leven van de ontwikkelaar makkelijker maken.  
Deze worden hieronder opgelijst~\cite{Inc.}.  

\subparagraph{Sencha Animator}
Dit is een desktop applicatie om CSS3 animaties te ontwerpen.  
Deze animaties worden enkel in WebKit browsers ondersteund.

\subparagraph{Sencha Architect}
Dit is een andere desktop applicatie waarmee je makkelijk een UI kan ontwikkelen met behulp van \term{drag-and-drop} commando's.  

\subparagraph{Sencha GXT}
Sencha GXT is een uitbreiding op Google Web Toolkit (GWT).  
De compiler van GWT laat toe applicaties in Java te schrijven en ze te compileren naar geoptimaliseerde,  \term{cross-browser} HTML5 en JavaScript.  Sencha GXT voegt grafieken,  widgets, etc. toe aan GWT.

\subparagraph{Sencha.IO}
Deze uitbreiding zorgt voor \term{cloud} services binnen mobiele applicaties.  

\paragraph{Documentatie}
Alle documentatie voor \st{} 2.1.1 is te vinden op \url{docs.sencha.com/touch/2-0}.  
Een zoekfunctie voor objecten,  eigenschappen en methoden is aanwezig om snel zaken op te zoeken.  
De meeste functionaliteiten zijn voorzien van codevoorbeelden samen met het resultaat hoe de browser de code rendert.  
Verder biedt de Sencha website ook een groot aanbod om Sencha te leren gebruiken \url{www.sencha.com/learn/touch/}.  
Hier staan handleidingen,  introductie video etc..

%Door de snelle ontwikkeling van Sencha blijft de documentatie niet altijd up-to-date.  Zo zijn vele methoden verouderd maar staat er geen alternatief vermeld. 
%TODO dit is misschien eerer subjectief?
Een ander handig raadslagwerk is de ‘Kitchen Sink'~\cite{Inc.2013}.  
Dit is een webapplicatie,  geschreven in \st{},  die de belangrijkste functionaliteiten bevat samen met de bijhorende code.  

\paragraph{Marktadoptatie}
Volgens de Sencha website is 50\% van de Fortune 100 - een lijst van de grootste Amerikaanse bedrijven gerangschikt op jaaromzet - een Sencha klant~\cite{Inc.}.  
Enkele van hun grootste klanten zijn CNN,  Samsung,  Cisco en  Visa.

\paragraph{Licenties}
\st{} is gratis binnen een commerciële context waarbij het bedrijf in kwestie de broncode niet deelt voor zijn gebruikers.  
Wanneer je dit wel wil doen bestaat er ook een gratis \term{open-source} versie van \st{}.  
Deze komt met een GNU GPL v3 \term{open-source} licentie wat wil zeggen dat je de vrijheid hebt om aanpassingen aan de broncode te maken en te verspreiden,  zolang je zelf je code maar gratis verspreid voor alle gebruikers.
  
Voor de ontwikkeling van eigen raamwerken of SDKs betaal je een \term{original equipment manufacturer} (OEM) licentie.  
Dit wil zeggen dat bedrijven hun producten gaan verkopen onder hun eigen merk en naam, maar gebruik maken van Sencha.  
Omdat het gebruik hiervan per gebruiker verschilt,  worden OEM licenties op maat gemaakt~\cite{Inc.}.

\subsection{Code en ontwikkeling}
Zoals reeds vermeld moet alle code in JavaScript worden geschreven en dient één HTML bestand slechts als container om de bestanden in te laden.  Sencha valt dus onder JavaScript gebaseerde raamwerken.  
De keuze voor deze aanpak heeft twee belangrijke motivaties.  
Enerzijds is \st{} gebouwd op Ext JS,  wat op zich een JavaScript raamwerk is.  
Anderzijds zorgt het voor een betere ondersteuning voor toestellen met verschillende resoluties.  
Samen met SASS en Compass kan Sencha lay-outs definiëren per device (zie sectie \ref{sec:sencha-aanpasbaarheid}).  
De \code{Ext.env.Browser} en \code{Ext.env.OS} eigenschappen en \code{Ext.Viewport.getOrientation} en \code{Ext.feature.has} methoden kunnen de vereisten bepalen en de juiste lay-out kiezen~\cite{JohnEClark2012}.

Om het de ontwikkelaars makkelijker te maken biedt Sencha ook SDK tools aan.  
Momenteel bevinden deze zich nog in bèta.  
Concreet zijn deze tools commando's voor de terminal die onder andere nieuwe projecten kunnen aanmaken, JavaScript bestanden kunnen optimaliseren maar vooral de webapplicatie kunnen omzetten naar native applicaties voor iOS en Android.

\paragraph{Debugging}
Het debuggen van je code gebeurt voornamelijk in de browser zelf.  
Tools als de Safari Web Inspector,  Chrome Developer Tools of Firebug moeten de fouten kunnen opsporen.  
De broncode van \st{} kan ook ingeladen worden met \code{sencha-touch-debug.js} als bibliotheek.  
Deze versie is niet gecomprimeerd en bevat commentaar en documentatie om makkelijker te zoeken waar in de code de fout zich juist bevond.

\subsection{Functionele kenmerken}
Net zoals \jqm{} heeft \st{} ook een hele hoop functionaliteiten om eenvoudig UI elementen te genereren.  
\st{} bevat alle elementen van de UI als JavaScript objecten.  
Net zoals alle objectgerichte programmeertalen maken deze objecten gebruik van een klassesysteem,  iets wat slechts vanaf \st{} 2 werd ingevoerd.  
Op die manier kunnen klassen worden gedefinieerd (\code{Ext.define}) en aangemaakt (\code{Ext.create}).  
Hierbij is ook overerving mogelijk.  
De basisklasse van alle objecten is \code{Ext.Component}.  
Componenten kunnen gerenderd worden, zichzelf tonen of verbergen,  centreren op het scherm en zichzelf aan- of uitzetten.   
Het aanmaken van componenten kan compacter door het gewenste component als \code{xtype} te definiëren.  

Een andere belangrijke component is \code{Ext.Container}.  
Containers kunnen subcomponenten bevatten en een lay-out specificeren.  
Alle componenten krijgen een naam die verwijst naar een namespace.  
Dit is handig om conflicten te vermijden tussen je eigen objecten en de standaard objecten van het raamwerk.  

Voor een opsomming van alle raamwerk componenten verwijzen we naar de documentatie~\cite{Inc.2013a}.

%jQuery subsecties:
%Pagina's en dialoogvensters
%werkbalken
%knoppen
%inhoud vormgeven
%elementen voor formulieren
%lijsten

\paragraph{Model}
Data kan intern worden voorgesteld met models.  
Dit is iets wat hoort bij het MVC patroon (zie sectie \ref{sec:sencha-programeerbaarheid}).  
Een model specificeert een lijst van velden die bij het model horen waarbij een veld een naam en een type heeft.  
Optioneel kunnen validaties bij de velden worden toegevoegd om data consistent te houden.  

\paragraph{Store}
\code{Ext.data.Store} is de klasse om instanties van een model op te slaan.  
Een \term{store} wordt voorzien van een \term{proxy}.  
Deze kan data aan de client of server zijde opslaan.  
Een \term{proxy} voor opslag aan client zijde kan zowel in het RAM geheugen als in de \term{local storage} van de browser opslaan.  
Een \term{proxy} voor server opslag kan data verzenden via AJAX (zelfde domein) of JSONP (verschillende domeinen).  
Een \term{proxy} kan ook nog voorzien worden van een \term{reader} die aangeeft hoe de ontvangen data gelezen moet worden.

\paragraph{View}
Een \term{view} is de benaming voor objecten die aan de gebruiker kunnen getoond worden.  
Een voorbeeld hiervan zijn lijsten,  waar vaak de data van een \term{store} wordt in weergegeven.  
Zo'n lijst kan makkelijk gefilterd of gesorteerd worden op basis van velden uit het model.  
Hiervoor moeten we \term{filters} of \term{sorters} aan de \term{store} toevoegen. 
De lay-out van één lijstitem bepalen kan via een \code{XTemplate}.  
Het sjabloon bepaalt de HTML structuur van elk item.  
Alle gedefinieerde velden van het model kunnen in de template worden opgeroepen of gemanipuleerd.

%TODO controllers?

\subsection{Niet-functionele kenmerken}
\paragraph{Performantie}
In vergelijking met versie 1.1 van \st{} is de performantie gestegen om wille van verschillende factoren.  
De introductie van het klasse systeem,  zoals besproken in de vorige sectie,  laat toe objecten dynamisch te laden. 
Het grote verschil tussen \code{Ext.define} en \code{Ext.create} is dat objecten enkel in het geheugen worden geladen na creatie.  
Het is dus de taak van de programmeur om objecten enkel te construeren wanneer ze nodig zijn.

Verder kwam versie 2.0 met een nieuwe lay-out \term{engine} die vooral het verwisselen van oriëntatie van het toestel versnelde.  
Ook een verbetering in performantie op Android toestellen,  voornamelijk bij scrollen en animaties,  werd ingevoerd~\cite{Inc.}.

Een benchmark voor deze verbeteringen zijn de opstarttijden van de Kitchen Sink applicatie.  
Het opstarten gebeurde met de verschillende \st{} versies en op verschillende toestellen.  
De resultaten zijn terug te vinden op figuur \ref{fig:sencha_performance}.  
Op bijna elk toestel blijkt \st{} 2.0 ongeveer één seconde sneller te werken~\cite{SenchaInc.2013}.

\begin{figure}
  \centering
  \includegraphics[width=0.8\textwidth]{figuren/sencha-touch-startup-times.png}
  \caption{\st{} Kitchen Sink opstarttijden~\cite{SenchaInc.2013}.}
  \label{fig:sencha_performance}
\end{figure}

\paragraph{Aanpasbaarheid}
\label{sec:sencha-aanpasbaarheid}
Elke component binnen het raamwerk moet overerven van \code{Ext.Component}.  
Deze voorziet een attribuut \code{ui}.  De waarde hiervan is een CSS klasse die bepaald hoe de component er zal uitzien.  
encha heeft al twee CSS klassen voorzien:  \code{light} en \code{dark}.  
Andere componenten kunnen deze lijst uitbreiden.  
Een knop kan bijvoorbeeld \code{normal},  \code{back},  \code{round},  \code{small},  \code{action} of \code{forward} als \code{ui} waarde hebben.

Het is ook mogelijk om eigen waarden voor \code{ui} te definiëren of de standaarden van Sencha aan te passen.  
Hiervoor moet je gebruik maken van SASS en Compass om je CSS bestanden aan te maken.  
SASS staat voor Syntactically Awesome Stylesheets en breidt CSS uit met variabelen,  geneste structuren,  mixins en overerving~\cite{Eppstein2013}.  
Mixins groeperen enkele CSS eigenschappen en kunnen worden herbruikt.  
Compass is een raamwerk bovenop SASS en CSS.  Het compileert SCSS (Sassy CSS) naar CSS bestanden~\cite{Eppstein2013a}.        

Sencha thema's bestaan allemaal uit een set van \term{mixins}.  
Door zelf \term{mixins} te creëren of reeds bestaande te manipuleren kunnen we eigen thema's creëren en ze aan de \code{ui}-waarde van een component toekennen.

\paragraph{Programmeerbaarheid}
\label{sec:sencha-programeerbaarheid}
Zoals reeds aangehaald ondersteund \st{} het MVC (Model-View-Controller) patroon.  
Dit patroon vermijdt lange JavaScript bestanden door ze logisch op te delen.  
Modellen groeperen velden tot een beschrijving van data-objecten,  views definiëren de weergave van componenten en controllers verbinden beide op basis van \term{events}.

In theorie zou het verschil tussen mobiele websites en applicaties enkel in de views terug te vinden zijn.  
Echter,  dit wordt nog niet volledig ondersteund en raadt men dus aan om hiervoor aparte projecten te voorzien.

\paragraph{Ondersteuning browser}
\st{} steunt op de WebKit browser \term{engine} dus moet de browser deze bevatten.  
Hoewel dit bij de meeste browsers geen probleem meer vormt vallen toch enkele populaire browsers uit de boot.  
\st{} is bijvoorbeeld niet compatibel met FireFox Mobile en Opera Mobile~\cite{JohnEClark2012}.

Zoals reeds vermeld zijn er ook methoden voorzien om informatie op te vragen over de context die gehanteerd wordt (browser, OS, toestel, etc.).  Verder kan \st{} ook vragen naar de ondersteuning van specifieke kenmerken (audio,  canvas,  CSS3, …),  analoog als Modernizr.  

Op de Secha website zijn voor sommige browsers en bijhorend besturingssystemen scorecards voorzien om hun compatibiliteit met HTLM5 en \st{} te bespreken~\cite{Inc.}.


\section{\kendo}
\label{sec:raamwerk-kendo}

\subsection{Omkadering}
\subsection{Code en ontwikkeling}
\subsection{Functionele kenmerken}
\subsection{Niet-functionele kenmerken}

\section{\lungo}
\label{sec:raamwerk-lungo}

\subsection{Omkadering}
\subsection{Code en ontwikkeling}
\subsection{Functionele kenmerken}
\subsection{Niet-functionele kenmerken}

\section{Tabel}
\label{sec:raamwerken-tabel}

\chapter{Evaluatie}
\label{chap:evaluatie}

In dit hoofdstuk voeren we de vergelijking uit en bekijken we de bekomen resultaten.
Enerzijds vergelijken we in \ref{sec:evaluatie-poc} de implementatie van de POC in de verschillende raamwerken.
Anderzijds vergelijken we in \ref{sec:evaluatie-criteria} de raamwerken op basis van de vergelijkscriteria opgesteld in .

%%%%%%%%%%%%%%%%%%%%%%%%%%%%%%%%%%%%%%%%%%%%%%%%%%%%%%%%%%%%%%%%%%%%%%%%

\section{POC}
\label{sec:evaluatie-poc}
In deze sectie bekijken we de problemen die we hebben tegengekomen bij het implementeren van de POC in de verschillende raamwerken.
Er wordt een onderscheid gemaakt tussen enerzijds de functionele vereisten als anderzijds vereisten met betrekking tot layout.

\subsection{Functionele vereisten}

\subsubsection{Formulieren}

\paragraph{jQuery Mobile} 
Voor het toevoegen van placeholders in de formuliervelden kon beroep worden gedaan op het \code{placeholder} attribuut in HTML5. Er dienden geen labels te worden gezet bij de velden. Deze labels zijn echter wel verplicht in jQuery Mobile, maar kunnen onzichtbaar worden gemaakt met de \code{ui-hide-label} CSS klasse~\cite{JQuery2013}.

Sommige velden waren verplicht in te vullen, terwijl andere niet. Hiervoor werd eerst gedacht om het \code{required} attribuut in HTML5 te gebruiken. Het probleem is echter dat er geen ondersteuning is voor mobiele browsers~\cite{Deveria2013}. Daarnaast was het ook nodig om de velden te valideren op hun waarde. Validatie is echter niet standaard aanwezig in jQuery Mobile. Als oplossing werd de plugin van Jörn Zaefferer gebruikt~\cite{Zaefferer2013}. Deze plugin loste ook het probleem met de verplichte velden op. Deze plugin kan op twee manieren gebruikt worden: enerzijds schrijven van CSS klassen in de HTML-code ofwel anderzijds door programmatie in de JavaScript-code. Beide aanpakken werden getest doorheen de POC. 
% TODO verder uitleggen hoe slim de plugin is

\subsubsection{Opladen van bewijs}

\paragraph{jQuery Mobile} 
Het opladen van een bestand kan gebeuren door \code{file} als invoertype van het formulierveld te gebruiken. Voor het kan worden doorgestuurd naar de backend, moet het bewijs eerst lokaal worden omgevormd naar base64. Dit werd geïmplementeerd met de FileReaderAPI en het canvas, wat beide HTML5 specificaties zijn. Het aangeklikte bestand wordt gelezen door middel van de FileReaderAPI, waarna het als afbeelding wordt opgeslagen en geïmporteerd wordt op het canvas. Eenmaal geïmporteerd, kan men de \code{.toDataURL()} oproepen op het canvas om de geïmporteerde afbeelding om te vormen naar base64. Deze aanpak werkt correct op recente mobiele apparaten. De FileReaderAPI wordt echter niet ondersteund op Android versies 2.3 en lager of iOS versies lager dan 6.0~\cite{Deveria2013a}.

\subsubsection{Handtekening}

\paragraph{jQuery Mobile} 
Er werd gezocht naar een plugin om deze functionaliteit te bekomen, doordat jQuery Mobile dit niet standaard aanbiedt. Eerst werd gewerkt met Signature Pad van Thomas Bradley~\cite{Bradley2013}. Door de lange tijd die werd besteed aan het aanpassen van layout, werd overgestapt naar jSignature van Willow Systems~\cite{Systems2013}. Deze laatste gaf ook het voordeel dat de breedte van het gebied om te handtekening in te zetten, zich automatisch naar 100\% schaalde. De plugin maakt gebruik van het HTML5 canvas element en de \code{.toDataURL()} methode, maar deze wordt niet ondersteund op Android versies 2.3 en lager~\cite{Systems2013}.

%%%%%%%%%%%%%%%%%%%%%%%%%%%%%%%%%%%%%%%%%%%%%%%%%%%%%%%%%%%%%%%%%%%%%%%%

\section{Vergelijkscriteria}
\label{sec:evaluatie-criteria}
\chapter{Besluit}
\label{chap:besluit}

\section{Geleerde lessen}
% beter en vollediger uitwerken van criteria op voorhand! niet enkel bekijken als je effectie moet gaan evalueren
% het aantal ongewenste verrassingen blijft zo beperkt

\section{Verbeteringen}
% POC updaten (pull-to-refresh,  meer items laden,  ...) HTML5 features meer toevoegen (GPS, audio,  drag and drop (herorden lijst, lang duwen), carousel met swipe, push eventes 
% toggle entries beter bijhouden


\section{Toekomstig werk}
% ook het criterium uitbreidbaarheid erbij betrekken, want nu komen ST en Kendo niet helemaal tot uiting in onze spidergraph
% onderzoeken ophalen icons in cache 
% onderzoeken crash kendo op ios
% nieuwe frameworks toevoegen + updates van huidige frameworks blijven controleren (resultaten ook updaten)
% methodologie blijven verder toetsen
% subjectieve gebruikservaringstesten met > 5 mensen
% het finale resultaat van de framework bekijken (look-and-feel van kendo, nice dialogs van lungo,...)
% is battery use an issue for web applications?
% vergelijking web / hybrid / native
% windows en blackberry ondersteunen?

\section{Conclusie}
%

%%% Local Variables: 
%%% mode: latex
%%% TeX-master: "masterproef"
%%% End: 



%% Indien er bijlagen zijn:
%\appendixpage*          % indien gewenst
%\appendix
%\include{app-A}
%% ... en zo verder tot
%\include{app-n}

\backmatter
% Na de bijlagen plaatst men nog de bibliografie.
% Je kan de  standaard "abbrv" bibliografiestijl vervangen door een andere.
\bibliographystyle{abbrv}
\bibliography{../Referenties/evaluatie-tex,../Referenties/vergelijking-tex,../Referenties/literatuurstudie-tex}

\end{document}

%%% Local Variables: 
%%% mode: latex
%%% TeX-master: t
%%% End: 
