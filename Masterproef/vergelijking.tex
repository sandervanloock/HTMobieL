\chapter{Vergelijking}
\label{chap:vergelijking}

In dit hoofdstuk bekijken we hoe we de mobiele HTML5 raamwerken vergelijken.
Hoofdzakelijk zullen we dit doen aan de hand van een \term{proof of concept} (POC).
Deze wordt geïntroduceerd in sectie \ref{sec:poc} en zal hoofdzakelijk de gekozen vergelijkingscriteria in sectie \ref{sec:vergelijkingscriteria} drijven.


\section{POC} % TIM (iets van 2 bladzijden)
\label{sec:poc}
In samenspraak met Capgemini werd gekozen voor een applicatie die het mogelijk maakt voor werknemers om hun onkosten via hun mobiel apparaat door te sturen.
Een werknemer kan een foto nemen met de camera 

\section{Vergelijkingscriteria} % SANDER (iets van 2x de bloglengte)
\label{sec:vergelijkingscriteria}

\subsection{Omkadering}
\subsection{Productiviteit}
\subsection{Gebruik}
\subsection{Ondersteuning}
\subsection{Performantie}